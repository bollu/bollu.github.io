\documentclass[11pt]{book}
%\documentclass[10pt]{llncs}
%\usepackage{llncsdoc}
\usepackage[sc,osf]{mathpazo}   % With old-style figures and real smallcaps.
\linespread{1.025}              % Palatino leads a little more leading
% Euler for math and numbers
\usepackage[euler-digits,small]{eulervm}
\usepackage{physics}
\usepackage{amsmath,amssymb}
\usepackage{graphicx}
\usepackage{makeidx}
\usepackage{algpseudocode}
\usepackage{algorithm}
\usepackage{listing}
\usepackage{minted}
\usepackage{tikz}
\usepackage{tikz-cd}
\usepackage{mathtools}

\evensidemargin=0.20in
\oddsidemargin=0.20in
\topmargin=0.2in
%\headheight=0.0in
%\headsep=0.0in
%\setlength{\parskip}{0mm}
%\setlength{\parindent}{4mm}
\setlength{\textwidth}{6.4in}
\setlength{\textheight}{8.5in}
%\leftmargin -2in
%\setlength{\rightmargin}{-2in}
%\usepackage{epsf}
%\usepackage{url}

\usepackage{booktabs}   %% For formal tables:
                        %% http://ctan.org/pkg/booktabs
\usepackage{subcaption} %% For complex figures with subfigures/subcaptions
                        %% http://ctan.org/pkg/subcaption
\usepackage{enumitem}
%\usepackage{minted}
%\newminted{fortran}{fontsize=\footnotesize}

\usepackage{xargs}
\usepackage[colorinlistoftodos,prependcaption,textsize=tiny]{todonotes}

\usepackage{hyperref}
\hypersetup{
    colorlinks,
    citecolor=blue,
    filecolor=blue,
    linkcolor=blue,
    urlcolor=blue
}

\usepackage{epsfig}
\usepackage{tabularx}
\usepackage{latexsym}

\DeclarePairedDelimiter{\ceil}{\lceil}{\rceil}

\newcommand\ddfrac[2]{\frac{\displaystyle #1}{\displaystyle #2}}
\newcommand{\N}{\ensuremath{\mathbb{N}}}
\newcommand{\R}{\ensuremath{\mathbb R}}
\newcommand{\coT}{\ensuremath{T^*}}
\newcommand{\Lie}{\ensuremath{\mathfrak{L}}}
\newcommand{\pushforward}[1]{\ensuremath{{#1}_{\star}}}
\newcommand{\pullback}[1]{\ensuremath{{#1}^{\star}}}

\newcommand{\inj}{\hookrightarrow}
\newcommand{\mono}{\inj}


\newcommand{\sur}{\twoheadrightarrow}
\newcommand{\epi}{\sur}

\newcommand{\pushfwd}[1]{\pushforward{#1}}
\newcommand{\pf}[1]{\pushfwd{#1}}

\newcommand{\boldX}{\ensuremath{\mathbf{X}}}
\newcommand{\boldY}{\ensuremath{\mathbf{Y}}}


\newcommand{\G}{\ensuremath{\mathcal{G}}}
\newcommand{\Set}{\ensuremath{\mathbf{Set}} }
% \newcommand{\braket}[2]{\ensuremath{\left\langle #1 \vert #2 \right\rangle}}


\def\qed{$\Box$}
\newtheorem{corollary}{Corollary}
\newtheorem{theorem}{Theorem}
\newtheorem{definition}{Definition}
\newtheorem{lemma}{Lemma}
\newtheorem{observation}{Observation}
\newtheorem{proof}{Proof}
\newtheorem{remark}{Remark}
\newtheorem{example}{Example}
\newtheorem{exercise}{Exercise}



\title{Topoi, Sheaves, Logic}
\author{Siddharth Bhat}
\date{Monsoon 2019}

\begin{document}
\maketitle
\tableofcontents

\chapter{Topoi of $M$-sets and automata}

Finite automata is just an $M$-set where $M$ is the alphabet.
Subobject classifier for $M$ for a set $A \subseteq X$, $\Xi_A$
maps elements $ x \in X$ 
to the left ideal $\Xi_A(x) \equiv \{ m \in M : m \curvearrowright x \in A \}$.

So, given a "subset automata" $A \subseteq X$, which is a state of states closed under
reachability, $\Xi_A$ assigns to that subset the entire language, and to
other states $x \in X$ the set of strings with prefixes necessary to reach the subset $A$.

What is conjunction and disjunction? Existential and universal quantification?

The adjunction between $Set$ and $Rel$ should allow us to define $M-Rel$, as
well as derive the NFA to DFA algorithm.

\chapter{7 sketches}
I'm following Seven sketches of compositionality. This is a transcription
of the four parts of lecture 7, which promises to cover sheaves, topoi, and
internal logics.

Consider a plane, that has a relationship $\texttt{dial} \mapsto \texttt{thruster}$.
We want this to represent some logical statement 
$\forall t \in Time, @_t (\texttt{dial} = \texttt{bad}), \exists r \in \R, 0 < r < 1, st. @_{t + r} (\texttt{thursters} = \texttt{on})$
We need some category where this logical statement lives. How do we define this
$@$ operator and things like that in the category?

We begin by studying the category of sets, where we normally interpret logic.
Then, we move to richer logics, for example, those with truth values other than
$\top, \bot$. For example, we can have things like $(\top~\text{ for 0 < t < 10})$.

\chapter{Logic in the category Set}

Objects are all sets, arrows are functions.

\section{Properties of \Set}
\begin{itemize}
    \item 1. \Set has limits and colimits. (eg. empty set / initial object, disjoint union / coproduct, pushouts)
    \item 1.1 \Set has terminal objects: the single element set. eg: $1 = \{ \star \}$. (it's unique upto unique isomorphism).
    \item 1.2 \Set has products: $X, Y \in Set \implies X \times Y \in Set$. $X \times Y \equiv \{ (x, y) | x \in X, y \in Y \}$
    \item 1.3 \Set has pullbacks: Given functions $f : X \to A, g: Y \to A$, we can create 
                $X \times_A Y \equiv \{ (x, y) | f(x) = g(y) \}$

        \begin{tikzcd}
            X \times_A Y \arrow[d, dotted, "!"] \arrow[r, dotted, "!"] \arrow[dr, phantom, "\lrcorner", very near start] & X \arrow[d, "f"]\\ 
            Y \arrow[r, "g"] & A
        \end{tikzcd}
        where the $\lrcorner$ means that the square is a pullback square.

    \item 2. Set has epi-mono factorizations: given $f: X \to Y$, we can get
        $epi: X \sur Im(f)$, $mono: Im(f) \inj Y$. $epi$ is surjective, $mono$ is injective.

        \begin{tikzcd}
            X \arrow[rd, "f"] \arrow[r, two heads] & Im(f) \arrow[d, hook]\\ & Y
        \end{tikzcd}
    \item 3. Internal hom: $\Set(A \times B, C) \simeq \Set(A, C^B)$ ($C^B \equiv \text{functions $B \to C$}$),
        where $Set(X, Y)$ is the hom-set.
    \item 4. Subobject classifiers: a \textbf{subobject} of $X \in C$ is (a subset in Set), 
        is an equivalence class of monomorphisms $A \hookrightarrow X$. 
        Given two monomorphisms $f: A \inj X, g: B \inj X$, The subobject will specify if $(f \sim g)$.

        \begin{tikzcd}
            A \arrow[rd, hook] \arrow[d, equal] \\ B \arrow[r, hook] & X
        \end{tikzcd}

        A \textbf{subobject classifier} in $C$ is an object $\Omega$ and a map $1 \rightarrow \Omega$, where $1$ is the terminal object,
        such that for all subobjects, $m: A \mono X$, there exists $\ceil m: X \to \Omega$ such that:

        \begin{tikzcd}
            A \arrow[r, "!"] \arrow[hook, d, "m"]
            \arrow[dr, phantom, "\lrcorner", very near start] & 1 \arrow[d, "true"] \\
            X \arrow[r, "\ceil m"] & \Omega
        \end{tikzcd}

        Where $\ceil m$ is called as the \emph{classifier} of $m: A \rightarrow X$
\end{itemize}

The subobject classifier in $\Set$ is $\Omega = \{ true, false \}$. $(true: 1 \rightarrow \Omega; * \mapsto true)$

We now classify a subobject, the even numbers of \N. let $E = \{ 0, 2 \dots \} \subseteq \N$.
To classify this, we have the commuting square:

\begin{tikzcd}
E \arrow[r, "!"] \arrow[hook, d, "m"]
\arrow[dr, phantom, "\lrcorner", very near start] & 1 \arrow[d, "true"] \\
\N \arrow[r, "\ceil m"] & \Omega
\end{tikzcd}

What is $\ceil m: \N \rightarrow \Omega$? It's going be 
$\ceil e = \begin{cases} true & e \mod 2 = 0 \\ false & \text{otherwise} \end{cases}$.

We need to check that this is indeed a pullback. This clearly commutes, but
we need to check that it's the most general solution.

(It needs to be a pullback so we have a one-to-one correspondence between $E$ and $\ceil e$, apparently.
I don't see it.)

We call a morphism $(X \rightarrow \Omega)$ as a \textbf{predicate}. (This is clear in $\Set$).
Intuitively, previously, $\ceil e$ was a predicate.  The subobject classifier
allows us to find $E$ given the $\ceil e$, thereby find the semantics (as it
were) from the predicate.

\section{Logical operations}

\subsection{And}
\begin{tikzcd}[row sep=large, column sep=15ex]
    1 \arrow[d, hook, "{(true , true)}"]  \arrow[r, equal] \arrow[dr, phantom, "\lrcorner", very near start]  & 1 \arrow[d, "true"]\\
    \Omega \times \Omega \arrow[r, "{ \land \equiv \ceil{(true, true)}}"] & \Omega
\end{tikzcd}

We first draw the LHS, and we get the RHS by apply the subobject classifier
onto the LHS!  $(\land: \Omega \times \Omega \rightarrow \Omega)$. 


\textbf{Question:} How do we compute $\land(false, false)$?

We know that the pullback is ging to be
\end{document}
