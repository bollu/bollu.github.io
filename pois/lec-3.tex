\section{Information Theory}

\subsection{shannon's perfect secrecy (1949)}
Shannon framed a secrecy theory based on information theory. If no
information is revealed to the other person, it is secure.

Cipher = $<Gen, Enc, Dec, M>$.

$M$ is the message space.

$Gen :: KeyLength -> Key$. Set of all keys $Gen$ can output
($Image(Gen)$) is called the key space. Key space ($K$)is asked to be finite.


$Enc :: M -> K -> C$. $C = ciphertext$. $Image(Enc)$ is called the ciphertext space.
$Enc$ takes a message and a key, and returns a ciphertext.

$Dec :: C -> K -> M$. such that $\forall (m: M) (k: K) Dec(k, Enc (m, k)) = m$.

 
A cipher scheme is secure iff: $\forall p = \text{probability distributions over the message space}$
(we don't know the exact probability distribution over
plaintext).
$\forall m \in M, \forall c \in C, P(C = c > 0) => P[M=m] = P[M=m | C=c]$.

What we know about the message before looking at the
ciphertext is the same as what we know about the message \textbb{after} we know the ciphertext.
We make sure that we do not take degenerate $c$ ($c$ that does not ever occur) to prevent nastiness in
conditional probability.
