\documentclass[11pt]{article}
%\documentclass[10pt]{llncs}
%\usepackage{llncsdoc}
\usepackage{amsmath}
\usepackage{graphicx}
\usepackage{makeidx}
\usepackage{algpseudocode}
\usepackage{algorithm}
\evensidemargin=0.20in
\oddsidemargin=0.20in
\topmargin=0.2in
%\headheight=0.0in
%\headsep=0.0in
%\setlength{\parskip}{0mm}     
%\setlength{\parindent}{4mm}
\setlength{\textwidth}{6.4in}
\setlength{\textheight}{8.5in}
%\leftmargin -2in
%\setlength{\rightmargin}{-2in}
%\usepackage{epsf}
%\usepackage{url}
\usepackage{epsfig}
\usepackage{tabularx}
\usepackage{latexsym}
\newtheorem{lemma}{Lemma}
\newtheorem{observation}{Observation}
\newtheorem{proof}{Proof}
\newcommand\ddfrac[2]{\frac{\displaystyle #1}{\displaystyle #2}}

\def\qed{$\Box$}
\def\proof{\textit{Proof. }}
\newtheorem{corollary}{Corollary}
\newtheorem{theorem}{Theorem}

\begin{document}
\section{Agreement in a distributed system}

\section{Introduction: Philosophy of today's class}
This is the beginning of ``clash of philosophies''.
Agreement in a distributed system can be impossible! (which is why it is part of this course).
If one-way-functions exist, then much more is possible.

Different adversaries:
\begin{itemize}
\item Computational Adversary - All modern crypto
\item Practical Adversary - Noise based crypto (last class)
\item Natura Adversary - Quantum crypto (after mid-2)
\item Philosophical Adversary - ??? (So kannan, man)

  Our problem is a fundamental problem in distributed systems. Problems of agreement abound.
  There are instances where distrubuted systems textbooks give up, and give proofs of no-solutions.
  However, this is POIS class, so impossibility proofs are where we start our trade.

  
  Djikstra's paper in 1980 where consensus is not possible: three nodes connected in a synchronous network.
  We want to simulate a broadcast channel over P2P. That is, we wish to broadcast information over P2P.

  The problem is called ``byzantine agreement problem''.

  Each party has a single bit of input $x_i$. Each will give a single bit of output $b_i$ ($i \in \{ 1, 2, 3 \}$).
  Agreement  requirement - All non-faulty nodes have the same output (all $b_i$ are equal).

  If this was the \textit{only goal}, then this is trivial. We just have all of them output a constant.
  Validity requirement - The output of all non-faulty nodes is equal to the input of \textit{some} non-faulty node.


  We can simulate broadcast. Run the protocol that has agreement and valitiy. We want to simulate broadcast. We send $0$ to everybody,
  and then simualate the protocol that we have crafted. Since everyone has started with the same input, everyone will end with the
  same output.

  Is this actually such a tough problem? Why can't we just say ``send your input to everybody''? Why is this not equivalent to simulating
  a broadcast channel?


  Let's see what happens. Let $A$ be a source. Say $A$ sends message $m$ to $B$ and $C$. this looks like $A$ has broadcasted message $m$.
  If this were happening over a \textit{broadcast channel}, then $m$ would have been received by $B$ and $C$.

  However, on a unicast channel, it is possible that $A$ fails between $B$ receiving the message and $C$ receiving the message.
  Now, this is \textit{not broadcast}.

  We need broadcast to be atomic - either everyone gets the message, or no one gets the message!

  assume $A$ is adverserial: that is, $A$ is able to send $m_1$ to $B$, $m_2$ to $C$, $m_1 \neq m_2$. So, we need to make sure
  that people can't fuck up broadcast over unicast. However, if we had a physical broadcat channel, this fuck up can't happen.


  So, the problem \textit{is complicated}. We wish to simulate an atomic broadcast channel over a unicast channel, given
  adverserial nodes.
\end{itemize}

\section{Three nodes, one of them is byzantine faulty, no way to have both agreement and validity}
\subsection{Intutition}
Say $A$, which is byzantine faulty sends $0$ to $B$ and $1$ to $C$. $B$ and $C$ now exchange values,
they realise agreement has not actually happened.

If they \textit{know} that $A$ was faulty, then they could agree that A was faulty.

However, $B$ does not know whether $C$ is faulty or not. It could be that $A$ had sent
$0$ to $C$, but $C$ was faulty and thus chose to broadcast $1$ to $B$.

So, $B$ only knows that \textit{one of} $A$ and $C$ is faulty.
Similarly, $C$ only knows that \textit{one of} $A$ and $B$ is faulty.

\subsection{Proof}
Let $\Pi$ be a byzantine agreement protocol for this case.

Running with inputs $x_1, x_2, x_3$, we get outputs $b_1, b_2, b_3$, we have the guarantee that
1. all b_i are equal
2. b_i is equal to some x_i


We now show that such a $\Pi$ cannot exist.

The original proof is difficult. Then, at MIT, a new proof technique came out in distributed systems
which proved this. (Proof is called "hexagon proof"). 

Proof strategy:
\begin{itemize}
    \item Given that $\Pi$ is a byzantine agreement protocol.
    \item We create some other problem with some other network, and show that if $\Pi$ is a byzantine
        agreement protocol in this network, then that protocol should do *something* in the other network.
    \item We show that the protocol does not do anything (is not consistent)
\end{itemize}


Consider a network of six nodes in a ring topology (hexagon) (Call this network $Hex$)
Call the original network $Tri$.

The code delegated to three parties in $Tri$ is $\Pi = < \Pi_1, \Pi_2, \Pi_3>$.

$\Pi'$ is the program for $Hex$.  $\Pi' = <\Pi_1, \Pi_2, \Pi_3, \Pi_1, \Pi_2, \Pi_3>$
(That is, node $i$ gets program $\Pi'_i$).


In $Tri$, $\Pi_1$ got messages from $\Pi_2 , \Pi_3$. This is
the same structure in $Hex$. So, structurally, the networks are the same.
Syntactically, the code will work in $Hex$ - because the
"local" network topology is the same. So, $\Pi_$ running at node $1$ cannot 
distinguish if it is at $Tri$ or $Hex$.

Proof by induction on rounds. At round $0$, we cannot distinguish since
structurally, the locales are the same. Then, induction on rounds.

We now show that $Hex$ is an inconsistent protocol.

Input $0$ for nodes $1, 2, 3$, $1$ for nodes $4, 5, 6$.


Consider $(2, 3)$ in $Tri, Hex$. $(2, 3)$ in $Tri$ will be talking to $1$.

Say $1$ is faulty in $Tri$ (we can do this since Pi is a BA protocol).
$1_tri$  will send what goes from $4_hex$ to $3_hex$ to $3_tri$.
$1_tri$  will send what goes from $1hex$ to $2_hex$ to $2_tri$.

So, $2, 3$ do not know if there are working in $Tri$ or in $Hex$, because
the messages are the same!

We know that in $Tri$, there is a BA protocol, so it should terminate.
Moreover, it should terminate with $0$ since we assumed that the protocol
is a BA protocol.

Also, this means that output will be $0$ in $Hex$ for $2, 3$ as well, since
they are structurally isomorphic.


Now, let us look at $4, 5$ in $Hex$ (which has protocols $\Pi_1, \Pi_2$).
Let us say that $3'$sends $\alpha$ to $2'$ and \beta to 1'.

This is equivalent to a faulty 3 in Tri.

We know that the output must be 1 for 1', 2' from our starting conditions.


Now, looking at $(1, 3)$, once again, if 2 is faulty, make outputs the
same way such that they can't distinguish if they are running in Tri or in Hex.

In this case, 1' will have to output 0, since 3 outputs 0. However,
in the last case, we had agreed that 1' would have to output 1.

Conditions:
- 3 outputs 0
- 1' outputs 1
- output(3) = output(1').

Hence, such a $\Pi'$ cannot exist. Hence, $\Pi$ cannot exist.


\section{Clash of philosophies}

\subsection{Distributed systems spirit}
We use the term "non faulty" when defining Agreement, Validity. Originally,
the definition was, "if I gave a node \Pi, then it should execute \Pi,
not something else (\Pi')".

Kannan - anyone who contributes to the execution deserves the output

\subsection{Clash}

Assume there is a player who is participating in two programs, and
is non-faulty.

Suppose it is a byzantine agreement program (1 out of 3), which is impossible.

We force everyone to digitally sign their messages. Now, impossibility becomes
possible.

Now, if I was supposed to run Pi (which says sign your messages), so I run it.


In the background, my friend asks me to share my secret key with him, which I do.
This person is also part of the BA network, which causes my key to be revealed,
and thus the adversary can forge my signature, thus the protocol breaks.


Should the node in question be called faulty or non-faulty?

Assume we call this node faulty. This makes sense, because he is breaking our
crypto assumptions of key-privateness.

However, distributed systems is of the opinion that anyone who shares their
resources must reap the rewards. So, from a distributed systems viewpoint,
the node is non-faulty.

Now, the notion of faulty and non faulty depends on the implementation detail!


\subsection{Non modular attacks}
Assume a byzantine agreement protocol between 3 people. It solves the problem
using digital sinatures. Call this protocol $\Pi$.


They're running two byzantine agreement protocols in parallel.

Call the parties "1 = 1', 2 = 2', 3 = 3'", so we have $\Pi$ running
between $1, 2, 3$, and another instance of $\Pi$ running between
$1', 2', 3'$. (makes sense, two agreement protocols are running in parallel).


Let the input be $0$ for (1, 2), and $3$ is corrupt in $\Pi$.

Let the input be $1$ for (1', 3') and $2'$ is corrupt in $\Pi'$.


Adversary can fail \textit{both} protocols (Why?)

The adversary has access to the private key of 2' (which is the private key
of 2 as well). Since all processes receive the same private key, the adversary
in 2' has access to the private key of 2.

For example, 2' creates (1, signed 1). The adversary in 2' will drop the packet
that 2 tried to send to 1 (which was supposed to be (0, signed 0)),
and sends the (1, signed 1).

\subsection{Processes}

We assume that 2 processes do not have the same PID.


Consider three distributed processes $1, 2, 3$. We have no global PID
assuming one of these are faulty, since we showed that it is impossible to agree
on even a single bit.
So now, we have three PIDs, $PID_1, PID_2, PID_3$.


Let us say server, client are connected across two ports P, P'. In theory,
an adversary can send messages from P to P', and P' to P (flip P and P').


\subsection{Correctness versus Security}

Correct HW on top of a correct OS on top of a correct compiler on top of a 
correct program will give us a correct program.

In seurity,
Secure HW on top of a secure OS on top of a secure compiler on top a secure
program will give us a secure program (?)

This is indeed untrue.
Consider the distributed process with same PIDs case. the attack is neither
fully OS based nor network based. the router can arbitrarily swap PIDs 
acting as an adversary. We cannot construct consensus due to the byzantine
agreement theorem we have.

We can have cross-model attacks.

\section{Crypto, enter the stage}

The idea is that the adversary needs to simulate the hexagon protocol to 
construct a contradiction. However, if we can ensure that this process
is computationally intensive, then life is chill, because we cannot
have such an adversary.

So, we force people to sign their messages, so that the adversary
can't simulate.




\end{document}
