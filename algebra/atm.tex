\documentclass{book}
\usepackage{amsmath}
\usepackage{amssymb}
\newcommand{\R}{\ensuremath{\mathbb{R}}}
\newcommand{\C}{\ensuremath{\mathbb{R}}}
\begin{document}
Atiyah macdonald solutions. Borcherds videos:
\url{https://www.youtube.com/watch?v=QOTf8KfrZFU&list=PL8yHsr3EFj53rSexSz7vsYt-3rpHPR3HB}

\chapter{Ch1}

1. $1/(x+1) = 1 - x + x^2 + ...$. Series truncates because $x$ is nilpotent,
   gives us an honest inverse. To show that the sum of nilpotent and unit is
   unit, consider $u + n$. Write as $u(1 + u^{-1}n)$. $u^{-1}n$ is nilpotent
   (ring is commutative, take large power and rearrange to exhibit nilpotence), so $(1 + u^{-1}n)$ has an inverse,
   hence $u(1 + (u^{-1}n)$ as the product of unit is a unit.

2. Let $f = \sum_{i=0}^n a_i x^i$ be a unit. Let $g = \sum_{j=0}^m b_j x^j$ be the inverse of $f$.
   Thus, $fg = 1$. But this means $\sum_{i,j \geq 0; i + j = k} a_i b_j = [k = 0]$ (that is, using
   Iversion notation, $[k = 0] \equiv 1$ if $k = 0$, and $0$ otherwise).
   Thus, set $k = 0$. We get that $a_0 b_0 = 1$, or that $a_0$ is a unit.

3. Nilradical = sqrt(0) = zero divisors. Jacobson radical = intersection of all maximal ideals.

\chapter{Ch2}

\section{Proposition 2.3}
$M$ is finite generated over $A$ iff $M$ is isomorphic to a quotient of $A^n$.

Let $M$ be finitely generated by $n$ generates $m[1], \dots, m[n]$. create a mapping
$\phi: A^n \rightarrow M; \phi(\vec a) = \sum_i a[i] m[i]$. This is surjective on $M$
as every element in $M$ can be written as an $A$-linear combnation of $m[i]$. Hence,
$Im(\phi) = M \simeq A^n/ker(\phi)$. So, $M$ is a quotient of $A^n$.


Let $M$ be isomorphic to a quotient of $A^n$. Pick the elements $g_i \equiv \delta_i^j \in A^n$.
That is, it is the element with a $1$ at $i$ and $0$ everywhere else. Any element of $A^n$
of the form $(a[1], a[2], \dots, a[n])$ can be written as $a[1]g[1] + a[2]g[2] + \dots a[n]g[n]$
and hence $M$ is finitely generated by the $g[i]$.



\section{Proposition 2.4}
\subsection{Theorem}
Let $M$ be finitely generated. Let $J$ be an ideal of $A$, and let $\phi: M \rightarrow M$ be an $A$-module
endomorphism of $M$ such that $\phi(M) \subseteq JM$. Then $\phi$ satisfies an equation

$$
\phi^n + j[1] \phi^{n-1} + \dots + j[n] = 0
$$
where the $j[1], j[2], \dots \j[n] \in J$.  
That js, jf $\phi$ takes all of $M$ to be entjrely wjthin $JM$, we have a polynomjal $q \in J[x]$ such that $q(\phi) = 0$.


\subsection{Making sense}
To understand this, take some abelian group $G$ as a $\mathbb Z$ module. Now if $\phi$ takes $G$ to $(j)G$ where
$(j) = J$ is some ideal, then we should be able to write a $J = (z)$ based relationship between $\phi$, so something
like

$$
\phi^n + (jz[1]) \phi^{n-1} + \dots + jz[n] = 0
$$
where the $jz[\cdot] \in J, z \in \mathbb Z$ (every element of $J = (j)$ is of the form $j\mathbb Z$).

This seems eminently reasonable: live by the $J$, die by the $J$. If your entire action is concentrated inside $J$,
J (heh) should be able to kill you with $J$-based relationships.


\subsection{Proof}

The idea is that since we have a basis for $M$, we can write $\phi$ as a matrix. Since the action of $\phi$
is to take elements into $JM$

\section{Corollary 2.5}
Let $M$ be a finitely generated $A$-module and let $J$ be an ideal of $A$ such that $AJ = J$. Then
there exists a $a \equiv 1 (\mod J)$ such that $aM = 0$.


\subsection{Making sense}
For example, imagine that $A \equiv C[X, Y]$, and let $J \equiv (x) \subseteq A$, points on
on the $y$-axis (or functions that vanish on the $y$-axis, as you like). If $AJ = J$, then
this means that the vector fields (module) in $A$ also vanish on $J$, because making $AJ$ kills all vectors
along the $x$-axis, but this keeps $A$ the same.
So $A$ is a module of vector fields that vanishes on the $y$-axis. 
The the corollary asserts there must exist an element $a$ that does not vanish on $J$ (ie, $a \equiv 1 (\mod J)$)
such that $aM = 0$. This means that $a$ should vanish everywhere on the *support* of all vector
fields in $M$. That is, $a$ gives us a "bump function" that is 1 over $I$, and zero over all vector fields
in $M$.

\subsection{Proof}
Set $\phi: A \rightarrow A; \phi(a) \equiv a$. Since $JM = M$, and $\phi(M) = M = JM$, then since $\phi $ lives by the $J$,
it must die by the $J$, and so there are coefficients $j_i \in J$ such that
$\phi^n + j_1 \phi^{n-1} + \dots + j_n = 0$. See that

\begin{align*}
&\phi^n + j_1 \phi^{n-1} + \dots + j_n = 0 \\
&(\phi^n + j_1 \phi^{n-1} + \dots + j_n)(a) = 0a = 0 \\
&(\phi^{n}(a) + j_1 \phi^{n-1}(a) + \dots + j_n a) = 0 \\
&(a + j_1 a + \dots + j_n a) = 0 \\
&(1 + j_1 + \dots + j_n a) = 0 \\
\end{align*}

Call the element $1 + j_1 + \dots + j_n \equiv x$. Clearly, $x \equiv 1 (\mod J)$, and 
this function annhilates $M = JM$, as it annhilates $J$. This gives us our
bump function of interest.

Alternatively, just use Cayley-hamilton on rings, because the above theorem is cayley-hamilton on rings `:P`

\section{Corollary of vanishing: vanish at an ideal is vanish at a function}

If $M$ is finitely generated and $I$ an ideal such that $IM = M$ then there is an element $i \in I$
such that $i \equiv 1 (\mod I)$ such that $iM = 0$.

\subsection{Proof}

Pick $\phi(x) \equiv x$ and pick $x \equiv 1 + \sum_i a_i$.

\section{Nakayama's lemma, form 0}
Let $M$ be a finitely generated $A$ module. Let $I$, an ideal of $A$, be contained in the Jacobson
radical $R$ of $A$ (That is, $I \subseteq R$).  Then $IM = M$ implies $M = 0$.

\subsection{Making sense}

Take functions that vanish on all "real points" (jacbonson radical is the intersection of all maximal 
ideals, so functions in the jacobson radical are those that vanish at "all real points"),
call these $I$. If scaling the module by these, that is, annhilating the module at "all points"
preserves the module, then the module is identically zero.

Alternatively, think of a graded ring like $R[x]$ and a graded module $M$.
If $I$ is nontrivial, like $I = (x)$, then multiplying
by $I$ will push the grading of $M$. So the "smallest element" in $M$ will no longer be reachable by $IM$.
Hence, if $IM = M$, $M$ does not possess a (non-trivial) "smallest element", and thus $M = 0$.

\subsection{Algebraically making sense}

If we have $IM = M$, then we rearrange into $(1 - I)M = 0$. But if $I$ is in the Jacobson radical,
then we can imagine that $(1 - I)$ is "a unit" (as in the Jacobson radical, this happens elementwise),
and thus we deduce that $M = 0$, since multiplying by a unit can't destroy information; Let $J$ be the
inverse to $(1 - I)$, and thus $J(1 - I)M = J0 = 0 \implies M = 0$.

\subsection{Proof 1: Apply Cayley Hamilton}

Since $IM = M$, $M$ lives by the $I$. It must thus die by the $I$. So, we must have
a bump function that is $1$ outside $I$ that kills $M$: so there exists
an $x \equiv 1 (\mod I)$ such that $xM = 0$. Hence, 

\subsection{Proof 2: Generators}

Let $M \neq 0$, and let $u_1, \dots, u_n$ be a minimal set of generators for $M$. Then $u_n \in IM$ by 
hypothesis. Hence we have an equation $u_n = i_1 u_i  + \dots i_n u_n$ will all $i_n \in I$.
Hence,

$$
(1 - i_n) u_n = i_1 u_1 + \dots + i_{n-1} u_{n-1}
$$

Since $i_n \in R$, we must have that $(1 - i_n) \in \mathbb R$ by characterization of Jacobson.
Hence, $(1 - i_n)$ is a unit. So, $u_n$ is un-necessary, as it is generated by $u_1, \dots, u_{n-1}$, which is
a contradiction.


\subsection{Characterization of jacbonson radical}

Let $R$ be a ring. For every element $j \in J$ (the jacbonson radical), $(1 - j)$ is a unit.
Define $I \equiv (1 - j)$ the ideal generated by $(1 - j)$.

The non-nuke proof is to consider the element $(1 - j)$. If $(1 - j) \in m$ for some maximal ideal $m$, then
since $j \in m$ for all maximal ideals, $(1 - j) + j \in m$ for the maximal ideal $m$,
hence $1 \in m$, contradicting maximality. Thus $(1 - j)$ is not in any maximal ideal.
Consider the ideal $I \equiv ((1 - j))$. Since $(1 - j)$ is not in any maximal ideal, the
ideal $I$ too cannot be contained in any maximal ideal (For contradiction, assume $((1 - j )) \in I \subseteq m$.
Then $(1 - j) \in m$, a contradiction). Thus, $I = R$, hence $(1 - j)$ is a unit.

The nuke proof is to consider the exact sequence:
$0 \rightarrow I \rightarrow R \rightarrow R/I \rightarrow 0$. We wish to
show that $I = R$ or $R/I = 0$. To show this, we will show that this is true at the localization of
each maximal ideal $m$.  If something holds for each maximal ideal, then it holds everywhere.
The exact sequence for the ring vanishing is
$0 \rightarrow I_m \rightarrow R_m \rightarrow R_m / I_m \rightarrow 0$. $j$ is in every maximal ideal, 
so $j \in I_m$, so $j$ will be contained in the only ideal of $I_m$. Now consider $(1 - j)$. If $(1 - j) \in I_m$,
then $(1 - j) + j \in I_m$ or $1 \in I_m$. This collapses the ring, and thus $1 - j = 0 = 1$ and is thus
a unit. If $(1 - j) \not \in I_m$, then it's a unit because everything outside of $I_m$ has been localized, and is
thus a unit. So, $(1 - j)$ is a unit locally for each maximal ideal, and is thus a global unit.


\section{Lift of vector space basis}

Let $A$ be a local ring with only maximal ideal $m$.
Let $x_i$ be elements of $M$ whose images in $M/mM$ form a basis of this vector
space (why is this a vector space? Answer: because it's a vector space over $R/m$). 
Then the $x_i$ generate $M$ --- so we can go from "basis of the vector space at
a point" to "generating set of vector fields in a neighbourhood" of module of
vector fields.

\subsection{Making sense}

What is $M/mM$? It should be the quotient module obtained by quotienting $M$ with $mM$. I'm not sure how to prove that
this is a maximal ideal.
Let $N$ be the submodule of $M$ generated by $x_i$.

\subsection{$IM$ is a submodule of $M$ for ideal $I$}

Let $R$ be a ring, $I$ an ideal of $R$, $M$ a module. We define $IM \equiv gen(\{ im : i \in I, m \in M \})$.
That is, generate a module from these elements.
Let $i_1 m_1, i_2 m_2 \in IM, r_1 r_2 \in R$. Consider:

\begin{align*}
& r_1 (i_1 m_1) + (r_2) (i_2 m_2) \\
& = (r_1 i_1) m_1 + (r_2 i_2) m_2 \\
&=  i_1' m_1 + i_2' m_2 \in IM
\end{align*}

The final expression is in $IM$ since it's the sum of two elements in $IM$.

\subsection{Quotient module}

Let $M$ be a module, $N \subseteq M$ a submodule of $M$. Define the equivalence relation on
to be  $m_1 \sim m_2 \equiv m_1 - m_2 \in N$. Let's check that this well defined for addition and scaling with the ring:

\begin{align*}
&m_1 \sim m_1'; m_2 \sim m_2' \\
&m_1 - m_1' = n_1 ; m_2 - m_2' = n_2 \\
&(m_1 + m_2) - (m_1' + m_2') = n_1 - n_2 \in N \\
&m_1 + m_2 \sim m_1' + m_2' \\
\end{align*}

For scaling:

\begin{align*}
&m_1 - m_1' = n_1 \\
&rm_1 - rm_1' = r(m_1 - m_1') = rn_1 \in N \\
&rm_1 \sim rm_1'
\end{align*}

\subsection{Module over quotient ring}

If $M$ is an $R$-module, and $I$ is an ideal of $R$, then we want to define a module structure on $M/IM$ as a module
on $R/I$. We need to check that the action is consistent with the equivalence relation. We define
the addition to be $(r + I)(m + IM) \equiv rm + IM$. This is consistent across the equivalence classes. Handwavy:

$$
\begin{align*}
&(r + I)(m + IM) \\
&= rm + rIM + IM + IIM \\
&= rm + (rI)M + IM + (II)M \\
&= rm + IM + IM + IM \\
&= rm + IM
\end{align*}
$$


\subsection{Proof of lift}

$A$ is a local ring with maximal ideal $m$. So intuitively, we're looking at some local neighbourhood.
We have elements $x_i$ such for $\phi: M \rightarrow M/mM$, it's $\phi(x_i)$ that generate the vector space
$M/mM$ over field $R/m$ (is a field because quotient by maximal ideal; This is the tangent space).
Let $X \subseteq M$ be the submodule generated by the $x_i$ (This is the local vector field module).
Now consider the map $X \rightarrow M \rightarrow M/mM$. It maps $X$
onto $M/mM$ (since all the generators $x_i$ of $M/mM$ are in $X$). We claim that $M = X + mM$.
Consider an element in $M$. Its image in $M/mM$ is determined by some value in $X$. The "leftover data"
is eliminated by $mM$, of which a copy we need to recover the precise element in $M$. 

Now, since $M = X + mM$, we want to conclude that $X = M$. Quotient by $X$. We get $M/X = X/X + mM/X$.
This gives us $M/X = mM/X$. By nakayama (Why is $m$ in the Jacobson radical of $A$? because $A$ is a local ring), $M/X = 0$, or $X = M$.

\section{Variants of Nakayama}
\href{http://math.stanford.edu/~conrad/210BPage/handouts/math210b-Nakayama-Lemmas.pdf}

\section{Nakayama's lemma, form 2: $IM = M \implies iM = M$.}
If $M$ is finitely generated such that $IM = M$ then there exists an $i \in I$ such that $iM = M$.

From this version, we can rewrite as $(1 - i)M = 0$. If $(1 - i)$ is a unit (which is the case if $i$ is in
the jacobson radical), then we get $M = 0$.

\url{https://math.stackexchange.com/a/3177237/261373}


\section{Exact sequences}


\end{document}
