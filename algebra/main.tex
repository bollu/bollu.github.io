\documentclass{book}
\usepackage{mathtools}
\usepackage{amsmath}
\usepackage{amssymb}
\usepackage{amsthm}
\newcommand{\N}{\ensuremath{\mathbb{N}}}
\newcommand{\Z}{\ensuremath{\mathbb{Z}}}
\newcommand{\Q}{\ensuremath{\mathbb{Q}}}
\newcommand{\C}{\ensuremath{\mathbb{C}}}
\newcommand{\frakp}{\ensuremath{\mathfrak{p}}}
\newcommand{\fraka}{\ensuremath{\mathfrak{a}}}
\DeclarePairedDelimiter\ceil{\lceil}{\rceil}
\DeclarePairedDelimiter\floor{\lfloor}{\rfloor}
\DeclarePairedDelimiter\fracpart{\{}{\}}
\theoremstyle{definition}
\newtheorem{theorem}{Theorem}
\newtheorem{example}[theorem]{Example}
%\newtheorem{proof}[theorem]{Proof}
\begin{document}
\chapter{Gaussian integers}
% math e222 L30 20031201 
\section{Recap: Euclidian Algorithm}
For any $a, b \in Z$ with $|b| < |a|$, we can decompose $a$ as
$a = \alpha \cdot b + r$ where $0 \leq r \lneq |d|$. This immediately implies
certain facts about the structure of ideals in \Z. 

\begin{theorem}
every ideal $I \neq 0$ in \Z~is principal. The generator of $I$
is the smallest positive integer in the ideal. Formally: $I = (\min\{ d \in I : d > 0 \})$.
\end{theorem}


\begin{proof}
Let $i \in I$ be a general element. Find its decomposition into $d$ using the
Euclidian algorithm as $i = \alpha \cdot d + r$.
Reasoning by ideals:

\begin{align*}
&\forall i \in I, \exists \alpha, r \in \Z, |r| \lneq d, \quad i = \alpha \cdot d + r \\
&\{ \text{writing in ideal notation,} \} \\
&\exists r \in \Z, r \not \in I,  \quad I \subseteq Z \cdot d + r \\
&\{ \text{since $I = (d)$,} \} \\
&\exists r \in Z, r \not \in \Z, \quad I \subseteq  I + r \\
&\implies r = 0
\end{align*}
\end{proof}

\begin{theorem}
Ideal $I = (a, b)$ is a principal ideal $I = (gcd(a, b))$.
\end{theorem}
\begin{proof}
We already know that every ideal $I$ is generated by its smallest positive number $d$.
We will show that $d = gcd(a, b)$. We first show that $d$ is a divisor of $a$, and
a divisor of $b$.
Since $a \in (a, b) = I = (d)$, we know that $a = \alpha \cdot d$
for some $\alpha \in \Z$. Hence $d$ divides $a$. Similarly, $d$ divides $b$.
To show that $d$ is the \emph{greatest common divisor}, let there be another
divisor common divisor $d'$ which divides $a$ and $b$:

\begin{align*}
& d \in I = (a, b) \implies d = ma + nb \quad \text{(Any element in $I$ can be written as $ma + nb$)} \\
&d' | a \implies d' | m a,d' | b \implies d' | n b \\
&d' | m a \land d' | n b \implies d' | [(m a + n b) = d]\\
& d' \leq d \quad \text{(A divisor of a number must be less than or equal to the number)}
\end{align*}             

Hence, $d = gcd(a, b)$.
\end{proof}


\begin{theorem}
If $p$ is a prime and $p | ab$ then $p | a$ or $p | b$.
\end{theorem}
\begin{proof}
We know that $gcd(a, p) = p \lor gcd(a, p) = 1$, since the only divisors of
$p$ are 1 and $p$ itself. If $p | a$ then we are done.
If $p \not| a$, then $gcd(a, p) \neq p$, and we must have $gcd(a, p) = 1$.
This means that $1 = \alpha a + \beta p$. Multiplying throughout by $b$, 
we get that $b = \alpha (ab) \beta (pb)$. We know that $p | ab$, and clearly
$p | pb$. Hence, we must have that $p|(ab + pb)$. Therefore, $p|b$.
\end{proof}


\begin{theorem}
Every integer $z$ has a unique decomposition into a product of primes of
the form ${z = \pm p_1 p_2 \dots p_n}$.
\end{theorem}
\begin{proof}
Proof by induction on the number of factors and using the property that if
$p|ab \implies p|a \lor p|b$. We prove this by induction on the size of
the number. It clearly holds for $2$ since $2$ is prime. Now, let us assume
it holds till number $n$. Now we consider $(n+1)$. If $(n+1)$ is prime,
then the decomposition is immediate. Assume it is not. This means that
$(n+1) = \alpha \beta$, for $\alpha, \beta \leq n$. We know that $\alpha, \beta$
have unique factorization. We can easily show that the product of two unique
factorizations also has a unique factorization. Hence proved.
\end{proof}

So really, given the Euclidian algorithm, we get this kind of prime decomposition
and the unicity of factorization.

\section{$\Z[i]$: The Gaussian integers}
The size function is the absolute value $\delta(a+bi) \equiv |a+bi|^2 = a^2 + b^2$.
A corollary of this is that every ideal of $Z[i]$ is principal. In particular,
the ideal $I_p$ such that $\Z[i]/I_p \simeq \Z/p\Z$ where $p \equiv 1 \mod 4$ is
principal, and is generated by a single element $a_p + b_p i$, and also that
$a_p^2 + b_p^2 = p$. This is Fermat's theorem, which shows that every
prime $p \equiv 1 \mod 4$ can be written as a sum of squares.

\section{$\delta(r) = |r|$ is a size function}

Let's try to show that $\delta$ is a good size function. Let us pick $B, A \in \Z[i]$.
We can write $B = A \cdot w$, where $w = \alpha + \beta i$ where $\alpha, \beta \in \Q$.
This is easy to do because in the complex numbers, we know that $B/A = B\bar{A}/(A\bar{A})$,
where $\bar A$ is the complex conjugate.  Hence $w = B/A = B\bar{A}/(A\bar{A})$.
We split $\alpha, \beta$ into their integer and fractional parts by
writing $\alpha = \alpha_0 + r_0$, $\beta = \beta_0 + s_0$ where
$\alpha_0, \beta_0 \in \Z$ and $-1/2 \leq r_0, s_0 < 1/2$. This gives us:

\begin{align*}
B = Aw = A(\alpha + \beta i) = A(\floor{\alpha} + i \floor \beta) + A(r_0 + s_0 i) \\
\end{align*}
Note that $A(\floor{\alpha} + i \floor \beta) \in \Z[i]$. What
we have leftover is $r \equiv A(r_0 + s_0 i)$, the remainder. 
We claim that $\delta(r) < \delta(A)/2$. To prove this, we note that $\delta$
which is the absolute value is multiplicative: $\forall u, b \in \C, |ub| = |u||b|$.
Hence, we get that $\delta(Ar) = \delta(A)\delta(r) = \delta(A)(r_0^2 + s_0^2)$.
Hence we can conclude that:

\begin{align*}
& \delta(Ar) = \delta(A)(r_0^2 + s_0^2) \leq \left[ \delta(A)(1/2^2 + 1/2^2) = \delta(A)(1/4 + 1/4) = \delta(A)/2 \right] \\
& \delta(Ar) \leq \delta(A)/2
\end{align*}

Note that the above trick of writing things in terms of $\alpha + \beta i = (\alpha_0 + \beta_0 i) + (r_0 + s_0i)$
does not allow us to show that all rings of the form \Z~with stuff adjoined is Euclidian. For
a concrete non-example, take $\Z[\sqrt{-5}]$. Here, the factorization works
out to be $(r_0 + 5 s_0 i) \leq 1/4 + 5/4$ which \emph{does not decrease} the
size. More drastically, $Z[\sqrt{-5}]$ cannot be a Euclidian domain for any
choice of size function, since unique factorization fails. $6 = 2 \times 3 = (1+ \sqrt{-5})(1-\sqrt{-5})$.

\section{Ideal of $\Z[i]$}

\begin{theorem}
If $I \neq (0)$, then $Z[i]/I$ is finite. That is, $I$ has finite index in $Z[i]$.
\end{theorem}
\begin{proof}
Let $I$ be a non-zero principal ideal generated by $\alpha$: 
$I = (\alpha)$. Then $\alpha \bar \alpha = a^2 + b^2 = n \in \N^+$.
This integer $n \in I$, since $\alpha \in I, \bar \alpha \in Z[i]$, and the ideal
is closed under multiplication with the rest of the ring. So $I \subseteq (n)$.
We claim that $(n) \subseteq I \subseteq R$, and that $(n)$ has finite index
in $R$, and therefore $I$ must have finite index in $R$. $(n)$ has finite
index in $R$ because $(n) = \{ n a + n b i : a, b \in \Z \}$. The cosets
of $R/(n) = \{ a + bi : 0 \leq a < n, 0 \leq b < n \}$. There are $n^2$ such
cosets.
\end{proof}

\begin{theorem}
If $I \neq (0)$, $I = (\alpha)$, then the index of $I$ in $R$ denoted by
$\#(R/I)$ is equal to $\delta(\alpha)$,
which is exactly how it works for the integers as well.
\end{theorem}
\begin{proof}
We write $\alpha = re^{i \theta}$. Now we know that $\delta(\alpha) = r^2$.
We want to find $\alpha \Z[i] = \alpha \Z + i \beta \Z$. Notice that
what we've done is to rotate the lattice by an angle $\theta$, and scale the lattice by $r$.
The index of a sublattice in a lattice is the square of the scaling factor. 

The size of a basic parallelogram is $1$. On scaling, we get have area $r^2$.
Each element in the fundamental lattice is a coset, because after this
the lattice repeats.
\end{proof}

Every Gaussian integer can be written as a unique factorization into primes
upto the units, since it's a UFD. The primes are elements such that the ideal $(p)$ is maximal
with respect to the principal ideal. But in this ring, all ideals are principal
ideals. Hence, $(p)$ must be a maximal ideal. That is. $Z[i]/(p)$ must be
a finite field. The problem is that we don't know what the units are, and we don't
know what the primes are. 

\section{Units of the $\Z[i]$}
$\delta: \Z[i] \rightarrow \Z_{\geq 0}$. $\alpha \mapsto \alpha \bar \alpha$. This cannot
be a ring homomorphism because it is not additive. A different way of looking
at it is that the image $\Z_{\geq 0}$ is not a group, so it can't be
a ring homomorphism. However, it is multiplicative: $\delta(\alpha \cdot \beta) = \delta(\alpha) \delta(\beta)$.
This is thanks to complex multiplication. With that note done,
let's begin chipping away at the units.

\begin{theorem}
(1) $\alpha$ is a unit if and only if (2) $\delta(\alpha) = 1$.
\end{theorem}
\begin{proof}
We first show $(2) \delta(\alpha) = 1 \implies (1)$ $\alpha$ is a unit.
Assume that $\delta(\alpha) = 1$. Hence, $|\alpha|^2 = 1$.
So, it can be written as $e^{i \theta} = \cos(\theta) + i \sin(\theta)$. The only
such numbers with  $\cos(\theta), \sin(\theta) \in \Z$ are $\pm 1, \pm i$. 
These are all units. 
\end{proof}

\begin{proof}
We wish to show (1) $\alpha$ is a unit~$\implies$~ (2) $\delta (\alpha) = 1$, 
Since $\alpha$ is a unit, there exists some element $\beta$ such that $\alpha \beta = 1$.
Now apply $\delta$ on both sides:
\begin{align*}
&\delta(\alpha \beta) = \delta(1) \\
&\delta(\alpha) \delta(\beta) = 1 \\
\end{align*}
Since $\delta(\alpha), \delta(\beta) \in \Z_{\geq 0}$ whose product is 1, we must have that
$\delta(\alpha) = \delta(\beta) = 1$. 
\end{proof}


\begin{proof}
A more complicated version of (1) $\alpha$ is a unit $\implies$ (2) $\delta (\alpha) = 1$.
Since $\alpha$ is a unit, we know that $1 \in (\alpha)$ since $\alpha \times \alpha^{-1} \in (\alpha)$
as $(\alpha)$ is closed under multiplication. However, if $1 \in (\alpha)$,
then every number is in the ring, since $z \cdot 1 \in (\alpha)$. Formally:

\begin{align*}
&\forall z \in Z[i], \forall i \in (\alpha), zi \in (\alpha) \\
&\text{pick $z = \alpha^{-1}, i = \alpha$:} \\
&\alpha^{-1} \cdot \alpha = 1 \in (\alpha) \\
&\text{pick $z$ as an arbitrary $z_0 \in Z[i]$, and $i = 1$:} \\
&z_0 \cdot 1 =  z_0 \in (\alpha) \\
&R = (\alpha)
\end{align*}

Therefore, $(\alpha) = Z[i]$. Now, we calculate $\delta(\alpha)$:
$$\delta(\alpha) = \#(R/(\alpha)) = \#(R/R) = 1$$
\end{proof}

We now know the unit group of the ring. $Z[i]^\times = \{ 1, i, i^2, i^3 \}$
which has order 4 in $\Z[i]$.

\section{Primes of $\Z[i]$}

We will use the letter $\pi$ to denote a prime. We know that we need $\Z[i] / (\pi)$
is a finite field. Every finite field has order $p^n$ for some prime $p \in \Z$
and $n \geq 1$. In our case, we claim that the dimension $(n = 1 \lor n = 2)$.
\begin{theorem}
Consider the quotient $F = \Z[i]/ (\pi)$. This must be finite since it has finite order $\delta(pi)$,
and is a field since $\pi$ is prime. We claim that this finite field $F$ of
characteristic $p$ with $p^n$ elements has \textbf{size $p^1$ or $p^2$}. That is,
it is a vector space of dimension $1$ or $2$ over $Z/pZ$ but no larger.
\end{theorem}
\begin{proof}
Let $F = \Z[i]/(\pi)$ have characteristic $p$, and let $\phi: \Z[i] \rightarrow \Z[i]/(\pi)$
be the canonical map $\phi(z) \equiv z + \pi$. Now, we know that $p \in \Z[i]$,
and also that $\phi(p) = 0$ since $F$ is char. $p$. Therefore, $p \in \Z[i]/(\pi)$.
This tells us that there is an inclusion of ideals $(p) \subseteq (\pi) \subsetneq \Z[i]$.
Hence, $\#(Z[i]:(\pi)) \leq \#(Z[i]:(p))$ --- intuitively, on squashing $(p)$,
we squash less elements than squashing $(\pi)$. Hence, the number of elements
in the quotient of $(\pi)$ is upper-bounded by number of elements in the
quotient in $(p)$. Now recall that $\#(Z[i]:(p)) = \delta(p) = p^2$. Hence:

\begin{align*}
&|F| = p^n \#(Z[i]:(\pi)) \leq \#(Z[i]:(p)) = \delta(p) = p^2 \\
&|F| = p^n \leq p^2 \implies |F| = p^1 \lor |F| = p^2
\end{align*}
Hence proved.
\end{proof}

This is where number theory starts. We have two cases. 

\begin{theorem}
If $R/(\pi)$ has order $p^2$. Then $(\pi) = (p)$
\end{theorem}
\begin{proof}
We argue by ideal-size-containment. Since 
$$
(p) \subseteq (\pi) \subseteq \Z[i]
$$

If $\#(\Z[i]:(\pi)) = p^2$ and $\#(\Z[i]:p) = \delta(p) = p^2$, then we know
that $\#(\Z[i]:p) = \#(\Z[i]:(\pi)) \times \#((\pi):p)$, or $p^2 = p^2 \cdot ((\pi):p)$.
This means that $((\pi):p) = 1$ or $(\pi) = (p)$. Hence, an ideal that's generated
by a prime $p$ in $\Z$ continues to be prime in $\Z[i]$.
\end{proof}

\begin{theorem}
If $R/(\pi)$ has order $p$, then TODO fill in structure!
\end{theorem}
\begin{proof}
In this case, $\Z[i]/(p)$ is not a field, so there are non-trivial ideal $(\pi)$
between $(p)$ and $\Z[i]$, such that $Z[i]/(\pi) \simeq \Z/p\Z$ (since it's a field of order $p$).
\end{proof}

To each Gaussian prime $\pi$ we can associate a rational prime $p$ as the
characteristic of the field $\Z[i]/(\pi)$. We now try to make explicit
the relationship between $\pi$, $p$, and the order of the field $\Z[i]/(\pi)$.
Really, we should study the finite ring $R/(p)$. If it's a field, we are
done. If it continues to be a ring, then there are ideals $(pi)$ in it
that generate fields.

\section{The ring $Z[i]/(p)$}

We study $\Z[i]/(p)$. We write:

\begin{align*}
&\Z[i]/(p) = (\Z[x] / (x^2+1)) / (p) \\
& = \Z[x] / (x^2+1, p) \\
& = \Z[x] / (p, x^2+1)  \\
& = (\Z[x] / (p)) / (x^2+1) \\
& = \Z/p\Z[x] / (x^2+1)
\end{align*}

The quotient ring of $\Z/p\Z[x] / (x^2+1)$ is a field if $(x^2+1)$
to be an irreducible over $\Z/p\Z$.
(TODO: link theorem). For it to be irreducible over $\Z/p\Z$, we need $x^2+1$ to not have roots
over $\Z/p\Z$. That is, we need $x^2 \equiv (-1) \mod p$ to have \textbf{no solutions}.

\begin{example}
Over $p=2$, we can write $x^2+1 \equiv (x+1)^2 \mod 2$. It has a repeated root $x = 1$.
In this case, there is a unique prime $\pi = 1 + i$ with
$(2) \subset (\pi) \subset Z[i]$
\end{example}

\begin{theorem}
If $p \equiv 3 \mod 4$, then $x^2+1$ is irreducible modulo $p$,
and $\Z[i]/(p)$ is a field.
\end{theorem}
\begin{proof}
If $p \equiv 3 \mod 4$, then:
\begin{align*}
|\Z/p\Z^\times| = p - 1 = (4k + 3) - 1 = 4k + 2 = 2(2k + 1) = 2 \cdot \text{odd}
\end{align*}

Let $r$ be a root of $x^2 + 1$ in $\Z/p\Z$. 
\begin{itemize}
\item[1.] Since $r \neq 0$, $r$ is invertible in $\Z/p\Z$ ($\Z/p\Z$ is a field). So $r \in \Z/p\Z^\times$.
\item[2.] $r^2 + 1 = 0 \implies r^2 = -1$.
\item[3.] $r$ has order 4: $r^4 = (r^2)^2 = (-1)^2 = 1$.
\item[4.] $\Z/p\Z^\times$ has no elements of order $4$, since the order of an element
 must divide the order of the group, but $|\Z/p\Z^\times| = 2\cdot \text{odd}$,
 and hence is not divisible by $4$.
\item[5.] Hence, $r \not \in \Z/p\Z^\times$. Contradiction with (1).
\end{itemize}
Hence, there is no root $r$ of $x^2 + 1$.
\end{proof}

\begin{theorem}
    If $p \equiv 1 \mod 4$, then $x^2+1$ factors as $(x-a)(x+a)$, where
    $a^2 \equiv (-1) \mod p$.
\end{theorem}
\begin{proof}
    $$|Z/pZ|^\times = p - 1 = 4k + 1 - 1 = 4k = 2^n \quad \text{where $n \geq 2$}$$
    Hence the Sylow-2 subgroup of $|Z/pZ|^\times$ has order $2^n$ (where $n \geq 2$).
    We claim that the only elements of order $2$ is $\pm 1$. Let us assume
    we have an element of order $2$. This means that $a^2 = 1$. Hence $a^2 - 1 = 0$,
    or $p | a^2 - 1$. Hence, $p | (a^2 - 1) (a^2 + 1)$. Since $p$ is prime,
    $p$ has to divide either $(a^2 - 1)$ or $(a^2+1)$. Hence $a^2 = \pm 1$.

    Now that we know this, we need more elements in $|Z/pZ|$ since it has order $2^n$
    but we have only found 2 elements of order 2. So the other elements
    must have order 4 or larger. We can always take powers of such an element
    to create an element of order 4.

    Spelling out the details, if an element $r \in Z/pZ^\times$ has order
    $4 \cdot m$, then $r^{4m} = 1$. So $(r^m)^4 = 1$. $r^m$ is the element
    of order 4 we are looking for.
\end{proof}

Consider $1 - \frac{1}{3} + \frac{1}{5} - \frac{1}{7} + \frac{1}{9} = \frac{\pi}{4}$.
We will show that this is a theorem about Gaussian numbers.

% https://www.youtube.com/watch?v=M56MTc4q-wU&feature=emb_logo

% Visual group theory: Galois theory (https://www.youtube.com/watch?v=UwTQdOop-nU&list=PLwV-9DG53NDxU337smpTwm6sef4x-SCLv)
\chapter{Atiyah MacDonald, Ch1 exercises}

\section{Q11}
$2x = x + x = (x+x)^2 = x^2 + 2x + x^2 = x + 2x + x = 2 \cdot 2x$. This gives
us the equation $2x = 2 \cdot (2x)$, and hence $2x = 0$.

\section{Q15}

Let $X$ be the set of prime ideals of the ring $A$. 
We will denote elements of $X$ as $x$, and when thinking of them as ideals,
we will write them as $\frakp_x$, though they are the same as sets ($x = \frakp_x$).    

Let $V(E)$ be the set of all points in $X$ that contain $E$. That is,
$V(E) = \{ \frakp_x \in X : E \subseteq \frakp_x \}$. We need to show:

\subsection{If $\fraka$ is generated by $E$, then $V(E) = V(\fraka)$}

$V(E) = \{ \frakp_x \in X : E \subseteq \frakp_x \}$
$V(\fraka) = \{ \frakp_x \in X : \fraka \subseteq \frakp_x \}$. The idea
is to exploit that since we are collecting ideals when building $V(E)$, and ideals
are closed under inclusion. if $e_1 \in \frakp_x, e_2 \in \frakp_x$, then
all combinations $a_1 e_1 + a_2 e_2 \in \frakp_x$. On the other hand, clearly 
the generated ideal will contain all elements of the original generating set.
Hence, the points of $x$ that we collect will be the same either way.


More geometrically, recall that for every (subset of $A$/polynomial) $E$, we let
$V(E)$ to be the points over which $E$ vanishes. That is, $x \in V(E) \iff E \xrightarrow{\frakp_x} 0$, 
where $E \xrightarrow{frakp_x} \cdot$ is rewriting $E$ using the fact that every element in
$\frakp_x$ is zero.


Now, notice that if we have that $E$ rewrites to zero, then all elements
in the ideal generated by $E$ also rewrite to zero, since $a_1 e_1 + a_2 e_2 \xrightarrow{\frakp_x} a_1 0 + a_2 0 = 0$.
Similarly, if the ideal generated by $E$ rewrites to zero, then so does $E$, 
because $E$ is a subset of the ideal generated by $E$.

\subsection{If $\fraka$ is generated by $E$, then $V(\fraka) = V(radical(\fraka))$}

Recall that the radical of an ideal is defined as $radical(\fraka) \equiv \{ a \in A : a^n \in \fraka \}$.
$X$ consists of \emph{prime} ideals. Prime ideals contain the radicals of all of
their elements. Recall that if $a^n \in \frakp$ where $\frakp$ is prime, then $a \cdot a^{n-1} \in \frakp$, hence
$a \in \frakp \lor a^{n-1} \in \frakp$ by definition of prime ideal. Induction on $n$ completes
the proof. Therefore, the additional elements we add when we consider $radical(\fraka)$
don't matter; if $a \xrightarrow{\frakp} 0$, then $a \in \frakp$, $radical(a) \subseteq \frakp$,
so $radical(\fraka) \xrightarrow{\frakp} 0$.

\section{Q17}

For each $f \in A$, We denote $X_f \equiv V(f)^\complement$ where we have $X = Spec(A)$.
We first collect some information about these $X_f$ and how to psychologically
think of them. First, recall that $V(f)$ will contain all the points $x \in X$ such
that $f$ vanishes over the point $x$: $f \xrightarrow{\frakp_x} 0$. Hence,
the complement $X_f$ will contain all those point $x' \in X$ such that $f$
does \emph{not} vanish over $x'$: $f \xrightarrow{\frakp_{x'}} \neq 0$. So
we are to imagine $X_f$ as containing those points $x'$ over which $f$ does
not vanish.

We will first show that we can union and intersect these $X_f$, and we will
then show how that these $X_f$ form an open base of the Zariski topology.

\subsection{$X_f \cap X_g = X_{fg}$}
$X_f \cap X_g$ contains all the points in $X$ where neither $f$ nor $g$ vanish.
If neither $f$ nor $g$ vanish, then $fg$ does not vanish. Conversely, if $fg$
does not vanish at $x$, since the point $x$ is prime, neither $f$ nor $g$ vanish
over $x$ (elements that do not belong to the prime ideal are a multiplicative subset: 
$xy \not \in \frakp \implies x \not \in \frakp  \land y \not \in \frakp$).

Hence, the set where $f$ and $g$ do not both vanish, $X_f \cap X_g$ is equal
to the set where $fg$ does not vanish.

\subsection{\textbf{Incorrect conjecture}: $X_f \cup X_g = X_{f + g}$}
$X_f \cup X_g$ contais all the points in $X$ where either $f$ or $g$ do not vanish.
But that does not mean that $f + g$ has to not vanish. For example, let the
the ring be $\mathbb R[X]$, and let $f = x^2+1$, $g = -x^2-1$. Both of these 
do not vanish over all of $\mathbb R$, and yet $f + g = 0$ which vanishes
everywhere. So it's \emph{not true} that $X_f \cup X_g = X_{f + g}$ because addition
can interfere with non-vanishing.

\subsection{$X_f = \emptyset \iff f \text{is nilpotent}$}
$(\impliedby)$: Let $f$ be nilpotent. We want to show that $X_f = \emptyset$.
Recall that $X_f = \{ x \in X: f \xrightarrow{\frakp_x} \neq 0 \}$. If
$f$ is nilpotent, then $f$ belongs to every prime ideal: $\forall x \in X, f \in \frakp_x$.
Thus $f$ vanishes on all prime ideals: $\forall x \in X, f \xrightarrow{\frakp_x} = 0$.
Hence, $X_f$, which contains prime ideals $x$ where $f$ \emph{does not vanish}, is empty.

$(\implies)$: Let  $X_f = \emptyset$. We wish to show that $f$ is nilpotent.
This means that $\forall x \in X, f \in \frakp_x$. But recall that the intersection
of all prime ideals in a ring is the nilradical. Hence $f$ is a nilpotent. We recollect
the proof that the intersection of all prime ideals is the nilradical. \textbf{TODO!}

\subsection{the sets $X_f$ form a basis (base) of open sets for the Zariski  topology}


\end{document}
