%% start of file `template.tex'.
%% Copyright 2006-2013 Xavier Danaux (xdanaux@gmail.com).
%
% This work may be distributed and/or modified under the
% conditions of the LaTeX Project Public License version 1.3c,
% available at http://www.latex-project.org/lppl/.

% \documentclass[11pt,a4paper,sans,colorlinks,linkcolor=true]{moderncv}        % possible options include font size ('10pt', '11pt' and '12pt'), paper size ('a4paper', 'letterpaper', 'a5paper', 'legalpaper', 'executivepaper' and 'landscape') and font family ('sans' and 'roman')
\documentclass[10pt,a4paper,sans,colorlinks]{moderncv}        % possible options include font size ('10pt', '11pt' and '12pt'), paper size ('a4paper', 'letterpaper', 'a5paper', 'legalpaper', 'executivepaper' and 'landscape') and font family ('sans' and 'roman')



% moderncv themes
\moderncvstyle{classic}                             % style options are 'casual' (default), 'classic', 'oldstyle' and 'banking'
\moderncvcolor{blue}                               % color options 'blue' (default), 'orange', 'green', 'red', 'purple', 'grey' and 'black'

%\renewcommand{\familydefault}{\sfdefault}         % to set the default font; use '\sfdefault' for the default sans serif font, '\rmdefault' for the default roman one, or any tex font name
%\nopagenumbers{}                                  % uncomment to suppress automatic page numbering for CVs longer than one page

% character encoding
\usepackage[utf8]{inputenc}                       % if you are not using xelatex ou lualatex, replace by the encoding you are using
%\usepackage{CJKutf8}                              % if you need to use CJK to typeset your resume in Chinese, Japanese or Korean

% adjust the page margins
\usepackage[scale=0.8]{geometry}
\setlength{\hintscolumnwidth}{4cm}                % if you want to change the width of the column with the dates
%\setlength{\makecvtitlenamewidth}{10cm}           % for the 'classic' style, if you want to force the width allocated to your name and avoid line breaks. be careful though, the length is normally calculated to avoid any overlap with your personal info; use this at your own typographical risks...

% personal data
\name{Siddharth}{Bhat}                              % optional, remove / comment the line if not wanted
\address{64 Storey's Way, Churchill College}{CB30DS}{Cambridge, United Kingdom}% optional, remove / comment the line if not wanted; the "postcode city" and and "country" arguments can be omitted or provided empty
\phone[mobile]{+44 07442785227}                   % optional, remove / comment the line if not wanted
\email{siddharth.bhat@cl.cam.ac.uk}
\homepage{pixel-druid.com}                         % optional, remove / comment the line if not wanted
% to show numerical labels in the bibliography (default is to show no labels); only useful if you make citations in your resume
%\makeatletter
%\renewcommand*{\bibliographyitemlabel}{\@biblabel{\arabic{enumiv}}}
%\makeatother
%\renewcommand*{\bibliographyitemlabel}{[\arabic{enumiv}]}% CONSIDER REPLACING THE ABOVE BY THIS

% bibliography with mutiple entries
%\usepackage{multibib}
%\newcites{book,misc}{{Books},{Others}}
%----------------------------------------------------------------------------------
%            content
%----------------------------------------------------------------------------------
\begin{document}
%\begin{CJK*}{UTF8}{gbsn}                          % to typeset your resume in Chinese using CJK
%-----       resume       ---------------------------------------------------------
\makecvtitle


\section{Expertise}

\textbf{Formal Verification (Lean) \& Static Analysis (LLVM/MLIR)}
\begin{itemize}
\item Given >4 talks @ LLVM dev meeting (Formal Semantics for LLVM \& MLIR).
\item Led the \texttt{lean-mlir} project for verifying optimizations in MLIR.
\item Top 20 contributors to the Lean theorem prover, \#2 contributor to Lean's bitvector theory.
\item Coauthor of the Lean4 metaprogramming book.
\end{itemize}

\textbf{Algorithms (SMT Solving for \texttt{QF\_BV})}
\begin{itemize}
\item Published new, formally verified decision procedures for parametric bitvector theory.
\item Fixed various bugs in Polly's implementation of polyhedral algorithms.
\item Designed and implemented algorithms for proving memory (non-)interference in symbolic simulators for ARM.
\end{itemize}

\textbf{Compiler Design (Functional), Implementation (LLVM, MLIR), HPC (Loop Optimization):}
\begin{itemize}
\item \#3 Contributor to Polly, LLVM's polyhedral loop optimizer. 121 commits to Polly, 6000 LoC to LLVM, added support for polyhedral GPU code generation.
\item Google Summer of Code Mentor for LLVM (2016).
\item \#2 contributor to Asterius, a Haskell $\to$ WebAssembly compiler.
\item I implemented a Haskell-style runtime on top of the WASM runtime. Contributions merged into Asterius, and eventually into GHC proper.
\item Primary author of Lean's LLVM backend.
\item Papers on designing MLIR based IRs for quantum \& functional compilation.
\end{itemize}


\textbf{AI for Maths (Lean, F$\ast$):}
\begin{itemize}
\item RL for proofs in the F$\ast$ proof assistant @ Microsoft Research ("Towards neural synthesis for SMT-assisted proof-oriented programming"@ICSE 2025, best paper).
\item Established stronger baselines for AI based geometric theorem proving.
\end{itemize}

\section{Education}
\cventry{PhD in Computer Science}{University of Cambridge}{}{}{}{(2024 - 2026 March (tentative))}.
\cventry{PhD in Computer Science}{University of Edinburgh (\textit{moved to Cambridge})}{}{}{}{(2022 - 2024)}.
\cventry{Master by Research in Computer Science }{International Institute of Information Technology Hyderabad India}{}{}{}{(2020 - 2021)}.
\cventry{B.Tech in Computer Science}{International Institute of Information Technology Hyderabad India}{}{}{}{2015 - 2020}.

\section{Internships}
\cventry{Sep-Nov '24}{Amazon Web Services, Automated Reasoning Group}{Austin}{}{}{Deciding memory (non)interference in \texttt{lnsym}, a Lean-based ARM symbolic simulator}
\cventry{Jul-Sep '23}{Microsoft Research}{Redmond}{}{}{Retrieval Augmented theorem proving for the Fstar proof assistant.}
\cventry{July 1-10 '23}{Adjoint School}{Glasgow}{}{}{Researched Markov categories and their relationship to probabilistic programming.}
% \cventry{Winter 2019}{Teaching Assistant for Natural Language: Applications}{IIIT-H}{}{}{Monitored projects, took sessions on word embeddings, involving \texttt{word2vec}, \texttt{GloVe}, \texttt{fasttext}. }
\cventry{May-Jul '19}{Intern at Tweag.io}{Paris, France}{}{}{Re-implemented portions of GHC(Glasgow Haskell Compiler) runtime for \href{https://github.com/tweag/asterius/commits?author=bollu}{\texttt{Asterius} (link)}, a Haskell to WebAssembly compiler. Involved Haskell, C, and WebAssembly.}
% \cventry{Winter 2018}{Teaching Assistant for Principles of Programming Languages}{IIIT-H}{}{}{Course covers the book "Essentials of Programming Languages" by Dan Friedman. Helped write lecture notes, set assignments, graded assignments and exams.}
\cventry{Summer 2018}{ Visiting research intern at ETH Zurich}{Zurich, Switzerland}{}{}{Investigating formal verification of polyhedral compilation. \href{http://github.com/bollu/polyir}{\texttt{PolyIR (Link)}} is a formal specification of polyhedral programs.}
\cventry{Summer 2018}{GSoC mentor, Polly Labs}{}{}{}{Mentoring a project to enable Polly's loop optimisations into Chapel.}
\cventry{Mar-Dec '17}{ETH Zurich, Research Intern at SPCL}{Zurich, Switzerland}{}{}{Worked on Polly, a polyhedral loop optimizer for LLVM.}


% \cventry{Jan-Mar '17}{Course content contributor}{IIIT-H}{}{}{Wrote lecture notes for the \href{http://pascal.iiit.ac.in/itws2/docs/\#orgf49f2a9}{Intro to programming course (link)}}

\cventry{May-Jul '16}{Research Intern}{IISC Bangalore}{Bangalore}{}{Worked on PolyMage, DSL compiler for optimising loop transforms. Contributed to ISL and PLUTO. Implemented tiling patterns, optimised PolyMage for stencils.}%

\cventry{Summer 2016}{Selected for GSoC 2016}{Google}{}{}{Binding SymEngine, a symbolic math library to Haskell. Had to drop this to intern at IISc, Bangalore. Still maintain the library (symengine.hs)}

\cventry{Summer 2015}{GSoC 2015}{Google}{}{}{Worked on VisPy, a pure Python graphics library which uses OpenGL internally for performance. Successfully completed.}


\section{Publications}

\subsection{First Author Papers}
\cvitem{}{Certified Decision Procedures for Width-Independent Bitvector Predicates: \textbf{Siddharth Bhat (1st)}, L\'eo Stefanesco, Chris Hughes, Tobias Grosser. OOPSLA 2025}
\cvitem{}{Verifying Peephole Rewriting in SSA Compiler IRs: \textbf{Siddharth Bhat (1st)}, Alex Keizer, Chris Hughes, Andres Goens, Tobias Grosser. ITP 2024}
\cvitem{}{Lambda the Ultimate SSA: \textbf{Siddharth Bhat (1st)}, Tobias Grosser. CGO 2022}
\cvitem{}{Word Embeddings as Tuples of Feature Probabilities: \textbf{Siddharth Bhat (1st)}, Alok Debnath, Souvik Banerjee, Manish Shrivastava Representation Learning for NLP, 2020}
\subsection{Collaborations}
\cvitem{}{Interactive Bit Vector Reasoning using Verified Bitblasting: Henrik B\"oving, \textit{Siddharth Bhat}, Alex Keizer, Luisa Cicolini, Leon Frenot, Abdalrhman Mohamed, L\'eo Stefanesco, Harun Khan, Josh Clune, Clark Barrett, Tobias Grosser. OOPSLA 2025}
\cvitem{}{Verifying Wu's Method can Boost Symbolic AI to Rival Silver Medalists and AlphaGeometry to Outperform Gold Medalists at IMO Geometry: 
Shiven Sinha, Ameya Prabhu, Ponnurangam Kumaraguru, \textit{Siddharth Bhat}, Matthias Bethge. NeurIPS 2024 Workshop MATH-AI}
\cvitem{}{Towards Neural Synthesis for SMT-Assisted Proof-Oriented Programming: Saikat Chakraborty, Gabriel Ebner, \textit{Siddharth Bhat}, Sarah Fakhoury, Sakina Fatima, Shuvendu Lahiri, Nikhil Swamy. ICSE 2024}
% \cvitem{}{Mutable Grammars: Abhinav Menon, Jatin Agarwal, \textbf{Siddharth Bhat}, Andres Goens, Tobias Grosser. Submitted to ITP 2024}
\cvitem{}{Rewriting Optimization Problems into Disciplined Convex Programming Form: Ramon Fernandez Mir, \textit{Siddharth Bhat}, Andres Goens, Tobias Grosser. CICM 2024}
\cvitem{}{Guided Equality Saturation: Thomas Koehler, Andres Goens, \textit{Siddharth Bhat}, Tobias Grosser, Phil Trinder, Michel Steuwer. POPL 2024}
\cvitem{}{QSSA: An SSA based IR for Quantum Computing: Anurudh Peduri, \textit{Siddharth Bhat}, Tobias Grosser. CC 2021}
\cvitem{}{Optimizing Geometric Multigrid Computation using a DSL Approach: Vinay Vasista, Kumudha KN, \textit{Siddharth Bhat}, Uday Bondhugula.  Supercomputing (SC), Nov 2017}



\section{Open Source Contributions}

\cvitem{\href{https://github.com/leanprover/lean4/issues?&q=author\%3Abollu}{Lean4}}{Co-developed the bitblasting theory for Lean's bitvector automation, wrote the LLVM backend for the compiler.}
\cvitem{\href{https://github.com/coq/coq/issues?&q=author\%3Abollu}{Rocq}}{Submitted issues, bug-fixes, helped improve developer documentation.}
\cvitem{\href{https://github.com/vellvm/vellvm/issues?&q=author\%3Abollu}{VE-LLVM}}{Collaboration with VE-LLVM, a formal semantics of the LLVM compiler toolchain in Coq} 
\cvitem{\href{https://polly.llvm.org/}{Polly}}{Implementing support for Fortran, added unified memory abilities to the CUDA backend within Polly, a polyhedral loop optimiser for LLVM. \href{https://reviews.llvm.org/p/bollu/}{(Link to commits)}}

\cvitem{\href{https://github.com/symengine/symengine.hs}{Symengine.hs}}{GSoC 2016. Haskell bindings to SymEngine, a C++ symbolic manipulation library.}

\cvitem{\href{https://github.com/vispy/vispy/commits?author=bollu}{VisPy}}{GSoC 2015. Rewrote scene graph for performance. Added visuals, high level API for easy use of plotting. Implemented auto-resizing with \textbf{Cassowary}, a linear optimisation library.}

\cvitem{Rust}{Contributed to the Rust compiler and ecosystem. Found compiler errors, fixed libraries. Was part of \href{https://github.com/PistonDevelopers}{\emph{Piston}}, group of Rust programmers who experimented with writing game engines.}

\cvitem{Haskell}{Contributed to the Haskell ecosystem. Reported and fixed bugs in \emph{stack}, \emph{stackage}, \emph{diagrams}, \emph{GHC}, etc. \href {https://phabricator.haskell.org/p/bollu/}{(Link to GHC commits).}}

\cvitem{PLUTO}{Source to Source C optimiser for loop nests. Improved the PLUTO API that had gone out of sync with master. Discovered bugs in PLUTO for diamond tiling transforms}

\cvitem{\href{http://mcl.csa.iisc.ac.in/polymage.html}{PolyMage}}{DSL compiler that generates C code. Uses \textbf{Polyhedral Compilation} Extended the compiler to add stencils, time iterated-stencils.}


\cvitem{\href{https://github.com/hrydgard/ppsspp/commits?author=bollu}{PPSSPP}}{C++ open source PSP emulator. Wrote most of the touch handling code. Implemented atomic locks for audio performance.}

\subsection{Personal Projects (Compilers)}

\cvitem{\href{https://github.com/opencompl/lean-mlir}{{Lean-MLIR}}}{Formal semantics for the MLIR compiler framework, defined within the Lean4 proof assistant.}

\cvitem{\href{https://github.com/bollu/lz}{{lz}}}{An MLIR based compiler backend for the Lean4 proof assistant.}


% \cvitem{\href{https://github.com/bollu/lean-to}{{\footnotesize Lean-to}}}{Jupyter kernel for the Lean4 proof assistant.}

\cvitem{\href{http://github.com/bollu/simplexhc}{Simplexhc}}{A custom compiler for a subset of Haskell. The goal is to try and apply \emph{polyhedral compilation} ideas to compile a lazy, pure, functional programming language with LLVM as a backend.
Has \textbf{64 stars} on github.}

\subsection{Personal Projects (Formal Verification)}

\cvitem{\href{https://github.com/arthurpaulino/lean4-metaprogramming-book}{{Lean4 Metaprogramming Book}}}{A textbook on metaprogramming in Lean4. I wrote the chapters on tactics and metaprogramming for embedded DSLs.}


% \cvitem{\href{https://github.com/bollu/sublimebookmark}{Sublime Bookmarks}}{A plugin for sublime text to quickly jump between pieces of your codebase. \textbf{26k downloads} and counting.}

% \cvitem{\href{https://www.github.com/bollu/cellularAutomata}{Cellular Automata}}{A collection of Cellular Automata written in Haskell. Uses \textbf{Comonads} for abstraction. \textbf{130 stars} on Github.}


% \cvitem{\href{http://bollu.github.io/teleport}{Teleport}}{A simple tool to switch between projects written in Haskell. Shows how to write “real world Haskell". Published as a \textbf{Literal Haskell tutorial}. \textbf{90 stars} on github}

% \cvitem{\href{http://github.com/bollu/timi}{TIMi}}{A visual interpreter of the \textbf{template instantiation machine} to understand evaluation of lazy functional languages. \textbf{51} stars on github.}

\subsection{Talks \& Presentations}
\cvitem{\href{https://github.com/bollu/barvinok/blob/master/slides.pdf}{Barvinok}}{Talk at ETH Zurich: Slides describing the Barvinok algorithm to count lattice points in polyhedra}

\cvitem{\href{https://confengine.com/user/siddharth-bhat}{FunctionalConf '19}}{Talk on implementing embedded probabilistic programming languages in Haskell  \href{https://github.com/bollu/functionalconf-2019-slides-probabilistic-programming}{(Slides)}}

\cvitem{\href{https://skillsmatter.com/skillscasts/14910-smallpt-hs-porting-a-raytracer-s-performance-to-haskell}{Haskell Exchange 2020}}{Talk on optimizing \texttt{smallpt-hs} (a port of a raytracer to haskell) to beat \texttt{C++} performance \href{https://github.com/bollu/slides-haskell-exchange-2020-smallpt}{(Slides)} }

% \cvitem{\footnotesize \href{https://www.youtube.com/watch?v=cfdII1jDJYU}{FPIndia}}{Talk on egg: fast and extensible equality saturation. \href{https://github.com/bollu/notes/blob/master/egg-equality-saturation-slides/slides.pdf}{(Slides)}} 

% \cvitem{\href{https://faculty.iiit.ac.in/~theory/seminar/talks/trisecting-ruler-compass/}{\footnotesize Theory seminar, winter '19}}{Talk on impossibility of compass-straightedge constructions using field theory.}

% \cvitem{\href{http://math.stackexchange.com/users/261373/siddharth-bhat}{math.se}}{Answer on \textbf{math.stackexchange. 8312 reputation, \textbf{top 4\% overall}}. Abstract algebra and differential/algebraic geometry.}.



\cvitem{Euro LLVM Dev 2025}{\href{https://www.youtube.com/watch?v=WtsInfbzxjs}{How to trust your peephole rewrites: automatically verifying them for arbitrary width!}}.
\cvitem{US LLVM Dev 2024}{\href{https://www.youtube.com/watch?v=4lh2NnLOxvQ}{lean-mlir: A workbench for formally verifying peephole optimizations in MLIR}.}
\cvitem{US LLVM Dev 2023}{\href{https://www.youtube.com/watch?v=VJORFvHJKWE}{(Correctly) Extending dominance to MLIR Regions}.}
\cvitem{US LLVM Dev 2023}{\href{https://www.youtube.com/watch?v=6bDKasLZyxU}{MLIR Side Effect Modelling}.}
\cvitem{Euro LLVM Dev 2022}{\href{https://www.youtube.com/watch?v=cyMQbZ0B84Q}{MLIR for Functional Programming}.}
\cvitem{FPIndia 2021}{\href{https://www.youtube.com/watch?v=cfdII1jDJYU}{Equality Saturation}.}
\cvitem{Functiona Conf 2019}{\href{https://confengine.com/conferences/functional-conf-2019/proposals}{Monad-bayes: Probabilistic programming in Haskell}.}

\section{Awards}

\cvitem{\href{https://www.renaissancephilanthropy.org/mathbench-towards-evaluating-natural-language-proofs}{Renaissance Philanthropy, AI for Maths}}{'Towards evaluating Natural Language Profs' One of 30 research groups awarded out of over 280 applicants.}

\end{document}


%% end of file `template.tex'.
