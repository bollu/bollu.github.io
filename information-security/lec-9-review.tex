- Historical ciphers:  (breaking)
+ ceasar and shift
+ Monoalphabetic substitution cipher
+  Vigenere cipher

- 17th century: Kerchoff's Principle: (don't use obscurity)
+ Shannon's pessimistic theorem 
+ One time pad is perfectly secure
+ |M| <= |K| (limitation of perfect security)


- Two relaxations
+ PPTM adversary
+ Negligible p of error
+ $f(n)$ is negligible iff $\forall p \in R[x], \exists N_0, \forall n \geq N_0, f(n) < 1 / p(n)$.
++ examples: $f(n) = \frac{1}{2^n}$. $f(n) = \frac{1}{eps^n}$ where $eps > 0$.


- PRG
- Secure encryption

- CPA
- CPA secure : Adversary has free access to encryption oracle
- So, we need probabilistic encryption to offer CPA security.
- PRF
- CPA secure encryption

- CBC
- IFC
- Random counter mode

- Convert Pseudo random function to pseudo random permutation. If both
  forward and backward are efficient, then it's a block cipher. We did this
  using a "Feistel structure".

$f' :: Z x Z \to Z x Z$
$f' = (x, y) \to (y, (F_k(y)~\texttt{xor}~x))$

This function is invertible.
Each application is a "fiestel round".
Apply this as many times as wanted, at least 4 is recommended.


- 3 DES. 2 keys of 56 bits each.


- CCA secure (chosen ciphertext attack)
  Adversary does not know what the message is. He can actively modify the 
  *ciphertext*.

- Semantic security


- MAC : message authentication code - solves problem of data integrity.

CPA secure + MAC => CCA secure.

c -> cpa secure(c) + mac (c)


What is information?

- Shannon,  Kolmogrov, Lenin
- Randomness, Space, Time.
