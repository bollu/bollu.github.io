
\chapter{Lecture 2 - More philosophy - Amazing Advantages of Additional Adversity}

Textbook is
\begin{itemize}
    \item Introduction to Modern Cryptography
\end{itemize}

\subsection{Ceasar Cipher}

rotate letters by a certain amount.

%% $\{ letter -> number -> (number + delta) % 26 -> letter \}$.
crypto goes to FUBSWR.

\section{Kerckhoff's Principle}
Security of system depends on secrecy of the key and not on the
obscurity of the algorithm.

\subsection{Password Shadows}
Password is $\{x\}$, we store $\{ f(x) \}$.

It is possible to reverse-engineer $f$ to discover $x$. So, we should not depend
on $f$ being secure.

\subsection{}

\subsection{Shift Cipher}

We can brute force this, we can brute force keys.

Principles learnt from shift ciphers
\begin{itemize}
    \item Key space needs to be large. for shift cipher, key space is 26.
\end{itemize}

$p_i$ probability of letter in plaintext.
$q_i$ probability of letter in ciphertext.

${ \exists delta, \forall x in Letter, p_i = q_{i+k} }$

${ pi \cdot p_{i+ k} = p_i ^ 2 }$ if we wind the right $k$. So, we need to find
the right $k$.

So, large key space is not enough.  We need to ensure that frequency is also fudged.


\subsection{Monoalphabetic Substitution Cipher}

Create a bijection $\{ f : Letter \leftarrow Letter \}$.
This has a large key space, $\{ 26! \}$.


Attack is based on frequency. $\{ \forall x \in Letter, freq(x) = freq(f(x)) \}$.
So, one can match $x$ with $f(x)$.

Again, we need to fudge frequency.

\subsection{Polyalphabetic sustitution cipher}

This needs a passphrase, for example, Cat.

Add passphrase to plaintext.

${c r y p t o + c a t c a t = \cdots }$

Frequencies are not maintained, because different text is added each time
to the same plaintext.

\begin{itemize}
    \item Step 1 - Given length, we break the cipher.
    \item Step 2 - Length is susceptible to brute force attack.
\end{itemize}

\subsubsection{Breaking given length}
Assume the length of passphrase is known, say, $k$.

Let ciphertext be $c_0 c_1 c_2 c_3 c_4 ..$.  Let us look at ciphertext at
lengths of 3.

This will give us a \textit{shift cipher}, since the text is all shifted by the
\textit{same} letter in the passphrase. Now, we can perform the frequency
attack.

If we screw up the partition, then the frequency spectra will be gibberish.
zsh:1: command not found: :w

\subsubsection{What we learnt by breaking}
This was also broken. So, they learnt that "security is hard, forget it!".
Or, complication does not imply security.


\subsection{What is an Unbreakable cipher? Or, shannon enters the scene}
\subsubsection{A preamble, the thought process}
\begin{itemize}
    \item We need to specify what it means to have a good cipher. Where do
        we stop? We need a formal spec. (Definition of security).
    \item Precise assumptions involved must be known. (Hardness assumption).
    \item The truth of security, and the trade-offs involved (Shannon's Proof)
\end{itemize}
