\documentclass[11pt]{book}
%\documentclass[10pt]{llncs}
%\usepackage{llncsdoc}
\usepackage{amsmath,amssymb}
\usepackage{graphicx}
\usepackage{makeidx}
\usepackage{algpseudocode}
\usepackage{algorithm}
\usepackage{listing}
\evensidemargin=0.20in
\oddsidemargin=0.20in
\topmargin=0.2in
%\headheight=0.0in
%\headsep=0.0in
%\setlength{\parskip}{0mm}
%\setlength{\parindent}{4mm}
\setlength{\textwidth}{6.4in}
\setlength{\textheight}{8.5in}
%\leftmargin -2in
%\setlength{\rightmargin}{-2in}
%\usepackage{epsf}
%\usepackage{url}

\usepackage{booktabs}   %% For formal tables:
                        %% http://ctan.org/pkg/booktabs
\usepackage{subcaption} %% For complex figures with subfigures/subcaptions
                        %% http://ctan.org/pkg/subcaption
\usepackage{enumitem}
%\usepackage{minted}
%\newminted{fortran}{fontsize=\footnotesize}

\usepackage{xargs}
\usepackage[colorinlistoftodos,prependcaption,textsize=tiny]{todonotes}

\usepackage{hyperref}
\hypersetup{
    colorlinks,
    citecolor=black,
    filecolor=black,
    linkcolor=black,
    urlcolor=black
}

\usepackage{epsfig}
\usepackage{tabularx}
\usepackage{latexsym}
\newcommand\ddfrac[2]{\frac{\displaystyle #1}{\displaystyle #2}}

\def\qed{$\Box$}
\newtheorem{corollary}{Corollary}
\newtheorem{theorem}{Theorem}
\newtheorem{definition}{Definition}
\newtheorem{lemma}{Lemma}
\newtheorem{observation}{Observation}
\newtheorem{proof}{Proof}

%\newcommand{\P}{\texttt{P}}
%\newcommand{\NP}{\texttt{NP}}
%\newcommand{\PSPACE}{\texttt{PSPACE}}
%\newcommand{\NPSPACE}{\texttt{NPSPACE}}
%\newcommand{\TQBF}{\texttt{TQBF}}

\newcommand{\cobpp}{\texttt{co-BPP}}
\newcommand{\ip}{\texttt{IP}}
\newcommand{\dip}{\texttt{DIP}}
\newcommand{\zkp}{\texttt{ZKP}}

\newcommand{\textbb}[1]{$\mathbb{#1}$}
\newcommand{\nats}{\mathbb{N}}
\newcommand{\reals}{\mathbb{R}}


\newcommand{\hashsat}{\texttt{\#SAT}}
\newcommand{\tqbf}{\texttt{TQBF}}


\newcommand{\pptm}{\texttt{PPTM}}
\newcommand{\dtm}{\texttt{dtm}}


\title{Computational Complexity Theory}
\author{Siddharth Bhat}
\date{}

\begin{document}

\maketitle
\tableofcontents

% http://mirrors.ibiblio.org/CTAN/macros/latex/contrib/physics/physics.pdf
\newcommand{\qdot}{{\dot q}}

\chapter{Lagrangian, Hamiltonian mechanics}

Mechanics in terms of generalized coords.
\section{Lagrangian}
Define a functional. $L$ over the config. space of partibles $q^i$, $qdot^i$.
$L = L(q^i, qdot^i)$.  We have an explicit dependence on $t$.



$L = KE - PE$

Assuming a 1-particle system of unit mass,
$$L = \frac{1}{2} \qdot^2 - V(q)$$

Assuming an n-particle system of unit mass,
$$L = \sum_i \frac{1}{2} {qdot^i}^2 - V(q^i)$$ 

\section{Variational principle}

Take a minimum path from $A$ to $B$. Now notice that the path that is
slightly different from this path will have some delta from the minimum.

Action
$$S(t0, t1) = \int L \dd t = \int_{t0}^{t1} L(q^i, qdot^i) \dd t$$.
Least action: $\delta S = 0$

% \begin{align*}
%     \delta S &= \delta \int L(q^i, qdot^i) \dd{t} \\
%              &= \int \delta L(q^i, qdot^i) \dd{t} \\
%              &= \int \pdv{L}{q^i} \delta q^i + \pdv{L}{qdot^i} \delta qdot^i \dd{t} \\
% \begin{align*}




%% What is the document class I need?
\documentclass{article} 
%% Some recommended packages.
\usepackage{booktabs}   %% For formal tables:
                        %% http://ctan.org/pkg/booktabs
\usepackage{subcaption} %% For complex figures with subfigures/subcaptions
                        %% http://ctan.org/pkg/subcaption
\usepackage{enumitem}
\usepackage{minted}
\newminted{fortran}{fontsize=\footnotesize}

\usepackage{xargs}
\usepackage[colorinlistoftodos,prependcaption,textsize=tiny]{todonotes}


\begin{document}
\section{Lecture 2 - Signals and Systems: Signals}
Dennis Freeman, MIT lectures are supposedly very good.

\subsection{Brief Overview}

If $x(t)$ was our continuous time function, we wish to create
$x[n]$. $x[n]$ is discrete.

\subsection{Recovering back signals}
\begin{itemize}
    \item Zero order hold
        Keep previous value we had till the next value
    \item Nearest Neighbour (NN)
        For each point, pick nearest neighbour and use that.
    \item Linear interpolation
    \item $\frac{sin(x)}{x}$
\end{itemize}


\subsection{Operations on signals}

\begin{itemize}
    \item Shifting: $y[n] = x[n - k]$. Delay by $k$ steps
    \item Flipping: $y[n] = x[-n]$.
    \item Scaling: $y[n] = \cdot x[k * n]$. This would lead to loss of information.
\end{itemize}

Order of operations: Shift, flip, scale. For counterexample, consider
$y[n] = x[3n + 1]$. Proof by induction on $ax + b$?

Eg: $x[-2n + 2]$. Perform as:

$y[n] = x[n + 2]$.

\subsection{Characteristics of signals}

\subsubsection{Even, Odd}

Even signal $x[n] = x[-n]$
Odd signal $x[n] = -x[-n]$

Every signal can be decomposed into sum of even and odd signals
$x[n] = \frac{x[n] + x[-n]}{2} + \frac{x[n] -x[-n]}{2}$

\subsubsection{Periodic}
$\exists N \in Z, \forall n \in Z, x[n] = x[n + N]$.

\subsubsection{Energy and power}

$Energy = \sum_{k=-\infty}{\infty} |x[k]|^2$


$Power = lim_{N \larrow \infty} \frac{1}{2N + 1} \sum_{k=-N}{N} |x[k]|^2$

\subsubsubsection{Unit Step}

u(n) = 1 if n >= 0, 0 otherwise.
Energy is \infty, power is \frac{1}{2}.


\subsection{Special Signals}

\subsubsection{Dirac delta}
$\delta[n] = 0 if n \neq 0, 1 if n = 0$.

\subsubsection{Unit Step}
u[n] = 1 if n >= 0, 0 otherwise.


\subsubsection{Ramp function}
r[n] = n if n >= 0, 0 otherwise.

\subsubsection{Exponential function}
e[n] = a^n u[n]

a is a parameter.


\subsection{Writing unit step in terms of delta}
u[n] = \sum_{k=0}{\infty} delta[n - k]

\subsection{Writing delta in terms of unit step}
delta[n] = u[n + 1] - u[n]

\subsection{Writing any signal $x[n]$ in term of $delta[n]$}
$x[n] = \sum_{k=-\infty}{infty} delta[n - k] \cdot x[k]$

\chapter{Tensor product states}

\section{Postulates of QM}
\begin{itemize}
\item Associated to any isolated physical system is a complex vector space
with inner product. This space is called as the state space of the system.
This system is completely described by its state vector which is a unit
vector in the state space.
\end{itemize}

\section{Tensor product}

Let $A$ and $B$ be vector spaces with bases $A_{basis}, B_{basis}$.
$A \tensor B$ is a \emph{new vector space}, whose basis vectors are $a_i \tensor b_j$
where $a_i \in A_{basis}, b_i \in B_{basis}$.

Properties of the tensor product:
\begin{itemize}
    \item For any arbitrary scalar $z$ and element $v \in H_a$, $w \in H_b$,
        $z (\ket v \tensor \ket w) = (z \ket v) \tensor \ket w = \ket v \tensor (z \ket w)$
    \item $(\ket v_1 + \ket v_2) \tensor \ket w = \ket v_1 \tensor \ket w + \ket v_2 \tensor \ket w$
    \item $\ket w \tensor (\ket v_1 + \ket v_2)= \ket w \tensor \ket v_1 + \ket w \tensor \ket v_2$
    \textbf{TODO: what is an easy way to get correctly sized brackets?}
    \item Suppose $\ket v \in H_a, \ket w \in H_b$, and $A$ and $B$ are linear
        operators on $H_a$ and $H_b$ respectively. 
        $(A \tensor B) (\ket v \tensor \ket w) \equiv (A \ket v) \tensor (B \ket w)$.

    \item Let $C = \sum_i c_i A_i \tensor B_i$, where $A_i, B_i$ are linear
        operators on $H_a, H_b$. Now, $C (\ket v \tensor \ket w) = \sum_i c_i ((A_i \ket v) \tensor (B_i \ket w))$
    \item $\ket x = \sum_i a_i \ket v_i \tensor \ket w_i$. $\ket y = \sum_j b_j \ket v_j \tensor \ket w_j$.
        Now, $\bra{x}\ket{y} = (\sum_i a_i^* \bra v_i \tensor \bra w_i)(\sum_j b_j \ket v_j \tensor \ket w_j)$,
        which is equal to $\sum_i \sum_i a_i^* b_j \bra{v_i}\ket{v_j'} \bra{w_i}\ket{w_j'}$
\end{itemize}

This is way too redundant, \textbf{TODO:} write down the slick definition of tensor
product spaces seen in John Lee's intro to smooth manifolds, or the definition
seen in Tensor Geometry: The Geometric Viewpoint and its uses.

\begin{align*}
    \tr(A \ket \psi \bra \psi) = 
    \sum_i \bra i A \ket \psi \bra{\psi}\ket{i} =  
    \sum_i (\bra{\psi}\ket{i}) \cdot (\bra i A \ket \psi) = 
    \sum_i \bra{\psi}(\ket{i} \bra i) A \ket \psi = 
    \bra{\psi} A \ket \psi 
\end{align*}


\begin{theorem}
    Two operators $A$, $B$ are simeltanelously diagonalizable iff $[A, B] = 0$,
    where $[A, B] = AB - BA$. That is, there exists a basis where both $A$
    and $B$ are diagonal matrices.
\end{theorem}
\begin{proof}
    One direction of the proof is easy. If two operators are simeltanelously
    diagonalizable, then we can simply write both operators in this common
    basis. Diagonal matrices commute, hence $[A, B] = 0$.

    Let $\ket{a, j}$ be an orthonormal basis for the eigenspace $V_a$ of $A$
    with eigenvalue $a$ and index $j$ to label repeated eigenvalues.

    $AB \ket{a, j} = BA \ket{a, j} = a B \ket {a, j}$. Hence,
    $A (B \ket {a, j} = a (B \ket {a, j}$. Hence, $B \ket {a, j}$ is an
    eigenvector of $A$. Therefore, $B\ket{a, j} \in V_a$. 

    Define projector $P_a$ onto $V_a$. Now, define $B_a = P_a B P_a$.
\end{proof}

\chapter{Gauge theories}
We construct a 1-dimensional gauge theory and study its symmetries.

\section{Euler-Lagrange equations for a field}
Consider a Lagrangian:
$$\Lag(\phi, \phi', \dot \phi) = {\dot \phi}^2  - \phi'^2$$

When written in terms of $M^4$ (minkowski space), we know that
$$
\partial_\mu \equiv (\partial_{t}, - \grad)
$$

Hence, in minkowski space, the lagrangian becomes a function of \textit{only}:
$$
\Lag(\phi, \partial_\mu \phi) \equiv \dots
$$

So we managed to unify the space derivative and the time derivative (yay).

Now, we can consider the action of this Lagrangian:

\begin{align*}
    S[\phi] = \int L(\phi, \partial_\mu \phi)~\dd^4x
\end{align*}

Minimising the functional $S$,
\begin{align*}
    &\delta S[\phi] = 0 \\
    & \delta \int L(\phi, \partial_\mu \phi)~\dd^4x = 0 \\
    %
    &\text{For an analogy, consider $dL = \pdv{L}{\phi} \dd \phi + \pdv{L}{\psi} \dd \psi + \dots$} \\
    %
    & \int \bigg[\pdv{L}{\phi} \var \phi +
    \pdv{L}{\partial_\mu \phi} \var (\partial_\mu \phi) \bigg] \dd^4 x  = 0\\
    %
    &\text{Using linearity of integration, and commuting of $\delta, \partial_\mu$}\\
    %
    & \int \pdv{L}{\phi} (\var \phi) \dd^4 x +
      \int \pdv{L}{\partial_\mu \phi}  (\partial_\mu \var \phi) \dd^4 x = 0 \\
    %
    &\text{Using $\int U \dd V = UV - \int V \dd U$,
        $V = \delta \phi$,
    $U = \pdv{L}{\partial_\mu \phi}$ } \\
    \\
    %
    & \int \pdv{L}{\phi} (\var \phi) \dd^4 x +
       \eval{\pdv{L}{\partial_\mu \phi}  (\var \phi)}_{endpoints}
      -  \int \delta \phi \bigg(\partial_\mu \fdv{L}{\partial_\mu \phi} \bigg) \dd^4 x  = 0\\
  %
  &\text{Since fields decay at endpoints, forget the integral} \\
  %
  & \int \pdv{L}{\phi} (\var \phi) \dd^4 x
  -  \int \delta \phi \bigg(\partial_\mu \pdv{L}{\partial_\mu \phi} \bigg) \dd^4 x = 0\\
  %
  &\text{Refactoring to pull the common $\delta \phi$,}\\
  %
  & \int \bigg(\pdv{L}{\phi} -  \partial_\mu \pdv{L}{\partial_\mu \phi} \bigg) (\var \phi) \dd^4 x  = 0 \\
  %
  & \text{Since this is true for all perturbations $\delta \phi$ implies that:} \\
  %
  &\pdv{L}{\phi} -  \partial_\mu \pdv{L}{\partial_\mu \phi} = 0
\end{align*}


Hence, we have the Euler-Lagrange equation for \textbf{scalar fields}:
\begin{equation}
    \boxed{\pdv{L}{\phi} -  \partial_\mu \pdv{L}{(\partial_\mu \phi)} = 0}
\end{equation}

\section{Lagrangian for the massless scalar field}
So next, let's consider a Lagrangian (which is supposedly the free particle
analogue):
(\textbf{TODO: find out why this is free particle KE})

\begin{align*}
    &(\partial_\mu \phi)^2 \equiv \partial_\mu \phi \partial^\mu \phi~\text{(This is notation)} \\
    &\text{Note that:} \\
    &\partial_\mu \phi \partial^\mu \phi = \partial_\mu \phi \eta^{\mu \nu} \partial_\nu \phi
    ~\text{(where $\eta^{\mu \nu}$ is the metric; lowering indeces)} \\
    %
    \\
    &\text{We define the Lagrangian as:}\\
    &\Lag(\phi, \partial_\mu \phi) = \frac{1}{2}(\partial_\mu \phi)^2
\end{align*}

Calculating the terms in the EL equation:
\begin{align*}
    &\pdv{L}{\phi} = 0 \\
    \\
    \\
    &\pdv{L}{(\partial_\sigma \phi)}
    = \pdv{(\partial_\sigma \phi)} \bigg( \frac{1}{2}(\partial_\mu \phi)^2  \bigg) \\
    %
    &=\pdv{(\partial_\sigma \phi)}  \bigg( \frac{1}{2}\eta^{\mu \nu} (\partial_\mu \phi) (\partial_\nu \phi) \bigg) \\
    %
    &=\frac{1}{2}\eta^{\mu \nu} \pdv{(\partial_\mu \phi)}{(\partial_\sigma \phi)} (\partial_\nu \phi) +
    \frac{1}{2}\eta^{\mu \nu} (\partial_\mu \phi) \pdv{(\partial_\nu \phi)}{(\partial_\sigma \phi)} \\
    %
    &\textbf{(TODO: understand why $\pdv{(\partial_\mu \phi)}{(\partial_\sigma \phi)}  = \delta_{\mu}^\sigma$)}
    \\
    &=\frac{1}{2}\eta^{\mu \nu} \delta_{\mu}^{\sigma} (\partial_\nu \phi) +
    \frac{1}{2}\eta^{\mu \nu} (\partial_\mu \phi) \delta_{\nu}^{\sigma} \\
    %
    &\text{(Contracting on $\mu$)}\\
    %
    &=\frac{1}{2}\eta^{\sigma \nu} (\partial_\nu \phi) +
    \frac{1}{2}\eta^{\mu \sigma} (\partial_\mu \phi) \\
    %
    &\text{(Replacing dummy index $\mu \equiv \nu$)} \\
    %
    &=\eta^{\nu \sigma} \partial_{mu} \phi \\
    %
    &\textbf{(TODO: understand how this happens, something about $\eta$'s signature)} \\
    %
    &=\partial^\sigma \phi
    \\
\end{align*}
    Hence, the inner part of the second term in the EL equation is:
\begin{equation}
    \pdv{L}{(\partial_\sigma \phi)}  = \partial^\sigma \phi
\end{equation}
Now, the second term of the Lagrangian is:

\begin{align*}
    &\partial_\mu \pdv{L}{(\partial_\mu \phi)} = \partial_\mu (\partial^\mu \phi)
    =~\partial_0 \partial^0  \phi - \laplacian \phi = \Box \phi
\end{align*}

Hence, finally, our Euler-Lagrange equation is:
\begin{align*}
    &\pdv{L}{\phi} -  \partial_\mu \pdv{L}{(\partial_\mu \phi)} = 0 \\
    &0 + \Box \phi  = 0
\end{align*}

So, finally, The Euler-Lagrange equations for a scalar field governed by $L =
\frac{1}{2} (\partial_\mu \phi)^2$ (which is massless since it has no $\phi^2$
term) is:
\begin{equation}
    \boxed{\Box \phi  = 0}
 \end{equation}
\section{Lagrangian for a massive scalar field}
We now draw analogies to classical and relativistic mechanics to find
the mass term in the scalar field equation. We know that
$E^2 = p^2 c^2 + m^2 c^4$, which on performing the QM substitution
(\textbf{TODO: understand precisely why this is the substitution}), provides
us with the equation:  $\Box + m^2 = 0$.

Acting on a scalar field, we get:
\begin{align*}
    &\text{The \textit{Klein-Gordon} equation:}\\
    &(\Box + m^2) \phi = 0
\end{align*}

We try something in the Lagrangian to get something like the Klein-Gordon
equations (\textbf{TODO: AFAICT, this is ad-hoc observation by physicists. Find deeper reason})

\begin{align*}
    &\Lag = \frac{1}{2} (\partial_\mu \phi)^2 + \alpha m^2 \phi^2~\text{($\alpha$ is a coefficient to be determined)} \\
\end{align*}

The Euler-Lagrange equations work out to:

\begin{align*}
    &\text{First term:}
    ~\pdv{L}{\phi} =  2 \alpha m^2 \phi\\
    %
    &\text{Second term:}
    ~\partial_\mu \pdv{L}{(\partial_\mu \phi)} = \Box \phi \\
    %
    &\text{Full EL equations:} \\
    &2 \alpha m^2 \phi - \Box \phi = 0 \\
    &(2 \alpha m^2 - \Box) \phi = 0 \\
    %
    &\text{Comparing with Klein-Gordon equation, $(\Box + m^2) \phi = 0$, $\alpha = \frac{-1}{2}$}
\end{align*}

Hence, the Lagrangian for a scalar mass field works out to:
\begin{equation}
    \boxed{\Lag = \frac{1}{2} (\partial_\mu \phi)^2 - \frac{1}{2} m^2 \phi^2}
\end{equation}

\section{Symmetries of a scalar field Lagrangian}
Now, we look at the scalar field with mass, and try to study the gauge
theory (ie, the theory of symmetries) for this object.
$$
\Lag = \frac{1}{2} (\partial_\mu \phi)^2 - \frac{1}{2} m^2 \phi^2
$$

We replicate the Lagrangian, giving us $\Lag_1$ and $\Lag_2$, whose "interaction"
we study to arrive at the gauge.

\begin{align*}
\Lag_1 &= \frac{1}{2} (\partial_\mu \phi_1)^2 - \frac{1}{2} m^2 \phi^2 \\
&=\frac{1}{2} \eta^{\mu \nu} \partial_\mu \phi_1 \partial_\nu \phi_1 - \frac{1}{2} m^2 \phi_1^2 \\
%
\Lag_2 &= \frac{1}{2} \eta^{\mu \nu} \partial_\mu \phi_2 \partial_\nu \phi_2 - \frac{1}{2} m^2 \phi_2^2
\end{align*}

We make a new Lagrangian by adding the two previous Lagrangians:
\begin{align*}
&\Lag(\phi_1, \phi_2, \partial_\mu \phi_1, \partial_\mu \phi_2) = \Lag_1 + \Lag_2 \\
%
&=~\bigg(\frac{1}{2} \eta^{\mu \nu} \partial_\mu \phi_1 \partial_\nu \phi_1 - \frac{1}{2} m^2 \phi_1^2\bigg) +
\bigg(\frac{1}{2} \eta^{\mu \nu} \partial_\mu \phi_2 \partial_\nu \phi_2 - \frac{1}{2} m^2 \phi_2^2\bigg) \\
%
&=~\frac{1}{2} \eta^{\mu \nu}(\partial_\mu \phi_1 \partial_\nu \phi_1 + \partial_\mu \phi_2 \partial_\nu \phi_2)
- \frac{1}{2} m^2 (\phi_1^2 + \phi_2^2) \\
%
&\textbf{(TODO: how did it become $\partial_\mu (\phi_1 + \phi_2) \partial_\nu (\phi_1 + \phi_2)$?)} \\
%
&=\frac{1}{2} \eta^{\mu \nu} \partial_\mu (\phi_1 + \phi_2) \partial_\nu (\phi_1 + \phi_2)
- \frac{1}{2} m^2 (\phi_1^2 + \phi_2^2)
\end{align*}

Now, we define a \textit{new} $\phi \equiv \phi_1 + i \phi_2$, since the existence of
$\phi_1^2 + \phi_2^2$ hints at some kind of rotational symmetry. Now, this will allow
us to explore the $U(1)$ rotational symmetry which possibly exists. Note that because of this definition:

\begin{align*}
&\phi \equiv \phi_1 + i \phi_2 \\
&\phi^* \equiv \phi_1 - i \phi_2 \\
&\phi_1 \equiv \frac{\phi + \phi^*}{2} \\
&\phi_2 \equiv   \frac{\phi - \phi^*}{2i} \\
\end{align*}

Plugging these back into the Lagrangian \textbf{(TODO: work this out!)}
\begin{align*}
L = \frac{1}{2}(\partial_\mu \phi \partial^\mu \phi^*) - \frac{1}{2}m^2 \phi \phi^*
\end{align*}

Now, we see a symmetry of $\phi \to e^{i \theta}\phi$. This will leave the Lagrangian invariant, hence
$U(1)$ symmetry (This is shown later, that the Lagrangian is invariant, but we're trying to give a taste here)


\chapter{Hamiltonians, Creation and annhilation, etc (page 61 - 72)}
$$H(p, q) = \frac{p^2}{2m} + V(q)$$

\chapter{Noether's theorem}

Consider $L(q_i, \dot q_i, t)$, where $i \in [1\dots N]$.
If the $L$ is independent of a particular coordinate $i$, then the generalized momentum
corresponding to this is conserved.

Let $L(r, \dot r, t) \to L(r + \epsilon, \dot r, t)$. Now:

\begin{align*}
\delta L = L(r + \epsilon, \dot r, t) - L(r, \dot r, t) = \pdv{L}{r} \epsilon
\end{align*}

If this is a symmetry, then $\delta L = 0$. \textbf{TODO: why?}. Since this is
true for \textit{arbitrary} $\epsilon$, Hence, $\pdv{L}{r} = 0$, which means
we now have a symmetry!

Let's consider the Euler-Lagrange equations for $\Lag$:

\begin{align*}
 &\pdv{L}{r} - \pdv{t} \bigg( \pdv{L}{\dot r} \bigg) = 0 \\
&\text{We have already shown that $\pdv{L}{r} = 0$, so:} \\
&\pdv{t} \bigg( \pdv{L}{\dot r} \bigg) = 0 \\
&\pdv{L}{\dot r} \propto \dot r~\text{(We know the lagrangian is the kinematic Lagrangian)} \\
&\pdv{t} \dot r = 0 \implies \pdv{p}{t} = 0~\text{(p is a constant!)}
\end{align*}

\subsection{Rotationally invariant Lagrangian}

Consider a Lagrangian:
$$\Lag = \sum_i \frac{1}{2} m \va{r_i}^2 - \sum_j V(|\va r_i - \va r_j|)$$

This is rotationally invariant. Let's rotate the system
through an angle $\epsilon$ about an axis $\hat n$.
$$
\va r_i \to r_i + \epsilon (\hat n \cross \va r_i),~
\dot{r_i} \to \dot{r_i} + \epsilon (\hat n \cross \dot{  r_i})
$$

\begin{align*}
&\delta \Lag = \Lag(r_i + \epsilon (\hat n \cross r_i), \dot{r_i} + \epsilon (\hat n \cross \dot{ r_i}, t)
- \Lag(r_i, \dot r_i, t) \\
&\pdv{L}{r_i} \delta r_i + \pdv{L}{\dot r_i} \delta {\dot r_i}
\end{align*}

\section{Derving the force of the EM-field from the Lanrangian}

Recall that $B = \curl A$, $E = \grad \phi - \pdv{A}{t}$, and the force
on a particle is $\va F = q (\va E + \va v \times \va B)$.

\begin{align*}
ma &= e (\grad \phi - \pdv{A}{t}) + e (v \times (\curl A)) \\
ma &= e (\grad \phi - \pdv{A}{t}) + e (\div (v \vdot A) - (v \vdot \grad) A)
\end{align*}

Note that $(v \cdot \grad) A$ is:
\begin{align*}
v \vdot \grad = \dv{x}{t} \pdv{x} + \dv{y}{t} \pdv{y} + \dv{z}{t} \pdv{z} \\
(v \vdot \grad) A = \dv{x}{t} \pdv{A}{x} + \dv{y}{t} \pdv{A}{y} + \dv{z}{t} \pdv{A}{z} \\
\end{align*}

However, now let us compare $\dv{A}{t}$ and $(v \cdot \grad) A$:

\begin{align*}
\dv{A}{t} = \pdv{A}{t} + \dv{x}{t}\pdv{A}{x} + \dv{y}{t}\pdv{A}{y} \dv{z}{t}\pdv{A}{z} \\
\dv{A}{t} = \pdv{A}{t} + (v \cdot \grad) A
\end{align*}

Now, rewriting $ma$,
\begin{align*}
ma &= e (\grad \phi - \pdv{A}{t}) + e (\div (v \vdot A) - (v \vdot \grad) A) \\
ma &= e (\grad \phi - \pdv{A}{t}) + e (\div (v \vdot A) - (v \vdot \grad) A)
\end{align*}


\chapter{Charged particle interaction in fields, or, how maxwell's equations have $U(1)$ symmetry}
(written in red ink pen)

Consider a scalar field $\phi$, and the lagrangian:

\begin{align*}
    &\Lag(\phi, \partial_\mu \phi) = \frac{1}{2} (\partial_\mu \phi) (\partial_\mu \phi) - \frac{1}{2} m^2 \phi^2
\end{align*}

Next, we want to consider charged particle interactions, which comes from
$H = \frac{(p - eA)^2}{2m}$, which creates the lorentz force $e \va v \times \va B = e \va v \vdot \va A$.

We know that $\Lag$ is invariant under global rotation $e^{i \theta}$. Now we
study local gauge invariance by making $\theta$ a function of space. That is,
$\theta \to \theta(x)$.

We use the complex form of the lagrangian:

\begin{align*}
    \Lag = (\partial_\mu \phi) (\partial^\mu \phi)^* - m^2|\phi|^2
\end{align*}

We now consider the transform:

\begin{align*}
    phi \to e^{i e \theta(x)} \phi
\end{align*}

This implies the transforms:
\begin{align*}
    &\partial_\mu \phi =
    \partial_\mu (e^{i e \theta(x)} \phi) =
    e^{i e \theta(x)} \partial_\mu \phi + i e e^{i e \theta(x)} \phi \partial_\mu \theta = \\
    &e^{i e \theta(x)}(\partial_\mu \phi + i e \phi \partial_\mu \theta)
\end{align*}

Hence, for invariance, we get:

\begin{align*}
    &(\partial_\mu \phi ) (\partial^\mu \phi )^*  \\
    =~&(e^{i e \theta(x)}(\partial_\mu \phi + i e \phi \partial_\mu \theta))
    (e^{i e \theta(x)}(\partial^\mu \phi + i e \phi \partial^\mu \theta))^* \\
     =~&(\partial_\mu \phi + i e \phi \partial_\mu \theta)(\partial^\mu \phi^* - i e \phi^* \partial^\mu \theta)
\end{align*}

We introduce:
\begin{align*}
    D_\mu \phi \equiv (\partial_\mu - i A_\mu) \phi (\partial^\mu + i A^\mu) \phi^*
\end{align*}

We shall check that this transforms in a lorentzian way. In the new coordinate
system:
\begin{align*}
    \bar D_\mu \bar \phi \equiv &([\partial_\mu - i \bar A_\mu) \bar \phi]
        [(\partial^\mu + i \bar A^\mu) \bar \phi^*] \\
    %
    =~&[(\partial_\mu - i \bar A_\mu) (e^{i e \theta(x)}) \phi]
    [(\partial^\mu + i \bar A^\mu) (e^{i e \theta(x)}\phi)^*]  \\
    %
    =~&[\partial_\mu (e^{i e \theta(x)} \phi) - i \bar A_\mu e^{i e \theta(x)} \phi]
    [\partial_\mu (e^{-i e \theta(x)} \phi^*) + i \bar A_\mu e^{-i e \theta(x)} \phi^*] \\
    %
    &\text{evaluate $\partial_\mu(UV)$} \\
    =~&[e^{i e \theta(x)} (\partial_\mu \phi) + i e  e^{i e \theta(x)} (\partial_\mu \theta)\phi - i \bar A_\mu e^{i e \theta(x)} \phi]
    [e^{- i e \theta(x)} (\partial_\mu \phi^*) - i e  e^{- i e \theta(x)} (\partial_\mu \theta) \phi^* + i \bar A_\mu e^{- i e \theta(x)} \phi^*] \\
    %
    =~&[e^{i e \theta(x)} (\partial_\mu \phi + i e (\partial_\mu \theta) \phi - i \bar A_\mu  \phi)]
    [e^{- i e \theta(x)} (\partial_\mu \phi^* - i e (\partial_\mu \theta) \phi^* + i \bar A_\mu  \phi^*)] \\
    %
    &\text{cancelling $e^{i e \theta(x)}$ with $e^{- i e \theta(x)}$} \\
    =~&[\partial_\mu \phi + i e (\partial_\mu \theta) \phi - i \bar A_\mu  \phi]
    [\partial_\mu \phi^* - i e (\partial_\mu \theta) \phi^* + i \bar A_\mu  \phi^*] \\
    %
    =~&[\partial_\mu + i e (\partial_\mu \theta) - i \bar A_\mu  ]\phi
    [\partial_\mu  - i e(\partial_\mu \theta) + i \bar A_\mu ] \phi^* \\
\end{align*}

Comparing:
\begin{align*}
    D_\mu \phi &\equiv (\partial_\mu - i A_\mu) \phi (\partial^\mu + i A^\mu) \phi^* \\
    %
    \bar D_\mu \bar \phi &\equiv [\partial_\mu + i e (\partial_\mu \theta) - i \bar A_\mu  ]\phi
    [\partial_\mu  - i e (\partial_\mu \theta)  + i \bar A_\mu ] \phi^* \\
    %
    -i A_\mu &= + i e (\partial_\mu \theta) - i \bar A_\mu \\
    -A_\mu &= e (\partial_\mu \theta) - \bar A_\mu \\
    \bar A_\mu &= e (\partial_\mu \theta) + A_\mu \\
   &\textbf{TODO: This is not what mukku got! mukku got $\bar A_\mu = A_\mu - e(\partial_\mu \theta)$. HOW?}
\end{align*}

So now, we know how $A_\mu$ transforms:
\begin{equation}
    \boxed{\bar A_\mu = A_\mu - e(\partial_\mu \theta)}
\end{equation}

So, now, our modified lagrangian is:

\begin{align*}
    \Lag (\phi, \phi^*, \partial_\mu \phi, \partial_\mu \phi^*, A_\mu) =
    [(\partial_\mu - i A_\mu) \phi][(\partial^\mu - i A^\mu) \phi]^* - \frac{1}{2} m |\phi|^2
\end{align*}



We want to understand if we can add interaction terms for $A_\mu$. If not,
what the obstacles are. We would like the interaction terms to be a Lorentz
tensor, and we would like it to respect local symmetries.

\begin{align*}
    &\mu^2 A_\mu A^\mu \to \mu^2 \bar A_\mu \bar A^\mu = \mu^2 (A_\mu + e \partial_\mu \theta) (A^\mu + e \partial^\mu \theta)
\end{align*}
So clearly, we cannot have mass terms of the form $\mu^2 A_\mu A^\mu$. We must
try other things and check if they are interesting.

We now try to inspect some invariants of $A_\mu$.  For example, we can
construct:
$$F_{\mu\nu} = \partial_\mu A_\nu - \partial_\nu A_\mu$$

We already know that $F$ is a Lorentz tensor. Let's check if it's gauge
invariant.

\begin{align*}
    \bar F_{\mu\nu} =~&\partial_\nu \bar A_\mu - \partial_\mu A_\nu \\
    =~&\partial_\nu(A_\mu - e (\partial_\mu \theta)) - \partial_\mu (A_\nu - e \partial_\nu \theta)) \\
    %
      &\text{Since partial derivatives commute:} \\
    %
    =~&\partial_\nu A_\mu - \cancel{e \partial_\nu \partial_\mu \theta} - \partial_\mu A_\nu - \cancel{e \partial_\mu \partial_\nu \theta})) \\
    =~&\partial_\nu A_\mu - \partial_\mu A_\nu = F_{\mu \nu}
\end{align*}

Hence, $F_{\mu \nu}$ is also gauge invariant.

Here, there is some sentence of how "there are only two invariants possible
for $F_{\mu \nu}$:

\begin{itemize}
    \item $F^{\mu \nu} F_{\mu \nu}$
    \item $e^{\mu \nu \alpha \beta} F_{\mu \nu} F_{\alpha \beta}$ (what is $e$ in this context?)
        Supposedly, we can ignore this since it relates to divergence (??)
        \textbf{TODO: figure this out}
\end{itemize}

So, extending the lagrangian with the $F_{\mu \nu}$ term gives us:

\begin{align*}
    &\Lag (\phi, \phi^*, \partial_\mu \phi, \partial_\mu \phi^*, A_\mu) =
    \frac{1}{4} F^{\mu \nu} F_{\mu \nu} + [D_\mu \phi][D^\mu \phi]^* - \frac{1}{2} m |\phi|^2 \\
    %
    &\text{where:}\\
    %
    &D_\mu \equiv (\partial^\mu - i A^\mu) \\
    &F_{\mu\nu} \equiv \partial_\mu A_\nu - partial_\nu A_\mu
\end{align*}

Now, performing variational calculus on this lagrangian, we should theoretically
regain maxwell's equations:

\textbf{TODO: follow the last part of the stuff written in red ink pen to
understand what in the actual fuck happened}


\section{Information Theory, CPA security - Lecture 5}


Most theorems will read as: if X is true, then the protocol $\Pi$ is secure.

\section{Our first example of circumventing an impossibility}

\section{PRNGs - Pseudo random number generators}
This allows us to break $|K| \geq |M|$.
This is still a one-time pad, but it allows us to create $|K| << |M|$.


Deterministic program $G$.  Takes as input $n$-bit string, returns $l(n)$ bit string. We have two
assumptions.
\begin{itemize}
\item 1. $l(n) > n$. Expansion.
\item 2. Pseudorandomess.
\end{itemize}

\subsubsection{Pseudorandomness}
for all PPTM(probabilistic polynomial turing machine) $D$,
$$ |P[D(r) = 1] - P[D(G(s)) = 1]| \leq \negl(|s|) \quad  r \leftarrow \binary^{l(n)} \quad  s \leftarrow \binary^n$$
$r, s$ are chosen uniformly at random. The probability distribution is over the
random coins used by $D$, along with the uniform distributions of $r$ and $s$.


\begin{itemize}
\item Strings of length $l(n)$. pick one at random. probability of picking one of them is
\item Strings of length $n$, and then we inject into $l(n)$ with $G$. Clearly, $|Im(G)| < 2^{l(n)}|$.
So, we can sample all $|Im(G)|$. If we are in a pseudo-random world, it will repeat for sure (with $P = 1$).
If we are in the non-PRNG world (true randomness), the chance that something repeats will be negligibly small.
\item We cannot distinguish with polynomial samples, however. So, PPTM is a good choice for a distinguisher.
\end{itemize}


Given that we have to assume PRNGs exist, there are different ways to proceed:
\begin{itemize}
\item Heuristics - Assume that the PRNG we write is a true PRNG, and then get to work.
\item Specific mathematical assumptions - Assume that certain problems are hard. Build PRNGs from this mathematical assumption.
\item Provable Security - If there exists even one hard problem $P$, then we can use that to build a PRNG.
\item Proven security - prove PRNGs exist.
\end{itemize}


\subsubsection{Assume PRNGs exist. We will build a secure encryption scheme}

% \begin{align*}
%     &M = \binary^{l(m)} \quad K = \binary^m \\
%     &Gen: k = {0, 1}^n$ \\
%     &Enc_k(m) = m \xor G(k) \quad Dec_k(c) = c \xor G(k)
% \end{align*}


Note that this is just one time. If they attacker can see two ciphertexts, they can XOR the ciphertexts to get the XOR of the cleartexts.

\paragraph{Proof that this is sane}.
If the adversary can differentiate between $M_0$ $M_1$, we will use it to break the PRNG (as in, distinguish between PRNG and RNG).
Call the adversary A. It can generate 2 messages $M_0$ and $M_1$. When given $encryption(M_b) = G(k) xor M_b$, he can guess $b = 0 \/ b = 1$ with non-negligible probability.
Call the distinguisher $D$. $D$ has to distinguish between truly random and pseudo random world for our proof.
Given a string $w$ and ask if $w$ can be distinguished by $A$. We can pick $w$ from the PRNG world or the RNG world.

If $A(w xor M0, w xor M1)$ gives us the  correct value(can distinguish), then we are using the PRNG. Otherwise, it is the RNG.


\subsubsection{CPA secure}

Adversary gets to pick $M_0$ and $M_1$, we choose a bit $b$ at random and give encryption of $M_b$. He has an oracle that has oracle access to the encryption algorithm. Even with this, he should not be able to guess $b$.

\paragraph{No determistic algorithm can be CPA secure}

The adversary will ask for encryption of $M_0$ and encryption of $M_1$. He gets back $C_0$ and $C_1$. Then, we can compare that to our result, and find the random bit $b$.


\paragraph{How to create CPA secure}
$C = <R, c xor enc(R)>$ where R is a random string.
Decryption will never fail. if we know R, we can xor twice.
Encryption will not fail because encryption of random data is still random.

We have a problem of length doubling: For one length of data, we need R as well.

\subsubsection{Indexable PRNGS}
A PRNG that we can index at a point, and it will start generating from that index. They are called ``pseudorandom functions''.

Consider $Z/pZ^x$. All numbers except $1$ in $Z/pZ$ are generators.

Discrete log: Given $g^x mod p$, given $g$, given $p$, find $x$. (log in a group). We know that Discrete log is hard. Let us try and build a PRNG.


Step 1. Given a PRNG that expands 1 bit, we can use it to create a PRNG that expands any number of bits $n$
$s = seed$. 
$G(s), G(G(s)), G(G(G(s))), G^n(s)$, take the extra bits from each $G^i(s)$. This is a PRNG.

This is a PRNG.

Assume we can break this PRNG. $s_1 s_2 ... s_n$ = stuff from PRNG is distinguishable from $r_1 r_2 r_3 ... r_n$ = Random info.

Construct $s_0 s_1 s_2 .. s_n$, $r_0 s_1 s_2.. s_n$, $r_0 r_1 s_2 s_3 ... s_n$. $r_0 r_1 r_2 r_3 .. r_n$. We know that we can
distinguish first from last. Hence, there must be an adjacent set of strings that can be distinguished, since ``distinguishable'' is transitive (why?)
so, if $r_i  dist r_{i+2}$, we need to have either $r_i dist r_{i + 1}$ or $r_{i + 1} dist r_{i + 2}$. However, between these strings, we have only edited $s_i$.
So, we are able to distinguish one bit extra. This means we can actually distinguish the output of $G$.


Step 2. if we can find $MSB(x)$, we can find x in polynomial time. So, all we need to do is to break $MSB(x)$.


Step 3. Create PRNG that produces one bit output using discrete log.

Take seed s. output $MSB(s_1 = g^s mod p)$. So we now have a PRNG that can create one bit. Second output: $MSB(s_2 = g^{s_1} mod p )$
Third output: $MSB(s_3 = g^{s_3} mod p)$.

Hence, if discrete log is hard, we can get a PRNG. 

\subsection{Multi message Indistinguishability experiment}

This is defined for an encryption scheme $(\gen, \enc, \dec)$.
\begin{itemize}
\item Adversary outputs a pair of vector of messages $(\vec m_0, \vec m_1)$. Each
    vector contains the same number of messages, and the $i$th messages have the
    same length. That is, $|m_0[i]| = |m_1[i]|$.
\item A random key is created: $ k \leftarrow \gen$, and a random bit
    $b \leftarrow \binary$ is chosen. For all $i$, $c[i] \leftarrow \enc_k(m[i])$
    is computed. $\vec c$ is given to the adversary $A$.
\item The adversary $A$ outputs a bit $b'$. The output of the experiment is $1$
    if $b = b'$, and $0$ otherwise.
\end{itemize}

Security definition of the cryptosystem remains unchanged.


\paragraph{Weakness of one time pads under this threat model}
Note that one time pads will fall to this threat model, since repeatedly
ciphering data with a one-time pad will allow us to extract data from the
one-time pad. Indeed, any deterministic scheme can be attacked under
this threat model. So, we now need probabilistic encryption schemes.

\paragraph{Attacking all deterministic cryptosystems under multi message threat model}
Let $m_0 \equiv (0^n, 0^n), m_1 \equiv (0^n, 1^n)$. Run this through the experiment. We will
be given $c \equiv (c_0, c_1)$. If $c_0 = c_1$, then  we know that the message
was $(0^n, 0^n) = m_0$, and is $m_1$ otherwise. We know this since the encryption
function is deterministic, and  hence $\enc_k(m_0) = \enc_k(m_1) \implies m_0 = m_1$.


\section{Information Theory - Lecture 5}
\section{Our first example of circumventing an impossibility}

One way function: hard one way, easy the other
Trapdoor one way: One way function with a trapdoor that makes the hard way easy with the key.

\section{Exploring discrete log problem}

Take a seed, that is a member of $Z_p^x$.

Construct the following sequence of bits: 
$\lbrack MSB (x1)~MSB (x2) ~\dots~MSB(x_i) \rbrack$

\texttt{|| = concatenation}
$x_j =  g^x_{j - 1}$ in $Z_p^x$


Given $(g^x mod p, p, g)$ what is the $MSB(x)$? Is this actually as tough
as trying to find $x$?


\subsection{Theorem: $LSB(x)$ is easy to get}
\subsubsection{Proof}


Fermat's little theorem: $\forall x \in Z_p^x, x^{p - 1} = 1$
\subsubsection{Proof of fermat's little theorem (raw number theory)}
Everything happens in $x^{p - 1}$.

$S = {a, 2a, 3a, ... (p - 1)a}$.
We show that this is a permutation of
$S' = {1, 2, 3, ..., (p - 1) }$

$a \neq 0$. So, $a \cdot x = 0 implies x = 0 since Z_p^x is integral domain$

Suppose two elements are not distinct in $S$. This means that $ai - aj = 0$
Hence, $p | a(i - j)$. But, $a < p$, $(i - j) < p$. Hence, their product 
cannot be divisible by $p$ (product of two numbers less than a prime numbers.


Multiplying all numbers in S should be equal to multiplying all numbers in S'


$a^(p - 1) (p - 1)! =(congruent mod p)= (p - 1)!$
Hence $a^(p - 1) =(congruent mod p)= 1$

\subsubsection{Continuing proof of LSB of discrete log is easy}

$(g^x)^(p - 1) = 1$
$(g^x)^{\frac{p - 1}{2}} = +-1$

When $x$ is even, this will be $+1$. When $x$ is odd, this will be $-1$

In some sense, we are computing $(g^x|p)$ (legendre symbol)

Hence, we can find $LSB(x)$. Note that this will fail if $g^(p - 1)/2 = 1$,
but $g$ is a generator of $Z_p^x$ so it can't happen (g has order |p - 1|).


\subsection{Can we not use this to "peel bits" off? We can peel more than just LSB}

If $4 | p - 1$, then we can reapply the same method to get *two* bits.


$g^{x^{\frac{p - 1}{4}}}$ =

$1. if x == 0 mod 4 -> 1$
$2. if x == 1 mod 4 -> ?$
$3. if x == 2 mod 4 -> ?$
$4. if x == 3 mod 4 -> ?$

\subsection{Hardness given ability to get MSB}

Assume there is an algorithm to find $MSB(x)$ given $y = g^x$ (everything in $Z_p$)

We want to find $sqrt(y)$. That is, it finds $z$ such that $z^2 = y$. Suppose 
sqrt(y) \textit{does exist}.

Note: algorithm to find roots of polynomial in a FF efficiently (?) Look this up. If we 
have this, we can nuke this problem.

SQRT-WHEN-SQRT-EXISTS(y):
Compute $a = y^\frac{p + 1}{4}$.
$a^2 = y^\frac{p + 1}{4})^2 = y^(\frac{p + 1}{2})$.

We know that $y$ is a quadratic residue, so $y = g^2k$
So, $a^2 = (g^2k^\frac{p + 1}{2}) = g^(k * \frac{p + 1})$
<Lost, do the arithmetic yourself>.


\texttt{x -> x/2} if x is even
\texttt{x -> x - 1} if x is odd.

If we have the trace of the function fixpoint (0), then we can reconstruct x.

Going to $x/2$ is difficult because we have two square roots in $Z_p^x$.

\subsubsection{Brilliant: }
If we have an MSB algorithm, then this step can be *made unique*.
If the number is between $0..(\frac{p-1}{2})$, then MSB = 0. Otherwise,
if it is in the other portion, $MSB = 1$. So, we can use $MSB$ because
we know that the sqrt will be $g^\frac{p - 1}{2} = -1$ multiplicative
factor away from each other (the roots of x are $c+-k$)

So, given MSB, we can find discrete log.
Therefore, MSB is just as hard as discrete log, because:

discrete log == MSB algorithm + LSB algorithm (WTF)


\subsection{One way function / Permutation}

$F: \{0, 1\}^n \to \{0, 1\}^n$

There exists PPTM such that $P \lbrack M(x) == F(x) \rbrack = 1 - negligible$.


For all PPTM A, forall x chosen at random from domain(f), 
$P[A(f(x)) \in f^{-1}(f(x))] = negligible$

\subsection{Hard-core predicate of a one-way function $f$}

$H:{0, 1}^n -> {0, 1}$ is a hard core predicate of $f$ if 

1. x -> h(x) is easy,


$\forall PPTM A, \forall random x in dom(f), P[A(f(x)) = h(x)] = negligible + \frac{1}{2}$

As in, should be negligible from random guess (since range is ${0, 1}$).



\subsection{Convert one-way function to PNG}

PNG(s) $h(s1) || h(s2) || h(s3) ... || h(s_n)$

$|| = concatenation$

$s_i = f(s_{i - 1})$
$s_0 = s$


\subsection{General construction of hard-core predicates}

For a one-way function f, the XOR of a random subset of bits will be a hardcore
prediate.

Let $I$ be the index set, $I \subset \lbrack 1 \dots n \rbrack$.
$H(x) = \texttt{XOR} x_i, i \in I$ will be a hardcore predicate.


\subsection{Exercise}
if $p = s.2^r, maximum r$, LSB is $0$th bit, $r$th bit is a hardcore predicate.

\chapter{Quantum Computing: Shor's algorithm}

We have $pq = N$. We wish to find $x$ such that $y = a^x \mod N$.

\begin{align*}
    &s_0 = \ket 0^{\tensor n} \\
    &s_1 = H^{\tensor n} s_0  = \frac{1}{2^n} \sum_i \ket{i} \\
    &s_2 = a^{s_1} \mod N = \frac{1}{2^n} \sum_i \ket{a^i \mod N} \\
\end{align*}

Let us now consider the function $f(x) = a^x \mod N$. This function will
be periodic with period $r$. Let us assume that $f: [0, Q-1] \to [0, Q-1]$ where
$Q$ is the domain of the function / the maximum value that is fed to $f$.

Now, note that since the function is periodic, $\l[\forall y, |f^{-1}(y)| = Q/r\r]$.

\begin{align*}
    &s_3 = measure(s_2) = \frac{1}{\sqrt\frac{Q}{r}} \l(\ket{a_0} + \ket {a_0 + r} + \dots \r)
\end{align*}

At this point, the states in $s_3$ will consists of inputs $\l[ a_0, a_0 + r, \dots a_0 + \delta r \r]$
such that $f(a_0 + \delta r) = m_0$.

We now wish to extract the $r$ from the superposition of states. A non solution
is to try and repeatedly measure the values, then what we can get is a set of
values $\l[a_0 + \delta_0 r, a_1 + \delta_1 r, a_2 + \delta_2 r, \dots \r]$.
Recovering $r$ from this set is difficult, so we try another solution.

because the Fourier transform is a change of basis, it's a unitary matrix,
and can hence be implemented as a quantum circuit. Since the function $f$
periodic and $r$ is the perid, feeding $f$ into a fourier transform will
allow us to find $r$. 

On applying the fourier transform, the function becomes a new function
such that $g \equiv FFT(f)$ such that $g(0) = g(Q/r) = g(2Q/r) = g(\lambda Q/r) = 1, g(\_) = 0~\text{otherwise}$.

    

\chapter{Probabilistic proofs}
\section{Completeness and Soundness}
\begin{definition}
Completeness: For every true assertion, there is a valid proof.
\end{definition}

\begin{definition}
Soundness: For every false assertion, no valid proof exists.
\end{definition}

A good proof system must also be such that the verifier is efficient
(that is, polynomial time).

If we ask that a proof system must be sound and complete, there is no 
scope for error! Further, it is not clear if the verifier and the
prover can "talk" to each other. If we choose to allow interactions, what
are the implications?


We relax the assumptions this way --- Relaxed compleness states that
for every true assertion, there is a
proof strategy that will convince the verifier with probability 
at least $> \frac{1}{2}$.  
Similarly, relaxed soundness states that for every false assertion,
every proof strategy fails to convinve the verifier with probability
at least $> \frac{2}{3}$. 

The formalization is as follows:
\begin{definition}
Interactive proof systems
\begin{itemize}
\item An interactive proof system for a language $L$ consists of two
entities: a prover $P$ and a verifier $V$.
$P$ and $V$ share common input, and work for $R \in \mathbb{N}$ rounds.

\item In each round, the prover can send the verifier a message that 
is polynomial in the length of the input.

\item The verifier can send a polynomial length reply to the prover.

\item The verifier is a randomized polynomial time turing machine. Time
is measured as a function of the length of the input.

\item \textbf{Completeness}: $\forall x \in L$, there exists a prover strategy
so that the verifier accepts with probability $> \frac{2}{3}$.

\item \textbf{Soundness}: $\forall x \notin L$, any prover strategy will lad
the verifier to accept with probability  $< \frac{1}{3}$.
\end{itemize}
\end{definition}

\chapter{B\"uchi Automata 1}

\section{$\omega$ regular languages}
$\omega$ regular expressions are regexes plus the $\omega$ operator.

$$L^\omega \equiv \{ w_1 w_2 \dots : \w_i \in L \forall i \geq 1 \}$$

an $\omega$ regular expression is of the form $\gamma \equiv \alpha_1 \beta_1^\omega + \alpha_2 \beta_2 ^\omega + \dots + \alpha_i \beta_i ^ \omega$.

The semantics is of this is going to be:

$$
L(\omega) \equiv \bigcup_{1 \leq i \leq n} L(\alpha_i) L(\beta_i)^\omega
$$


\begin{itemize}
\item elements of $L \equiv (A^\star B)^\omega$ is the set of infinite words containing infinitely many $B$s.
\item elements of $L \equiv (A^\star B)^\omega + (B^\star A)^\omega$ is the set of infinite words containing infinitely many $B$ or infinitely many $A$s.
\item infinite words where each $A$ is followed by a letter $B$. To build this, we case split on whether $A$ occurs finitely or infinitely many times,
 If $A$ occurs finitely many times, then $(B^\star A B)^\star B^\omega$ works.
 If $A$ occurs infinitely many times, then $(B^\star A B) ^\omega$ works.
\end{itemize}

% use amsfonts for box, diamond.
See that we can encode all of the usual LTL properties. Let $L$ be a language with alphabet $\Sigma$, and let us extend the regex operators with $\Box$, $\Diamond$.
Note that we are consuming infinite words as properties, so our expressions must consume infinite words! So we shouldn't use something like $\Sigma^\star$
since that only consumes finitely many words.

\begin{itemize}
\item Always $A$: $\Box A \equiv \llbracket A \rrbracket^\omega$.
\item Eventually $A$: $\Diamond A \equiv \Sigma^* \llbracket A L^\omega \rrbracket$.
\item Eventually Always $A$: $\Diamond (\Box A) \equiv \Sigma^* \llbracket A^\omega \rrbracket$.
\item Always Eventually $A$: $\Box (\Diamond A) \equiv (\Sigma^* \llbracket A \rrbracket)^\omega$.
\end{itemize}


\section{automata for  $\omega$ regular languages}

A natural question presents itself: can we have a notion for automata that accepts words from an $\omega$ regular language?
Since the only extension we have made is the $\omega$ operator, if we find a clean way to handle the $\omega$ operator, then we are done!
Recall that $A^\omega \equiv \{ w_1 w_2 \dotss : w_i \in L \forall i \geq 1 \}$.
Intuitively, we want to say that we "accept" $w_1$, and then we "accept" $w_2$, and so on.


\section{Public key crypto}

Before Mid-1, we created a CCA-secure scheme.
We assumed that the key $K$ is pre-shared between sender, receiver.

Diffie Hellman key exchange.


Can the key be made public, such that converting from public (encryption) key to the
private (decryption) key is hard?

So, we can publish an encryption key that is public, thereby allowing everyone to
communicate with us.


\subsection{Diffie Hellman SKE(Secret Key Exchange)}
\begin{itemize}
\item There is a group $G$ and a generator $g$ of $G$. Eg: $Z_p^x = <g>$. These are *public*.
\item Alice chooses a random element $a \in G$. Alice sends $g^a$ to Bob.
\item Eve is eavesdropping, all she can see is $g^a$.
\item Bob chooses $b$, sends $g^b$ to Alice.
\item Eve sees $g^b$, cannot find $b$.
\item Key is $g^{ab} = g^a \cdot g^b$
\item key for Eve is $g^{b^a}$ ($g^b$ came from Bob).
\item Key for Bob is $g^{a^b}$ ($g^a$ came from Alice).
\item Now, they both have the key, while an eavesdropper cannot find the key.
\end{itemize}

This is insecure if it is possible to get $g^{ab}$ from $g^a$, $g^b$. Even if discrete log is
hard, there could be some way to use group structure to do this.

This assumption is called 'CDH assumption': given $g^a, g^b$, computing $g^ab$ is hard.


\subsection{Does this satisfy our need? Or, RSA}
This solves key exchange, but not the way we wanted to. We wanted to *publish* the encryption key.

\subsubsection{RSA}
\begin{itemize}
\item $p$ and $q$ are two *large* primes of nearly same length. (today, ~512 bits).
  $n = p \cdot q$. $e \in [1..(p-1)]$, $(e, (p - 1)(q - 1)) = 1$
  $d$ such that $ed = 1 mod (p - 1)(q - 1)$

  Public key: $<N, e>$
  Private key: $<p, q, d>$

\item $Encryption(m) = m^e (mod N)$
\item $Decryption(c) = c^d (mod N)$
\end{itemize}

\subsubsection{Correctness of RSA}
\begin{align}
  dec(enc(m)) &= \\
  c^d(mod N) &= \\
  (m^e)^d (mod N) &= \\
  m^{ed} (mod N)
  = m \text(since ed = (p - 1)(q - 1) = \phi(n))
\end{align}

\paragraph{$phi(N)$}
$phi(N) = pq \text{(all numbers)} - q \text{(multiples of p)} - p \text{(multiples of q)} + 1 \text{(double subtraction of $N$)}$

\paragraph{$a^\phi(n) = 1 (mod n)$}
consider the set $S' = {i_1 a , i_2 a, i_3 a, i_\phi(n) a}$, $S = {i_1, i_2, ..., i_\phi(n)}$.
Show that $S$ and $S'$ are permutations.
QED. (TODO: how to box?)


However, here, we don't publish the key.

\paragraph{RSA assumption}

Given Encrypted message $m^e (mod N)$, and the public key $<N, e>$ we cannot get $m$.


\paragraph{Textbook RSA does not work}
\begin{itemize}
\item RSA is deterministic. Hence, we do not have CPA security.
\item Small key, small N: If Key is small, then, for example, let $m^3 = N$. Now, we can compute cube root $m < \sqrt[3](N)$.
\item Small key:  $c_1 = m^3 mod (N_1)$. $c_2 = m^3 mod (N_2)$. $c_3 = m^3 mod (N_3)$ We are multi-casting
  this message to two people, both of whom have chosen $3$ as their exponent.
  Use chinese remainder theorem. Find $m^3$ in $0 \leq m^3 \leq N_1N_2N_3$.
\end{itemize}

\paragraph{Chinese Remainder Theorem}

Given a family of congruence equations $x = a_i (mod N_i)$, all $N_i, N_j, i \neq j$ are pairwise coprime,
Then we can find $x \in N_1N_2N_3...N_k$.

That is, there is a ring isomorphism:

$Z/N_1Z x Z/N_2Z, Z/N_3Z x ... x Z/N_kZ ~= Z/(N_1N_2N_3 \cdots N_k)Z$

Backwards is obvious, just take modulo $N_1, N_2, N_3, \cdots, N_k$.

Simplest case:

Assume $a_1 = 1$, all other $a_k = 0$. In this case we must set, $x = q.N_2N_3N_4..N_k$.
Let $q = (N_2N_3..N_k)^{-1} (mod N_1)$. Hence, $a_1 = 1 (mod N_1)$. (since $q.N_2N_3N_4..N_k = 1$ (mod $N_1$)).
Also, this number $x$ modulo any *other* $N_k, k \neq 1$ will be $0$ since $x$ is a multiple of that $N_k$.

Similarly, the vector
$[0, 1, 0, 0, ....0]$ = Let  $ Sol= Pi_{i = 1}^k N_i / N_2$. Now, the number we need is $x = Sol*(Sol^{-1}) (mod N_2)$.

So, we can in generate construct our ``basis vectors'' $[1, 0, 0, ...], [0, 1, 0, 0, ..], [0, 0, 1, ...]$.
So, write any number as:
$a_1 [1, 0, 0, ...] + a_2 [0, 1, 0, ...] + a_3 [0, 0, 1, ...]$ (I am somewhat confused, how do we *find* $a_i$?)


\subsection{PKCS v1.5 (Public Key standard)}
We give a probabilistic version of RSA to give it CPA security. To encrypt a message $m$:

\begin{itemize}
\item $Enc(m) = (0000~0000~||~0000 0010~||~r$ (at least 8 bytes, $r \neq all-zeroes$) $||~0000~0000 || m)^e (mod N)$ (|| = concatenation.)
\item We lose *at least* 11 bytes of performance. ($1 + 1 + (\geq 8) + 1$). If we fix length of $r$ to be $8$ (or some other constant),
  then the scheme is insecure! (Homework assignment).
\item for RSA, LSB is the hard core predicate. That is, it is hard to get LSB(x) given $x^e (mod N)$. The first 16 bits are very easy to get (apparently).
  So, we standardise something in the first 16 bits. (WTF? Read the proof of this). The bits after that are right after (where $r$) sits is
  also weaker than LSB. So, we keep the randomness in the weaker bits, and the *actual message* in the stronger bits (remember, LSB is hard!).
\item 
\item Dec(m)
\end{itemize}

Theoretical version of this is called $RSA-OAEP$. (OAEP = optimal asymmetric encryption padding).
This has a proof in the random oracle model that $RSA-OAEP$ is secure. There is no such proof of $PKCS$.

\subsection{El Gamal scheme}
New public key scheme that is based on discrete log, but uses the public key template.
Has a proof of CPA-security in the standard model.


Let $G$ be a group. Let the message be an *element* of the group $m \in G$.
Let $r \in G$, $r$ random. Let the cipher text $c = m \cdot r$. $c \in G$.

We want $r$ to look like $g^{xy}$. We know from diffie-hellman that $g^xy$ cannot be found from
$g^x, g^y$.

Group $G$ is published. Generator $g$ is published ($G = <g>$), $|G|$ is public.
There is a *secret element $x$*. We publish $g^x$.

$PK = <G, g, |G|, h = g^x>$
$SK = x$

$Enc(m) = Choose y \in G. <g^y, h^y * m>$

$Dec(g^y, h^y * m) = (g^x)^y * m = (g^{y^x}) * m$
$m = h^y * m / (g^y)^x.$

So, we can get $m$.

We get $CPA$ security since it is probabilistic (choice of $y$ is probabilistic).


\subsubsection{Homomorphic Encryption with El-Gamal}
Note that El-Gamal is homomorphic WRT group operation.
$<g^y1, h^y1 * m1>, <g^y2, h^y2 * m2>$. Then multiply pointwise. $<g^{y_1 + y_2}, h^{y_1 + y_2}*m_1*m_2>$. Hence, we have encrypted $m_1 * m_2$.





\chapter{\ip}

\section{The class \ip}
TODO: copy notes from the algorithms class, they're identical. Unfortunately,
I missed this because I was sick.

We know that \dip~(deterministic IP) ~= \nptime. We also conjecture that
\ptime~= \bpp. So, clearly, randomization and interactivity alone don't
give much.

It is surprising that $\nptime~\subseteq \ip$ (since GNI is in \ip), since we
are combining two powers that are useless on their own. So, it's shocking
that \ip~= \pspace!


\section{Arithmetization}

We define a new problem, $$\hashsat \equiv \{ (\phi, k)~\vert~ \phi~\text{is a 3-CNF boolean formula with exactly $k$ satisfying clauses} \}$$.

Note that $\overline{\texttt{3-SAT}} = \hashsat(0)$. Hence, \hashsat~ is a
generalization of a problem that is not known to be in NP ($\overline{\texttt{3-SAT}}$).  We will use \textit{arithmetization} to show that $\hashsat \in \ip$.

We have a prover $P$, and a formula $\phi$, with $k$ satisfying clauses.
$P$ knows $(\phi, k)$, and wants to convince the verifier $V$ of this fact.

We have a boolean formula $\phi$ with boolean variables. If we choose to
work in a larger field $F_p$, $\{0, 1\} \in F_p$. A boolean function
can be viewed as a polynomial in a larger field, which agrees with the
boolean formula on $\{0, 1\}$. For instance, if $\phi = x_1 \land x_2$,
we can create a polynomial $q(x_1, x_2) = x_1 x_2$. Clearly, $q$ and $\phi$ agree
on $x_1, x_2 \in \{0, 1\}$.

\begin{itemize}
\item $\phi(x) = \lnot x \leftrightarrow q_\phi(x) = (1 - x)$
\item $\phi(x_1, x_2) = x_1 \land x_2 \leftrightarrow q_\phi(x_1, x_2) = x_1 x_2$
\item $\phi(x_1, x_2) = x_1 \lor x_2 \leftrightarrow q_\phi(x_1, x_2) = 1 - (1 - x_1)(1 - x_2)$
\end{itemize}
Now, we can express every boolean formula as a polynomial.


\subsection{Rewrite \hashsat in terms of arithmetization}
$$
(\phi, k) \in \hashsat \leftrightarrow k = 
    \sum_{x_1 \in \{0, 1\}} \sum_{x_2 \in \{0, 1\}} \dots \sum_{x_n \in \{0, 1\}}
    Q_\phi(x_1, x_2, \dots, x_n)
$$



\subsection{Interactive proof for $k$ in terms of $Q_\phi$}
We provide a recursive solution to verify
\begin{align*}
k = 
    \sum_{x_1 \in \{0, 1\}} \sum_{x_2 \in \{0, 1\}} \dots \sum_{x_n \in \{0, 1\}}
    Q_\phi(x_1, x_2, \dots, x_n)
\end{align*}

Define $h(x_1) \equiv Q_\phi(x_1, b_2, b_3, \dots b_n)$, where $b_i \in \{0, 1\}$ are constant.
Now, $h(x_1)$ is a univariate polynomial of degree $d$.

If we consider all possibilities of $b_2 \dots b_n$, we get $2^{n - 1}$ variants
of the $h(x_1)$ polynomial.  We consider all of them by summing over all possiblities,
and collecting all of them in $H$.

\begin{align*}
H(x_1) = 
     \sum_{b_2 \in \{0, 1\}} \dots \sum_{b_n \in \{0, 1\}} Q_\phi(x_1, b_2, b_3, \dots b_n)
\end{align*}

Note that $H(x_1)$ still has degree $d$, since it is the sum of many $h(x_1)$
polynomials.  However, see that $H(x_1)$ is an exponential sum (it has exponential number of terms), and therefore
the verifier can't simply construct $H(x_1)$ it \textbf{needs} the prover
to proxy for $H(x_1)$ in some sense. This is where we need the unbounded prover.


We can now see that the original statement is the same as saying
\begin{align*}
k = H(0) + H(1)
\end{align*}

\subsection{Using this property to verify}
\subsubsection{$n = 1$}
If $n = 1$, then $V$ checks that $k = Q_{phi}(x_1)$.

\subsubsection{$n > 1$}
\begin{itemize}
\item If $n > 1$, $V$ asks $P$ to give $H(x_1)$. $P$ gives $S(x)$.

\item $V$ checks if $k = S(0) + S(1)$. If not, reject. If success, then we need
to verify that $S =_? H$.

\item $V$ chooses $a \in_{random} F_p$, and computes $S(a)$.

\item Recursively ask for a proof that $S(a) = \sum_{b_2} \sum_{b_3} \dots \sum_{b_n} Q_\phi(a, b_1, b_2, \dots b_n)$.
This is a recursive step since I can write this as:

\begin{align*}
S(a) &= 
     \sum_{x_2 \in \{0, 1\}} \dots \sum_{x_n \in \{0, 1\}}
    Q_\phi(a, x_2, \dots, x_n) \\
    &\text{Compare to} \\
k &= 
    \sum_{x_1 \in \{0, 1\}} \sum_{x_2 \in \{0, 1\}} \dots \sum_{x_n \in \{0, 1\}}
    Q_\phi(x_1, x_2, \dots, x_n)
\end{align*}

and now $Q'_\phi(x_2, \dots x_n) \equiv Q_\phi(a, x_2, \dots, x_n)$ is a polynomial
of degree $n - 1$.


\subsection{Completeness}
The prover $P$ that can actually construct the correct $H$ will cause the verifier
to accept with probability $1$.

\subsection{Soundness}
We are currently checking of $(H - S)(a) = 0$. We had chosen $a \in_{random} F_p$.
Only if $a$ is a root of $(H - S)$, will the prover escape undetected.

There are only $d$ roots for the polynomial $(H - S)$, and $F_p$ has $p - 1$
elements. So, the prover will be undetected with probability $P < \frac{d}{p - 1}$.

Probability he is caught in the first round is $d/p$. 

So, the probability a cheating prover will be detected will be $\big(1 - d/p\big)^{n}$.
\end{itemize}

\section{\ip~= \pspace}

We show that $\tqbf \in \ip$, hence $\pspace \subseteq \ip$, since \tqbf~is
a \pspace-complete problem.

To show that $\ip \subseteq \pspace$, notice that 





\section{How to query a DB without revealing data / Oblivious Transfer}

1. We shoul not reveal the query
2. Database reveals nothing except for the query answer to the query


\section{Simplification of the problem}
Consider an array of $n$ bits, $Arr = b_0 b_1 \cdots b_n$. The query is an index query $i$.
The task is that the querier $B$ should get to know $b_i$, should not get to know
$b_j, j \neq i$. $A$ should not *know $i$*!

Intuitively, it seems like ``information deadlock'' should take place. Since neither
party knows what to do, there is some sort of deadlock. Hence, impossibility. (Kannan
is happy here, since now we can information theory this, as he puts it).

Supposedly, one-way-functions will work.

$x \rightarrow f(x) is easy$.
$f(x) \rightarrow x is hard$. If we know trapdoor information,
$f(x) \rightarrow x$ is easy.

\section{Construction of the scheme}

$f: [1, n] -> [1, n]$ is a trapdoor one-way *permutation* (unique decryption).

\begin{itemize}
\item 1. $B$ chooses $n$ bits at random - $r_1 r_2 \cdots r_n$.
\item 2. $B$ applies $f$ only at $r_i$ to obtain $Z = r_1 r_2 \cdots r_{i -1} f(r_i) r_{i + 1} \cdots r_n$.
  $f$ may not be applicable to single bits. In this case, we can use the hadcore predicate and $XOR$ $r_i$ with the
  hardcore predicate.
\item 3. $B$ sends $Z$ to $A$.
\item 4. $A$ decrypts (find inverse) of $Z$ and obtains $Y = f^{-1}(r_1), f^{-1}(r_2), \cdots r_i, f^{-1}(r_{i + 1}), f^{-1}(r_n)$
\item 5. Let array be $Arr = b_0 b_1 \cdots b_n$.
\item 6. Perform $Arr XOR Y = b_0 XOR f^{-1}r_1, \cdots , b_{i} XOR r_i, \cdots b_n XOR f^{-1}(r_n)$.
\item 7. A sends $Arr XOR Y$ to $B$.
\item 8. $B$ obtain $b_i = (Arr XOR Y)[i] XOR r_i = (b_{i} XOR r_i) XOR r_i = b_i$.
\end{itemize}

Note that $B$ cannot see $b_j$ where $j \neq i$, since in some sense, we have ``encoded'' the $b_j$ with  $f^{-1}$.

$A$ cannot know which index is the correct index, since there is no ``marker'' for the correct index.





\section{Solving the universal problem}

$A$ has input $X$. $B$ has input $Y$. we wish to compute $Comp(X, Y)$. Either both of
them want $Comp(x, y)$, or one of them want $Comp(x, y)$.

$A$ and $B$ are unwilling to reveal their information to each other.

So, how does one solve this? Generalize our specialized construction.

Andrew Yao was awarded the Turing award in 2000 for posing the general problem and solving it.


\section{Yao's millionaire problem}

There are two millionares (Kannan quip: let me make them billionaires, because inflation). They wish to
find out who is richer, without revealing their bank balance to each other. Can we solve this?


\subsection{Weird kannan style generalization}

Given two machines $A$ and $B$, can we construct a virtual machine $S$, such that $A, B$ do not know what
$S$ is computing, but they virtually simulate $S$?

That is, $Mem(s) = Mem(a) XOR Mem(b)$

Can a cluster of insecure machines simulate a secure machine? (Neat!)

Sid question: Can we not construct FHE by keeping some data on the client as well? It could be redundant data,
but it would still be part of the algorithm? I guess this does not really give you FHE, because the full data is not
owned by one party.





\subsubsubsection{Kannan tangent - Teaching ethics instead of teaching crypto}
It is better to teach ethics and forego the area of crypto, rather than teach crypto and allow people
to forget ethics.

I'm not keen on crypto solving dishonesty problems.

\begin{itemize}
\item 1. There will be dishonest people in the world
\item 2. Software will have bugs for dishonest people to exploit.
\end{itemize}

Outside of Earth, it has already caught on.

The first major implementation of our solution was performed by satellites (?) (what in the hell, TIL).
They don't want to collide in mid-space, but they **do not want to reveal where they are**.

As satellite traffic increased, there was a genuine chance that collision would take place.
They ran this protocol between satellites so they can prevent collisions.


\subsubsection{Finally, the solution}

We wish to perform an instruction $z <- x + y$ on $S$, where $x, y$ are also stored in $S$.

\subsubsubsection{Constructing XOR}
$x$ is stored in $S$ means that $x_a$ is in $A$, $x_b$ is in $B$$, $x = x_A XOR x_b$.
(note that in our scheme, $x_a$ and $x_b$ are stored *at the same memory loc at $A$ and $B$.
That is $A_{ram}[i] = x_a, B_{ram}[i] = x_b$.


Party $A$ performs:
$z_a= x_a XOR y_a$

Party $B$ performs:
$z_b = x_b XOR y_b$



\subsubsubsection{Construct AND}
We know what the output should look like:
$z_a XOR z_b = (x_a XOR x_b) /\ (y_a XOR y_b)$


Code for A:
$z_a <- random {0, 1}$. 
A creates an array of length $4$, which stores values of $z_b$ corresponding to values of $x_b, y_b$.
(Look at $z_A XOR z_b$, and see what the value should look like). So, consider all 4 cases, corresponding
to the indeces of $A_i$.

0. If x_b = 0, y_b = 0,
z_b = (x_a /\ y_a) XOR z_a. = Arr_0


1. If x_b = 0, y_b = 1,
z_b = (x_a /\ (NOT y_a)) XOR z_a = Arr_1.


2. If x_b = 1, y_b = 0,
z_b = ((NOT x_a) /\ y_a)) XOR z_a = Arr_2.


3. If x_b = 1, y_b = 1,
z_b = ((NOT x_a) /\ (NOT y_a)) XOR z_A = Arr_3.


Code for B:
Run oblivious transfer with $n = 4$. $A$ has a linear database of size 4. $B$ has the index.
Hence, $B$ gets $z_b$.


\subsubsubsection{Note: why pick $z_a$ randomly?}
$z_a$ acts as one-time-pad in the array table construction. This obscures
what $B$ can see about $x_a$, $y_a$.

\subsubsubsection{Exploiting multiple machines - Byzantine situations}

With many machines, we can XOR the data between many machines. This gives us much higher security.
However, we lose out on fault-tolerance. People have explored this fully, and we can ask for arbitrary fault tolerance and security,
and we can then recieve a protol to be used for that setting.

We can have at most $n/3$ parties that were colluding and disrupting among a cluster of $n$ machines, and still have
fault tolerance and security.



\chapter{Review}
\section{PH}
\subsection{$\nptime = \conptime$ implies that PH collapses}
\subsection{$\text{\ppoly} \subset \nptime$ implies PH collapses to level 2}
intuitiion: ppoly's advice string lets us do something like quantifier
exchange, since it gives us an outer there exists for the advice string,
which can be guessed by and $NP^{NP}$ oracle. Also relies on the trick that
$\exists = \lnot \forall$, and you can not in this case due to being oracle call.
\section{BPP}
\subsection{BPP subset P/poly}
Intuition: can guess random seeds that are good for \emph{all} inputs.
\subsection{$BPP \subset \Sigma_p^2 \cap \Pi_p^2$}
Intutiion: can show we can find some $u_i$ to spread randomness around.
If we are on the accept side, the good seeds are large enough that we
can use randomness to cover the full set. If we are on the bad side, the
spreading with $u_i$ will not let us cover enough ground. So, we can 
convert it to $\exists u_i, \forall x_i, ...$ which is a PH style problem.

\end{document}
