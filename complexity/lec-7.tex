\newcommand{\ptime}{\texttt{P}}
\newcommand{\logspace}{\texttt{L}}
\newcommand{\nlogspace}{\texttt{NL}}
\newcommand{\ph}{\texttt{PH}}
\newcommand{\pathproblem}{\texttt{PATH}}
\chapter{LogSpace}
\section{The languages \logspace~\& \nlogspace}

$\logspace = DPSACE(\log (n))$, $\nlogspace = NSPACE(\log(n))$


\textbf{Open problem: \logspace~versus \nlogspace}.

$\logspace \subset \ptime$ since configuration space is $2^{O(\log(n))} = O(n)$,
hence we can brute force the configuration space.

\subsection{\nlogspace-completeness}
A language $A$ is \nlogspace-complete if:
\begin{itemize}
    \item $A \in \nlogspace$
    \item $B \in \nlogspace, B \leq_L A$
\end{itemize}

Note that $B \leq_L A$ uses a mapping reduction: $\exists f: \Sigma^* \to \Sigma^*$,
$f$ converts inputs of $B$ to inputs of $A$ in log-space.

$f$ must be \textbf{implicitly computable}: That is, the problem of generating
the $i$-th bit should be done in $\logspace$.


\section{The \pathproblem~problem}
$\pathproblem = \{ \langle G, s, t \rangle~\vert~\exists s \to t~\text{path in}~G \}$

\begin{theorem}
    $\pathproblem \in \nlogspace$
\end{theorem}

\begin{proof}
We perform the naive solution --- We try to explore all possible path possibilities for $s \to t$.
\end{proof}

\begin{theorem}
    $A \in \nlogspace \implies A \leq_{\logspace} \pathproblem$
\end{theorem}
\begin{proof}
    Since $A \in \nlogspace$, there exists an NDTM $M$ which decides $A$ by
    using $O(\log(n))$ space.

    We try to walk through the configuration graph of $M$ (henceforth called $Config_M$, from which
    we can try to find the path from the initial configuration to the
    final configuration. We can then try to solve
    $\pathproblem(Config_M, c_{init}, c_{final})$, which will tell us
    whether an accepting path exists.
\end{proof}

