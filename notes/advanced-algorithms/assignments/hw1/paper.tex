\documentclass{article}
\usepackage{amsmath}
\usepackage{amsfonts}
\begin{document}
\newcommand{\threesat}{\texttt{3-SAT}~}
\newcommand{\pspace}{\texttt{PSPACE}}
\newcommand{\np}{\texttt{NP}}
\section{Q1. Deterministic TMs}

\subsection{$f(n) = 2^n$}

the idea is that we encode both the input and the output in unary. We strike
out inputs' digits one-by-one. When we do, we copy the output digits (that is,
we double the current output) that is currently on the tape. We then go back to
the input and repeat this process.


\subsection{$f(v, b) = \log_b(v)$}

One simple way to encode the log is to  implement $g(i) = b^i$. We then 
compute $g(i)$, $i \geq 0$ till $g(i) \geq v$. We then return $i - 1$

To implement $g(i)$, we copy the string $b$ $i$ times, similar to how we copy the
string once in the implementation of $2^n$.

\section{Q2. \np $\subseteq$ \pspace}
since \threesat is NP-complete, we show that \threesat can be computed in
\pspace. Hence, \np $\subseteq$ \pspace.

We can simply enumerate all possible $2^n$ assignments of the given \threesat
problem in $\log(2^n) = n = poly(n)$ space, and verify whether they satisfy
the given problem.

\section{Q3. Nodes not in path between two nodes}

Note that a node $v$ is in the path $s \rightarrow t$, if there exists paths
$s \rightarrow v$, $v \rightarrow t$.

We can first compute all pairs shortest path using floyd-warshall, which is 
$O(V^3)$, after which for each $v$, we can look for the existence of paths
$s \rightarrow v \rightarrow t$, in $O(1)$ time. The total time works out to be 
$O(V^3) + O(1)O(V) = O(V^3)$


\section{Q4. Tshirts}

We first compute probabilites which are used to compute expectation.
\begin{align*}
&P(\text{k people not receiving t shirts}) = \\
&P(\text{picking $(n - k)$ people from $n$ people, $n$ times (with reptition)}) = \\
&{\bigg(\frac{n - k}{n}\bigg)}^n
\end{align*}

Now that we know the probability of $k$ people not receiving t-shirts, we can
compute the expectation of this:

\begin{align*}
&\mathbb{E}[\text{number of participants with no tshirt}] =  \\
&\sum_{i = 0}^n i  P(\text{i people not receiving t-shirts}) = \\
&\sum_{i = 0}^n i {\bigg(\frac{n - i}{n}\bigg)}^n 
\end{align*}

\end{document}
