\newcommand{\gni}{\texttt{GNI }}
\chapter{Probabilistic proofs}

\section{\ip --- interactive proofs}
\begin{definition}
Completeness: For every true assertion, there is a valid proof.
\end{definition}

\begin{definition}
Soundness: For every false assertion, no valid proof exists.
\end{definition}

A good proof system must also be such that the verifier is efficient
(that is, polynomial time).

If we ask that a proof system must be sound and complete, there is no 
scope for error! Further, it is not clear if the verifier and the
prover can "talk" to each other. If we choose to allow interactions, what
are the implications?


We relax the assumptions this way --- Relaxed compleness states that
for every true assertion, there is a
proof strategy that will convince the verifier with probability 
at least $> \frac{2}{3}$.  
Similarly, relaxed soundness states that for every false assertion,
every proof strategy fails to convinve the verifier with probability
at least $> \frac{2}{3}$. 

The formalization is as follows:
\begin{definition}
Interactive proof systems
\begin{itemize}
\item An interactive proof system for a language $L$ consists of two
entities: a prover $P$ and a verifier $V$.
$P$ and $V$ share common input, and work for $R \in \mathbb{N}$ rounds.

\item In each round, the prover can send the verifier a message that 
is polynomial in the length of the input.

\item The verifier can send a polynomial length reply to the prover.

\item The verifier is a randomized polynomial time turing machine. Time
is measured as a function of the length of the input.

\item \textbf{Completeness}: $\forall x \in L$, there exists a prover strategy
so that the verifier accepts with probability $> \frac{2}{3}$.

\item \textbf{Soundness}: $\forall x \notin L$, any prover strategy will lead
the verifier to accept with probability  $< \frac{1}{3}$.
\end{itemize}
\end{definition}

Note that the power of the prover in unspecified in this definition ---
we are implicitly saying that finding a proof is generally much harder
than verifying a proof. Hence, the prover has no real bounds on the power,
while the verifier does.

We also have the value $R \in \mathbb{N}$, which lets us setup the number
of rounds. This is a knob we can twiddle, that allows us to change the hardness
of the problem.



\begin{definition}
The \ip hieararchy: Let $r: \mathbb{N} \to \mathbb{N}$ be the "number of rounds" function.
Define $IP(r)$ to be the set of languages such that there exists an interactive
proof system using at most $r(|x|)$ rounds on input $x$.

For a class of functions $R \subset \{ \mathbb{N} \to \mathbb{N} \}$, we can then define $\ip~(R) = \cup_{r \in R} ~\ip~(r)$.
\end{definition}


Note that $\nptime \subset \ip$.  Also, the number of rounds cannot be more than 
polynomial --- the verifier is poly bounded in time, so the verifier
cannot work more than poly rounds.  So, we denote $\ip \equiv \ip(O(poly(n))$.

Both \textbf{randomness} and \textbf{interaction} are essential to the definition.


When randomness is removed but only interaction is present, this will be
like \nptime. The prover can arrive by itself the set of messages the
verifier would send to the prover.


When interaction is removed but randomness is remained, the verifier is
similar to that of \nptime, but the verifier can now be \textbf{probabilistic}.
This class of languages is likely beyond \nptime.

\section{Graph non-isomorphism (GNI)}
Two graphs $G$, $H$ are isomorphic (denoted $G \sim H$), iff there exists
a bijection such that $\forall x, y \in V_1, (x, y) \in E_1 \implies (f(x), f(y)) \in E_2$.

Using this, we define \gni, the problem of checking if two graphs
are not isomorphic:
\begin{align*}
\gni \equiv \{ \langle G, G' \rangle ~\vert~ G \nsim G' \}
\end{align*}

Graph isomoprhism is in \nptime since the witness will just be the bijection.
Hence, \gni is in \conptime, and it is not known whether \gni is in \nptime.

In an interactive proof system for \gni, the verifier asks the prover to
distinguish between isomorphic graphs.

\begin{itemize}
\item $G_1, G_2$ are given to both prover, verifier.

\item The verifier chooses a random  $r \in \{1, 2\}$ uniformly at random.

\item The verifier picks a random permutation $\pi$ of the set $\{1, 2,\dots, |V(G_1)|\}$

\item the verifier constructs the graph $H$ as the permutation of $G_r$ under $\pi$.
The graph $H$ is sent to the prover. That is, $H \equiv \pi(G_r)$.

\item the prover P replies with $r' \in \{1 2\}$. The reply $r'$ is 1
if $H$ is isomorphic to $G_1$, and $2$ otherwise.

\item The verifier accepts if $r = r'$.
\end{itemize}

Note that $H \sim G_r$. Now if $G_r \sim G_{other}$, then $H \sim G_r \sim G_{other}$, and
so the prover has to literally guess between $G_r$ and $G_{other}$, and at best
it can simply guess. (Even though the prover has \textit{unbounded computation},
it is unable to distinguish between $G_r$ and $G_{other}$). In two rounds,
the probability of the guesses of the prover being right is $\frac{1}{2}^2 = \frac{1}{4}$,
which fulfils our soundness guarantee ($\frac{1}{4} < \frac{1}{3}$).

On the other hand if $G_r \nsim G_{other}$, then if the prover knows how to solve
\gni , it can check between $H$, $G_r$, and $G_{other}$ to consistently
report $G_r$. In this case, the prover will \textit{always} be correct,
so this will pass compleness (since $1 > \frac{2}{3}$).

This is very interesting, because the verifier \textbf{does not know} whether
$G_1 \sim_? G_2$. The verifier tries to engage with the prover, to understand
whether $G_1 \sim_? G_2$.

\begin{theorem}
$\gni \subset \ip (2)$
\end{theorem}

\begin{theorem}
$\conptime \subset \ip$
\end{theorem}

\begin{theorem}
$\ip = \pspace$
\end{theorem}
