% https://www.youtube.com/watch?v=aHW6HKVC-SY&list=PL2SOU6wwxB0uP4rJgf5ayhHWgw7akUWSf&index=8
\chapter{Online algorithms}

Making decisions in the face of uncertainty. We don't want to regret things
too much when we look back from the future. We want to bound versus the
best in hindsight.


We use an algorithm \ALG to make decisions. \OPT is the optimal decision
given the future. We say an algorithm is $r$ competitive is for all sequences
$\sigma = [\sigma[1], \sigma[2], \dots]$:
$$cost(\ALG(\sigma)) \leq r \cdot cost(\OPT(\sigma)) + O(1)$$

Analysing these kinds of algorithms is called competitive analysis by Slater
and Tarjan (CACM '85).


\textbf{Toy example 1:  the Ski rental problem}: What is unknown is $M$, the number
of days we'll be spending at the vacation. I don't know how long we'll be spending at a vacation. Each
day, when I wake up, I am told whether I am staying another day.
We can either rent the ski for a day which costs $\$1$, or buy the skis which costs $\$B$.
The optimal solution when we know $M$ is to buy the ski of $\$1 \cdot M > \$ B$, and
rent the ski otherwise. \OPT pays at most $B$.
Our solution (\ALG), knowing $M$ is to buy the skis on day $(B+1)$. \ALG pays
at most $2B$.

\textbf{Toy example 2: The Free Pizza problem}:
There's a long hallway with $n$ evenly spaced rooms, numbered $[-n/2, \dots, 0, n]$
We're at room number $0$. The pizza is at room $t$.
\OPT knows where the pizza is so it goes straight there, so he pays $t$ cost.
Visit rooms $[1, -2, 4, -8, 16, \cdots]$. We pay:
$$
(1 + 1) + (2 + 2) + \cdots + (2^m + 2^m) + t
$$

which is bounded by something like $9t$.

\section{List update problem}

We imagine we have a linked list. We have repeated requests to access items
in the linked list. It takes $i$ time to get to the $i$th element since
we need to walk the list. Once we access an element in the linked list, we can
move the item to anywhere we walked past.

\section{Paging}
we have cache versus RAM. Cache has bounded size.


