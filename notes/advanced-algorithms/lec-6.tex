\chapter{Gaps in space and time}

We wish to study what is not computable given some resource.
If there resource is time, we want to understand what can be solved
in $t(n)$ but not in smaller than $t(n)$ --- in the sense of $o(t(n))$.

We can try to construct a hierarchy of problems that can be solved
given increasing time. 

\begin{align*}
f(n) \in o(g(n)) &\equiv \lim_{n \to \infty} \frac{f(n)}{g(n)} = 0 \\
f(n) \in O(g(n)) &\equiv \lim_{n \to \infty} \frac{f(n)}{g(n)} \in  O(1)
\end{align*}

\section{Space Hierarchy}

A function $f: \mathbb{N} \to \mathbb{N}$ is said to be \textbf{space constructible}
if there exists a turing machine that on input $1^n$, it computes $f(n)$
using space $O(f(n))$. So the output can be $1^{f(n)}$ say, since that uses
space $O(f(n))$.

Most common functions such as polynomials, exponentials, and logarithms
are all space constructible.

\begin{theorem}
Let $f$ be a space-constructible function. There exists a language $L$ which
can be decided in $O(f(n))$ space, but not in $o(f(n))$ space.
\end{theorem}
\begin{proof}
The proof is to \textbf{construct} a language which can be decided on $O(f(n))$
space, but not in $o(f(n))$ space. Such a language tends to be artificial due
to the construction having to work \textit{for all $f$}.

We need two properties for this language $L$ we create:

\begin{itemize}
\item It is \textbf{not decidable} in $o(f(n))$ space.
\item It \textbf{is} decidable in $O(f(n))$ space.
\end{itemize}

We will use diagonalization to show an construct an $L$ that 
\textbf{cannot be decided} in $o(f(n))$ space. List each TM that runs in 
$o(f(n))$ space. This collection of all TMs (viewed as strings) is written as:

$$ALLTM = \cup_{i=0}^\infty \{0, 1\}^i$$


We will define a language $L$ which cannot be decided by \textbf{any} TM
on the above list.

We will create a matrix of the form $DECIDE(i, j) = M_i(\langle M_j \rangle)$.
That is, we feed $M_i$ the string of $M_j$.($\langle M_j \rangle$ interprets
the machine $M_j$ as a string).

Now, create a language $L$:

\begin{align*}
L \equiv \{ M~\vert~M ( \langle M \rangle ) = 0 \}
\end{align*}

Note that $L$ is \textbf{not decidable} in $o(f(n))$ space. Proof by contradtiction:
Assume such a machine $M_c$ ($c$ for contradiction) exists. We now ask if $\langle M_c \rangle \in L$?

\begin{itemize}
\item If $\langle M_c \rangle \in L$, then $M_c (\langle M_c \rangle) = 0$ (by the definiton of $L$).
But since $M_c$ \textbf{decides} $L$,
$M_c (\langle M_c \rangle) = 0 \implies  \langle M_c \rangle \notin L$. \textbf{Contradiction}.

\item On the other hand, say that $\langle M_c \rangle \notin L$, then $M_c (\langle M_c \rangle) = 1$
(by the definition of $L$).
But since $M_c$ \textbf{decides} $L$, $M_c (\langle M_c \rangle) = 1 \implies \langle M_c \rangle \in L$. 
\textbf{Contradiction}.
\end{itemize}


We now move to show that $L$ \textbf{can be decided} in $O(f(n))$ space.
Consider a machine \texttt{INTERPRET} that does this:

\begin{minted}{python}
def INTERPRET(w):
    Mw = convert_to_TM(w)

    # Naive solution: Try to run Mw, see what happens.
    # flag = Mw.run(w)

    # Problem 1: How do we know it runs in o(f(n)) space?
    # flag = Mw.run_with_bounded_space(w, space_bound=f(n))

    # Problem 2: How do we know that Mw halts?
    # Count the size of the config. space, and reject if Mw
    # takes more steps than the configuration space size.
    flag = Mw.run_wth_bounded_space_and_steps(w, space_bound=f(n), 
                                                 steps_bound=Mw.config_space_size())


    return !flag
\end{minted}
\textbf{For more details, read Sipser chapter 9}
\end{proof}



\begin{corollary}
For two functions $f1, f2 : \mathbb{N} \to \mathbb{N}$, if
$f1 \in o(f2)$, then $\dspace~(f1) \subsetneq \dspace~(f2)$. 
(\textbf{Sid note: we do not need the condition that $f1 \neq f2$ thanks to the fact
that in $o(n)$, the limit tends to $0$})
\end{corollary}

\section{Time Hierarchy}

\begin{theorem}
Let $f$ be a time-constructible function. There exists a language $L$ which
can be decided in $O(f(n))$ time, but not in $o \bigg(\frac{f(n)}{\log(f(n)} \bigg)$ time.
\end{theorem}
\begin{proof}
Proof is the same as that of space hierarchy (roughly).

We get the $\log$ factor for us to simulate a $f(n)$ time turing machine.
We do not know how to perform the simulation with constant overhead.
\end{proof}

\begin{corollary} $\texttt{P} \subsetneq \texttt{EXPTIME}$ \end{corollary}


\section{Polynomial Hierarchy}

One interesting thing to study is the power of oracles (non-uniform computations).
One can try to study the nature of languages, given oracle access.

\begin{definition}
Let $M$ be a turing machine, $A$ be a language. \textbf{The language $L(M^A)$} is the
set of strings accepted by the machine $M$ with oracle access to $A$.
\end{definition}

We can deneralize this by giving access to a \textit{class of languages}!

\begin{definition}
Let $M$ be a turing machine, $C$ be a class of languages. 
\textbf{The language $L(M^C)$} is the
set of strings accepted by the machine $M$ with oracle access to any language
in $C$.

$$
M^C = \{ L(M^A)~\vert~A \in C \}
$$
\end{definition}

\begin{definition}
Let $C_1, C_2$ be classes of languages.
\textbf{The language $L(C_1^{C_2})$} is the
set of strings accepted by some machine in $C_1$ given oracle access to 
some machine in $C_2$.

$$
C_1^{C_2} = \{ L(M_1^{M_2})~\vert~M_1 \in C_1, M_2 \in C_2 \}
$$
\end{definition}

We will use $M^\phi$ to denote oracle access to an "empty" oracle. Hence, $M^\phi \sim M$.

An example would be $\conptime \subset \ptime^{\nptime}$, because the $\ptime$
oracle can flip the answer of the $\nptime$ oracle.


