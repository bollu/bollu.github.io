\chapter{B\"uchi Automata 1}

\section{$\omega$ regular languages}
$\omega$ regular expressions are regexes plus the $\omega$ operator.

$$L^\omega \equiv \{ w_1 w_2 \dots : \w_i \in L \forall i \geq 1 \}$$

an $\omega$ regular expression is of the form $\gamma \equiv \alpha_1 \beta_1^\omega + \alpha_2 \beta_2 ^\omega + \dots + \alpha_i \beta_i ^ \omega$.

The semantics is of this is going to be:

$$
L(\omega) \equiv \bigcup_{1 \leq i \leq n} L(\alpha_i) L(\beta_i)^\omega
$$


\begin{itemize}
\item elements of $L \equiv (A^\star B)^\omega$ is the set of infinite words containing infinitely many $B$s.
\item elements of $L \equiv (A^\star B)^\omega + (B^\star A)^\omega$ is the set of infinite words containing infinitely many $B$ or infinitely many $A$s.
\item infinite words where each $A$ is followed by a letter $B$. To build this, we case split on whether $A$ occurs finitely or infinitely many times,
 If $A$ occurs finitely many times, then $(B^\star A B)^\star B^\omega$ works.
 If $A$ occurs infinitely many times, then $(B^\star A B) ^\omega$ works.
\end{itemize}

% use amsfonts for box, diamond.
See that we can encode all of the usual LTL properties. Let $L$ be a language with alphabet $\Sigma$, and let us extend the regex operators with $\Box$, $\Diamond$.
Note that we are consuming infinite words as properties, so our expressions must consume infinite words! So we shouldn't use something like $\Sigma^\star$
since that only consumes finitely many words.

\begin{itemize}
\item Always $A$: $\Box A \equiv \llbracket A \rrbracket^\omega$.
\item Eventually $A$: $\Diamond A \equiv \Sigma^* \llbracket A L^\omega \rrbracket$.
\item Eventually Always $A$: $\Diamond (\Box A) \equiv \Sigma^* \llbracket A^\omega \rrbracket$.
\item Always Eventually $A$: $\Box (\Diamond A) \equiv (\Sigma^* \llbracket A \rrbracket)^\omega$.
\end{itemize}


\section{automata for  $\omega$ regular languages}

A natural question presents itself: can we have a notion for automata that accepts words from an $\omega$ regular language?
Since the only extension we have made is the $\omega$ operator, if we find a clean way to handle the $\omega$ operator, then we are done!
Recall that $A^\omega \equiv \{ w_1 w_2 \dotss : w_i \in L \forall i \geq 1 \}$.
Intuitively, we want to say that we "accept" $w_1$, and then we "accept" $w_2$, and so on.

