\documentclass[11pt]{article}
\usepackage[sc,osf]{mathpazo}   % With old-style figures and real smallcaps.
\linespread{1.025}              % Palatino leads a little more leading
% Euler for math and numbers
\usepackage[euler-digits,small]{eulervm}
\usepackage{amsmath}
\usepackage{amssymb}
\usepackage{physics}
\usepackage{tikz}
\usepackage{fancyhdr}
\author{Siddharth Bhat(20161105)}
\title{Optimization assignment --- Simplex theory 2}
\date{\today}

\pagestyle{fancy}
\fancyhf{}
\lhead{Siddharth Bhat (20161105)}
\rfoot{Page \thepage}

\begin{document}
\maketitle
\thispagestyle{fancy}
\paragraph{Q1. If $\vec x$ is an optimal solution, then no more than $m$ of its components
can be positive, where $m$ is the number of equality constraints}

False, if $m$ counts the \textit{number of equality constraints
in the linear program} . Consider a linear program:
\begin{align*}
\text{maximise $x + y$} \qquad \text{subject to:}~x \leq 1;~y \leq 1;~x, y \geq 0
\end{align*}
The optimal solution is the point $(x = 1, y = 1)$, which has $\mathbf{2}$ 
positive components, but the program has $\mathbf{0}$ equality constraints.

\paragraph{Q2. A variable that has just entered the basis cannot leave in the
very next iteration}
False. Consider the linear program:
\begin{align*}
    \text{maxmimize $Z = x + y$} \qquad \text{subject to:}~x + 3y + s = 3;~x,y,s, \geq 0
\end{align*}
\begin{itemize}
    \item The initial basis $B_0 = \{s\}$. 
    \item We will next pick $y$ since it gives us the greatest
        per-variable increase. $y$ enters, $s$ leaves.
    \item Our new state is:
        \begin{align*}
            \text{maxmimize $Z = 2 + \frac{x}{3} - \frac{2s}{3}$} \qquad \text{subject to:}~y = 1 - \frac{x}{3} - \frac{s}{3}
        \end{align*}
\item We can improve $Z$ by increasing $x$. $y$ leaves, $x$ enters
\end{itemize}
Clearly, $y$ entered in an iteration and left in the next iteration.
\end{document}

