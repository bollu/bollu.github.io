\documentclass[11pt]{book}
%\documentclass[10pt]{llncs}
%\usepackage{llncsdoc}
\usepackage{amsmath,amssymb}
\usepackage{graphicx}
\usepackage{makeidx}
\usepackage{algpseudocode}
\usepackage{algorithm}
\usepackage{listing}
\evensidemargin=0.20in
\oddsidemargin=0.20in
\topmargin=0.2in
%\headheight=0.0in
%\headsep=0.0in
%\setlength{\parskip}{0mm}
%\setlength{\parindent}{4mm}
\setlength{\textwidth}{6.4in}
\setlength{\textheight}{8.5in}
%\leftmargin -2in
%\setlength{\rightmargin}{-2in}
%\usepackage{epsf}
%\usepackage{url}

\usepackage{booktabs}   %% For formal tables:
                        %% http://ctan.org/pkg/booktabs
\usepackage{subcaption} %% For complex figures with subfigures/subcaptions
                        %% http://ctan.org/pkg/subcaption
\usepackage{enumitem}
%\usepackage{minted}
%\newminted{fortran}{fontsize=\footnotesize}

\usepackage{xargs}
\usepackage[colorinlistoftodos,prependcaption,textsize=tiny]{todonotes}

\usepackage{hyperref}
\hypersetup{
    colorlinks,
    citecolor=blue,
    filecolor=blue,
    linkcolor=blue,
    urlcolor=blue
}

\usepackage{epsfig}
\usepackage{tabularx}
\usepackage{latexsym}
\newcommand\ddfrac[2]{\frac{\displaystyle #1}{\displaystyle #2}}

\def\qed{$\Box$}
\newtheorem{corollary}{Corollary}
\newtheorem{theorem}{Theorem}
\newtheorem{definition}{Definition}
\newtheorem{lemma}{Lemma}
\newtheorem{observation}{Observation}
\newtheorem{proof}{Proof}

%\newcommand{\P}{\texttt{P}}
%\newcommand{\NP}{\texttt{NP}}
%\newcommand{\PSPACE}{\texttt{PSPACE}}
%\newcommand{\NPSPACE}{\texttt{NPSPACE}}
%\newcommand{\TQBF}{\texttt{TQBF}}

\newcommand{\cobpp}{\texttt{co-BPP}}
\newcommand{\ip}{\texttt{IP}}
\newcommand{\dip}{\texttt{DIP}}
\newcommand{\zkp}{\texttt{ZKP}}

\newcommand{\textbb}[1]{$\mathbb{#1}$}
\newcommand{\nats}{\mathbb{N}}
\newcommand{\reals}{\mathbb{R}}


\newcommand{\hashsat}{\texttt{\#SAT}}
\newcommand{\tqbf}{\texttt{TQBF}}


\newcommand{\pptm}{\texttt{PPTM}}
\newcommand{\dtm}{\texttt{dtm}}


\title{Computational Complexity Theory}
\author{Siddharth Bhat}
\date{}

\begin{document}

\maketitle
\tableofcontents

\chapter{Lecture 1: Introduction}

Taught in collaboration with MSR Redmond for the Q\# bits.

Topics:
\begin{itemize}
    \item Intro: Transition from Classical to Quantum: Stern Gerlash, 
        Sequential Stern Gerlash, Rise of randomness.
    \item Foundations of Quantum Theory: States, Ensembles, Qubits, Pure and
        Mixed states, Multi qubit states, Tensor products, Unitary transforms,
        Spectral decomposition, SVD, Generalized measurements, Projective
        measurements, POVM, Evolution of quantum state, Krauss Representation.
    \item Quantum Entropy: Subadditivity of Entropy, Avani-Licb(?) Inequality,
        Quantum channel, Quantum channel capacity, Data compression,
        Benjamin Schumahur(?) theorem.
    \item Quantum Entanglement: EPR paradox, Schmidt decomposition, 
        Purification of entanglement, Entanglement separability problem,
        Pure and mixed entangled states, Measures of Entanglement.
    \item Quantum information processing protocols:
        Teleportation, Superdense coding, Entanglement swapping.
    \item Impossible operations in quantum information theory:
        No cloning, No deleting, No partial erasure.
    \item Quantum Computation: Introduction to Quantum Computating,
        Pauli gates, Hadamard gates, Universal gates, Quantum algorithms
        (Shor, Grover search, machine learning and optimisation).
    \item Quantum programming: Programming quantum algorithms, Q\# progtramming
        language, quantum subroutines.
\end{itemize}
Books:
\begin{itemize}
    \item Quantum computation and Quantum information --- Nielsen and Chuang.
    \item Preskill lecture notes.
\end{itemize}

Grading:
\begin{itemize}
    \item Possibility of open book take-home open ended exam for the finals.
    \item Mid 1: 15\%
    \item Mid 2: 15\%
    \item End sem (open book?) : 30\%
    \item Assignments: 15\%
    \item Projects: 25\%
\end{itemize}

\section{Stern-Gerlach: A brief, morally correct construction of qubits}
\[
\footnotesize
\verb|light rays ---> [z] ---> (z+, z-) --block (z-) --> [x] --- (x+, x-) -- block (x-) --> [z] ---> (z+, z-?!)|
\]

$[z]$ represents a polarizer along that axis. 

\begin{itemize}
    \item Since we first polarized along $z$, how did we manage to get out 
        light rays in the $x$ direction? The polarization should have killed
        everything.

    \item Since we blocked $z-$, How did we get back $z-$ after passing stuff through
        $[x]$? Something has changed drastically from our classical picture.
\end{itemize}

We can consider $\qb{z+}$ to be something like:
\[
    \qb{z+} \equiv_? \frac{1}{2}\qb{x+} + \frac{1}{2}\qb{x-}
\]
Where \qb{x+} and \qb{x-} are basis vectors for some vector space
over \R.

If we were to pass the $z+$ light rays through $[y]$, then we would get
$\qb{y+}, \qb{y-}$. So, \qb{z+} is also:
\[
    \qb{z+} \equiv_? \frac{1}{2}\qb{y+} + \frac{1}{2}\qb{y-}
\]

\subsection{Analogy with polarization of light}
Consider a monochromatic light wave in the $z$ direction. A linearly
polarized light with polarization in the $x$ direction which we call
$x$ polarized light is given by:
\[
    E_x = E_0 \hat x \cos (k z - \omega t)
\]
$\omega \equiv \text{frequency} \equiv ck$, $c \equiv \text{speed of
light}$, $k \equiv \text{wave number}$.

Similarly, $y$ polarized light is given by:
\[
    E_y = E_0 \hat y \cos (k z - \omega t)
\]

Consider the case where we have $x$ filters along direction \texttt{-}, $x'$
filter along direction \texttt{/}, $y$ filters along direction \texttt{|}.
In this case, we can have $x, x', y$ filters arranged sequentially 
give us non-zero output (contrast with just having $x, y$).

We can express the $x'$ polarization as:

\[
    E_0 \hat{x'} cos (k z - \omega t) 
    = \frac{E_0}{\sqrt 2} \hat x \cos (k z - \omega t) + \frac{E_0}{\sqrt 2} \hat y \cos (k z - \omega t)
\]

By analogy, we write:
\[
    \qb{z_+} \equiv  \frac{1}{\sqrt 2} \qb{x_+} + \frac{1}{\sqrt 2} \qb{x_-}
\]

However, we now have probability $\frac{1}{\sqrt 2}$, but we want $\frac{1}{2}$.
So, we define the probability as:
\[
    \bra{x+}\ket{x_-}^2 = \frac{1}{2}
\]
\begin{align*}
    &z_+ \equiv \text{$x$ polarization} \\
    &z_- \equiv \text{$y$ polarization} \\
    &x_+ \equiv \text{$x'$ polarization} \\
    &x_- \equiv \text{$y'$ polarization} \\
\end{align*}

This problem  can be solved again by polarization of light. This time,
we consider circularly polarized light which can be obtained by letting
linearl polarized light passing through a quarter wave plate (?)

When we pass such circularly polarized light through an $x$ or $y$ filter,
we again obtain either an $x$ polarized beam, or a $y$ polarized beam
of equal intensity. Yet, everybody knows that circularly polarized light
is totally different from $45^\circ$ linearly polarized light.

A right circularly polarized light is a linear combination of $x$ polarized
light and $y$ polarized light, where the oscillation of the electric field
for the $y$ component is $90^\circ$ out of phase with the $x$ polarized component.

\begin{align*}
    &E = \frac{E_0}{\sqrt 2} \hat x \cos (k z - \omega t) + 
    \frac{E_0}{\sqrt 2} \hat y \cos (k z - \omega t + \frac{n}{2}) \\
    %
    &\frac{E}{E_0} = \frac{1}{\sqrt 2} \hat x e^{i(kz - \omega t)} + 
        \frac{i}{\sqrt 2}\hat y e^{i (k z - \omega t)}
    \end{align*}

Similarly, left circularly polarized light is:

\[
    E = \frac{E_0}{\sqrt 2} \hat x \cos (k z - \omega t) -
    \frac{E_0}{\sqrt 2} \hat y \cos (k z - \omega t + \frac{n}{2})
\]

\section{Observable}
An observable is something that we observe.

$$
Z \ket{z+} = \frac{hbar}{\sqrt 2} \ket{z+} \qquad
Z \ket{z-} = \frac{hbar}{\sqrt 2} \ket{z-} 
$$


TODO: try to construct an operator that takes a vector $\ket{v}$ to a
vector that is orthogonal to it.

\section{Operators}

\subsection{Projectors --- $P$}

Suppose $W$ is a $k$-dimensional vector subspace of the $d$-dimensional 
vector space $V$. 

Using Gram-Schmidt, it is possible to construct an orthonormal basis
$\ket{1}, \ket{2}, \dots \ket{d}$ for $V$ such that $\ket{1} \dots \ket{k}$
is an orthonormal basis for $W$. Then the projector $P$ is defined as:
\begin{align*}
    P_W \equiv \sum_{i=1}^k \ketbra{i}
\end{align*}

\begin{itemize}
\item $P^\dagger = P$ (Immediate from writing in $\ket{i}$ basis)
\item $P^2 = P$ (Immediate from writing in $\ket{i}$ basis)
\end{itemize}

$Q = I - P$ is the projector onto orthogonal complement of the subspace that $P$.
projects into. This projects onto the $\ket{k+1} \dots \ket{d}$ basis.

\subsection{Normal operator}
\begin{align*} A A^\dagger = A^\dagger A \end{align*}


\begin{theorem}
Spectral theorem for normal operators:
Any normal operator $M$ on a vector space $V$ is diagonal with respect to some
orthonormal basis for $V$.
\end{theorem}
\begin{proof}
Let $\lambda$ be an eigenvalue of $M$. $P_\lambda$ is the projector onto
$\lambda$'s eigenvector. $Q_\lambda = P_\lambda^\bot$ is the orthogonal complement projector
of $P$.

We first establish a fact about $P M Q$:
\begin{align*}
&M M^\dagger \ket \lambda = M^\dagger (M \ket \lambda) = \lambda M^\dagger \lambda \\
&\text{Hence, $M^\dagger v \in P$.} \\
&Q (M^\dagger P) = 0 \implies (P M Q)^\dagger = 0 \implies P M Q = 0
\end{align*}

Next, we prove some properties of $QM$ and $QM^\dagger$
\begin{align*}
QM = QM(P + Q) = QMP + QMQ = QMQ \\
QM^\dagger = QM^\dagger(P + Q) = QM^\dagger P + QM^\dagger Q = (PMQ)^\dagger + QM^\dagger Q
\end{align*}

\begin{align*}
&\text{QMQ is normal:} \\
&(QMQ)^\dagger(QMQ) = Q^\dagger M^\dagger Q^\dagger Q M Q = Q M^\dagger Q M Q = Q M^\dagger M Q \\
&(QMQ)(QMQ)^\dagger = (Q M Q) (Q^\dagger M^\dagger Q^\dagger) = Q M Q M^\dagger Q = 
Q M M^\dagger Q = Q M^\dagger M Q = (QMQ)^\dagger QMQ
\end{align*}

\begin{align*}
&M = (P + Q) M (P + Q) \\
&M = P M P + P M Q + Q M P + Q M Q \\
&M = P M P + Q M Q \\
&M = \lambda_i \ketbra{i} + Q M Q \\
&\text{Since $Q M Q$ is normal, and we are performing induction on dimension, and $P \bot Q$,} \\
&M = \lambda_i \ketbra{i} + \sum_k \lambda_k \ketbra{k} \\
&\text{Hence M is normal}
\end{align*}
\end{proof}

\begin{theorem}
Any diagonalizable operator is normal
\end{theorem}
\begin{proof}
Let $M$ be diagonal with respect to basis $\ket{i}$.
Then, $M \equiv \sum_i \lambda_i \ketbra{i}$.
Now, $M^\dagger= \sum_i \lambda_i^* \ketbra{i}$. 
\begin{align*}
&M M^\dagger = \bigg(\sum_i \lambda_i \ketbra{i}\bigg)
    \bigg(\sum_j \lambda_j^* \ketbra{j}\bigg) \\
&M M^\dagger = \sum_i \lambda_i^* \lambda_i \ketbra{i} \\
&\text{Similarly,}  \quad M^\dagger M = (\sum_i \lambda_i^* \lambda_i \ketbra{i}) 
\end{align*}
\end{proof}

\subsection{Unitary operator}
\[ U U^\dagger = U^\dagger U = I \]
\begin{itemize}
\item unitary operator is normal.
\item unitary operator preserves inner products.
\begin{align*}
\bra{b'} \ket{a'} = \bra{b} U^\dagger U \ket{a} = \bra{b} I \ket{a}
\end{align*}
\end{itemize}

\subsection{Positive operator}
Special class of Hermitian operator.

\begin{align*}
 \forall v \in V, \bra v A \ket v \geq 0
\end{align*}

If the inner product is strictly greater than zero, then such an operator
is called as \emph{positive definite}. If it is greater than or equal
to zero, it is called \emph{positive semidefinite}.

\begin{theorem}
A positive operator is Hermitian
\end{theorem}
\begin{proof}
\textbf{TODO}. Proof most likely follows real case, where we use
cholesky to write it as $A^T A$ and then show that it is normal. We then
use the fact that its eigenvalues are greater than or equal to zero
to establish that it is Hermitian.
\end{proof}


\chapter{Maxwell's equations in Minkowski space}
% http://www.physics.ucc.ie/apeer/PY4112/Tensors.pdf

Let us first review Maxwell's equations:

\begin{align*}
&\div E = \frac{\rho}{\epsilon_0}~\text{(Electric charges produce fields)}\\
&\div B = 0~\text{(Only magnetic dipoles exist)}\\
&\curl E = - \pdv{B}{t}~\text{(Lenz Law / Faraday's law - time varying magnetic field induces current that opposes it)} \\
&\curl B =  \mu_0 \bigg(J + \epsilon_0 \pdv{E}{t} \bigg)~\text{(Ampere's law + fudge factor)}
\end{align*}

\section{Constructing $F$, or Tensorifying Maxwell's equations}

Begin with the equation that $\div B = 0$. This tells that $B$ can be written
as the curl of some other field:

\begin{equation}
    \boxed{B \equiv \curl A}
\end{equation}

Expanding this equation of $B$ in tensorial form:
\begin{equation}
    \boxed{ B^i = \levicevita^{ijk}  \partial_j A^k }
\end{equation}

Next, take $\curl E = - \pdv{B}{t}$.


\begin{align*}
&\curl E = - \pdv{B}{t} = \pdv{(\curl A)}{t} = \curl{\pdv{A}{t}} \\
&\curl (E + \pdv{A}{t}) = 0 \\
&\text{writing this as the gradient of some field $\phi$ scaled by $\alpha : \reals$} \\
&E + \pdv{A}{t} = \alpha \big(\grad \phi\big) \\
&E = \alpha \grad \phi - \pdv{A}{t}
\end{align*}

Since electrostatics is time-independent, we choose to think of $\alpha = -1$, 
so we can interpret $\phi$ as the potential.

\begin{equation}
     E^i = - \pdv{\phi}{x^k}  g^{ik} - \pdv{A}{t}^i
\end{equation}

A slight reformulation (since we know that in Minkowski space, $\partial_t = \partial_0$)
we get the equation:


\begin{equation}
    \boxed{ E^i = - g^{ik} \partial_k \phi - \partial_0 A^i}
\end{equation}

We get the metric $g^ik$ involved to raise the covariant $\pdv{\phi}{x^k}$
into the contravariant $E^i$.

(\textbf{Sid question:} how does one justify switching $\curl$ and $\partial$? It feels like some algebra)

\textbf{Here be magic!} We define A new rank-$2$ tensor in Minkowski space-time,
called $F$ (for Faraday),

\begin{equation}
    \boxed{F_{\mu \nu} \equiv \partial_\mu A_\nu - \partial_\nu A_\mu}
\end{equation}

(\textbf{Sid question:} why is this object $F_{\mu \nu}$ covariant? What does this \textit{mean}?)

\begin{lemma}
$F_{\mu \nu}$ is antisymmetric.
\end{lemma}

\begin{lemma}
$F_{\mu \nu}$ has 6 degrees of freedom
\end{lemma}
\begin{proof}
Number of degrees of freedom of $F$: 
\begin{align*}
\frac{4^2~\text{(total)} - 4~\text{(diagonal)}}{2~\text{(anti-symmetry)}} = 6
\end{align*}
\end{proof}

Notice that $F$ is a 1-form!

\section{Expressing $B$, $E$ in terms of $F$}
We now wish to re-expresss $B^{ij}$ and $E^{ij}$ in terms of $F$, so that
this $F$ captures all of maxwell's equations.

\begin{align*}
    B^i &= \levicevita^{ijk}  \partial_j A^k = \levicevita^{ikj} \partial_k A^j \tag*{by $k$, $j$ being free variables} \\
    B^i &= \frac{1}{2} \bigg( \levicevita^{ijk} \partial_j A^k + \levicevita^{ikj} \partial_k A^j \bigg) \\
        &\text{Substituting $\partial_j A_k - \partial_k A_j = F_{jk}$, } \\
    B^i &= \frac{1}{2} \levicevita^{ijk} F_{jk}
\end{align*}


So, $B$ in terms of $F$ is:
\begin{equation}
    \boxed{B^i = \frac{1}{2} \levicevita^{ijk} F_{jk}}
\end{equation}

Similarly, we wish to write $E$ in terms of $F$. The algebra is as follows:
\begin{align*}
    E^i &= -g^{ik} \partial_k \phi - \partial_0 A^i \\
    E^i &= -g^{ik} \partial_k \phi - \partial_0 g^{ik} A_k  \tag*{Is this allowed? Am I always allowed to insert the $g_{ik}$?} \\
    E^i &= -g^{ik} (\partial_k \phi + \partial_0 A_k) \\
\end{align*}

Since $k = \{1, 2, 3\}$ ($k$ is spacelike coordinates), and we would like to
relate $\phi$ with $A$ (to unify $E$), we \textbf{set}:

\begin{equation}
    \boxed{A_0 \equiv - \phi}
\end{equation}

Continuing the derivation,



\begin{align*}
    E^i &= -g^{ik} (\partial_k (- A_0) + \partial_0 A_k) \\
    E^i &= -g^{ik} (\partial_0 A_k - \partial_k A_0 ) \\
    E^i &= -g^{ik} F_{0k}
\end{align*}


So, finally, the relation is:

\begin{equation}
    \boxed{E^i = -g^{ik} F_{0k}}
\end{equation}


Let us reconsider what we believed $E$ to be. We had:
\begin{align*}
    E &= - \grad \phi - \pdv{A}{t}
\end{align*}
However, comparing dimensions, space derivative of $\phi$ = time
derivative of $A$. This means that 
$\frac{\delta \phi}{\delta x} = \frac{\delta A}{\delta y}$, and so
$\frac{\delta \phi}{\frac{\delta x}{\delta t}} =  \delta A$. We arbitrarily
pick $c$ as our measuring stick for $\frac{\delta x}{\delta t}$.
Also, in minkowski space, our measuring stick is actually $(ct, x, y, z)$,
so $\partial_0 = \partial_{ct}$ So, when we write the equation for $E$, we should actually write

\begin{align*}
    E &= c \bigg(- \frac{\grad \phi}{c}  - \pdv{A}{ct}\bigg)
\end{align*}

which becomes:
\begin{equation}
    \boxed{E^i = c F^{i0}}
\end{equation}

\section{Rewriting Maxwell's equations in terms of $F$}
Now that we have constructed the Faraday tensor $F$, we wish to re-expresss
Maxwell's equations in terms of this object. This will give us a compact
form of the laws which are invariant under coordinate transforms.

\subsection{Combining (1) $\grad E = \frac{\rho}{\epsilon_0}$, (4) $\curl B = \mu_0 J + \pdv{E}{t}$}
\subsubsection{1. Using (4) $\curl B = \mu_0 J + \pdv{E}{t}$}

We consider the 4th Maxwell equation:

\begin{align*}
    \curl B &= \mu_0 J + \epsilon_0 \mu_0 \pdv{E}{t} \\
    \curl B &= \mu_0 J + \frac{1}{c^2} \pdv{E}{t} \\
            &\text{Converting to indices,}\\
    (\curl B)^i &= \mu_0 J^i + \frac{1}{c} \pdv{E^i}{ct} \tag{From $\partial_{ct} = \frac{1}{c} \partial_t$} \\
                &= \mu_0 J^i + \frac{1}{c} \pdv{E^i}{X^0} \\
                &= \mu_0 J^i + \pdv{F^{i0}}{X^0} \tag{From $E^i = c F^{i0}$} \\
                &= \mu_0 J^i + \partial_0 F^{i0}
\end{align*}

Now, we start to simplify the LHS, $\curl B$:

\begin{align*}
    &(\curl B)^i = \levicevita^{ijk} \partial_j B_k \\
    %
    &\text{Since $B^k = \frac{1}{2} \levicevita^{kmn} F_{mn}$,} \\
    %
    &\text{$B_k = \frac{1}{2} \levicevita_{kmn} F^{mn}$,} \tag{\textbf{TODO:} this is scam} \\
    %
    &(\curl B)^i = \levicevita^{ijk} \partial_j \bigg( \frac{1}{2} \levicevita_{kmn} F^{mn} \bigg) =
    \frac{1}{2} \levicevita^{ijk} \levicevita_{kmn} \partial_j F^{mn}\\
\end{align*}

\textbf{Aside: We need to know how to evaluate $\levicevita^{ijk} \levicevita_{kmn}:$}
\begin{align*}
    \levicevita_{i_1, i_2, \dots, i_n} \levicevita_{j_1, j_2, \dots j_n} =  
    \det{
    \begin{vmatrix}
        \delta_{i_1 j_1} & \delta_{i_1 j_2} &\dots &\delta_{i_1 j_n} \\
        \delta_{i_2 j_1} &\delta_{i_2 j_2} &\dots &\delta_{i_2 j_n} \\
        \vdots           &\vdots  & \ddots & \vdots \\
        \delta_{i_n j_1} & \delta_{i_n j_2} & \dots & \delta_{i_n j_n}
\end{vmatrix}}
\end{align*}

$\levicevita^{ijk} \levicevita^{imn} = -1 (\delta_j^m \delta_k^n - \delta_j^n \delta_k^m)$


He argued that we get a $-1$ factor here due to the presence of the
metric. I'm not fully convinced, but I can handwave this using the
magic words "tensor density".


Plugging both equations together,

\begin{align*}
    &\frac{1}{2} \levicevita^{ijk} \levicevita_{kmn} \partial_j F^{mn} =  \mu_0 J^i + \partial_0 F^{i0}  \\
    %
    &\text{(Since $kij$ is an even permutation of $ijk$):} \\
    %
    &\frac{1}{2} \levicevita^{kij} \levicevita_{kmn} \partial_j F^{mn} =  \mu_0 J^i + \partial_0 F^{i0}  \\
    %
    &\text{(Using  $\levicevita^{kij} \levicevita^{kmn} = -1 (\delta_i^m \delta_j^n - \delta_i^n \delta_j^m)$):}\\
    %
    &\frac{1}{2} \big[ 
   - \big(\delta^i_m \delta^j_n - \delta^i_n \delta^j_m\big) \big]
   \partial_j F^{mn} =  \mu_0 J^i + \partial_0 F^{i0} \\
    %
   &- \frac{1}{2} \big[ \partial_n F^{in} - \partial_m F^{mi}  \big] = \mu_0 J^i + \partial_0 F^{i0}   \\
   %
   &\text{($F$ is anti-symmetric, so rewriting $\partial_m F^{mi} = -\partial_m F^{im}$):} \\
   %
   &-\frac{1}{2} \big[ \partial_n F^{in} + \partial_m F^{im} \big] = \mu_0 J^i + \partial_0 F^{i0}   \\
   %
   &\text{(Replacing $\partial_m F^{im} \equiv \partial_n F^{in}$ since $m$ is free):} \\
   %
   &-\big[ \partial_m F^{im} \big] = \mu_0 J^i + \partial_0 F^{i0}   \\
   % 
   &\mu_0 J^i + \partial_0 F^{i0}  + \partial_m F^{im}  = 0 \\
   % 
   &\mu_0 J^i + \partial_\mu F^{i\mu} = 0 \tag{$\mu = \{0, 1, 2, 3 \}$}
\end{align*}

This gives us a continuity-style equation, linking the current density $J$ to
the rate of change of $F$.
\begin{equation}
    \boxed{ \mu_0 J^i + \partial_\mu F^{i\mu} = 0 \tag{$\mu = \{0, 1, 2, 3 \}$} }
\end{equation}


\subsubsection{Second part, using 1st equation}

\begin{align*}
    &\grad E = \frac{\rho}{\epsilon_0} \\
    %%
    &\partial_i E^i = \frac{\rho}{\epsilon_0} \\
    %%
    &\text{(Substituting $E^i = c F^{i0}$, $c^2 = \frac{1}{\mu_0 \epsilon_0}$): } \\
    %%
    &c \partial_i F^{i0} = \frac{\rho}{\epsilon_0}  = \frac{\rho \mu_0}{\mu_0 \epsilon_0} = \rho \mu_0 c^2 \\
    %%
    &\partial_i F^{i0} = \mu_0 c \rho \\
    %%
    &\text{(Since $F$ is anti-symmetric, $F^{00} = 0$, Hence):}\\
    %%
    &\partial_0 F^{00} + \partial_i F^{i0} = \mu_0 c \rho \\
    %%
    &\partial_\mu F^{\mu 0} = \mu_0 c \rho
\end{align*}

\begin{equation}
    \boxed{ \partial_\mu F^{\mu0} = \mu_0 c \rho}
\end{equation}

\subsubsection{Combining part 1 and part 2:}


\begin{align*}
    \mu_0 J^i + \partial_\mu F^{i\mu} = 0 \tag{From $B$}  \\
    \partial_\mu F^{i\mu} = -\mu_0 J^i 
    \partial_\mu F^{\mu 0} = \mu_0 c \rho \\
    \partial_\mu F^{0 \mu} = - \mu_0 c \rho \\
\end{align*}

To combine these equations, \textbf{we set:}
\begin{equation}
    \boxed{J^0 \equiv c \rho}
\end{equation}
We arrive at the unified equation:

\begin{align*}
    \partial_\mu F^{\nu \mu} = - \mu_0 J^{\nu}
\end{align*}

Choose units such that $c = \frac{h}{2 \pi} = G_n = 1$, which gives us:


\begin{align*}
    &\partial_\mu F^{\nu \mu} = -  J^{\nu} \\
    &\text{$F$ is antisymmetric, so flipping indices} \\
    &\partial_\mu F^{\mu \nu} =  J^{\nu} \\
\end{align*}

\begin{equation}
    \boxed{ \partial_\mu F^{\mu \nu} =  J^{\nu} }
\end{equation}

Note that this is \textbf{Ampere's law!}

\subsection{Combining (2) $\curl E = - \pdv{B}{t}$, (3) $\grad B = 0$}

\begin{align*}
    %%
    &\curl E = - \pdv{B}{t} \\
    %%
    &(\curl E)^i = \levicevita^{ijk} \partial_j E_k = - \partial_0 B \\
    %%
    &\levicevita^{ijk} \partial_j E_k = - \partial_0 (\frac{1}{2} \levicevita^{ijk} F_{jk}) \\
    %%
    &\levicevita^{ijk} \partial_j E_k  + \partial_0 (\frac{1}{2} \levicevita^{ijk} F_{jk})  = 0 \\
    %%
    &2\levicevita^{ijk} \partial_j E_k  + \partial_0 (\levicevita^{ijk} F_{jk})  = 0 \\
\end{align*}

Now we begin from the other direction, and start the derivation.

We know that the equation we want is:

\begin{equation}
    \boxed{\levicevita^{\alpha \beta \mu \nu}  \partial_{\beta} F_{\mu \nu} = 0}
\end{equation}

\subsubsection{$\alpha = 0$ case:}
First, set $\alpha = 0$. So now, the other $\beta, \mu, \nu$ are forced to be
become space components --- $(i, j, k)$.

Therefore, the equation now becomes:
\begin{align*}
    \levicevita^{0 i j k}  \partial_{i} F_{j k} = 0
\end{align*}

However, note that $\levicevita{0 i j k} = \levicevita{i j k}$, because if
$(i j k)$ is an even permutation, so will $(0 i j k)$, and vice versa for odd
(since $0 < i, j, k$).

Using this, the equation becomes

\begin{align*}
    \levicevita^{i j k}  \partial_{i} F_{j k} = 0 \\
    \partial_{i} ( \levicevita^{i j k} F_{j k}) = 0 \\
    \text{Since $B^i = \frac{1}{2} \levicevita^{ijk} F_{j k}$:} \\
    \partial_{i} \bigg( \frac{B^i}{2} \bigg) = 0 \\
    \partial_{i}  B^i = 0 \\
    \grad B = 0
\end{align*}

Hence, the above equation does encode $\grad B = 0$.

\subsubsection{$\alpha = m$ case:}
Let $\alpha$ be a spatial dimension $m = \{ 1, 2, 3 \}$.
\begin{align*}
    \levicevita^{\alpha \beta \mu \nu}  \partial_{\beta} F_{\mu \nu} = 0 \\
    \levicevita^{m \beta \mu \nu}  \partial_{\beta} F_{\mu \nu} = 0
\end{align*}

Once again, we get two cases, one where $\beta = 0$, and one where $\beta = n$
where $n$ is a spatial dimension. If $\beta = 0$, then the other dimensions
are forced to be spatial dimensions, which we shall denote as $\mu \equiv x$,
$\nu \equiv y$
\begin{align*}
    \levicevita^{m \beta \mu \nu}  \partial_{\beta} F_{\mu \nu} = 0 \\
    \levicevita^{m 0 x y}  \partial_{0} F_{x y} + \levicevita^{m n \mu \nu}  \partial_{n} F_{\mu \nu}  = 0 \\
\end{align*}

Now note that $\levicevita^{m 0 \mu \nu} = - \levicevita{0 m \mu \nu} = - \levicevita{m \mu \nu}$.

Using this, we can rewrite the above equation as:

\begin{align*}
    %%%
    \levicevita^{m 0 x y}  \partial_{0} F_{x y} + \levicevita^{m n \mu \nu}  \partial_{n} F_{\mu \nu}  = 0 \\
    %%%
    - \levicevita^{m x y}  \partial_{0} F_{x y} + \levicevita^{m n \mu \nu}  \partial_{n} F_{\mu \nu}  = 0 \\
\end{align*}

We now consider cases for $\mu$ in the second term, where either $\mu = 0$ or $\mu = o \in \{1, 2, 3\}$

If $\mu = 0$, then the other dimension $\nu$ must be a spatial dimension $p$.
If $\mu = q$, then the other dimension $\nu$ must be a time dimension $0$
(This is because we are not allowed to have 4 spatial dimensions, since the $\levicevita$
evaluates to 0 on repeated dimensions).


\begin{align*}
    - \levicevita^{m x y}  \partial_{0} F_{x y} + \levicevita^{m n \mu \nu}  \partial_{n} F_{\mu \nu}  = 0 \\
    \\
    %%%
    - \levicevita^{m x y}  \partial_{0} F_{x y} + \\
    %%% mu = 0, nu = p
    \levicevita^{m n 0 p}  \partial_{n} F_{0 p}  \tag{$\mu = 0$, $\nu = p$} \\
    %%% mu = q, nu = 0
    \levicevita^{m n q 0}  \partial_{n} F_{q 0} \tag{$\mu = q$, $\nu = 0$} \\
    = 0 
\end{align*}
Rearranging, and using the fact that $F_{0 p} = - F {p 0}$,
$\levicevita{m n 0 p} = \levicevita{0 m n p} = \levicevita{m n p}$,
$\levicevita{m n q 0} = - \levicevita{0 m n q} = - \levicevita{m n q}$,

\begin{align*}
    - \levicevita^{m x y}  \partial_{0} F_{x y} + 
    %%% mu = 0, nu = p
    \levicevita^{m n p}  (- \partial_{n} F_{p 0}) +
    %%% mu = q, nu = 0
    (- \levicevita^{m n q})  \partial_{n} F_{q 0}
    = 0 
\end{align*}

Multiplying throughout by $-1$, and noticing that since $p, q$ are dummy indeces,
we can set $p = q$. This allows us to get:



\begin{align*}
    \levicevita^{m x y}  \partial_{0} F_{x y} + 
    %%% mu = 0, nu = p
    2 \levicevita^{m n p}   \partial_{n} F_{p 0} = 0
\end{align*}

First, remember that $E_p = F_{p 0}$. So, we can replace the term $F_{p 0}$
(upto fudging of constant factors that we have always done), with $E_p$.

Now, compare

\begin{align*}
    &\levicevita^{m x y}  \partial_{0} F_{x y} + 
    2 \levicevita^{m n p}   \partial_{n} E_p = 0 \tag{Our equation} \\
    \\
    &2\levicevita^{ijk} \partial_j E_k  + \partial_0 (\levicevita^{ijk} F_{jk})  = 0 \tag{Previous equation} \\
\end{align*}

Note that the two equations are identical upto variable naming, and are
hence considered equal. So, we have encoded both of Maxwell's
laws into this particular equation:
\begin{equation}
    \boxed{\levicevita^{\alpha \beta \mu \nu}  \partial_{\beta} F_{\mu \nu} = 0}
\end{equation}

\chapter{Quantum deletion}

\begin{align*}
    &\psi = \alpha \ket0 + \beta \ket 1 \\
    &\ket \psi \ket 0 \ket M \rightarrow \ket \psi \ket \psi \ket M_{\psi} \\
    &(\alpha \ket0 + \beta \ket 1) \ket 0 \ket M = (\alpha \ket{00} + \beta \ket{10}) \ket M \\
\end{align*}

Cloning is possible upto fidelity $0.83$. We get a similar theorem for
quantum deletion --- in that, we can perform approximate deletion.


If $\psi_1, \psi_2$ are two non-orthogonal states, then there is no deletion
machine by which we can delete one copy from two cpies of of $\psi_1$ and 
$\psi_2$

\begin{align*}
&\psi_1 \psi_1 \rightarrow \psi_1 \Sigma \\
&\psi_2 \psi_2 \rightarrow \psi_2 \Sigma \\
&\bra{\psi_1}\ket{\psi_2}^2 = \bra{\psi_1}\ket{\psi_2}\bra{\Sigma}\ket{\Sigma} \\
&(\bra{\psi_1}\ket{\psi_2} - 1) \bra{\psi_1}\ket{\psi_2} = 0
\end{align*}

Hence $\bra{\psi_1}\ket{\psi_2} = 0 \lor 1$


\section{No flipping}
One of the strongest impossible operations. Given a state $\ket{\psi}$, we cannot
make a state that takes it to an orthogonal state $\ket{\overline{\psi}}$.

(Take a state $a0 + b1$ to $-b0 + a1$?)


\section{No partial erasure}
$\ket{\psi(\theta, \phi)} \rightarrow \ket{\psi'(\theta)}\ket{\Sigma}$ is
impossible, where $\psi(\theta, \phi)$ is the parametrisation of a 2
qubit state on a bloch sphere.

\section{No splitting}
We cannot split quantum information.
$\ket{\psi(\theta, \phi)} \rightarrow \ket{\psi'(\theta)}\ket{\Sigma'(\phi)}$ is
impossible. That is, we cannot split the combined information in $(\theta, \phi)$
into two separate pieces of data.

\chapter{Clasical information theory}
Book recommendation: Elements of Information theory --- JJ Thomas and Thomas Cover.

\section{What is information}
\paragraph{Entropy}
Blah blah blah, define surprisal of a probability 
\begin{align*}
    I: [0, 1] \rightarrow \R \quad
    I(p) = - \log p
\end{align*}
Now, entropy of a random variable $X$ is:
\begin{align*}
    \H : \text{Random variable} \rightarrow \R \quad
    \H(X) \equiv \sum_{x \in X} p(x) I\l(p(x)\r)
\end{align*}

\paragraph{Conditional entropy}
\begin{align*}
    &\H : \text{Random variable} \times \text{Random variable} \rightarrow \R \quad
    \H(Y|X) \equiv \sum_{x \in X} p(x) H(Y|X=x) \\
\end{align*}

It can be shown that
    $\H(X, Y) = \H(X) + \H(Y|X)$
\paragraph{Mutual information}
\begin{align*}
    I(X; Y) &\equiv H(X) - H(X|Y) \\
            &= H(X) - [H(X, Y) - H(Y)] \\
            &= H(X) + H(Y) - H(X, Y) 
\end{align*}
It is a measure of the reduction of uncertainty in $X$ upon knowing $Y$.

\paragraph{Relative entropy / K-L divergence}
Suppose there are two probability distributions $P(x)$ and $Q(x)$. The
relative entropy is:

\begin{align*}
    H \l(p(x) || q(x) \r) \equiv \sum_{x \in X} p(x) \log \frac{p(x)}{q(x)}
\end{align*}

\begin{theorem}
    K-L divergence is always positive. That is, $H(p(x) || q(x)) \geq 0$,
    with $H(p(x) || q(x)) = 0 \iff p(x) = q(x)$
\end{theorem}
\begin{proof}
    \begin{align*}
        H(p(x) || q(x)) 
        &= \sum_{x \in X} p(x) \log \l( \frac{p(x)}{q(x)} \r) \\
        &= - \sum_{x \in X} p(x) \log \l( \frac{q(x)}{p(x)} \r) \\
\end{align*}

We know that $\log x \leq \frac{x - 1}{\ln 2}$.
Hence, $-\log x \geq \frac{1 - x}{\ln 2}$.

\begin{align*}
        H(p(x) || q(x)) 
        &= - \sum_{x \in X} p(x) \log \l( \frac{q(x)}{p(x)} \r) \\
        &\geq \frac{1}{\ln 2} \sum_{x \in X} p(x) \l( 1 - \frac{q(x)}{p(x)} \r) \\
        &\geq \frac{1}{\ln 2} \sum_{x \in X}\l(p(x) - q(x) \r) \\
        &\geq \frac{1}{\ln 2} (1 - 1) = 0 \\
\end{align*}
\end{proof}

%% What is the document class I need?
\documentclass{article} 
%% Some recommended packages.
\usepackage{booktabs}   %% For formal tables:
                        %% http://ctan.org/pkg/booktabs
\usepackage{subcaption} %% For complex figures with subfigures/subcaptions
                        %% http://ctan.org/pkg/subcaption
\usepackage{enumitem}
\usepackage{minted}
\newminted{fortran}{fontsize=\footnotesize}

\usepackage{xargs}
\usepackage[colorinlistoftodos,prependcaption,textsize=tiny]{todonotes}
\begin{document}
\section{Lecture 5 - 2d Convolution, Statistical signal processing}

- Using Gaussians for blurring.


\subsection{Moving Average}
Low pass conpoment: $movingaverage(x, N)$
High pass component: x - movingaverage(x, N)$

if $N$ is small, we will pick up on noise. if $N$ is large, we may smooth
way too much.

Now, we need to perform \textit{Statistics} on signals.

\subsection{Recursive moving average}
$y[n] = \frac{1}{N}\sum{m = 0}^{N - 1} x(n - m)$
$y[n] = \frac{x[n]}{N} + \frac{1}{N}\sum{m = 1}^{N - 1} x(n - m)$ + \frac{1}{N}x(n - N) - \frac{1}{N} x(n - N)
$y[n] = y[n - 1] + \frac{1}{N}(x(n) - x(n - N))$


\section{Statistical signal processing}

$Mean(n) = \frac{1}{N}\Sum{i = 0}{N - 1}x(i)$
$Mean(n) = mean_{N - 1} \cdot \frac{N - 1}{N} + \frac{1}{N} x(n - 1)$


$\sigma(n) = \frac{1}{N - 1} \sum{i = 0}{N - 1} (x(i) - \mu)^2$
$ = \frac{1}{N-1} \sum{i = 0}{N - 1} (x(i)^2 + \mu^2 - 2 \mu x(i))$
$ = \frac{1}{N-1} (\sum{i = 0}{N - 1} x(i)^2 + N \mu^2 - 2 \mu N \frac{\sum{i =0}^{N - 1} x(i)}{N} $
$ = \frac{N}{N - 1}(\frac{\sum x(i)^2}{N} - \mu^2)

(TODO: Write sigma recursively, just do this thing, not sure about computation)


\newcommand{\badpref}{\ensuremath{\textsf{BadPrefix}}}
\newcommand{\badprefix}{\badpref}

\newcommand{\tracesfin}{\ensuremath{\textsf{Traces}_{fin}}}

\chapter{Lecture 6: Liveness \& Fairness}

\begin{definition}
$E$ is a \textbf{Safety Property} iff for all words in $T \in E^c$, there is a finite bad prefix $A_0 \dots A_n$ such that \emph{no extension}
of this is in $E$. We write the set of bad prefixes for a safety property as $\badpref(E) \subseteq A^+$
\end{definition}
Formally, we write:

$$
T \models E \iff \tracesfin(T) \cap \badpref(E) = \emptyset
$$


we write $\badpref(E)$ to be the set of all finite words $A_1 \dots A_n \in A^+$ such that there is no extension which lives in $E$.

\begin{definition}
A \textbf{minimal bad prefix} is a bad prefix that itself contains no proper bad prefix.
\end{definition}


\begin{theorem}
Every invariant $E$ defined by a propositional formula $\phi$ is a safety property.
\end{theorem}
\begin{proof}
all finite words of the form $A_1 \dots A_n$ such that $A_n \not \models \phi$ is the bad prefix.
\end{proof}

\begin{definition}
The \textbf{prefix set} of an infinite word
$\sigma$ is the set of words
$pref(A_1 A_2 \dots) \equiv \{ A_1 \dots A_n : \forall n \geq 0 \}$.
\end{definition}


\begin{definition}
The \textbf{prefix set} of a property $E$ is the union of the prefix closures of all the words in it.  $pref(E) \equiv \bigcup_{\sigma \in E} pref(\sigma)$.
\end{definition}

\begin{definition}
The \textbf{prefix closure} of a property $E$ is:
$$
pref(E) \equiv \{ \sigma \in (2^{AP})^\omega : pref(\sigma) \subseteq pref(E) \}
$$
\end{definition}


\begin{theorem}
$E$ is a safety property iff $\badpref(E) \subseteq pref(E)$.
\end{theorem}
\begin{proof}
\end{proof}

\section{Safety Property as closed sets}

Let $X \equiv 2^{AP}$, our space from where we pick up events in the trace.
Define a metric on the space of infinite sequences $X^\omega$. Given two executions $\vec x, \vec y \in X^\omega$, 
we measure their similarity in the smallest index they differ (Idea from the paper ``LTL is Closed Under Topological Closure'').
We define a metric with $d(\vec x, \vec x) \equiv 0$, and $d(\vec x, \vec y) = 2^{-i}$ if $i$ is the smallest index such that $\vec x[i] \neq \vec y[i]$.
(Think why this obeys transitive).

The distance between a trace $\vec x$ and a property $S \subseteq X^\omega$ is the infimum of the distances from every element in $S$: $d(x, S) \equiv \inf_{y \in S} d(x, y)$.
Using this, we will show that safety properties correspond to closed sets, and liveness properties correspond to dense sets.

\subsection{Safety Properties}
Under this interpretation, a safety property is a closed set.
Intuitively, we are stating that every limit point of $S$ is in $S$.
Written differently, we are saying that $\forall \vec x \in X, d(\vec x, S) = 0 \implies \vec x \in S$. (Compare this to the closed interval $[0, 1]$ versus the open $(0, 1)$).
Alternatively, we can think in terms of limit points. $S$ contains all its limit points.
If we have a property $\vec x$, and we can write a sequence $\vec s_1, \vec s_2, \dots$,
where each $s_i \in S$, and $d(s_i, \vec x) < 2^i$,
then since $S$ is closed, we must have that $\lim_i \vec s_i = \vec x \in S$.
From our safety interpretation, this means that $s_1$ and $\vec x$ can diverge at step $2$, but this already tells us that $\vec x$ is safe upto 2 steps.
Similarly, $s_2$ and $\vec x$ diverge at step $4$, this tells us that $\vec x$ is safe upto 4 steps.
Repeating this, we can see that $d(s_i, \vec x) < 2^i$ establishes that $\vec x$ is safe for $2^i$ steps,
and thus it must be safe for all time.

\subsection{Liveness Properties}
Recall that a liveness property is that which can extend any finite trace.
This can be seen as a \emph{denseness} condition on the set, because intuitively, every trace is arbitrarily close to the liveness property. (Think of $\mathbb Q \in \mathbb R$).
Intuitively, suppose we have a trace $\vec x$, and let $L$ be a liveness property. Now, since every finite prefix $\vec x[:i] \in X^*$
must be extensible to a new property $\vec l_i \in X^\omega$ such that $\vec x[:i] = \vec l[:i]$ (i.e., $d(\vec x, \vec l_i) \leq 2^{-i}$), this implies that
in fact, the sequence $\vec l_1, \vec l_2, \vec l_3, \dots$ establishes that $\inf_i d(\vec x, \vec l_i) = 0$.
Therefore, any property $\vec x$ is arbitrarily close to $\vec L$.

\subsection{Decomposition Theorem}
We prove in trace semantics that any property can be written as the intersection of a safety and liveness property.
Is it true that any set of a metric space can be written as the intersection of a closed set and a dense set?
Yes.
For a given set $S$, let the closed set be its closure, $C_S \equiv \overline S$.
See that $C_S$ is an overapproximation, since it has added the limit points $C - S$. See that the set of limit points has empty interior,
so its complement will be dense. We define the dense set $D_S \equiv X - (C - S)$, or $X - \texttt{extra}$.

\chapter{Quantum Computing: Shor's algorithm}

We have $pq = N$. We wish to find $x$ such that $y = a^x \mod N$.

\begin{align*}
    &s_0 = \ket 0^{\tensor n} \\
    &s_1 = H^{\tensor n} s_0  = \frac{1}{2^n} \sum_i \ket{i} \\
    &s_2 = a^{s_1} \mod N = \frac{1}{2^n} \sum_i \ket{a^i \mod N} \\
\end{align*}

Let us now consider the function $f(x) = a^x \mod N$. This function will
be periodic with period $r$. Let us assume that $f: [0, Q-1] \to [0, Q-1]$ where
$Q$ is the domain of the function / the maximum value that is fed to $f$.

Now, note that since the function is periodic, $\l[\forall y, |f^{-1}(y)| = Q/r\r]$.

\begin{align*}
    &s_3 = measure(s_2) = \frac{1}{\sqrt\frac{Q}{r}} \l(\ket{a_0} + \ket {a_0 + r} + \dots \r)
\end{align*}

At this point, the states in $s_3$ will consists of inputs $\l[ a_0, a_0 + r, \dots a_0 + \delta r \r]$
such that $f(a_0 + \delta r) = m_0$.

We now wish to extract the $r$ from the superposition of states. A non solution
is to try and repeatedly measure the values, then what we can get is a set of
values $\l[a_0 + \delta_0 r, a_1 + \delta_1 r, a_2 + \delta_2 r, \dots \r]$.
Recovering $r$ from this set is difficult, so we try another solution.

because the Fourier transform is a change of basis, it's a unitary matrix,
and can hence be implemented as a quantum circuit. Since the function $f$
periodic and $r$ is the perid, feeding $f$ into a fourier transform will
allow us to find $r$. 

On applying the fourier transform, the function becomes a new function
such that $g \equiv FFT(f)$ such that $g(0) = g(Q/r) = g(2Q/r) = g(\lambda Q/r) = 1, g(\_) = 0~\text{otherwise}$.

    

\newcommand{\gni}{\texttt{GNI }}
\chapter{Probabilistic proofs}

\section{\ip --- interactive proofs}
\begin{definition}
Completeness: For every true assertion, there is a valid proof.
\end{definition}

\begin{definition}
Soundness: For every false assertion, no valid proof exists.
\end{definition}

A good proof system must also be such that the verifier is efficient
(that is, polynomial time).

If we ask that a proof system must be sound and complete, there is no 
scope for error! Further, it is not clear if the verifier and the
prover can "talk" to each other. If we choose to allow interactions, what
are the implications?


We relax the assumptions this way --- Relaxed compleness states that
for every true assertion, there is a
proof strategy that will convince the verifier with probability 
at least $> \frac{2}{3}$.  
Similarly, relaxed soundness states that for every false assertion,
every proof strategy fails to convinve the verifier with probability
at least $> \frac{2}{3}$. 

The formalization is as follows:
\begin{definition}
Interactive proof systems
\begin{itemize}
\item An interactive proof system for a language $L$ consists of two
entities: a prover $P$ and a verifier $V$.
$P$ and $V$ share common input, and work for $R \in \mathbb{N}$ rounds.

\item In each round, the prover can send the verifier a message that 
is polynomial in the length of the input.

\item The verifier can send a polynomial length reply to the prover.

\item The verifier is a randomized polynomial time turing machine. Time
is measured as a function of the length of the input.

\item \textbf{Completeness}: $\forall x \in L$, there exists a prover strategy
so that the verifier accepts with probability $> \frac{2}{3}$.

\item \textbf{Soundness}: $\forall x \notin L$, any prover strategy will lead
the verifier to accept with probability  $< \frac{1}{3}$.
\end{itemize}
\end{definition}

Note that the power of the prover in unspecified in this definition ---
we are implicitly saying that finding a proof is generally much harder
than verifying a proof. Hence, the prover has no real bounds on the power,
while the verifier does.

We also have the value $R \in \mathbb{N}$, which lets us setup the number
of rounds. This is a knob we can twiddle, that allows us to change the hardness
of the problem.



\begin{definition}
The \ip hieararchy: Let $r: \mathbb{N} \to \mathbb{N}$ be the "number of rounds" function.
Define $IP(r)$ to be the set of languages such that there exists an interactive
proof system using at most $r(|x|)$ rounds on input $x$.

For a class of functions $R \subset \{ \mathbb{N} \to \mathbb{N} \}$, we can then define $\ip~(R) = \cup_{r \in R} ~\ip~(r)$.
\end{definition}


Note that $\nptime \subset \ip$.  Also, the number of rounds cannot be more than 
polynomial --- the verifier is poly bounded in time, so the verifier
cannot work more than poly rounds.  So, we denote $\ip \equiv \ip(O(poly(n))$.

Both \textbf{randomness} and \textbf{interaction} are essential to the definition.


When randomness is removed but only interaction is present, this will be
like \nptime. The prover can arrive by itself the set of messages the
verifier would send to the prover.


When interaction is removed but randomness is remained, the verifier is
similar to that of \nptime, but the verifier can now be \textbf{probabilistic}.
This class of languages is likely beyond \nptime.

\section{Graph non-isomorphism (GNI)}
Two graphs $G$, $H$ are isomorphic (denoted $G \sim H$), iff there exists
a bijection such that $\forall x, y \in V_1, (x, y) \in E_1 \implies (f(x), f(y)) \in E_2$.

Using this, we define \gni, the problem of checking if two graphs
are not isomorphic:
\begin{align*}
\gni \equiv \{ \langle G, G' \rangle ~\vert~ G \nsim G' \}
\end{align*}

Graph isomoprhism is in \nptime since the witness will just be the bijection.
Hence, \gni is in \conptime, and it is not known whether \gni is in \nptime.

In an interactive proof system for \gni, the verifier asks the prover to
distinguish between isomorphic graphs.

\begin{itemize}
\item $G_1, G_2$ are given to both prover, verifier.

\item The verifier chooses a random  $r \in \{1, 2\}$ uniformly at random.

\item The verifier picks a random permutation $\pi$ of the set $\{1, 2,\dots, |V(G_1)|\}$

\item the verifier constructs the graph $H$ as the permutation of $G_r$ under $\pi$.
The graph $H$ is sent to the prover. That is, $H \equiv \pi(G_r)$.

\item the prover P replies with $r' \in \{1 2\}$. The reply $r'$ is 1
if $H$ is isomorphic to $G_1$, and $2$ otherwise.

\item The verifier accepts if $r = r'$.
\end{itemize}

Note that $H \sim G_r$. Now if $G_r \sim G_{other}$, then $H \sim G_r \sim G_{other}$, and
so the prover has to literally guess between $G_r$ and $G_{other}$, and at best
it can simply guess. (Even though the prover has \textit{unbounded computation},
it is unable to distinguish between $G_r$ and $G_{other}$). In two rounds,
the probability of the guesses of the prover being right is $\frac{1}{2}^2 = \frac{1}{4}$,
which fulfils our soundness guarantee ($\frac{1}{4} < \frac{1}{3}$).

On the other hand if $G_r \nsim G_{other}$, then if the prover knows how to solve
\gni , it can check between $H$, $G_r$, and $G_{other}$ to consistently
report $G_r$. In this case, the prover will \textit{always} be correct,
so this will pass compleness (since $1 > \frac{2}{3}$).

This is very interesting, because the verifier \textbf{does not know} whether
$G_1 \sim_? G_2$. The verifier tries to engage with the prover, to understand
whether $G_1 \sim_? G_2$.

\begin{theorem}
$\gni \subset \ip (2)$
\end{theorem}

\begin{theorem}
$\conptime \subset \ip$
\end{theorem}

\begin{theorem}
$\ip = \pspace$
\end{theorem}

\newcommand{\pram}{\texttt{PRAM}}
\chapter{Parallel Computing}

Moore's law, blocking factors:
\begin{itemize}
    \item Memory wall: memory latency was higher than compute.
    \item Power wall: Power leakage.
    \item ILP wall: ILP via branch prediction, out-of-order-execution, and
        speculative execution. Diminishing returns from instruction-level
        parallelism.
\end{itemize}

Interesting questions one can ask:
\begin{itemize}
    \item How do we analyze parallel algorithms? 
    \item Can every sequential algorithm be parallelized?
    \item What are the complexity classes related to parllel computning?
    \item Can sequential programs be automatically converted to parallel programs?
\end{itemize}

\textit{Concurrent data structures, course: Professor Govindarajulu.}

\section{\pram~model}

Global shared memory, shared by $n$ processors. Each processor has
individual bidirectional buses into the memory.

Also, note that we have \textit{random access} into the memory, which is
different from a turing machine, which only offers sequential access.

\textit{(Sid question: what is a problem that can be solved efficiently given
random access and not with sequential access?)}

Access to shared memory costs the same as one unit of computation.

Different flavours provide different semantics to concurrent access
of shared memory (EREW, CREW, CRCW).
\begin{itemize}
    \item EREW - Exclusive Read, Exclusive Write. No scope for memory contention,
        so algorithm design is tough.
    \item CREW - Concurrent Read, Exclusive Write. Allow processors to read
        simeltaneously, writes are still exclusive. Is practically feasible.
    \item CRCW - Concurrent Read, Concurrent Write
        Processors can read/write simeltaneously. So here, we need to specify
        semantics of concurrent writes!
\end{itemize}

Flavours of concurrent write semantics:
\begin{itemize}
    \item COMMON: Concurrent write is allowed as long as all the values being
        attempted are equal. Eg: boolean OR of $n$ bits. Each processor $p_i$
        will read $a[i]$. if $a[i] = 1$, then $p_i$ tries to write 1.
        it doesn't matter how many processors try to write 1, if any bit is
        1, then the output will be 1. We need to make sure the cell is
        initialized to 0, so that if all bits are 0, the answer is 0.
    \item ARBITRARY: In the case of a concurrent write, \textit{someone} wins
        and its write takes effect.
    \item PRIORITY: Assumes that processors have numbers that can be used
        to decide which write succeeds.
\end{itemize}

\section{Matrix multiplication}

\begin{itemize}
    \item Recursively block the matrices.
    \item Multiply strips (cannon's algorithm).
\end{itemize}

\subsection{Prefix computations}
Given an array $A$ of n elements and an associative operator $\circ$, we want
to compute $P(i) = \circ_{k \in [0\dots n]} A[k]$.
$P(0) = A(0), P(1) = A(0) \circ A(1) \dots$.


We can use the naive approach:
\begin{minted}{python}
def scan(A, op):
    out[0] = A[0]
    for i in range(1, length(A)):
        out[i] = op(A[i], out[i - 1])
\end{minted}

We have  linear RAW dependences:  $out[i] \to out[i - 1]$.

So, we crete a complete binary tree with processors at the internal nodes.
Input is at the leaf node. Each node performs the $\circ$ of its left and
right subtree.


\begin{minted}{python}
def sumfast(A, op, l, r):
    if (l == r):
        return A[l]
    else:
        mid = (l + r) / 2
        return op(scanfast(A, op, l, mid), scanfast(A, op, mid, r));

def sum(A, op):
    return sum(A, op, 0, length(A))
\end{minted}
Note that this will not give us \textit{prefix sums}.
We can finish in $\log(n)$ time given $2^n$ processors.

For a \textit{prefix sum}, we need a combination of upward and downward traversal.
First send data from bottom to top. Next, send down data from top to bottom
of the prefix sums towards the leaves.

\subsubsection{Analysis of prefix computation}
\begin{itemize}
    \item Step 1 can use $\frac{n}{2}$ processors in parallel, each using 1 unit
        of time.

    \item Step 2 is a recursive calls and takes $T(\frac{n}{2})$ time.

    \item Step 3 uses $n$processors each of which take 1 unit of time.
\end{itemize}

Work done by the algorithm: $W(n) = W(n/2) + O(n)$ ($O(n)$ for the first
and third step).
$W(n) = O(n)$ is the solution.

\subsubsection{Optimal parallel algorithm}
A parallel algorithm that does the same amount of work
as the best known sequential algorithm is called an \textit{optimal algorithm}.

This makes sense, because if we set $\texttt{num processors} = 1$, we want the asymptotics
to match the sequential algorithm.



\chapter{Design models of parallel algorithms}

\section{Partitioning}
This is similar to divide-and-conquer, but we don't need to \textit{combine}
solutions! We can treat problems independently and solve it in parallel.
Examples are parallel merging and searching.

We generate subproblems that are independent of each other.
Example is quicksort. Once we partition the array into two subarrays,
we sort the subarrays recursively.

\subsection{Merging in parallel by partitioning}
Two sorted arrays $A$ and $B$ are to be merged into an array $C$.

\subsubsection{suboptimal algorithm --- Time: $O(\log n)$, work: $O(n \log n)$}

We define a function $Rank(x_0, X) = |\{ x < x_0~\vert~x \in X \}|$. Note
that the position of $x_0$ in $sorted(X)$ is equal to $Rank(x_0, X)$.
\textbf{Claim:} $Rank(x, C) = Rank(x, A) + Rank(x, B)$.


For $x \in A$, $Rank(x, A)$ is immediately available (since $A$ is sorted).
We need to find $Rank(x, B)$, but we can find this using binary search through $B$.


Time for each binary search is $O(\log n)$. Total time for merging is
$O(\log n)$, since we are doing each binary search in parallel --- we just need
to read the array $B$, no need to update. The total work is $O(n \log n)$, since
we are performing $O(\log n)$ work for $n$ elements.


Note that this is \textbf{non optimal}. The sequential algorithm has
a time complexity of $O(n)$.


We are going to try and reduce the work to $O(n)$. 

\subsubsection{Merging, take 2, optimal --- time: $O(\log n)$, work: $O(n)$}
General technique is to solve a smaller problem in parallel, and then
extend the solution to the entire problem!

\begin{itemize}
    \item The problem size to be solved is guided by the factor of non-optimality
        in the current algorithm. We need to reduce the total work to $O(n)$.

        For input size $n$, we do $O(n \log n)$ work. So, for input size $n / \log n$,
        we do $O(n / \log n \times \log (n / \log n) \sim O(n) + O(\log(\log(n)) \sim O(n)$.

    \item We pick every $\log n$th element of $A$. We merge the selected elements
        of $A$ and $B$. However, we still perform binary search on the entireity of $B$.

    \item Pick elements $A[\log n], A[2 \log n], \dots, A[ n - \log n], A[n]$, and
        rank the, in $B$ (ie, find their corresponding positions in $B$.)

    \item Define $[B_{r(i)}, \dots, B_{r(i + 1)}] \equiv \text{portion of $B$ between $A[r \log i]$, $A[(r + 1) \log i]$ in $B$}$.

        \begin{minted}{py}
        A = (5) 6 9 12 (15) 18 19 (21) 23 26
        B = 1 4 (..5..) 7 8 10 11 12 (..15..) 16 17 20 (..21..) 22

        In the output array, we can merge
        the array of B between the (..) elements of A
        \end{minted}

        The problem is that the size of $\log n$ per chunk in $A$ does not mean
        that the size is $\log n$ in $B$.


        \begin{minted}{py}
        A = (5) 6 9 12 (15) ... (...) ...
        B = 6 6 6 6 6 6 6 ... 6

        In this case, the entireity of B is between [5, 15]
        \end{minted}

        So, if we can somehow control the size of $B$, so, we can perform
        binary search in $O(\log n)$, with $n / \log n$ processors.

        We then need to perform the merge with $O(\log n)$, 
        \textbf{under certain conditions}.  There are again $n / \log n$
        such merges.


        The work is $O(n)$.


        So now, the only thing we need to control is the size of partitions
        of $B$.


    \item If $[B_{r(i)}, \dots, B_{r(i + 1)}]$ is too large, then we can
        pick $\log n$ items of this section, and we can rank them in $A$!
        Each piece in $A$ will be smaller than $\log n$, since the partition
        of $A$ was already $\log n$.

    \item we can merge two sorted arrays of size $n$ in time $O(\log n)$
        with work $O(n)$.  This algorithm works in \texttt{CREW}.
        We can improve this  further, we will see this later.
\end{itemize}

\subsection{Searching faster --- time: $O(1)$, work: $O(\sqrt n)$}

Each binary search takes $O(\log n)$ time, and we have $O(n / \log n)$ subproblems,
each of size $O(\log n)$. 

Can we make search faster?

\begin{itemize}
    \item Consider a sorted array $A$ with $n$ elements. We want to search
        for an element $x$.
        Given $p$ processors, we can search at the indeces $1, n / p, 2n/p, \dots, n$.

    \item Record the result of each comparison as $1$ or $0$.
        $cmp[i] = 1 \equiv A[i] < x$, $cmp[i] = 0 \equiv A[i] \geq x$.
        More succinctly, \verb|cmp = map (\a -> a < x) A|.

    \item $cmp$ will either have all 0s, all 1s, or a shift from 1s to 0s.

    \item If we have a shift from 1s to 0s, we know that $x$ is likely
        in the $n/p$ segment corresponding to the shift from 1 to 0.

    \item So now, we can recursively search that small segment.

    \item $T(n) = T(n / p) + O(p)$. ($O(p)$ since $cmp$ has length $p$).
        Hence, $T(n) = T(n / p) + O(1)$. This gives us $O(\log n)$ when $p = 1$
        (make sure this is correct, there is some \textbf{off by one here}.

    \item When $p = O(\sqrt n)$, the time taken will be $O(\log n / \log (\sqrt n)) = O(1)$
        This looks useless from a work point of view, but we want to see what this is
        good for!
\end{itemize}

\subsection{From parallel search to merge --- time: $O(\log \log n)$, work: $O(???)$ }
\begin{itemize}
    \item We have two sorted arrays $A$ and $B$, which we want to merge.
    \item We want to rank some elements of $A$ to create paritions of $B$.
    \item Let us take $\sqrt n$ elements of $A$ in $B$.
    \item We have $n$ processors, so each search can use $ n / \sqrt n = \sqrt n$ processors.
    \item each search now finishes in $O(1)$  time.
    \item the problem is that the partitions of $A$ are much larger now (they are $\sqrt n$ large).
    \item we have a $\sqrt n$ sized piece of $A$, and we have a size of B that is of size $(?)$.
        Note that for each piece of $A$, we now choose to allocate $\sqrt n$ processors.
    \item So, we pick $n^\frac{1}{4}$ elements of $A$ in $B$, each of which
        uses $n^\frac{1}{4}$ processors. Size of each piece is now $n^\frac{1}{4}$.
    \item So, we pick $n^\frac{1}{8}$ elements of $A$ in $B$, each of which
        uses $n^\frac{1}{8}$ processors. Size of each piece is now $n^\frac{1}{8}$.
    \item We reduce the sequence $n \to \sqrt n \to n^{\frac{1}{4}} \to n^\frac{1}{8} \dots \to O(1)$.
        This can be done in $\log \log n$ steps!
\end{itemize}

\chapter{Parallel algorithms, part 2}

\section{Pointer jumping}
Pointer jumping is the technique of updating a successor with the
successor's successor. As this is repeated, the sucessor gets closer
to the root node. The distance between a node and its successor
doubles in each round trip.


\begin{minted}{python}
# F := Forest consisting of rooted, directed trees. F is specified using
# an array P

# P[i] := P[i] = j iff (i, j) is an edge in F. That is, j is a parent
# of i.

# P must contain self-loops at *each of the roots*. Each arc is
# specified by (i, P[i])

# output: a list S, containing the root of i at S[i]
def find_roots(P):
    for i in parallel([1, n]):
        S[i] = P[i]

        while S[i] != S[S[i]:
            S[i] = S[S[i]

    return S

\end{minted}

\section{List Ranking}

We have a list $L$ of $n$ nodes. $S[i]$ contains a pointer to the
node \textit{following} node $i$ on L, for $1 \leq i \leq n$. We assume
that $S(i) = 0$ when $i$ is the end of the list. The \textit{List-ranking problem}
is to determine the distance of each node $i$ from the end of the list.

\subsection{\textbf{non-optimal} list ranking using pointer jumping}

\begin{minted}{python}
def listrank(S):
    for i in parallel([1, n]):
        S[i] =  R[i] == 0 ? 0: 1


    for i in parallel([1, n]):
        Q[i] = S[i]
        while Q[i] != 0 && Q[Q[i]] != 0:
            R[i] = R[i] + R[Q[i]]
            Q[i] = Q[Q[i]]
\end{minted}

this takes time $O(\log n)$, using $O(n \log n)$ operations.

\subsection{Making our algorithm better}
We want to make our algorithm better, we have a work complexity of $O(\log n)$
which we are trying to eliminate.

There are also some implementation issues. In the PRAM model, syncrhonous execution
means that all $n$ processors execute each step in parallel. So, we can have
inconsistent results!

How do we pick a list of size $n / \log n$? Our input is in the form of an array
of successor elements. So, we can't take equi-distant parts of the array,
since it won't be a valid sub-list anymore.


What we can do is to pick \textit{independent nodes}. Formally, we want
to remove an independent set: vertices that share no edge amongst them.

\begin{minted}{python}
1 -> (8) -> 5 -> 11 -> (2) -> 6 -> (10) -> 4 -> 3 -> (7) -> 12 -> 9
on removal:
1 -> 5 -> 11 -> 6 -> 4 -> 3 -> 12 -> 9
\end{minted}
We can remove \texttt{8, 2, 10, 7} in parallel.

We want to go to a subset of size $n / \log n$, but by removing independent
nodes, we can go smallest to $n / 2$.

\begin{minted}{python}
a -> (b) -> c -> (d) -> e -> (f) -> ...
\end{minted}
There are no other elements in the above chain we can add to the independent set.
So, we will need to repeat our process to reach $n / \log n$.

\section{Detour: Independent sets}
In a graph $G = (V, E)$, a subset of nodes $U \subseteq V$ is called an
\textit{independent set} if:
$$U~\text{is an independent set of G} \equiv \forall (u_1, u_2) \in U, u_1 \neq u_2 \implies (u_1, u_2) \notin E$$.

Linked lists, when viewed as graphs, have large independent sets.

\subsection{Technique: Symmetry breaking}
The idea is to look at a symmetric setting, and then induce differences
between them. Independent sets are symmetric, because given two nodes
that are neighbours, they're both eligible to be in the independent set 
(modulo other obstructions). This algorithm is applicable for graph coloring.

Usually, this technique requires randomization. However, there are special
cases where fast, deterministic symmetry breaking is possible. Linked lists
and directed cyclic graphs are examples where this is possible.

We first construct a symmetry-breaking based graph coloring solution,
which is then used to find independent sets.

\subsection{Coloring by Symmetry breaking}
Considered a directed cycle of $n$ nodes $0 \dots n-1$.

Assume we have 8 nodes, which are labeled using 3 bits. We may not have
consecutive numbering of our nodes, so we assume that our nodes are randomly
numbered, from 0 to 7 (3 bits).


\begin{itemize}
    \item Initially, treat each number as a color for the vertex.
    \item We can reduce the number of colors to $\log n$ in one step:
    \begin{itemize}
    \item Compare color with the parent. $Newcolor(u) = 2 k + color(u)[k]$.
    \item $k$ is the index of the first bit position from LSB where $color(u)$ and $color(parent(u))$ differ.
    \item So, $color(u)[k]$ is indexing the k-th bit of $color(u)$ starting from LSB.
    \item note that $0 \leq k \leq \log n - 1$.
    \item such a $k$ will always exist, since we are guaranteed some unique
    labelling of the vertices when we start this process.
    \end{itemize}
\end{itemize}

\begin{minted}{python}
This table may not be fully accurate, re-check:

u   | v   | new color (mostly 2 bits)
110 | 000 | 11 (k = 1)
000 | 100 | 100 (k = 2)
100 | 111 | 00 (k = 0)
010 | 001 | 00 (k = 0)
001 | 011 | 10 (k = 1)
011 | 101 | 11 (k = 1)
111 | 010 | 01 (k = 0)
101 | 110 | 01 (k = 0)
\end{minted}

\subsubsection{Correctness proof}
Proof by contradiction.
Suppose we have an edge $(u, v)$, where $newcolor(u) = newcolor(v)$.
Let $newcolor(u) = 2k + color(u)[k]$, and $newcolor(v) = 2r + color(v)[r]$.

If $newcolor(u) = newcolor(v)$, then $2k + color(u)[k] = 2r + color(v)[r]$.
Rearranging, we get that $2(r - k) = color(u)[k] - color(v)[k]$.


If $k = r$, then we get that $color(u)[k] = color(v)[k]$. But this cannot
happen, because by definition, $k$ is the bit where $u, v$ first differ!


If $k \neq r$, then we get that $2(r - k) = color(u)[k] - color(v)[k]$.
By comparing magnitudes, we see that $\big|color(u)[k] - color(v)[k]\big| \leq 1$
(since we're subtracting bit values), while $\big|2(r - k)\big| \geq 2$. 
This can't happen either for two equal values!

\subsubsection{Analysing number of new colors}
In one iteration, we can reduce the number of colors from $n$ to $2 \log n$.
For the new colors, we only need $1 + ceil(\log \log n)$ bits.

\textbf{Can we repeat this technique? Yes, we can}. This technique reduces number
of colors from $t$ to $1 + ceil(\log t)$. At some point, $t < 1 + ceil(\log t)$,
at which point we will be forced to stop. 

This stopping point happens at $t = 3$. So, we repeat until only $8$ colors
are being used.

The total time is the solution to the recurrence $T(n) = T(\log n) + 1$.
We define the function that solves the recurrence as $\log^* n$.
$$\log^*n = i \equiv \log(\log(\dots \text{$i$ times} \dots (n))) = 1$$


\subsubsection{Reducing from $8$ to $3$ colors}
for $i$ in $[8..3]$, If node $u$ is colored $i$, then choose a color among
$\{1, 2, 3\}$ that is not the same as its neighbours.

\begin{minted}{python}
# color: map (vertex -> color)
# V: vertex set
for c in range(8, 3):
    for v in V:
        if color[v] == c:
            # we will always have one number here, since we have three 
            # colors, and we are only removing two colors
            newcolor[v] = rand ({1, 2, 3} - color[pred(v)] - color[succ(v)])
    newcolor = color
\end{minted}

This is always possible.


\subsection{Finding Indepenent sets using the coloring}
For bounded degree graphs colored with $O(1)$ colors, a coloring is equivalent
to finding a large independent set.

Iterate on each color and count the number of nodes with a given color.
Pick the subset of like colored nodes of the largest size. It is clearly
an independent set, and has size of at least some fraction of $n$.

\subsection{Algorithm outline}

\begin{minted}{python}
def rank(L):
    L1 = L

    for r in [2, R]:
        color the list with 3 colors
        pick an independent set U_i of nodes of size >= n /3
        L_i = remove nodes in U_i from L_{i - 1}

    Rank the List L_r using poiner jumping


    for i in [r, 1]:
        reinsert the nodes in U_i into L_i
\end{minted}

We are removing $n / 3$ nodes in each iteration, we want to stop at
$n / \log  n$ nodes. We need $O(\log \log n)$ iterations.


\subsection{total time taken}
Each iteration is $O(\log^* n)$. At $O(\log \log n)$ iterations, this takes
$O(\log^* n \log \log n)$ time.

To rank the remaining list, we take $O(\log n)$ time.

To reintroduce the removed elements, we take $r = O(\log \log n)$ iterations,
$O(\log \log n)$ time.


\subsection{Slowing down re-introduction to make this optimal}
We can reintroduce slower.

we can use only $n / \log n$ processor


\subsection{Slowing down independent set}


\section{Back to list ranking}
\begin{itemize}
    \item Anderson-Miller is in JaJa's book
    \item Hellman-JaJa is another popular approach (read the paper)
\end{itemize}

\chapter{Tree processing}
\section{Traversal via an Euler tour}
\begin{definition}
an \textbf{Euler tour} is a cycle of a graph that includes every edge of the
graph exactly once.
\end{definition}

\begin{lemma}
A directed graph $G$ \textbf{has an Euler tour} iff for every vertex,its in-degree
equals its out-degree.
\end{lemma}

For a tree $T = (V, E)$, to define an euler tour, we make it a directed graph.
$T_e = (V_e, E_e)$, where $V_e = V$, and $E_e = \cup_{(u,v) \in V} \{ (u, v), (v, u) \}$
That is, each $(u, v)$ in $E$ creates two edges $(u, v)$, and $(v, u)$ in $E_e$.
$T_e$ will have an Euler tour.


We have to define a successor function $s: E_e \to E_e$. Here, the successor for an edge.
For a node $u$ in $T_e$, order its \textbf{neighbours (both incoming and outgoing)}
$v_1, v_2, \dots v_d$. This can be done \textbf{independently at each node}. 
For $e = (v_i, u)$, set $s(e) = (u, v_{i + 1~\text{mod $d$}})$. This choice of $s$ is valid
since we always have both edges $(x, y)$ and $(y, x)$, and we are therefore
assured that $(v_i, u)$ will be an incoming edge, and $(u, v_{i + 1}$ will be
an outgoing edge. Also, compute $i + 1$ modulo $d$, so that we eventually
cycle.

\textbf{TODO: relabel vertices to $[0..(d - 1)]$ so that modulo works properly}
\textbf{TODO: add example}

\begin{theorem}
$s$ actually constructs a tour.
\end{theorem}
\begin{proof}
Induction on number of vertices. If $n = 1$, obviously true. 
If $n = 2$, at most one edge present. We will go along the edge and come back,
which is a valid tour.


\begin{itemize}
\item Let the tour be well defined for $n = k$. We will prove it for $n = k + 1$.
\item Every tree has at least one leaf, call it $l$. Create a tree $T' = T/\{l\}$.
\item Let $u$ be a neighbour of $l$ in $T$.
\item Let $N(u) = \{ v_0, v_1, \dots v_i = l, v_{i + 1}, \dots v_d \}$.
\item Set $s_{new}(u, v) \equiv (v, u)$. Set $s_{new}(v_{i - 1}, u) \equiv (u, v)$.
\item For all other vertices, $s_{new}(e) = s(e)$.
\end{itemize}
\end{proof}


\section{Using euler tours for traversal}
Operations on a tree such a rooting, preorder, and postorder traversal
can be converted to routines on an Euler tour.

\subsection{Rooting a tree}
Designate a node in a tree as the root. All edges in the tree are
directed towards (or away) from the root.

\begin{itemize}
\item let $\{v_1, v_2, \dots v_d\}$ be the neighbours of root node $r$. 
\item we mark the final edge of the tour as \texttt{NIL}, so we get an
Euler path, and not an Euler tour.
\item the edge $(r, v_i)$ appears before $(v_i, r)$.
\item so the edge $parent \to child$ appears before $child \to parent$
\item So, if $uv$ precedes $vu$, then set $u = parent(v)$. Orient the
edge $uv$ as $v \to u$ (that is, $child \to parent$), since we want all edges towards the root.
\end{itemize}

\subsection{Preorder traversal}
We have a rooted tree with $r$ as the root. In a preorder traversal, a node is
listed before any of the nodes in its subtrees.

In an Euler tour, nodes in a subtree are visited by entering subtrees,
and the exiting towards the parent.

If we can track the first occurence of a node in an euler path, this will
tell us the preorder traversal. Note that edges in the euler tour occur
first as $parent \to child$, and later as $child \to parent$. So, we can
look at the sequence of edges in the euler tour, and find the preorder
numbering.

\subsection{Expression tree evauation of binary trees}
Tree may not be balanced.

We use the \texttt{RAKE} technique to evaluate subexpressions. We rake the
leaves from the expression tree --- we remove the leaf node and its parent.

\begin{itemize}
\item $T = (V, E)$ is a tree rooted at root node $r$. $p: E \to E$ is the 
parent function.
\item One step of the rake operation at a leaf $l$ with $p(l) \neq r$ involves:
    \begin{itemize}
    \item Remove node $l$, $p(l)$ from the tree
    \item Make the sublings of $l$ as the child of $p(p(l))$. That is, graft
    the siblings of $l$ to the grandparent of $l$.
    \end{itemize}
\end{itemize}

Why is this a good technique? Can this be applied in parallel to several leaf
nodes? Yes, it can be applied to leaf nodes that don't share the same parent.
In general, there is a richness of leaf nodes in a tree, since there
are only $n - 1$ edges.

Each application of rake at all leaves reduces the  number of leaves by half.
Each application of \texttt{RAKE} is $O(1)$. So, total time is $O(\log n)$.


\begin{minted}{python}
def shrinkTree(R):
    compute labels for leaf nodes, store in array A (exclude leftmost
    and rightmost nodes in this A)

    for _ in range(k):
        apply rake operation to all odd numbered leaves that are
        the *left* children of their parent

        apply rake operation to all odd numbered leaves that are
        the *right* children of their parent

        update A to be the remaining even leaves
\end{minted}

Applying \texttt{Rake} means that we can process more than one leaf node
at the same time.

Fo expression evaluation, this may mean that an internal node with 
only one operand gets raked.

\begin{minted}{python}
          + g(u)
  
    + p(u) 

Y     X (u)


--After raking--

   + g(u)
Y
\end{minted}

\begin{itemize}
    \item Transfer the impact of applying the operaot at p(u) to the sibling of u
    \item $R_u = a_u X_u + b_u$
    \item $X_u$ is the result of the subexpression at node $u$ -- $X_u = f(left, right)$
    \item adjust $a_u$ and $b_u$ during any rake operation appropriately
    \item Initially, at each leaf node, $a_u = 1, b_u = 0$.
\end{itemize}



\begin{minted}{python}
          + g(u)
  
    + p(u) 
    X_w
    a_w
    b_w

v        (u) 5, 1, 0
X_v,
a_v,
b_v

--After raking--

     + g(u)
v
X_v'
a_v'
b_v'
\end{minted}

\begin{itemize}
    \item Before removing $p(u)$, the contribution of $p(u)$ to $g(u)$ will be $X_w a_w + b_w$.
    \item we want what $p(u)$ used to calculate to be what $v$ calculates after.
    \item $X_w = (X_u a_u + b_u) + (X_v a_v + b_v)= (X_v a_v) + (X_u a_u + b_u + b_v)$
    \item What $p(u)$ used to calculate is: $a_w X_w + b_w = a_w (a_v X_v + a_u x_u + b_u + b_v) + b_w = a_w a_v x_v + a_w (a_u X_u + b_u + b_v) + b_w$
    \item what $p(v)$ should be: $a_v' = a_w a_v$, $b_v' = a_w (\dots)$
\end{itemize}

For other operators, proceed in a similar fashion (\textbf{TODO: do this and send to kiko, he seems interested!})



\chapter{Questions}
$f (n) = f (n − 1) + 10$
$f (n) = f (n − 1) + n$
$f (n) = 2f (n − 1)$
$f (n) = f (n/2) + 10$
$f (n) = f (n/2) + n$
$f (n) = 2f (n/2) + n$
$f (n) = 3f (n/2)$
$f (n) = 2f (n/2) + O(n^2)$


\chapter{Review}
\section{PH}
\subsection{$\nptime = \conptime$ implies that PH collapses}
\subsection{$\text{\ppoly} \subset \nptime$ implies PH collapses to level 2}
intuitiion: ppoly's advice string lets us do something like quantifier
exchange, since it gives us an outer there exists for the advice string,
which can be guessed by and $NP^{NP}$ oracle. Also relies on the trick that
$\exists = \lnot \forall$, and you can not in this case due to being oracle call.
\section{BPP}
\subsection{BPP subset P/poly}
Intuition: can guess random seeds that are good for \emph{all} inputs.
\subsection{$BPP \subset \Sigma_p^2 \cap \Pi_p^2$}
Intutiion: can show we can find some $u_i$ to spread randomness around.
If we are on the accept side, the good seeds are large enough that we
can use randomness to cover the full set. If we are on the bad side, the
spreading with $u_i$ will not let us cover enough ground. So, we can 
convert it to $\exists u_i, \forall x_i, ...$ which is a PH style problem.

\end{document}
