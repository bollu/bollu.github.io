\newcommand{\ppoly}{\texttt{P/Poly}}
\newcommand{\bpp}{\texttt{BPP}}
\chapter{Other flavours of complexity classes:\bpp, \ppoly}

\section{Introduction}
\section{\ppoly}
If $\nptime \subset \ppoly$, then $\ph = \Sigma_2^P$.

\begin{theorem}
    Let $L = \{ 1^n ~\vert~ n \in \nats \land ~P(n) \}$, where $P: \nats \to \{0, 1\}$
    is a filtering predicate. All such languages $L \in \ppoly$.
\end{theorem}
\begin{proof}
    Check if $1^n \in_? L$. If $1^n \notin L$, Define $C_n \equiv \text{reject all inputs}$.
    If $1^n \in L$, define $C_n \equiv \text{accept all inputs}$.
\end{proof}

The problem is that unary languages of this kind can be \textit{uncomputable}.
\begin{align*}
    \texttt{UHALT} \equiv \{ 1^n~\vert~ \text{$n$ encodes $\langle M, w \rangle$, $M$ accepts} \}
\end{align*}

$\texttt{UHALT} \in \ppoly$ since it is a unary language, but unfortunately, this
language is undecidable.

The hardness is being pushed to the "construction of the circuit" --- this
feels pathological, so we may change this definition to allow us to
fix this.

\subsection{Karp-Lipton theorem}
\begin{theorem}
If $\nptime \subseteq \ppoly$, then $\ph = \Sigma^p_2$
\end{theorem}
\begin{proof}
    Recall that $\Sigma^p_2 = \Pi^p_2 \implies PH = \Sigma^p_2$. So, we
    choose to show that $\Sigma^p_2 = \Pi^p_2$.

    The $\Pi^p_2$-complete problem is this: $\forall u \exists v, \phi(u, v) =_? 1$,
    where $\phi$ is a boolean formula.

    We need to show that this problem is in $\Sigma_2^p$, if $\nptime \subseteq \ppoly$.

    If $\nptime \subseteq \ppoly$, then $\sat \in \ppoly$. So, given
    $\phi$, there exists a polynomial size Circuit family $\{ C_n \}$, 
    such that $C_n (\phi, u, v) \equiv \forall \phi, \forall u, \exists v, \phi(u, v) = 1$.

    assume we have a $\sat$ oracle. We can use it to find the satisfying assignment.
    Assume we have $\phi(x_1, x_2, \dots x_n)$. If it says "no", then we say "no".
    If it says "yes", then we need to find the satisfying assignment.
    Betweem $\phi(0, x_2, \dots, x_n)$, $\phi(1, x_2, \dots, x_n)$, one of
    these must be satisfiable (as $\phi$ is satisfiable). Do this $n$
    times, at which point we find the satisfying assignment.

    This entire process can be written as a family of boolean circuits,
    $\{ C_n' \}$, $C_n' (\phi, u) = v$.

    \begin{align*}
        \forall u \exists v, \phi(u, v) = 1 \iff \\
        \exists \{ C_n' \}, \forall u, \phi(u, C_n'(\phi, v)) = 1 \iff \\
        \exists w, \forall u, \phi(u, \langle \text{$w$ interpreted as a circuit} \rangle(\phi, v)) = 1 
\end{align*}

But the final statement is in $\Sigma^p_2$!
\end{proof}

\section{\bpp}

\subsection{\bpp = \ptime, the conjectured proof}
If true randomness can make our algorithm run fast, then a pseudorandomness
can also make our algorithm run fast, since \bpp is still polynomial,
but by definition, polynomial time adversary can't differentiate between
PRNG and a true RNG.

So, a deterministic turing machine can try to brute force all the seeds
of the PRNG.

However, for this, we need to know which seeds to look at. Somehow,
a coding theory idea called \textbf{list decoding} (ambiguity-allowed decoding)
will give us a polynomial number of seeds to try / brute force on.

We can brute force the list of seeds to try in poly-time, and we get poly
number of seeds, so we can "de-randomize" BPP into P.

Some of the steps in this sketch is conjectured (eg, existence of PRNG).  
For this to happen, we need some assumptions such as $\ptime \neq \nptime$,
(whose proof techniques we have been guaranteed by the end of the semester),
so $\bpp = \ptime$ is something we should try to solve.

\begin{theorem}
$\bpp \subset \ppoly$: Randomization does not help us go beyond circuit complexity
\end{theorem}
\begin{proof}
if $L \in \bpp$, with probability $\frac{2}{3}$ it will be correct. We can
use Chernoff bounds to repeat the experiment and take majority, given
polynomial number of rounds, we can take error down to \textbf{negligible (less
than any polynomial function)}.

$$
\forall w, P[M(w)=L(w)] \geq 1 - \frac{1}{2^{n + 1}}
$$

But for a length $n$, there are only $2^n$ strings.

We need to translate the definition of $\ppoly$ from circuit complexity, to
turing machines with advise.
\end{proof}
