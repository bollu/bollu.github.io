\chapter{NP}
\section{Cook Levin theorem}
\texttt{SAT} is \texttt{NP-hard}. 
\subsection{Proof}.
Unfold theorem statement into:
$\forall L \in \texttt{NP}, L \leq_p \texttt{SAT}$. 
Since this should work for all things in NP, let's just write down the definition:

there exists an \texttt{NDTM} $N$ such that $N$ accepts $w$, $\forall w
\in L$, in $|w|^k$ steps.

$N$ is an \texttt{NDTM}, so $N$ accepts $w$ means that there exists a branch
of N that acceeps $w$ is $|w|^k$ steps.

We should be able to construct a CNF such that $\phi(w)$ is \texttt{SAT} iff
there exists an accepting branch for $N(w)$.

\subsubsection{Caveats}

1. the construction of $\phi$ from $N$ should make sure that
$\phi$ has $poly(|w|)$ clauses --- otherwise, this is no longer a poly-time
reduction.  We know that $\langle \texttt{AND}, \texttt{OR}, \texttt{NOT} \rangle$ is
universal, so we can clearly construct any \texttt{TM} into a circuit. The problem
is that the CNF we construct from the truth-table of the \texttt{TM} will be
polynomial.

\textbf{Sid Q:} Proof that boolean circuits are universal?

\subsubsection{Proof sketch}
Consider the NDTM $N(n)$, we will now argue about its configuration.

We can cut off the turing tape after the first polynomial number of cells ---
since the \texttt{NDTM} can only access those many cells.

We should start will the initial state $q_{start}$.

We should get the accept state $q_{accept}$ in $n^k$ steps.

If we can pose this in terms of a CNF formula of poly-length, we are done.

\subsubsection{Setting up \texttt{SAT}}

\paragraph{\textbf{Variables - Cells of the tape}}
The state of the turing tape on the $i$th step at the $j$th position of the
turing tape for all $s \in alphabet(N)$ as $x_{i, j, s}$. $x_{i, j, s} = 1$
is interpreted as "at step $i$, on cell $j$, value $s$ is written.

For this to be valid, we need each cell to have exactly one symbol.


\paragraph{\textbf{Formula - Validity of cells}}
For every $(i, j)$ for at \textbf{least one} $s$ must be $1$:
$\phi_{cell least} = \bigwedge\limits_{i, j} (\lor_s x_{i, j, s})$


For every $(i, j)$ for at \textbf{most one} $s$ must be $1$. This is equivalent
to saying that for every $(s, t)$, one of them must be absent.
$\phi_{cell most} = \bigwedge\limits_{s, t, s \neq t} (\lnot{x_{i, j, s}} \lor \lnot{x_{i, j, t}})$.

$\phi_{cell} = \phi_{cell least} \land \phi_{cell most}$


\paragraph{\textbf{Formula - Initial state}}
The initial cells contain the correct letters, corresponding to the initial
input $w = \langle w_1 w_2 \dots w_n$, and the other cells must be blank.
$\phi_{init} = x_{1, 0, \#} \land x_{1, 0, q_{start}} \land (x_{1, 1, w_1} \land x{1, 2, w_2} \land \dots \land x_{1, n, w_n}) \land (x_{1, n + 1, blank} \land x_{1, n + 2, blank} \land \dots \land x_{1, n^k, blank})$

We use the $x_{\texttt{STATE}, \texttt{LOC}, \texttt{STATE ALPHABET}}$ to encode the current state
we are in. We adjoin this to the alphabet since we automatically get mutual
exclusion. The size of the \texttt{STATE ALPHABET} is constant, so it doesn't
matter. The \texttt{LOC} of this will correspond to the \textbf{location of the head}.
So, by placing the state symbol at a point, we understand that the head is at
($\texttt{LOC} + 1$) in the original tape.


\paragraph{\textbf{Formula - Final state}}
To encode a TM configuration, what we need to know is the state of the TM,
the position of the head, and the letters on the tape.

$\phi_{accept} = \lor_{i, k} x_{i, j, q_{accept}}$


\paragraph{\textbf{Formula - Transition}}
Every valid move in the TM can only touch two adjacent cells, since the memory
is not really 'random-access', it localizes computation.

So, every transition formula will view three adjacent cells at once. If we 
can view three are valid, we can "slide the window" to check the correctness
of all cells. 

$\phi_{move} = \bigwedge\limits_{\text{valid adjacent triplets}}$

To identify valid adjacent triplets, let the 
original config be \texttt{<a | b | c>}, 
let the new config be \texttt{<d | e | f>}.

The total number of legal windows will be a subset of $s^9$. So, for any
legal window, we can create a formula of some constant size.

$\phi_{move} = \bigwedge\limits{\text{legal window}}$.

The number of legal windows is $O(n^k)$, since we only have $n^k$ steps
and each legal window adds some constant number of formulas.



\paragraph{\textbf{Final $\phi$}}:
$\phi = \phi_{cell} \land \phi_{init} \land \phi_{accept} \land \phi_{move}$
If the NDTM accepts, then $\phi$ will be valid and vice versa.


\textbf{Sid Q:} Check that this reduction actually takes \textit{poly-time in n}! It
is clear to me that this inflates the size in poly, but it's not clear to me
that this takes only \textit{poly-time in n}.

Hence, SAT is \textbf{NP-hard} (well, NP-complete).




\section{3-SAT is NP-complete}
3-SAT: Every clause has 3 literals.

We reduce 4-SAT into 3-SAT, and then we show how to reduce \texttt{n-SAT} to \texttt{(n-1)-SAT}

\begin{align*}
    C_{\texttt{4-SAT}} &= x_1 \lor x_2 \lor x_3 \lor x_4 \\
    C'_{\texttt{3-SAT}} &= (x_1 \lor x_2 \lor z) \land (x_3 \lor x_4 \lor \lnot{z})
\end{align*}

If $C$ is true, We can always pick an assignment for $z$ to make the \textbf{other clause true}.
So, we can pick a correct $z$.

Similarly, if $C$ is false, $C'$ reduces to $C' = z \land \lnot{z}$. This 
is \texttt{UNSAT}, so this is not possible.

Now, in general, we reduce \texttt{n-SAT} into \texttt{(n-1)-SAT}.

\section{\texttt{CLIQUE} is NP-complete}
Does the graph contain a $k$ clique?

Proof.

Reduce 3-SAT to \texttt{CLIQUE}

$l$ clauses, $m$ variables.

For every clause and its complement, create a collection of vertices. Within
a triplet, there are no edges. All triplets are across. Connect everything
to everything, unless they are contradictory. 

The number of edges will be high.

Now, ask for an $l$ clique. If this exists, then we have $l$ assignments
which do not contradict each other.
