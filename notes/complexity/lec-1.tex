\chapter{Problems, Solutions, and Resources}

\section{Problems}
Alphabet set is finite, call it $\Sigma$. Strings must be finite length.

Given some input, and a computer that produces some output, the description
could be infinite --- both input and output.

However, the machine's \textit{description} (aka, the relationship between
input and output) must be finite.

So, the \textit{total input} can be infinite, but the input chunk must be
finite, and the response of the machine per \textit{input chunk} must be finite.

So, we can just use the language $L= \{0, 1\}$ for the machine.

Problems which have yes/no as answers are called decision problems.
Inputs are from $\Sigma^{*}$, outputs are from $\{0, 1\}$. The problem is a
mapping $f: \Sigma^{*} \to \{0, 1\}$. This is equivalent to providing the set
$\texttt{ACCEPT} \subset \Sigma^{*} = f^{-1}(1)$. Note that $\texttt{REJECT} =
\texttt{ACCEPT}^c$. The set $\texttt{ACCEPT}$ is called a language.

Now, we can study languages by looking at their grammars (welcome, Chomsky).

What about fractional bit problems? Is this useful? Could we exploit some
properties of fractional dimension?

\subsubsection{Cantor set}
take $S_0 = [0, 1]$ In each iteration, remove the middle one-third of each
continuous interval. Therefore,
\begin{itemize}
\item $S_0 = [0, 1]$
\item $S_1 = [0, \frac{1}{3}] \cup [\frac{2}{3}, 1]$
\end{itemize}

In $S_\infty$, \textit{uncountably infinite} points remain (However, this
set has \textit{measure $0$}).

So now, the question is, what is the dimension? We define Haussdorf dimension, 
and use this to exhibit fractional dimension of the Cantor set.

\texttt{TODO: fill this up!}

The total number of problems that can exist is $powerset(\Sigma^*)$.
$\texttt{RE}$ (recursively enumerable) Is a subset of $powerset(\Sigma^*)$
which computers can handle. The annoying thing is that there are \textit{finite
length problems} which computers cannot solve.

\subsection{Kannan}
If the universe is a machine, then it must have infinite description.

QM is the meeting point of universes?

\section{Solutions}

\subsection{Kannan}
\textbf{Question:} We study a lot of Science --- why? What is the ultimate goal
of science? Equivalently, what is the theory of everything we need to find
to halt on the journey of Science?

Assuming Science = God, we need to ask Science a question. Which language
will we use to query Science? Or, well, which language is \textit{enough} to
query Science? If the query alphabet is $\Sigma$, we can ask $\Sigma^*$ questions.
However, we can only reasonably pose questions of finite length (even though the
Science oracle can answer questions of infinite length).

In this case, have we achieved te ultimate goal of science?

\section{Resources}
