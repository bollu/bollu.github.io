\documentclass{article}
\usepackage{amsmath}
\usepackage{amssymb}
\begin{document}
% \newtheorem{example}

\section{What is a tenser?}
Let us define a tenser of rank $r \in \mathbb N$
to be the data:

\begin{itemize}
    \item A set of numbers $s_1, s_2, \dots s_r \in \mathbb N$, where
         $s_i$ is said to be the size of the tensor along dimension $i$.
    \item A function $F: [s_1] \times [s_2] \times \dots \times [s_r] \rightarrow
        (\mathbb R \rightarrow \mathbb R)$, where $[u] \equiv \{ 1, 2, \dots u\}$.
\end{itemize}

Given a tenser $T \equiv (r, S \equiv (s_1, s_2, \dots, s_r), F)$, we
denote by:

\begin{align*}
&T[ix_1, ix_2, \dots ix_r] : ([s_1] \times [s_2] \times \dots \times [s_r]) \rightarrow
    (\mathbb R \rightarrow \mathbb R) \\
&T[ix_1, ix_2, \dots ix_r] \equiv F(ix_1, ix_2, ix_3, \dots, ix_r)
\end{align*}

Let us now instantiate an honest to god tenser. We shall create what
plebes know as a "vector field" to be a tenser.

%\begin{example}
\subsection{Example: Vector field as a tenser}
Consider the vector field on $\mathbb R^2$ to be $V \equiv (sin(x), cos(y))$.

For us, this will correspond to a tenser of rank 1 $T \equiv (r \equiv 1, S \equiv (2), F)$ where:
$$
F(1) \equiv \lambda x. \sin x \quad F(2) \equiv \lambda x. \cos x
$$

\subsection{Example: Function as a tenser}
A function $f: \mathbb R \rightarrow \mathbb R$ is a tenser of rank 1
$T_f \equiv (r \equiv 1, S \equiv (1), F(1) \equiv f)$.
% \end{example}

\subsection{Example: Scalar as a tenser}
A scalar $r \in \mathbb R$ is a tenser of rank 1 $T \equiv (r\equiv 0, S\equiv(), F() \equiv \lambda \_. r)$

\subsection{Matrix field as a tenser}
The matrix field which maps each point $(x, y)$ to the matrix 
$
\begin{bmatrix}
x & 0 \\ 0 & y
\end{bmatrix}
$
is a rank 2 tenser:

\begin{align*}
&T \equiv (r\equiv 2, S \equiv (2, 2), F) \\
&F(1, 1) \equiv \lambda x. x \quad F(1, 2) \equiv \lambda \_. 0 \\
&F(2, 1) \equiv \lambda \_. 0 \quad F(2, 2) \equiv \lambda y. y \\
\end{align*}

\section{Tenser derivatives}

We are often interested in understanding how one tenser varies with respect
to another. But what does this question even mean? Well, I claim there is
only one sensible explanation. A tenser after all is just a collection of 
functions. So the derivative of one tenser with respect to another, say 
$\frac{\partial A[i_1, i_2, \dots i_n]}{\partial B[j_1, j_2, \dots j_m]}$ can
only be a new tenser  whose entries are the derivatives of \textit{each} function in $A$ with \textit{each}
function in $B$.

This instantly leads to the definition:

\begin{align*}
&A \equiv (n, S_A, F_a) \quad B \equiv (m, S_B, F_b) \\
&C \equiv (n+m, (S_A, S_B), F) \\
&F(i_1, i_2, \dots i_n, j_1, j_2, \dots j_m) \equiv 
    \frac{\partial A[i_1, i_2, \dots i_n]}{\partial B[j_1, j_2, \dots j_m]}
\end{align*}

Note that this is perfectly well defined, since
$A[i_1, i_2, \dots i_n]: \mathbb R \rightarrow \mathbb R$. Similarly,
$B[j_1, j_2, \dots j_m]: \mathbb R \rightarrow \mathbb R$, and we hopefully
know how to differentiate single variable functions.

The $C$ as written above is often colourfully written as:
\begin{align*}
&\frac{\partial A}{\partial B} \\
&\frac{\partial A[i_1, i_2, \dots i_n]}{\partial B[j_1, j_2, \dots j_m]}
\end{align*}

and many other abuses of notation. But never forget what it is doing: It is
simply creating a convenient way to consider the change of every component of $A$
relative to every component of $B$.


\section{The derivative $\frac{\partial x^T x}{\partial x}$}

Notice that if $x$ is an honest to god vector, the above expression makes no
sense. For example, let $x = (5, 5) \in \mathbb R^2$. Now, $x^T x = 50$,
leading to the absurd expression $\frac{\partial 50}{\partial (5, 5)}$,
which is quite senseless since differentiation is only defined for \emph{functions}.

Hence, whenever people write such expressions, they really mean a rank 1, shape (n) tenser.
that is, an $n$-tuple of scalar functions, each function describing the value
of the $n$th component of the vector, relative to some parametrization.


Let $x \equiv (r \equiv 1, S \equiv (2), F_x)$. Now, 
$x^T x \equiv (r\equiv 0, S\equiv(()), F_{xtx}\equiv F_x(1) F_x(1) + F_x(2)F_x(2) = F_x(1)^2 + F_x(2)^2)$

Let us now calculate $\frac{\partial x^Tx}{\partial x}$ as we have agreed upon above:

\begin{align*}
 &\frac{\partial x^Tx}{\partial x} (r \equiv 0+1, S \equiv 2, F_{der}) \\
 &F_{der}[][1] = \frac{\partial F_{xtx}}{\partial F_x(1)}  = \frac{\partial F_x(1)^2 + F_x(2)^2}{\partial F_x(1)}
    = 2 F_x(1) \\
 &F_{der}[][2] = \frac{\partial F_{xtx}}{\partial F_x(2)}  = \frac{\partial F_x(1)^2 + F_x(2)^2}{\partial F_x(2)}
    = 2 F_x(2) \\
\end{align*}

Hence, $\frac{\partial x^Tx}{\partial x} = 2 \cdot x$.


We can show that the collection of tensers form a vector space over $\mathbb R$,
since functions $\mathbb R \rightarrow \mathbb R$ form a vector space over $\mathbb R$,
and a tenser is a clever collection of such scalar functions. Hence, the
notation $2 \cdot x$ is interpreted in terms of this vector space structure.

\end{document}

# x^Tx := x_i x^i
# dx^Tx/dx := d{x_i x^i}{d x_j} = d{x_i x^i}{d x_i} . d{x_i}{x_j}
# dx^Tx/dx := d{x_i x^i}{d x_j} = \sum_i 2 x_i . δ(i,j)
                                = 2 x_j
