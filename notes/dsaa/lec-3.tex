%% What is the document class I need?
\documentclass{article} 
%% Some recommended packages.
\usepackage{booktabs}   %% For formal tables:
                        %% http://ctan.org/pkg/booktabs
\usepackage{subcaption} %% For complex figures with subfigures/subcaptions
                        %% http://ctan.org/pkg/subcaption
\usepackage{enumitem}
\usepackage{minted}
\newminted{fortran}{fontsize=\footnotesize}

\usepackage{xargs}
\usepackage[colorinlistoftodos,prependcaption,textsize=tiny]{todonotes}


\begin{document}
\section{Lecture 3 - Systems}

x[n] --> H --> y[n]


y[n] = \frac{1}{2p + 1}\sum_{k=-p}^{p} x[n - k].

\subsection{Causal / Non Causal}

\subsubsection{Causal system}

y[n] = f(x[k1], .. x[kn]) where k_i <= n.

System only has access to past and current state.

\subsubsection{Non Causal}
A system that is not causal is a non-causal system. Can access the future.


\subsection{Static / Dynamic}
\subsubsection{Static}

y[n] = f(x[n])

No access to past values.

\subsubsection{Dynamic}

Has access to past or future

y[n] = f(..., x[n+1], x[n], x[n - 1], ...)



\subsection{Linear / non-linear}
\subsubsection{Linear}
\subsubsubsection{Additivity}
x1[n] + x2[n] --H--> y1[n] + y2[n]

\subsubsubsection{Homogenity}
A x1[n] ---H---> A y1[n]

y[n] = x[n] + a


\subsection{Time Invariance}

Define H(x[n]) = y[n].
If H(x[n - n0]) = y[n - n0], then this is time-invariant

I need to formalize this in a much saner way, this is nuts. We're implicitly
talking about function spaces here, and this notation is terrible.

TODO: formalize this in Coq? this is neat.

If this system is followed, the system is said to be \textbb{time invariant}.

The right way to formalize is this, I think

```
H :: (Ix -> Value) -> (Ix -> Value)
H(x) = y

Time invariance:
Shifting input is the same as shifting output.

forall x0 \in Ix, H(\n -> x(n - n0)) = \n -> H(x)(n - n0)
```

\subsubsection{Example 1}
H(x[n]) = x[n] + x[n - 1] = y[n]
H(x[n - n0]) = x[n - n0] + x[n - 1 - n0] = y[n - n0]

\subsubsection{Non-Example 1}
y[n] = x[n^2]
1. y[n - n0] = x[(n - n0)^2]

2. Z[n] = x[n - n0]
   output = Z[n^2] = x[n^2 - n0]

\section{LTI systems - Linear Time invariant systems}
\subsection{Impulse response}

$H(\delta(n))$ is defined as impulse response.

The impulse response of an LTI system completely characterizes the LTI system.

Proof:

Let H be an LTI system.

$x[n] = \sum_{k=-\infty}^{\infty} x[k] \cdot \delta[n - k]$.
$H(x[n]) = \sum_{k=-\infty}^{\infty} H(x[k] \cdot \delta[n - k])$.

This is by linearity, use linearity of $H$ to push through $\Sigma$.


$H(x[n]) = \sum_{k=-\infty}^{\infty} x[k] * H(\cdot \delta[n - k])$.

This is by linearity, $x[k]$ is a scalar.


$H(x[n]) = \sum_{k=-\infty}^{\infty} x[k] * h[n - k]$.
where $h[n] = H(delta[n])$.  This is by time invariance, allowing us to "push" the
$n - k$ factor through $H$.




