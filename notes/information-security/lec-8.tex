\section{Information Theory - Lecture 8: CPA security}

Adversary has oracle access to the encryption machine, still can't decrypt it.

However, CPA secure is not really enough.


If we have a key scheme of the form $<r, f_k(r) XOR m>$, perhaps the adversary will
XOR m with $f_k(r)$, to make the key scheme $<r, flipMSB(f_k(r) XOR m)>$.



Now, we allow the adversary access to the decryptor as well.

CCA secure := chosen cyphertext attack.


\section{Data integrity}

\texttt{Sender(Alice) ---> Receiver(Bob)}

Both have a secret key.

Alice has a message M which she sends to bob.

Bob will receive M' that could be tampered. Bob should be able to tell
if $M' =? M$.

This is kind of impossible. If Bob could actually tell the difference,
then there is no need to transmit the message.


We want a MAC algorithm (message authentication code)


$<Gen, MAC, Verify>$

Verify(M, tag) returns Valid or Invalid
4
when tag is generated by $MAC_key(m)$, verifying algorithm should generate true.


CBCMAC
