\documentclass[11pt]{book}
%\documentclass[10pt]{llncs}
%\usepackage{llncsdoc}
\usepackage[sc,osf]{mathpazo}   % With old-style figures and real smallcaps.
\linespread{1.025}              % Palatino leads a little more leading
% Euler for math and numbers
\usepackage[euler-digits,small]{eulervm}
\usepackage{physics}
\usepackage{amsmath,amssymb}
\usepackage{graphicx}
\usepackage{makeidx}
\usepackage{algpseudocode}
\usepackage{algorithm}
\usepackage{listing}
\usepackage{minted}
\usepackage{tikz}
\evensidemargin=0.20in
\oddsidemargin=0.20in
\topmargin=0.2in
%\headheight=0.0in
%\headsep=0.0in
%\setlength{\parskip}{0mm}
%\setlength{\parindent}{4mm}
\setlength{\textwidth}{6.4in}
\setlength{\textheight}{8.5in}
%\leftmargin -2in
%\setlength{\rightmargin}{-2in}
%\usepackage{epsf}
%\usepackage{url}

\usepackage{booktabs}   %% For formal tables:
                        %% http://ctan.org/pkg/booktabs
\usepackage{subcaption} %% For complex figures with subfigures/subcaptions
                        %% http://ctan.org/pkg/subcaption
\usepackage{enumitem}
%\usepackage{minted}
%\newminted{fortran}{fontsize=\footnotesize}

\usepackage{xargs}
\usepackage[colorinlistoftodos,prependcaption,textsize=tiny]{todonotes}

\usepackage{hyperref}
\hypersetup{
    colorlinks,
    citecolor=blue,
    filecolor=blue,
    linkcolor=blue,
    urlcolor=blue
}

\usepackage{epsfig}
\usepackage{tabularx}
\usepackage{latexsym}
\newcommand\ddfrac[2]{\frac{\displaystyle #1}{\displaystyle #2}}
\newcommand{\E}[1]{\ensuremath{\mathbb{E} \left[ #1 \right]}}
\renewcommand{\P}[1]{\ensuremath{\mathbb{P} \left[ #1 \right]}}
\newcommand{\N}{\ensuremath{\mathbb{N}}}
\newcommand{\R}{\ensuremath{\mathbb R}}
\newcommand{\Rgezero}{\ensuremath{\mathbb R_{\geq 0}}}
\newcommand{\powerset}{\ensuremath{\mathcal{P}}}

\newcommand{\boldX}{\ensuremath{\mathbf{X}}}
\newcommand{\boldY}{\ensuremath{\mathbf{Y}}}


\newcommand{\G}{\ensuremath{\mathcal{G}}}
% \newcommand{\braket}[2]{\ensuremath{\left\langle #1 \vert #2 \right\rangle}}


\def\qed{$\Box$}
\newtheorem{question}{Question}
\newtheorem{theorem}{Theorem}
\newtheorem{definition}{Definition}
\newtheorem{lemma}{Lemma}
\newtheorem{observation}{Observation}
\newtheorem{proof}{Proof}
\newtheorem{remark}{Remark}
\newtheorem{example}{Example}

\title{Probabilistic graphical models, Assignment 1}
\author{Siddharth Bhat (20161105)}
\date{Jan 17th, 2020}

\begin{document}
We will denote the set $\{1, 2, \dots n\}$ as $[n]$.


\begin{question}
Prove that $\max_{A \subseteq [n]} |p(A) - q(A)| = \frac{\sum_{i=1}^n  |p(i) - q(i)|}{2}$
\end{question}

$$
|P(A) - Q(A)| \equiv \left|\sum_{a \in A} p(a) - q(a)\right|
$$

Note that to maximise the above quantity, we can choose to maximise either
positive values or negative values, since it is surrounded by  $|~|$. Let us
arbitrarily choose to maximise positive values (the solution is symmetric).

In that case, we need to ensure that we pick $a_0 \in [n]$ such that $[P(a_0) - Q(a_0) > 0]$.
This forces us to pick $A \equiv \{ a \in [n]~:~ P(a)-Q(a) > 0 \}$.
Now, define $\bar A \equiv \{ a \in [n]~:~ P(a) - Q(a) \leq 0 \}$. That is,
$A \cap \bar A = \emptyset$, $A \cup \bar A = [n]$.

Recall that $\sum_a p(a) = 1$, $\sum_a q(a) = 1$.

\begin{align*}
&\sum_{i=1}^n|p(i) - q(i)| \\
&= \sum_{a \in A} |p(a) - q(a)| + \sum_{\bar a \in \bar A} |p(\bar a) - q(\bar a)| \\
&= \sum_{a \in A} (p(a) - q(a)) + \sum_{\bar a \in \bar A}  (q(\bar a) - p(\bar a)
\end{align*}

\begin{question}
    Prove that on a finite DAG, at least one vertex has no incoming edges.
\end{question}

\emph{Proof sketch:} Build a chain by picking a vertex $v_0$. If this vertex
has no incoming edges, then we are done. If not, pick a predecessor $v_1$
such that $v_1 \rightarrow v_0$. Now, attempt to pick a predecessor of $v_1$.
If it has no predecessor, we are done. If not, we must have a new vertex $v_2$.
The crucial point is that this $v_2$ is \emph{not equal to $v_0, v_1$}. For if
it were, we would have a cycle of the form $(v_2 \rightarrow v_1 \rightarrow v_0 \rightarrow v_2)$.
or of the form $(v_2 \rightarrow v_1 \rightarrow v_2)$.

Hence, we have a \emph{decreasing measure}, the number of available vertices to
extend the chain is the number of vertices in the graph minus the length
of the chain. This is a finite number, and cannot be less than 0. Thus this
process must terminate, yielding a final vertex at some point that has no
predecessor.

\begin{question}
    Consider the three variable distribution $P(a, b, c) = P(a|b)P(b|c)P(c)$
    where all variables are binary. how many parameters are needed to specify
    a distribution of this form?
\end{question}
For each $c=0, c=1$, we need the values of $P(b=1|c=0), P(b=1|c=1)$ which is
2 parameters. The other two $P(b=0|c=0), P(b=0|c=1)$ can be calculated.


For each $b=0, b=1$, we need the values of $P(a=1|b=0), P(a=1|b=1)$ which is
2 parameters.

So, in total, we need $2+2+1 =5$ parameters.
\end{document}
