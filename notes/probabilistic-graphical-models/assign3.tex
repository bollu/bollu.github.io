% Section 3.5 (Q1 and Q2)
% Section 4.5 (Q1)
% Section 6.8 (Q1)


\documentclass[11pt]{article}
%\documentclass[10pt]{llncs}
%\usepackage{llncsdoc}
\linespread{1.025}              % Palatino leads a little more leading
\usepackage{physics}
\usepackage{amsmath,amssymb}
\usepackage{graphicx}
\usepackage{makeidx}
\usepackage{algpseudocode}
\usepackage{algorithm}
\usepackage{listing}
\usepackage{minted}
\usepackage{tikz}
\usepackage{bbold}
\evensidemargin=0.20in
\oddsidemargin=0.20in
\topmargin=0.2in
%\headheight=0.0in
%\headsep=0.0in
%\setlength{\parskip}{0mm}
%\setlength{\parindent}{4mm}
\setlength{\textwidth}{6.4in}
\setlength{\textheight}{8.5in}
%\leftmargin -2in
%\setlength{\rightmargin}{-2in}
%\usepackage{epsf}
%\usepackage{url}

\usepackage{booktabs}   %% For formal tables:
                        %% http://ctan.org/pkg/booktabs
\usepackage{subcaption} %% For complex figures with subfigures/subcaptions
                        %% http://ctan.org/pkg/subcaption
\usepackage{enumitem}
%\usepackage{minted}
%\newminted{fortran}{fontsize=\footnotesize}

\usepackage{xargs}
\usepackage[colorinlistoftodos,prependcaption,textsize=tiny]{todonotes}

\usepackage{hyperref}
\hypersetup{
    colorlinks,
    citecolor=blue,
    filecolor=blue,
    linkcolor=blue,
    urlcolor=blue
}

\usepackage{epsfig}
\usepackage{tabularx}
\usepackage{latexsym}
\newcommand\ddfrac[2]{\frac{\displaystyle #1}{\displaystyle #2}}
\newcommand{\E}[1]{\ensuremath{\mathbb{E} \left[ #1 \right]}}
\renewcommand{\P}[1]{\ensuremath{\mathbb{P} \left[ #1 \right]}}
\newcommand{\N}{\ensuremath{\mathbb{N}}}
\newcommand{\R}{\ensuremath{\mathbb R}}

\newcommand{\Rgezero}{\ensuremath{\mathbb R_{\geq 0}}}
\newcommand{\powerset}{\ensuremath{\mathcal{P}}}

\newcommand{\boldX}{\ensuremath{\mathbf{X}}}
\newcommand{\boldY}{\ensuremath{\mathbf{Y}}}


\newcommand{\G}{\ensuremath{\mathcal{G}}}
\newcommand{\D}{\ensuremath{\mathcal{D}}}
\renewcommand{\H}{\ensuremath{\mathcal{H}}}
\newcommand{\X}{\ensuremath{\mathcal{X}}}
\DeclareMathOperator{\vcdim}{VCdim}
\newcommand{\Vcdim}{\ensuremath{\vcdim}}
\newcommand{\VCdim}{\ensuremath{\vcdim}}
\newcommand{\VCDim}{\ensuremath{\vcdim}}
\newcommand{\vc}{\ensuremath{\vcdim}}
\newcommand{\VC}{\ensuremath{\vcdim}}
\newcommand{\zoset}{\ensuremath{\{0, 1\}}} % set \{0, 1\}

\def\qed{$\Box$}
\newtheorem{question}{Question}
\newtheorem{theorem}{Theorem}
\newtheorem{definition}{Definition}
\newtheorem{lemma}{Lemma}
\newtheorem{observation}{Observation}
\newtheorem{proof}{Proof}
\newtheorem{remark}{Remark}
\newtheorem{example}{Example}

\title{Probabilistic graphical models, Assignment 3}
\author{Siddharth Bhat (20161105)}
\date{March 21st, 2020}

% Assignment for VC Dimension:
% Understanding ML Textbook
% 
% Section 6.8 (Exercises of chapter 6): Problems 1, 2, 3, 5, 9
% Submit solutions to any 4 of 5 problems.


\begin{document}
\maketitle
\section*{6.8, Q1:}
Monotonicity of VC dimension

Let $\H' \subseteq \H$.  Show that $\vc(\H') \leq \vc(\H)$.

\subsubsection*{Answer}
Recall that the definition of \vc is is that \vc(\H) is the maximal size of
a set $C \subseteq \X$ which can be \emph{shattered} by \H.

Expanding the definition of shattering, we get that the \vc(\H) is the maximal size
of \emph{any} set $C \subseteq X$ such that \H~restricted to $C$ is the set of all
functions from $C$ to \{0, 1\}.

Now, If $C \subseteq \X$ is shattered by $\H' \subseteq \H$, then this means
that:

\[
|\{ f|_C : f \in H' \}| = 2^{|C|}
\]

Since $\H' \subseteq \H$, we can
replace $\H'$ with $\H$ in the above formula to arrive at:

\[
|\{ f|_C : f \in H \}| = 2^{|C|}
\]

So, clearly, $\vc(\H') \leq \vc(\H)$.
However, there might be a set that is \emph{larger} than $C$ that can be shattered
by $\H$. This lets us get the strict equality $\vc(\H) < \vc(\H)$ in certain cases
--- that is, we \emph{cannot} assert that $\vc(H) \leq \vc(H')$.
For example, if we choose $\H' = \emptyset$ where $\H$ is a hypothesis class with
$\vcdim(\H) = 1$. Then $\vcdim(\emptyset) = 0 < 1 = \vcdim(\H)$. 


\section*{6.8, Q2:}
Given a finite domain $\X$, and a finite number $k \leq |\X|$s, find and prove
the VC dimension of:

\subsection*{A. Functions that assign 1 to exactly $k$ elements of \X}
$\H \equiv \bigg\{ h \in \zoset^{X} : ~|\{x : h(x) = 1\}| = k \bigg\}$.

\subsubsection*{Solution}
There are three cases which we will prove:

\begin{align*}
    \Vcdim(\H) = 
    \begin{cases}
        0 & k = 0 \\
        0 & k > 0, |\X| < 2k \\
        k & k > 0, |\X| \geq 2k
    \end{cases}
\end{align*}


\textbf{Case 1. $k = 0$} 
When $k=0$, we can express only a single function:
$zero: \X \rightarrow \{0, 1\}; zero(\_) = 0$. Hence, we can only represent
the empty set. So, we can shatter no set that is larger than the empty set.
Therefore, $\vc(\H) = 0$ \qed.

\textbf{Case 2. $k > 0 \land |\X| < 2k$} 
If $k \neq 0$, Then let $S \subseteq \X$ such that $S$ is shattered by $\H$.
However, note that we must be able to express the function $f: S \rightarrow \{0, 1\}; f(\_) = 0$
using the hypothesis $h_S \in \H$. However, $h_S$ must have $k \neq 0$ entries of $1$,
which $f$ does not have. Therefore, $h_S$ could not have shattered $S$. \qed.


\textbf{Case 3. $k > 0 \land |\X| \geq 2k$} 
Let us say we are trying to shatter $S \subseteq X, |S| = k$. In this case,
for any given subset $T \subseteq S$, $T$ will have at most $k$ elements. We will
create a numbering called $count: X / S \rightarrow \N$, which enumerates elements
of $X/S$ in some arbitrary order. Note that the \emph{image} of $count$ is
guaranteed to have \emph{at least $k$ elements} --- this will be important for
our correctness proof. We will now build a hypothesis $h_T: \X \rightarrow \{0, 1\}$
which classifies this subset $T$ correctly:

\begin{align*}
    h_T(x) \equiv 
    \begin{cases}
        1 & x \in T \\
        1 & x \in X / S \land count(x) \leq k - |T|  \\
        0 & \text{otherwise}
    \end{cases}
\end{align*}

This function $h_T$ clearly assigns values correctly to elemets $x \in S$.
For elements outside of $S$, it arbitrarily marks $k - |T|$ of them $1$, to
comply with the requirement that $h \in \H$ must have $k$ 1s. For this
to work out, we need to have enough elements in $X / S$. In the worst case
when $T = \emptyset$, we will need $k - |\emptyset| = k$ elements in $X / S$.

\subsection*{B. Functions that assign 1 to at most $k$ elements of \X}
\subsubsection*{Solution}
In this case, the VC dimension is $\max(|X|, k)$ since we can 
only represent subsets of size upto $k$.
VC dimension is $k$.


\section*{6.8, Q3:}
Let $\X$ be the boolean hypercube $\{0, 1\}^n$. We define parity to be:
$$h_I: \X \rightarrow \{0, 1\}; h_I((x_1, x_2, \dots, x_n)) \equiv \sum_{i \in I} (x_i) ~ \mod 2.$$

What is the VC dimensions of the set of all parity functions? That is,
$$\H_{parity, n} \equiv \{ h_I : I \subseteq \{1, 2, \dots n\} \}$$

\subsubsection*{Solution}

Once again, unwrapping the definition, our hypothesis class can compute the
sum modulo 2 of \emph{all of the subsets of $\vec x \in \X$}. We need to
use this to find the \emph{largest} set $C \subseteq \X \equiv \{0, 1\}^n$ such
that $|\H_C| = 2^{|C|}$. 

We can interpret elements $(h \in H)$ as a vector 
$h_I \in \{0, 1\}^n$, where $h_I$ is a vector with $1$'s at each index $i \in I$,
and $0$ at other indexes. That is:

$$
h_I \in \{0, 1\}^n \qquad
h_I[i] \equiv \mathbb{1}[i \in I] = 
\begin{cases} 1 & i \in I \\ 0 & \text{otherwise} \end{cases}
$$


We can reinterpret the function $h_I(x)$ as $h^T x$ where we have
a vector space over the galois field $GF_2$, where $\oplus$ denotes XOR (recall
that addition mod 2 is XOR).

\begin{align*}
h_I(x) = \bigoplus_{i \in I} x_i = \bigoplus_{i=1}^n \mathbb{1}[i \in I] x_i = \bigoplus_{i=1}^n h_I[i] x[i] =  h_I^T x
\end{align*}


Now, we can reinterpret the question of finding the VC dimension as finding
the largest collection of vectors $C \subseteq \X = \{0, 1\}^n$ such that
the function $C_{act}$ has full image, where the function $C_{act}$ is:

\begin{align*}
&C_{act}: \H \rightarrow \{0, 1\}^{|C|} \\
&C_{act}: \{0, 1\}^n \rightarrow \{0, 1\}^|C| \\
&C_{act}(h) \equiv (h(c_0), h(c_1), h(c_2), \dots h(c_n)) \\
&\quad = (h^T c_0, h^T c_1, \dots, h^T c_n) \\
& \quad = h^T (c_0, c_1, \dots c_n)
\end{align*}

If we regard $(c_0, c_1, c_2, \dots c_n) \subseteq \mathbb R^{n \times |C|}$
as a matrix, then we can see that  $C_{act}$ is a \emph{linear function}.


Now, if the set $C$ shatters $\H$, then:
\begin{itemize}
\item[1] The function $C_{act}$ will produce every element in $\{0, 1\}^{|C|}$
\item[2] the function $C_{act}$ will have full image. 
\item[3] This is  only possible when the dimension of the domain is less than or equal to the dimension of the range.
\item[4] the largest set that can be shattered is the largest matrix  $C \subseteq R^{n \times |C|}$  
         such that the function $C_{act}$ has full range.
\item[5] Thus, $|C| \leq n$ for $C_{act}$ to have full range.
\item[6] We can achieve $|C| = n$ by picking $C = I_{n \times n}$. In other words,
          the element $c_i$ will be the $i$th row of the identity matrix. That
          is $c_i[j] = \mathbb{1}[i = j] = \begin{cases} 1 & \text{i = j} \\ 0 & \text{otherwise} \end{cases}$.
         Clearly, this $C$ is shattered since the function $C_{act}$ is the identity function which
         will produce every single output in $\{0, 1\}^{|C|} = \{0, 1\}^n = \{0, 1\}^{|\H|}$.         
\item[7] $|C|=n$ is the largest possible, since $C_{act}$ is a function, and
         the size of its image is at most the size of the domain. Since the domain $\H$
         has $2^n$ elements, the image too can have at most $2^n$ elements, which
         it does when $|C| = n$, since $|\{0, 1\}^{|C|}| = 2^{|C|}$.
\item[7.5] $|C|=n$ is the largest possible, since $C_{act}$ is linear.
            For a linear function to be surjective, we need $Dim(domain)  Dim(range)$.
            Hence, $Dim(Domain) = Dim(\H) = n \geq Dim(range) = Dim(\{0, 1\}^{|C|}) = c$.
            That is, $n \geq |C|$.
\end{itemize}

Hence, we conclude that $\vc(\H) = n$.

\section*{6.8, Q5:}
Let $\H^d$ be the class of axis-aligned bounding boxes in $\R^d$. Show that
the VC dimensions of $\H^d$ is $2d$.


\subsubsection*{Solution}
Formally, we have 
$$
h_{\vec l, \vec r}(\vec p) \equiv
\begin{cases}
1 &  l[i] \leq p[i] \leq r[i]~\text{for all $i \in \{1, 2, \dots, n \}$ } \\
0 & \text{otherwise}
\end{cases} \qquad
\H^d \equiv \left\{ h_{\vec l, \vec r}: \R^d \rightarrow \{0, 1\} \mid \forall \vec l, \vec r \in \mathbb R^d \right\}
$$

We claim that the set of points:

\begin{align*}
S &\equiv S^+ \cup S^- \\
S^+ &\equiv \{ p[i] = 2; p[j \neq i]  = 0 : i \in [d], p \in \R^d \} \\
S^- &\equiv \{ p[i] = -2; p[j \neq i] = 0 : i \in [d], p \in \R^d  \} \\
\end{align*}

shatters the hypothesis class $\H^d$.

\textbf{$H^d$ shatters $S$:}
We first show that $H^d$ restricted to $S$ expresses all functions $S \rightarrow \{0, 1\}$.
Note that the set $S$ has $2d$ points. We will consider the $d$ subspaces,
indexed by $i \in \{1, \dots \}$. We will build some notation for this
consideration:

\begin{align*}
&S[i, pos] \in S; \quad S[i, pos] \equiv p[i] = 2; p[j \neq i] = 0 : p \in \R^d  \\
&S[i, neg] \in S; \quad S[i, neg] \equiv p[-i] = -2; p[j \neq i] = 0 : p \in \R^d \\
&S[i, :] \subseteq S; \quad S[i, :] \equiv \{ S[i, pos] \cup S[i, neg] \}
\end{align*}


For all classifications $c: S \rightarrow \{0, 1\}$, we will show how to build a
hypothesis $BB_v \in \H^d$ such that $BB_c(s) = c(s) \forall s \in S$.
For each subset $S[i, :]$, we will build up a bounding box in $BB_c^i \in \H^d$.
We will then show that the convex hull of
all bounding boxes $conv(BB_c^1, \dots, BB_c^n)$ is the $BB_c$ we are looking for:
$BB_c = conv(BB_c^1, \dots, BB_c^n)$.

We first describe the $BB_c^i$:

$$
BB_c^i : \H^d \equiv conv\bigg(c(S[i, neg]) * S[i, neg]), c(S[i, pos]) * S[i, pos]\bigg) 
$$

That is, we try to cover the points which have $c(S[i]) = 1$ with a
convex hull. If a point is
not covered, then we will use $0 \times \vec p = \vec 0$. If a point is
indeed covered, then we use $1 \times \vec p = \vec p$ (the point itself). We then
take the convex hull of these. Thus, $BB_v^i$ only contains those
points in $S[i, :]$ that need to be covered.

\textbf{Claim 1: each $BB_c^i$ classifies $S[i, :]$ according to c} immediate from construction. \\
\textbf{Claim 2: The convex hull of correct classifiers of $BB_v^i$ classifies $S$:}
Consider some point $S[i, pos]$ (a similar argument will hold for $S[i, neg]$).
We know from Claim 1 that $BB_c^i$ correctly classifies $S[i, pos]$. The
other classifiers $BB_c^j$ cannot influence what happens in the $i$ dimension,
since they only attempt to cover the value $0$ along dimension $i$.
It is only $BB_c^i$ that can "expand" the cover in dimension $i$ to cover
$S[i, pos]$. Hence, the full convex hull will indeed cover the points of
interest.


\textbf{$H^d$ cannot shatter 2d+1:}

We are given a set $S \subseteq X$.$|S| = 2d+1$ into $S[i, pos]$,
$S[i, neg]$, $S[\star]$. We define:

\begin{align*}
S[i, pos] &\equiv \arg \max_{p \in S} p[i] \quad \text{(point with max. value in $i$ dimension)}\\
S[i, neg] &\equiv \arg \min_{p \in S} p[i] \quad \text{(point with min. value in $i$ dimension)}\\
S[\star]  &\equiv s \in S, s \neq S[i, pos], s \neq S[i, neg]~\text{for all $i \in \{1, \dots, n\}$} \quad \text{(leftover point)}
\end{align*}

Note that $S[\star]$ will always exist, since there are $2d$ points that we get
from all the $S[i, pos], S[i, neg]$, while $|S| = 2d+1$.  We note that
the points $S[i, pos], S[i, neg]$ together form the vertices of a bounding
box for $S$.

Now, let $f_S: S \rightarrow \{0, 1\}$ be a classifier such that:

$$
f: S \rightarrow \{0, 1\} \\
f_S(s) \equiv
\begin{cases}
1 & s = S[\star] \\
0 & \text{otherwise}
\end{cases}
$$

Since our hypothesis class consists of bounding boxes, for any $h\in \H$, if
the value of $h$ on the vertices of the bounding box must be the same
as the interior. However, $f_S(vertices) = 1$, while $f_S(interior) = 0$.
Hence, such an $f$ cannot be realised by any $h \in \H$. 

Therefore, no set of size $2d+1$ can be shattered.


\section*{6.8, Q9:}
Let $\H_{si}$ ($si$ for signed interval) be the class of signed intervals.
That is: $\H \equiv \{ h_{a, b, s} : a \leq b, s = \pm 1 \}$
where 
$$
h_{a, b, s}(x) \equiv
\begin{cases}
    s & a \leq x \leq b \\
    -s & \text{otherwise}
\end{cases}
$$

\subsubsection*{Solution}

We will first show that $\vc(\H_{si})$ is 3 by exhaustive enumeration. We will
then show a slicker method, by proving that if a hypothesis space $\H$
has $\vcdim(\H) = n$, then the VC dimension of the space that is $\H' \equiv \H \times {0, 1}$
where the $\{0, 1\}$ controls whether we should negate the output of $h \in \H$
will have $\vcdim(\H') = \H+1$. Now, clearly the above hypothesis class $H_{si}$
is $\H_{interval} \times{0, 1}$. We know that $\vc(H_{interval}) = 2$, and
hence $\vc(H_{si}) = 3$. \\

\textbf{Exhaustive enumeration}

Let us consider all possibilities for three points $\{ 1, 3, 5 \}$. We will 
write down for each subset the classifier to be used, thereby showing that
this set is shattered. For a subset, we will need to pick a classifier that
has value $+1$ on elements $s \in S$, and has value $-1$ on elements $s' \not \in S$.

\begin{align*}
    &\emptyset \mapsto h_{0, 0, 1} \\
    &\{ 1 \} \mapsto h_{0, 2, 1} \quad
    \{ 3 \} \mapsto h_{2, 4, 1} \quad
    \{ 5 \} \mapsto h_{4, 6, 1} \\
    &\{ 1, 3 \} \mapsto h_{0, 4, 1} \quad
    \{ 3, 5 \} \mapsto h_{2, 6, 1} \quad
    \{ 1, 5 \} \mapsto h_{2, 3, -1} \\
    &\{1, 3, 5\} \mapsto h_{0, 6, 1}
\end{align*}
Hence, the set is shattered.

Consider any set of size 4. For concreteness, we pick the set $\{1, 3, 5, 7\}$.
Since we will only make use of the \emph{ordering} of the elements, hence our argument
will work for any set of size 4 (and higher).
We claim that the subset $\{3, 7\} \subseteq \{1, 3, 5, 7\}$ cannot be classified
by any hypothesis $h \in H$ correctly. That is,  no hypothesis $h \in \H$ can
be such that  $h(1) = 1, h(3) = -1, h(5) = 1, h(7) = -1$.

This is because every function $h_{a, b, s} \in H$ can change its value twice, when
hopping from the boundary of being to the left of $(a, b)$ to entering $(a, b)$,
and then again exiting $(a, b)$ from the right:
\begin{align*}
h_{a, b, s}(x < a) = -s  &\mapsto h_{a, b, s}(a \leq x \leq b) = s \quad \text{change 1} \\
h_{a, b, s}(a \leq x \leq b) = s &\mapsto h_{a, b, s}(x \geq b) = -s \quad \text{change 2}
\end{align*}

However, in the case outlined above, to detect $\{3, 7\}$,
we would need to change sign three times:
once from $1 \mapsto 3$, once again from $3 \mapsto 5$, and finally from $5 \mapsto 7$.

So, sets of size 4 cannot be shattered by $\H$. For even larger sets, we can concentrate
what happens on any 4 elements and replicate the same argument.

Hence, $\vc(\H) = 3$.

\end{document}
