\documentclass{book}

%%% % https://tex.stackexchange.com/a/97128
\usepackage[sc,osf]{mathpazo}   % With old-style figures and real smallcaps.
%%% \linespread{1.025}              % Palatino leads a little more leading
%%% % Euler for math and numbers
%%% \usepackage[euler-digits,small]{eulervm}

% \usepackage{cmbright}


\usepackage{mathtools}
\usepackage{amsmath}
\usepackage{amssymb}
\usepackage{amsthm}
\usepackage{minted}
\usepackage{hyperref}
\usepackage{tikz}
\usepackage{tikz-cd}
\usepackage{cancel}
\usepackage{mathtools}

\newcommand{\F}{\ensuremath{\mathcal{F}}}
\newcommand{\G}{\ensuremath{\mathcal{G}}}
\newcommand{\N}{\ensuremath{\mathbb{N}}}
\newcommand{\Z}{\ensuremath{\mathbb{Z}}}
\newcommand{\Q}{\ensuremath{\mathbb{Q}}}
\newcommand{\C}{\ensuremath{\mathbb{C}}}
\newcommand{\R}{\ensuremath{\mathbb{R}}}
\newcommand{\CP}{\ensuremath{\mathbb{CP}}}
\renewcommand{\P}{\ensuremath{\mathbb{P}}}
\newcommand{\A}{\ensuremath{\mathbb{A}}}
\renewcommand{\O}{\ensuremath{\mathcal{O}}}

\DeclarePairedDelimiter\ceil{\lceil}{\rceil}
\DeclarePairedDelimiter\floor{\lfloor}{\rfloor}
\DeclarePairedDelimiter\fracpart{\{}{\}}
\theoremstyle{definition}
\newtheorem{theorem}{Theorem}
\newtheorem{example}[theorem]{Example}
\newtheorem{nonexample}[theorem]{Non Example}
\newtheorem{aside}[theorem]{Aside}
\newtheorem{definition}[theorem]{Definition}
\newtheorem{exercise}[theorem]{Exercise}
\newtheorem{note}[theorem]{Note}
\newtheorem{slogan}[theorem]{Slogan}
\newtheorem{Proposition}[theorem]{Proposition}

% https://tex.stackexchange.com/questions/469588/strike-out-an-arrow-with-a-small-oblique-segment-like-with-nrightarrow
% \newcommand*\Neg[2][0mu]{\Neginternal{#1}{\negslash}{#2}}
% \newcommand*\sNeg[2][0mu]{\Neginternal{#1}{\snegslash}{#2}}
% \newcommand*\negslash[1]{\m@th#1\not\mathrel{\phantom{=}}}
% \newcommand*\snegslash[1]{\rotatebox[origin=c]{60}{$\m@th#1-$}}

\begin{document}
\chapter{Lecture 1a: Introduction}
\begin{itemize}
\item \href{MIT 3.60: Symmetry, Structure, and Tensor Properties of Materials}{https://ocw.mit.edu/courses/materials-science-and-engineering/3-60-symmetry-structure-and-tensor-properties-of-materials-fall-2005/}
\item I'm reading this so I can gain intuition for semidirect products of
      crystallographic groups, stress-energy tensors, and tensegrity.
\item In the early part of the term, we're going to use plain old geometry.
\item Halfway through, we'll switch to linear algebra and eigenvalue problems.
      The first half of the course is one long process of synthesis.
\item ``It will blossom like an elegant filigreed structure''.
\item Textbook: Martin J Burger: ``Elementary crystallography''. Out of print
   for 15 years!
\item ``International tables for X-ray crystallography'': Published by
  international union for crystallography. Volume I is symmetry tables. Everything
  we will derive is tabulated here.
\end{itemize}

\section{Syllabus: Whirlwind tour}
Crystallography is divided into X-ray crystallography and optical crystallography.
Optical crystallography studies crystals using polarized light. We're going to be
talking about geometrical crystallography / symmetry theory. This is the first
month and a half.

A thing in a pattern is called as a motif.

A \textbf{translation} is given by a vector, has magnitude and direction, but no origin.
We can think of it as a rubber stamp. Rather, we think of operations that act on all of space.
\begin{definition}
An object or a space possesses symmetry when there is an operation or a set of
operations that maps it into congruence wit itself.
\end{definition}

For \textbf{rotation}, we need to know the point about which the rotation takes place,
call it $A$, which is the location of the axis. Then, we need to know the angle,
which will denote with a subscript $\theta$. So the full notation for a
rotation is $A_\theta$.

For \textbf{reflection}, we use the symbol $M$ for mirror. 

And this is all that we have in 2D: translation, rotation, reflection.

\chapter{Lecture 1b: Introduction}

\begin{definition}
a Symmetry Element is the locus of points that's left invariant/unmoved by the
operation.  What is the net consequence of a sequence of transformations? We
can answer by describing the symmetry element of a sequence of transformations.
\end{definition}


We now make a profound conclusion: Translation, Rotation, and reflection are the
only ones that can exist as single step transformations. A translation performs
$(x, y) \mapsto (x + a, y + b)$. A mirror plane moves $(x, y)$ to $(-x, y)$
if the mirror line is perpendicular to the $x$-axis and passes through the
origin. If we do this again, $(-x, y) \mapsto (x, y)$: we're left with the
identity. If we now rotate by 180 degrees through the origin, the point
$(x, y) \mapsto (-x, -y)$. If we perform it again, it would go to $(x, y)$.

In more general terms, if we change the sense no coordinate, that's translation.
If we change the sense of one coordinate, we get reflection. If we change the
sense of both coordinate, it's going to be a rotation.

We chose special cases to make it easy. But a mirror plane will always translate
one coordiate, while a rotation will translate one coordinate.


Extrapolating, in 1D, we can change the sense of no coordinate, that's
translation, or we change the sense of 1 coordinate, that's reflection. So we
don't have rotation in 1D.

Now extrapolating in 3D, we're going to have 4 distinct one step operations.

Are there are infinitely many operations on composition of operations? We'll see
what happens when we compose two operations next lecture.

\section{Patterns in 2D}

\subsection{Translation}
Translation has magnitude and direction, but it has no origin. We can summarize
the periodicity by taking some feducial points $P$, by saying that something hung
on one of the point $p$ will be hung on all of the other points $p' \in P$.
This kind of a point is called as a \textbf{lattice point}. This array of
ficticious point is called as a lattice. It is the array of ficticious point
that represents the translational symmetry of the crystal is the lattice.
                            
If we put atoms on the lattice, then it's called as a \textbf{structure}.

We can add another translational symmetry, let's say $T_2$ which is not along
$T_1$. So we have two non-collinear translations which give us a 2D space lattice,
at which motifs will be hung at $n T_1 + m T_2$ where $n, m \in \Z$. It's something
called as a lattice net.

How can we now define the area that is unique to one lattice point? We can
draw a parallelogram starting from one lattice point. This gives us the
unique part that is hung at a lattice point. This is called as the \textbf{unit cell}
or as the \textbf{cell}.

The existence of $T_1, T_2$ give us an array of lattice points, which define a
cell. But if you pick a cell, there are many $T_1, T_2$ that define
the exact same lattice! These are called \textbf{conjugate translations}.

So the implication is that we're going to need rules. We could pick ridiculous
cells that are long thin translations. We want to talk 'fat', short(est)
translations.  The second rule is to pick $T_1, T_2$ that displays the symmetry
(if any) of the lattice. The translational periodicity and the symmetry of the
lattice go together.  These are the only two rules we need to pick what's
called as the standard cell.

What happens if we start with a translatory lattice, with a rotation $A_\alpha$.
We can put the rotation at our designated lattice point. 

\begin{minted}{text}
        l
    *--------*
T  /          \
  /            \
 /              \  T
*----------------*
    T
\end{minted}

If our lattice looks like this, we need $l$ to be a multiple of $T$, because we need
the lattice to have translational symmetry of $T$. So let's set $l = pT$

\begin{minted}{text}
        l=pT
    *--------*
T  /|        |\
  / |        | \
 /a |        |a \  T
*---*--------*---*
       pT
<------T---------->
\end{minted}
From the horizational line at the bottom we get $2T \cos\alpha + pT = T$, which
gives us the equation $1 - 2 \cos \alpha = p$, which makes sense because nothing
in the construction depended on the size of what we took. This says that
$\cos \alpha = (1 - p)/2$. This gives us something we can plug and chug.

If we set $p = 3$, then we get $\cos \alpha = (3-1)/2 = -1$, which means
$\alpha = 180^\circ$. We can try this. We start with:

\begin{minted}{text}
*----T-->*
O        O'
\end{minted}

now let's rotate by 180 degrees counter clockwise about $O$, giving us $O_1$

\begin{minted}{text}
*<--T---*----T-->*
O1      O        O'
\end{minted}

Next we can rotate it 180 degrees clockwise about $O'$, giving us $O_2$:

\begin{minted}{text}
*<--T---*----T-->*<------*
O1      O        O'      O2
\end{minted}

Now it's indeed true that the distance between $O_1$ and $O_2$ is $3T$, as we picked
$p = 3$.  Only a small number of rotations will be compatible. We'll pick up on
that from the next lecture.

\end{document}

