\documentclass[11pt]{book}
%\documentclass[10pt]{llncs}
%\usepackage{llncsdoc}
\usepackage[sc,osf]{mathpazo}   % With old-style figures and real smallcaps.
\linespread{1.025}              % Palatino leads a little more leading
% Euler for math and numbers
\usepackage[euler-digits,small]{eulervm}
\usepackage{physics}
\usepackage{amsmath,amssymb}
\usepackage{graphicx}
\usepackage{makeidx}
\usepackage{algpseudocode}
\usepackage{algorithm}
\usepackage{listing}
\usepackage{minted}
\usepackage{tikz}
\usepackage{tikz-cd}
\usepackage{mathtools}

\evensidemargin=0.20in
\oddsidemargin=0.20in
\topmargin=0.2in
%\headheight=0.0in
%\headsep=0.0in
%\setlength{\parskip}{0mm}
%\setlength{\parindent}{4mm}
\setlength{\textwidth}{6.4in}
\setlength{\textheight}{8.5in}
%\leftmargin -2in
%\setlength{\rightmargin}{-2in}
%\usepackage{epsf}
%\usepackage{url}

\usepackage{booktabs}   %% For formal tables:
                        %% http://ctan.org/pkg/booktabs
\usepackage{subcaption} %% For complex figures with subfigures/subcaptions
                        %% http://ctan.org/pkg/subcaption
\usepackage{enumitem}
%\usepackage{minted}
%\newminted{fortran}{fontsize=\footnotesize}

\usepackage{xargs}
\usepackage[colorinlistoftodos,prependcaption,textsize=tiny]{todonotes}

\usepackage{hyperref}
\hypersetup{
    colorlinks,
    citecolor=blue,
    filecolor=blue,
    linkcolor=blue,
    urlcolor=blue
}

\usepackage{epsfig}
\usepackage{tabularx}
\usepackage{latexsym}

\DeclarePairedDelimiter{\ceil}{\lceil}{\rceil}

\newcommand\ddfrac[2]{\frac{\displaystyle #1}{\displaystyle #2}}
\newcommand{\N}{\ensuremath{\mathbb{N}}}
\newcommand{\R}{\ensuremath{\mathbb R}}
\newcommand{\coT}{\ensuremath{T^*}}
\newcommand{\Lie}{\ensuremath{\mathfrak{L}}}
\newcommand{\pushforward}[1]{\ensuremath{{#1}_{\star}}}
\newcommand{\pullback}[1]{\ensuremath{{#1}^{\star}}}

\newcommand{\inj}{\hookrightarrow}
\newcommand{\mono}{\inj}


\newcommand{\sur}{\twoheadrightarrow}
\newcommand{\epi}{\sur}

\newcommand{\pushfwd}[1]{\pushforward{#1}}
\newcommand{\pf}[1]{\pushfwd{#1}}

\newcommand{\boldX}{\ensuremath{\mathbf{X}}}
\newcommand{\boldY}{\ensuremath{\mathbf{Y}}}


\newcommand{\G}{\ensuremath{\mathcal{G}}}
\newcommand{\Set}{\ensuremath{\mathbf{Set}} }
% \newcommand{\braket}[2]{\ensuremath{\left\langle #1 \vert #2 \right\rangle}}


\def\qed{$\Box$}
\newtheorem{corollary}{Corollary}
\newtheorem{theorem}{Theorem}
\newtheorem{definition}{Definition}
\newtheorem{lemma}{Lemma}
\newtheorem{observation}{Observation}
\newtheorem{proof}{Proof}
\newtheorem{remark}{Remark}
\newtheorem{example}{Example}
\newtheorem{exercise}{Exercise}



\title{Cameras, Lights}
\author{Siddharth Bhat}
\date{Easter 2024}

\begin{document}
\maketitle
\tableofcontents

%
% Fundamentals of photography by CEK Mees


% https://monoskop.org/images/b/bd/The_fundamentals_of_photography_(1921).pdf
% physics of digital photography: andy rowlands.

% https://www.pencilofrays.com/lens-design-spreadsheet/
% % A pencul of rays.
\chapter{Fundamentals of Photography}

Exposure: Amount of light reaching per unit area of the sensor.


Shutter speed:
Units of time (sec). Faster shutter speed, easier to freeze action because light for a short period of time enters.
(I like to have low shutter speed).

Aperture: size of the hole. Units of f-stops. Larger the number, smaller the hole. 
Shooting at a small f-stop like 2.8 gives us maximum separation between subject and object, blurs background by a lot.
(Why? I guess that if f-stop is large, aperture is small, and this makes a pinhole camera.)
Calculated by dividing focal length by the diameter of the enterance pupil. $N \equiv f / D$.
As an example, if the focal length is $10$mm, and the diameter of the pupil is $5$mm, we would have $N = 2$, which would be 
written as having $f$-number "$f/2$". Informally, what this gives us is the way to caclculate the diameter of the pupil,
which is $D = f/2 = 10/2 = 5$. What a crazy setup. So, clearly, higher the $f$ number, smaller the diameter of the pupil.


Human eye varies between $f/2.1$ and $f/8.3$, wikipedia claims.
(I like to have low f-stop, since it gives us strong background blur and bokeh.)


ISO: light sensitivity of sensor.
What does ISO stand for?
Higher ISO: allows more light, but introduces more noise.

Focal length:
Measured in units of length.
In photography and telescopy, since object is basically infitely far away, assuming all incoming rays are parallel.

% https://en.wikipedia.org/wiki/Angle_of_view_(photography)
Field of view / Angle of vew:
Apparently, there is some crazy theory that says that if we want a sharp image, then the image plane must be at the focal length.
Now, for a fixed height of film-format, as we increase the length (focal length), the angle subtended, where $\tan \alpha = (d/2)/f$ decreases,
and thus the angle $\alpha$ decreases, whcih we have a smaller angle of view.
Thus, larger the focal length, smaller the angle/field of view.


Focal length v/s Field of view.

ISO in digital cameras: measures "speed" (in seconds) / time taken to expose a standard amount.

Film speed is found from a plot of optical density vs. log of exposure for the film, known as the D–log H curve or Hurter–Driffield curve.
There typically are five regions in the curve: the base + fog, the toe, the linear region, the shoulder, and the overexposed region.
For black-and-white negative film, the "speed point" m is the point on the curve where density exceeds the base + fog density by 0.1 when the negative is developed so
that a point n where the log of exposure is 1.3 units greater than the exposure at point m has a density 0.8 greater than the density at point m.
The exposure Hm, in lux-s, is that for point m when the specified contrast condition is satisfied. The ISO arithmetic speed is determined from:

Lower ISO = less light, higher ISO = more light.
(Based on the previous two choices, I should use low ISO, since I like letting a lot of light in.)



\chapter{$\alpha$6000: Modes to Use}

Use S / shutter setting mode to capture motion blur based photos.
Use A / aperture priority mode to capture portraits to setup a strong DoF / bokeh creation.




\end{document}
