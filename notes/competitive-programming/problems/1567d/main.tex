\section{Codeforces 1567d: Expression Evaluation Error}
\begin{itemize}
\item Let us study the meaning of carrying.
\item Suppose we have 1 digit
numbers $x, y$ with single digits $x_0, y_0$. Let $z = x + y$.
What is $z_0, z_1$? If $x_0 + y_0 < 10$, then $z_1 = 0; z_0 = x_0 + y_0$.
Otherwise, we have $z_1 = 1; z_0 = x_0 + y_0 - 10$.

\item This works because the value of the number $z$ is $x + y$, while
the value in digits is $10z_1 + z_0$, which in the first
case becomes $10 \cdot 0 + (x_0 + y_0) = x + y$, and in the second
case becomes $10 \cdot 1 + (x_0 + y_0 - 10) = x + y$.  The point
of doing this is that when $x_0 + y_0 \geq 10$, we have
$0 \leq x_0 + y_0 - 10 \leq 9$.

\item Now in the above question, we have a \emph{perturbation}, where we add \emph{as if} we have only digits $[0, 1, \dots, 9]$, but we live in
base 11. Thus, this means that if $x_0 + y_0 \geq 10$, we set $z_1 = 1; z_0 = x_0 + y_0 - 10$. Now the value of $z$ (in base 11) is $11 \cdot z_1 + z_0$ which equals $11 + x_0 + y_0 - 10$ which is $x_0 + y_0 + 1$. So, carrying, or having a digit in a higher place is very valuable in this adding system.

\item This brings us to the key idea: Suppose we have to create sum $s$ from $n$ numbers. If $n$ equals $1$, then we are done and we simply create a list of length 1: $[s]$.
\item Otherwise, we try to find the greatest power of 10 $10^k$ that we can
create from $s$ such that we still have enough leftover
to fill up $(n-1)$ slots. So we maximize $k$ such that $(s - 10^k) \geq (n-1)$.
\item This produces a number for us on the list, $10^k$. We recurse
 and produce the next number on $s' \equiv s - 10^k$ , $n' \equiv n - 1$.


\item For a correctness proof, suppose the optimal set of numbers is $\mathcal O$. Order these by descending order (so it matches with our algorithm), and label the numbers $o_1 \geq o_2 \geq \dots$. Let our
sequence be $g_1 \geq g_2 \dots$ (with $g$ for greedy). Let
the two sequences agree upto some $i$. So $i$ is the leftmost
index such that $o_i \neq g_i$.
\item Now, at this stage, the total sum is some $s$, and we produce $g_i \equiv 10^k$ such that either (a) $10^{k+1} > s$, or that (b) $s - 10^k < n-1$. Let's deal with (a) first. So we have that $g_i = 10^k$ such
that $g_i = 10^k < s$ and $10^{k+1} > s$.
\item We must have $o_i \geq g_i$. If not, then we will have that $o_i < g_i$, and since the sequences are decreasing, $\sum_{j \geq i} o_j < \sum_{j \geq i} g_j$ which contradicts the optimality of $\mathcal O$.


\item Thus, we use a greedy algorithm and write:

\begin{minted}{cpp}
vector<int> f(int sum, int n) {
    assert(sum >= n);
    vector<int> outs;
    while (n > 0) {
        if (n == 1) {
            // have used all numbers save 1.
            outs.push_back(sum); break;
        }
        int pow10 = 1;
        assert(pow10 <= sum);
        while (1) {
            const int nextpow10 = pow10 * 10;
            if (!(nextpow10 <= sum)) { break; }
            if (!(sum - nextpow10 >= n-1)) { break; }
            pow10 = nextpow10;
        }
        assert(pow10 <= sum);
        assert(sum - pow10 >= n-1);
        // assert(pow10 * 10 > sum);
        // [untrue; we may quit because we need to produce many numbers.
        // eg: sum = 10, n = 10. We will have pow10 = 1]
        outs.push_back(pow10);
        sum -= pow10;
        n--;
    }
    return outs;
}

\end{minted}

\end{itemize}

\newpage
