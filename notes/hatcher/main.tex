\documentclass[11pt]{book}
%\documentclass[10pt]{llncs}
%\usepackage{llncsdoc}
\usepackage[sc,osf]{mathpazo}   % With old-style figures and real smallcaps.
\linespread{1.025}              % Palatino leads a little more leading
% Euler for math and numbers
\usepackage[euler-digits,small]{eulervm}
\usepackage{bbding} % for flower. 
\usepackage{physics}
\usepackage{amsmath,amssymb}
\usepackage{graphicx}
\usepackage{makeidx}
\usepackage{algpseudocode}
\usepackage{algorithm}
\usepackage{listing}
\usepackage{minted}
\usepackage{cancel}
% \usepackage{quiver}
\evensidemargin=0.20in
\oddsidemargin=0.20in
\topmargin=0.2in
%\headheight=0.0in
%\headsep=0.0in
%\setlength{\parskip}{0mm}
%\setlength{\parindent}{4mm}
\setlength{\textwidth}{6.4in}
\setlength{\textheight}{8.5in}
%\leftmargin -2in
%\setlength{\rightmargin}{-2in}
%\usepackage{epsf}
%\usepackage{url}

\usepackage{booktabs}   %% For formal tables:
                        %% http://ctan.org/pkg/booktabs
\usepackage{subcaption} %% For complex figures with subfigures/subcaptions
                        %% http://ctan.org/pkg/subcaption
\usepackage{enumitem}
%\usepackage{minted}
%\newminted{fortran}{fontsize=\footnotesize}

\usepackage{xargs}
\usepackage[colorinlistoftodos,prependcaption,textsize=tiny]{todonotes}

\usepackage{hyperref}
\hypersetup{
    colorlinks,
    citecolor=blue,
    filecolor=blue,
    linkcolor=blue,
    urlcolor=blue
}

\usepackage{epsfig}
\usepackage{tabularx}
\usepackage{latexsym}
\newcommand\ddfrac[2]{\frac{\displaystyle #1}{\displaystyle #2}}
\newcommand{\N}{\ensuremath{\mathbb{N}}}
\newcommand{\R}{\ensuremath{\mathbb R}}
\newcommand{\coT}{\ensuremath{T^*}}
\newcommand{\Lie}{\ensuremath{\mathfrak{L}}}
\newcommand{\Vectorfield}{\ensuremath{\mathfrak{X}}}
\newcommand{\pushforward}[1]{\ensuremath{{#1}_{\star}}}
\newcommand{\pullback}[1]{\ensuremath{{#1}^{\star}}}
\newcommand{\vectorfield}{\ensuremath{\mathfrak{X}}}

\newcommand{\pushfwd}[1]{\pushforward{#1}}
\newcommand{\pf}[1]{\pushfwd{#1}}

\newcommand{\boldX}{\ensuremath{\mathbf{X}}}
\newcommand{\boldY}{\ensuremath{\mathbf{Y}}}


\newcommand{\G}{\ensuremath{\mathcal{G}}}
% \newcommand{\braket}[2]{\ensuremath{\left\langle #1 \vert #2 \right\rangle}}


\def\qed{$\Box$}
\newtheorem{theorem}{Theorem}
\newtheorem{corollary}[theorem]{Corollary}
\newtheorem{definition}[theorem]{Definition}
\newtheorem{lemma}[theorem]{Lemma}
\newtheorem{observation}[theorem]{Observation}
\newtheorem{proof}[theorem]{Proof}
\newtheorem{remark}[theorem]{Remark}
\newtheorem{example}[theorem]{Example}

\newcommand{\X}{\ensuremath{\mathfrak{X}}}


\newcommand*{\start}[1]{\leavevmode\newline \textbf{#1} }
\newcommand*{\answer}{\start{Answer}}
% \newcommand*{\answer}{\leavevmode\newline \textbf{Answer} }
\newcommand*{\qed}{\ensuremath{\blacksquare}}

\title{Algebraic topology: Hatcher}
\author{Siddharth Bhat}
\date{Spring of the second Year of the Plague}


\begin{document}
\maketitle
\tableofcontents
\chapter{Links}
\url{https://pages.uoregon.edu/njp/hw2solutions.pdf} contains solutions to asssigned homework from hatcher.
\chapter{Ch1}
\section{Ch1.1}
\subsection{Ex1}
If $f_1 ; g_1 \simeq f_2 ; g_2$ and $g_1 ; g_2$ we must show that $f_1 \simeq f_2$. Let $h: [0, 1] \times S^1 \rightarrow X$
witness $f_1; g_1 \simeq f_2; g_2$ and let $h': [0, 1] \times S^1 \rightarrow X$ witness $g_1 \simeq g_2$. Then
define $h'^{-1}$ as the witness of $g_1^{-1} \simeq g_2^{-1}$. Then say that $f_1 \simeq f_1; g_1; g_1^{-1} \simeq f_2; g_2; \simeq g_2^{-1} \simeq f_2$ via $h; h'^{-1}$.

\subsection{Ex2}
Consider $\beta_h(f) \equiv h \circ f \circ h^{-1}$. If $h \simeq h'$ via the homotopy $H$, then we have that $H; id; H^{-1}$ 
witnessing $h \circ f \circ h^{-1} \simeq h' \circ f \circ h'^{-1}$. Thus, $\beta_h(f) \simeq \beta_{h'}(f)$ for all $f$.

\subsection{Ex3}
\question{$\pi_1(X)$ is abelian iff $\beta_h$ depend only on endpoints of $h$.}

\start{First part:} Assume $\beta_h$ depends only on endpoints, prove that $\pi_1(X)$ is abelain. Consider $f, g$ loops at $x_0$
See that $\beta_f(g) = f; g; f^{-1}$, and $\beta_g(g) = g; g; g^{-1} \simeq g$. But $f(0) = g(0) = x_0$ and $f(1) = g(1) = x_0$
Hence $\beta_f = \beta_g$, thus $\beta_f(g) = \beta_g(g)$, thus $f; g; f^{-1} = g$, or $f; g = g; h$ thus establishing
that the group is abelian.

\start{Second part:} Assume that $\pi_1(X)$ is abelian. Show that $\beta_*$ depends only on endpoints. Let $f, g$ be two
loops with equal endpoints; $f(0) = g(0)$ and $f(1) = g(1)$. See that:

$$
\beta_f(g) = f; g; f^{-1} = g; f; f^{-1} = g
$$

Thus $\beta_f(g) \simeq g$. So this means that $\beta_f$ will perform no action on any path from $f(0)$ to $f(1)$.
Thus, the action of $\beta_f$ will be equl to all $\beta_g$ as long as $f, g$ have the same endpoints.

\subsection{Ex4}
\start{Idea:} Use compactness to finite finite subcover of star shaped regions. Within each star shaped region,
since star shaped is contractible, all paths are nullhomotopic. So we can homotope the section of the path into a point, and then expand
back into the line segment. [I always get weirded out that homotopy does not preserver cardinality of the image, or indeed any intuition I have about
``structure''].


rigorous seems hard.

\subsection{Ex5}

\end{document}
