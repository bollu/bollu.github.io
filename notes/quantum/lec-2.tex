\section{Operators}

\subsection{Projectors --- $P$}

Suppose $W$ is a $k$-dimensional vector subspace of the $d$-dimensional 
vector space $V$. 

Using Gram-Schmidt, it is possible to construct an orthonormal basis
$\ket{1}, \ket{2}, \dots \ket{d}$ for $V$ such that $\ket{1} \dots \ket{k}$
is an orthonormal basis for $W$. Then the projector $P$ is defined as:
\begin{align*}
    P_W \equiv \sum_{i=1}^k \ketbra{i}
\end{align*}

\begin{itemize}
\item $P^\dagger = P$ (Immediate from writing in $\ket{i}$ basis)
\item $P^2 = P$ (Immediate from writing in $\ket{i}$ basis)
\end{itemize}

$Q = I - P$ is the projector onto orthogonal complement of the subspace that $P$.
projects into. This projects onto the $\ket{k+1} \dots \ket{d}$ basis.

\subsection{Normal operator}
\begin{align*} A A^\dagger = A^\dagger A \end{align*}


\begin{theorem}
Spectral theorem for normal operators:
Any normal operator $M$ on a vector space $V$ is diagonal with respect to some
orthonormal basis for $V$.
\end{theorem}
\begin{proof}
Let $\lambda$ be an eigenvalue of $M$. $P_\lambda$ is the projector onto
$\lambda$'s eigenvector. $Q_\lambda = P_\lambda^\bot$ is the orthogonal complement projector
of $P$.

We first establish a fact about $P M Q$:
\begin{align*}
&M M^\dagger \ket \lambda = M^\dagger (M \ket \lambda) = \lambda M^\dagger \lambda \\
&\text{Hence, $M^\dagger v \in P$.} \\
&Q (M^\dagger P) = 0 \implies (P M Q)^\dagger = 0 \implies P M Q = 0
\end{align*}

Next, we prove some properties of $QM$ and $QM^\dagger$
\begin{align*}
QM = QM(P + Q) = QMP + QMQ = QMQ \\
QM^\dagger = QM^\dagger(P + Q) = QM^\dagger P + QM^\dagger Q = (PMQ)^\dagger + QM^\dagger Q
\end{align*}

\begin{align*}
&\text{QMQ is normal:} \\
&(QMQ)^\dagger(QMQ) = Q^\dagger M^\dagger Q^\dagger Q M Q = Q M^\dagger Q M Q = Q M^\dagger M Q \\
&(QMQ)(QMQ)^\dagger = (Q M Q) (Q^\dagger M^\dagger Q^\dagger) = Q M Q M^\dagger Q = 
Q M M^\dagger Q = Q M^\dagger M Q = (QMQ)^\dagger QMQ
\end{align*}

\begin{align*}
&M = (P + Q) M (P + Q) \\
&M = P M P + P M Q + Q M P + Q M Q \\
&M = P M P + Q M Q \\
&M = \lambda_i \ketbra{i} + Q M Q \\
&\text{Since $Q M Q$ is normal, and we are performing induction on dimension, and $P \bot Q$,} \\
&M = \lambda_i \ketbra{i} + \sum_k \lambda_k \ketbra{k} \\
&\text{Hence M is normal}
\end{align*}
\end{proof}

\begin{theorem}
Any diagonalizable operator is normal
\end{theorem}
\begin{proof}
Let $M$ be diagonal with respect to basis $\ket{i}$.
Then, $M \equiv \sum_i \lambda_i \ketbra{i}$.
Now, $M^\dagger= \sum_i \lambda_i^* \ketbra{i}$. 
\begin{align*}
&M M^\dagger = \bigg(\sum_i \lambda_i \ketbra{i}\bigg)
    \bigg(\sum_j \lambda_j^* \ketbra{j}\bigg) \\
&M M^\dagger = \sum_i \lambda_i^* \lambda_i \ketbra{i} \\
&\text{Similarly,}  \quad M^\dagger M = (\sum_i \lambda_i^* \lambda_i \ketbra{i}) 
\end{align*}
\end{proof}

\subsection{Unitary operator}
\[ U U^\dagger = U^\dagger U = I \]
\begin{itemize}
\item unitary operator is normal.
\item unitary operator preserves inner products.
\begin{align*}
\bra{b'} \ket{a'} = \bra{b} U^\dagger U \ket{a} = \bra{b} I \ket{a}
\end{align*}
\end{itemize}

\subsection{Positive operator}
Special class of Hermitian operator.

\begin{align*}
 \forall v \in V, \bra v A \ket v \geq 0
\end{align*}

If the inner product is strictly greater than zero, then such an operator
is called as \emph{positive definite}. If it is greater than or equal
to zero, it is called \emph{positive semidefinite}.

\begin{theorem}
A positive operator is Hermitian
\end{theorem}
\begin{proof}
\textbf{TODO}. Proof most likely follows real case, where we use
cholesky to write it as $A^T A$ and then show that it is normal. We then
use the fact that its eigenvalues are greater than or equal to zero
to establish that it is Hermitian.
\end{proof}

