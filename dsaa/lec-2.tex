%% What is the document class I need?
\documentclass{article} 
%% Some recommended packages.
\usepackage{booktabs}   %% For formal tables:
                        %% http://ctan.org/pkg/booktabs
\usepackage{subcaption} %% For complex figures with subfigures/subcaptions
                        %% http://ctan.org/pkg/subcaption
\usepackage{enumitem}
\usepackage{minted}
\newminted{fortran}{fontsize=\footnotesize}

\usepackage{xargs}
\usepackage[colorinlistoftodos,prependcaption,textsize=tiny]{todonotes}


\begin{document}
\section{Lecture 2 - Signals and Systems: Signals}
Dennis Freeman, MIT lectures are supposedly very good.

\subsection{Brief Overview}

If $x(t)$ was our continuous time function, we wish to create
$x[n]$. $x[n]$ is discrete.

\subsection{Recovering back signals}
\begin{itemize}
    \item Zero order hold
        Keep previous value we had till the next value
    \item Nearest Neighbour (NN)
        For each point, pick nearest neighbour and use that.
    \item Linear interpolation
    \item $\frac{sin(x)}{x}$
\end{itemize}


\subsection{Operations on signals}

\begin{itemize}
    \item Shifting: $y[n] = x[n - k]$. Delay by $k$ steps
    \item Flipping: $y[n] = x[-n]$.
    \item Scaling: $y[n] = \cdot x[k * n]$. This would lead to loss of information.
\end{itemize}

Order of operations: Shift, flip, scale. For counterexample, consider
$y[n] = x[3n + 1]$. Proof by induction on $ax + b$?

Eg: $x[-2n + 2]$. Perform as:

$y[n] = x[n + 2]$.

\subsection{Characteristics of signals}

\subsubsection{Even, Odd}

Even signal $x[n] = x[-n]$
Odd signal $x[n] = -x[-n]$

Every signal can be decomposed into sum of even and odd signals
$x[n] = \frac{x[n] + x[-n]}{2} + \frac{x[n] -x[-n]}{2}$

\subsubsection{Periodic}
$\exists N \in Z, \forall n \in Z, x[n] = x[n + N]$.

\subsubsection{Energy and power}

$Energy = \sum_{k=-\infty}{\infty} |x[k]|^2$


$Power = lim_{N \larrow \infty} \frac{1}{2N + 1} \sum_{k=-N}{N} |x[k]|^2$

\subsubsubsection{Unit Step}

u(n) = 1 if n >= 0, 0 otherwise.
Energy is \infty, power is \frac{1}{2}.


\subsection{Special Signals}

\subsubsection{Dirac delta}
$\delta[n] = 0 if n \neq 0, 1 if n = 0$.

\subsubsection{Unit Step}
u[n] = 1 if n >= 0, 0 otherwise.


\subsubsection{Ramp function}
r[n] = n if n >= 0, 0 otherwise.

\subsubsection{Exponential function}
e[n] = a^n u[n]

a is a parameter.


\subsection{Writing unit step in terms of delta}
u[n] = \sum_{k=0}{\infty} delta[n - k]

\subsection{Writing delta in terms of unit step}
delta[n] = u[n + 1] - u[n]

\subsection{Writing any signal $x[n]$ in term of $delta[n]$}
$x[n] = \sum_{k=-\infty}{infty} delta[n - k] \cdot x[k]$
