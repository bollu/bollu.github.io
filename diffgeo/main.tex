\documentclass[11pt]{book}
%\documentclass[10pt]{llncs}
%\usepackage{llncsdoc}
\usepackage{physics}
\usepackage{amsmath,amssymb}
\usepackage{graphicx}
\usepackage{makeidx}
\usepackage{algpseudocode}
\usepackage{algorithm}
\usepackage{listing}
\evensidemargin=0.20in
\oddsidemargin=0.20in
\topmargin=0.2in
%\headheight=0.0in
%\headsep=0.0in
%\setlength{\parskip}{0mm}
%\setlength{\parindent}{4mm}
\setlength{\textwidth}{6.4in}
\setlength{\textheight}{8.5in}
%\leftmargin -2in
%\setlength{\rightmargin}{-2in}
%\usepackage{epsf}
%\usepackage{url}

\usepackage{booktabs}   %% For formal tables:
                        %% http://ctan.org/pkg/booktabs
\usepackage{subcaption} %% For complex figures with subfigures/subcaptions
                        %% http://ctan.org/pkg/subcaption
\usepackage{enumitem}
%\usepackage{minted}
%\newminted{fortran}{fontsize=\footnotesize}

\usepackage{xargs}
\usepackage[colorinlistoftodos,prependcaption,textsize=tiny]{todonotes}

\usepackage{hyperref}
\hypersetup{
    colorlinks,
    citecolor=black,
    filecolor=black,
    linkcolor=black,
    urlcolor=black
}

\usepackage{epsfig}
\usepackage{tabularx}
\usepackage{latexsym}
\newcommand\ddfrac[2]{\frac{\displaystyle #1}{\displaystyle #2}}
\newcommand{\N}{\ensuremath{\mathbb{N}}}
\newcommand{\R}{\ensuremath{\mathbb R}}
\newcommand{\coT}{\ensuremath{T^*}}

\def\qed{$\Box$}
\newtheorem{corollary}{Corollary}
\newtheorem{theorem}{Theorem}
\newtheorem{definition}{Definition}
\newtheorem{lemma}{Lemma}
\newtheorem{observation}{Observation}
\newtheorem{proof}{Proof}
\newtheorem{remark}{Remark}

\title{General Relativity and Differential Geometry}
\author{Siddharth Bhat}
\date{Monsoon 2019}

\begin{document}
\maketitle
\tableofcontents

\chapter{Introduction}

I am following the following sources:

\begin{itemize}
    \item Susskind's General Relativity lectures as part of the Theoretical
        minimum: (\href{https://theoreticalminimum.com/courses/general-relativity/}{The link is here}).
        Note that the videos do not load. However, one can view the source to access the link
        to the iframe.
    \item Susskind's other General Relativity lectures, as part of his modern
        physics course: (\href{https://www.youtube.com/watch?v=hbmf0bB38h0&list=PL6C8BDEEBA6BDC78D}{link to playlist here}).
        These seem to be taught at a much gentler pace.
    \item The book Gravitation by Misner, Thorne and wheeler.
\end{itemize}

The notes as scribed here are a mix from all of these sources, as well
as tangential points I find interesting.

\section{The equivalence principle}

Gravity is in some sense the same thing as acceleration. First, an elementary
derivation which formalizes the intuition of the equivalence principle.

We consider an elevator moving upward. Let its distance from the bottom be $L(t)$.

\begin{align*}
    z' = z - L(t) \quad t' = t \quad x' = x
\end{align*}


% \begin{align*}
% F = m\frac{d^2 z'}{dt^2} \\
% F = m\frac{d^2 (z - L(t))}{dt^2} \\
% F = m\frac{d^2 z}{dt^2} \text{(If \frac{d^2L(t)}{dt^2} - 0)}
% \end{align*}

if $\frac{d^2 L(t)}{dt^2} = 0$, then the force is the same in the new coordinate
system as that of the old coordinate system

\section{Galileo's theory of flat space and gravitation}
Newton's laws:
\begin{align*}
    &\vec F = m \vec a \\
    &\vec F = m \frac{d^2 x}{dt^2} \\
\end{align*}

Galileo's gravitation, under the approximation that the earth is flat:
If we pick downwards to be negative direction along the $2$ dimension, then his
equation can be written as ${F_2 = - m g}$ where $g$ is a constant.

This is special, because the force is proportional to the mass, which is not
the case of things like electromagnetism.


Combining the two equations, we get ${m \frac{d^2x}{dt^2} = - m g}$, or
${\frac{d^2x}{dt^2} = - m g}$.


That the acceleration induced by the graviational force is independent of the
mass of the object is known as the \textit{equivalence principle}. At this
stage, we can say that gravity is equivalent across all objects independent of
their mass.

Let's now consider a collection of point masses --- A diffuse cloud of
particles, and have it fall. Different particles maybe heavy, light, large,
small. However, since all of them have the same acceleration, the point cloud
looks unchanged as it falls. That is, the object will have no stresses or
strains as it falls. We can't tell by looking at our neighbours that there is a
force being exterted on us, since all our neighbours are moving along with us!
We cannot tell the difference between being in free space versus being in a
graviational field.

\section{Newton's theory of gravity}
all bodies exert equal and opposite forces on each other. Given two bodies $a$
and $b$ of masses $m$ and $M$ with distance $R$, the force on $a$ is
${F_a \equiv \frac{GmM}{R^2}} \hat{r}$. where $\hat r$ is the direction from $a$
to $b$.

Again, we can prove that the acceleration of an object $a$ does not depend on
its own mass.

Now that gravitation depends on distance, we can actually feel something if we
are in a gravitational field, since different parts of a given object will have
a different force on it, due to the varying distance from (say) the earth.

Gravitational field is defined as the force exerted on a test mass at every
point in space.

\subsubsection{Gauss' theorem}
${\int  \grad \cdot A dx dy dz  = \int A_{\bot} d\sigma}$ where $\sigma$ is the
differential unit of surface area of the surface. 

Show that the divergence of a field in 3 dimensions will lead to an inverse
square law.


\section{Geometry and curvature}

To describe a geometry, all we need is the distance between neighboring points
on a blackboard. In general, given a parametrization, we can draw a possibly
distorted grid of lines of constant corrdinate. The distance between two points
${(x, y)}$ and ${(x + dx, y+ dy)}$ will be ${ds^2 = g_{11} dx^2 + 2 g_{12} d_x d_y + g_{22} d_y^2}$.


\chapter{Opimisation on Steifel Manifolds}
We have the space $O(n) = \{ X \in \R^{n \times n} | X^T X = I \}$. To find an
element of the tangent space at the point $P \in O(n)$, we parametrize a curve
$C(t) : \R \rightarrow O(n)$, such that $c(0) = P$.  Then, we differentiate the
curve and evaluate it at $0$. That is, $\frac{dc}{dt}\vert_{t=0} \in T_P O(n)$.


We consider $C(t) : \R \rightarrow O(n)$, such that $C(0) = P$.  Since $C(t) \in
O(n)$, we can write $C(t)^T C(t) = I$.  This in index notation, is $c^{ik}(t)
c^{jk}(t) = \delta^{ij}$. Differentiating with respect to $t$, we get:

\begin{align*}
&C^T(t) C(t) = I \\
&c^{ik}(t) c^{jk}(t) = \delta^{ij} \\
&\frac{d (c^{ik}(t) c^{jk}(t))}{dt}= \frac{d(\delta^{ij})}{dt} \\
& \dot c^{ik}(t) c^{jk}(t) + c^{ik}(t) \dot c^{jk}(t) = 0 \qquad \text{(chain rule)}\\
&\dot C^T(t) C(t) + C(t)^T \dot C(t) = 0
\end{align*}

Now, we know that $C(0) = P$, and hence $\dot C(0) \in T_P O(n)$. By evaluating
the above equation at $t = 0$, we obtain the relation:

\begin{align*}
&\dot C^T(0) C(0) + C(0)^T \dot C(0) = 0 \\
&\dot C^T(0) P + P^T \dot C(0) = 0
\end{align*}

Hence, we conclude that for all $Z \in T_P O(n)$, $Z^T P + P^T Z = 0$. Indeed,
we can characterize $T_P O(n)$ this way and prove the reverse inclusion (how?),
to show that:

\begin{align*}
     T_P O(n) \equiv \{ Z \mid Z^T P + P^T Z = 0 \}
\end{align*}

However, this equation for the $Z$ is "ineffective", in that it does not tell
us how to \textit{compute} the set of $Z$s. We can only \textit{check} if a
particular $Z_0 \in_? T_P O(n)$. 

We will solve the characterization of $Z$ by first solving a slightly easier
problem: $T_I O(n)$, where $I$ is the identity matrix. 

\begin{align*}
    T_I O(n) \equiv \{ Z \mid Z^T I + I^T Z = 0 \} = \{ Z \mid Z^T = -Z \}
 \end{align*}

 We now have a complete enumeration of the \textit{tangent space at the
 identity}: We know that this consists of all skew-symmetric matrices! 

 We now wish to transport this structure of $T_I O(n)$ to an arbitrary
 $T_P O(n)$.  Here, I will let you in on a secret about the structure of
 Lie groups, which we will later prove: The vector space at $T_P O(n)$ is obtained
 by multiplying by $P$ to $T_I O(n)$. 

 In this case, it tries to inform us that:

\begin{align*}
    T_P O(n) =  \{ PZ \mid Z^T = -Z \}
\end{align*}

We distrust this assertion at first, of course, so we can plug this equation back
in the characterization of $T_P O(n)$ that we had developed and see what pops out:

\begin{align*}
     &T_P O(n) \equiv \{ X \mid X^T P + P^T X = 0 \} \\
     &\text{Let $X = PZ$ where $Z^T = -Z$, and check that they satisfy the condition $X^T P + P^T X = 0$} \\
     & (PZ)^TP + P^TPZ = Z^TP^TP + P^TPZ = Z^T I + I^T Z = -Z + Z = 0
\end{align*}

Hence, they do satisfy the condition, and we can assert that at least:
$$\{ PZ \mid Z^T = -Z \} \subseteq  \{ X \mid X^T P + P^T X = 0 \}$$.

\chapter{Symplectic vector spaces}

% Symplectic geometry & classical mechanics, Tobias Osborne.
% videos link: https://www.youtube.com/watch?v=GdXyNYUD4cE

Let $V$ be a $m$-dimensional real vector space. Let $\Omega: V \times V \rightarrow V$
be a bilinear form on $V$ that is skew-symmetric:
$\forall a, b \in V, \Omega(a, b) = -\Omega(b, a)$.

\begin{theorem}
    Let $\Omega$ be a bilinear skew-symmetric map on $V$. Then there is a basis
    $u_1, u_2, \dots u_k$, $e_1, e_2, \dots e_n$, $f_1, f_2 \dots f_n$ such that:
    \begin{itemize}
        \item $\Omega(u_i, v) = 0 \quad \forall v \in V$
        \item $\Omega(e_i, e_j) =  \Omega(f_i, f_j) 0$
        \item $\Omega(e_i, f_j) = \delta_{ij}$
    \end{itemize}
\end{theorem}
\begin{proof}
    Generalize Gram-Schmidt process.
    Let $U = \{ u \in V | \Omega(u, v) = 0, ~\forall v \in V \}$
    $U$ is a subspace of $V$. Let $u_1, u_2, \dots u_k$ be a basis of $U$.

    Establish $e_j, f_j$ by induction. Let $V = U \oplus W$. Take $e_1 \in W$.
    Then, there must exist an $f_1 \in W$ such that $\Omega(e_1, f_1) \neq 0$. 
    Otherwise, we could expand the size of $U$ by adding $e_1$ to $U$. But
    we assumed that $U$ contains all such vectors. $U$ and $W$ share no non-zero
    vectors since $V = U \oplus W$.  Define $W_1 \equiv span(e_1, f_1)$.  
    Now build $W_1^\Omega \equiv \{ w \in W | \Omega(w, W_1) = 0 \}$.

    We need the folloing facts:
    \begin{itemize}
        \item $W_1^\Omega \cap W_1 = \{ 0 \}$
        \item $W = W \oplus W_1^\Omega$
    \end{itemize}

    So, recurse on $W_1$.

    Finally, $V = U \oplus W_1 \oplus W_1 \cdots \oplus W_n$.
\end{proof}

\begin{remark}
    $dim(U) = k$ is invariant for $(V, \Omega)$ since $U$ was defined in a
    coordinate free way. But, remember that $k + 2n = m$, and hence $n$
    is an invariant of $(V, \Omega)$. $n$ is called as the \emph{rank} of
    $\Omega$.
\end{remark}

$\Omega$ written in the basis of $\{ u_i, e_i, f_i\}$ gives 
$\Omega = \begin{bmatrix} 0 & 0 & 0 \\ 0 & 0 & I \\ 0 & -I & 0 \end{bmatrix}$

\subsection{Symplectic maps}

Let $V$ be an $m$ dimensional real vector space over \R. Let $\Omega: V \times V \rightarrow \mathbb{R}$
be a skew-symmetric bilinear map. We define 
$\tilde \Omega: V \rightarrow V^*$, $\tilde \Omega(v)(u) \equiv \Omega(v, u)$. 
The problem is that this map has a non-trivial kernel $U = \{ u_i \}$ in general,
so we cannot use it like a metric to identify the two spaces.

\begin{definition} A skew-symmetric bilinear map $\Omega: V \times V \rightarrow \R$ is Symplectic iff
    $\tilde \Omega$ is bijective. ie, $U = \{ 0\}$. $(V, \Omega)$ is then
    called a Symplectic vector space.
\end{definition}

\section{Properties of a Symplectic map}
\begin{itemize}
    \item $\tilde \Omega$ is an identification between $V$ and $V^*$.
    \item Since each $(e_i, f_i)$ come in pairs, the dimension of the vector space $V$ is divisible by $2$. ie, $m = 2n$.
    \item We have a basis $(e_i, f_i)$ called the Symplectic basis for $V$.
    \item $\Omega$ in matrix form with respect to the Symplectic basis is $\Omega = \begin{bmatrix} 0 & I \\ -I & 0 \end{bmatrix}$
\end{itemize}

\subsection{Subspaces of a Symplectic vector space}

\begin{definition}
    A subspace $W$ of $V$ is a Symplectic subspace if $\Omega|_W$ is non-degenrate.
    For example, take the subspace $span(e_1, f_1)$.
\end{definition}


\begin{definition}
    A subspace $W$ of $V$ is an Isotropic subspace if $\Omega|_W = 0$.
    For example, take the subspace $span(e_1, e_2)$.
\end{definition}

\subsection{Morphisms: Symplectomorphism}
\begin{definition}
    A map between $(V, \Omega)$ and $(V', \Omega')$ is a linear isomorphism
    $\phi: V \rightarrow V'$ such that $\phi^* \Omega' = \Omega$. That is,
    $\Omega'(\phi(v), \phi(w)) = \Omega(v, w)$. ($\phi^*$ is the pullback).
    The map $\phi$ is called a Symplectomorphism and the spaces $(V, \Omega)$ and $(V', \Omega')$
    are said to be Symplectomorphic
\end{definition}

(Question: why do we need to define this in terms of a pullback? Why not
pushforward? ie, $\Omega(v, w) = \Omega'(\phi(v), \phi(w))$?

\subsection{Prototypical example of Symplectic space}
Let $V = \R^{2n}$. Let $e_j = (0_1, \dots, 1_j, \dots 0_n; 0_{n+1} \dots 0_{2n})$.
Let $f_j = (0_1, \dots 0_n; 0_{n+1}, \dots 1_{n+j}, \dots 0_{2n})$.
take $\{e_j, f_j\}$  as a basis for $V$.
$\Omega_0 = \begin{bmatrix} 0 & I \\ -I & 0 \end{bmatrix}$

Let $\phi: V \rightarrow V$ be a Symplectomorphism of $V$. Then, we can 
show that $M_\phi^T \Omega_0 M_\phi = \Omega_0$. ($M_\phi$ is the matrix of $\phi$
associated to the $\{e_j, f_k\}$ basis). 

\subsection{Symplectic Manifolds}
We are generalizing our symplectic vector space. We are postulating a space
that locally looks like a symplectic vector space.

Let $\omega: T_p M \times T_p M \rightarrow \mathbb R$ be a 2-form on a manifold.
This is bilinear and skew-symmetric (by definition of begin a differential form).

We say that $\omega$ is closed iff $d\omega = 0$.

\begin{definition}
    The two form $\omega$ is a symplectic form if it is closed and $\omega_p$
    is symplectic for all $p \in M$. That is, we need $\omega_p$.
\end{definition}

\begin{definition}
    A symplectic manifold is a pair $(M, \omega)$ where $\omega$ is a symplectic
    form on $M$.
\end{definition}

\subsection{Prototypical example of Symplectic manifold}
Let $M = \R^{2n}$ with coordinates $X_1, \dots X_n, Y_1, \dots Y_n$.
$\omega = \sum_i dx_i  \wedge dy_i$.  $\omega$ is symplectic since it
has constant coefficients for each $dx_i \wedge dy_i$.

Symplectic basis for the tangent space 
$T_p \R^{2n} \equiv \{ \partial_{X_i}, \partial_{Y_i} \}$.

\subsection{2 sphere as a symplectic manifold}
Let $M = S^2$ as an embedded manifold in $\R^3$.
$S^2 \equiv \{ v \in \R^3, |v| = 1 \}$.
$T_p S^2 \equiv \{ w \in R^3 | p \cdot w = 0 \}$.
This works because $p$ is normal to the plane spanned by $T_p S^2$.
We define $\omega_p(u, v) \equiv p \cdot (u \times v)$. Clearly, this is
a 2-form since it is bilinear and anti-symmetric. $\omega$ is also closed,
since there cannot be any degree 3 forms on a 2D manifold! Hence, $d\omega = 0$.

We need to check that it is non-degenrate. For any point $p \in S^2$, 
$v \in T_p S^2$, we can take $u = p \times v$. Now, $\omega_p(u, v) = p \cdot(u \times v)$ will be
a non-degenrate area of a parallelopiped. The parallelopiped is non-degenrate 
as $v, p$ are perpendicular by definition. $u$ is perpendicular to both $v$
and $p$ by construction (cross-product). (TODO: draw picture).

\subsection{Mapping between Symplectic Manifolds}
Let $(M, \Omega)$ and $(M', \Omega')$ are Symplectomorphic if $\phi: M \rightarrow M'$
is diffeomorphic, and such that $\phi^* \Omega' = \Omega$.


\subsection{Symplectic manifolds are locally like $\R^{2n}$}
\begin{theorem} (Darboux): Let $(M, \omega)$ be a $2n$ dimensional symplectic
    manifold. Let $p \in M$. Then, there is a coordinate chart called 
    as the Darboux chart $U$ with basis $X_1, \dots X_n, Y_1 \dots Y_n$ such that
    the two-form $\omega = \sum_i dx_i \wedge dy_i$.
\end{theorem}

% Lecture 9: https://www.youtube.com/watch?v=hAX7ZCMM2kQ
\section{The cotangent bundle and Symplectic forms}

Let $X$ be an n-dimensional manifold. Let $M \equiv T^*X$ be its cotangent
bundle.  $M$ is also a manifold.

Define coordinate charts on $M$, $(T^*U: X_1, X_2, \dots X_n, \xi_1, \xi_2, \dots \xi_n)$,
where $U$ is a chart on $X$, $X_1, \dots X_n$ are coordinate functions for $U$,
and $\xi_1, \dots xi_n$ are coordinates for $T^*X_{x_0}$ where $x_0 \in X$, defined
by $\xi_{x_0} = \sum_i \xi_i {dx_i}_{x_0}$.

We need to show that the transition functions are smooth.

Let $p \in U \cap U'$. We need to express coordinates in one chart
as a smooth function of coordinates in another chart. $(U, X_1, \dots X_n)$
, $(U', X_1', \dots, X_n')$ are coordinates, and let $\xi \in T_p^* M$.
$(p, \xi) \in T^*X \equiv M$.


\begin{align*}
    \xi = \sum_i \xi_i dx_i = \sum_{i, k} \xi_i \frac{\partial x_i}{\partial x'_k} dx'_k = \sum_k \xi'_k dx'_k
\end{align*}

where the $\xi'_k = \xi_i \frac{\partial x_i}{\partial x'_k}$ are smooth since
the $\xi_i$ are smooth, and that the transition maps derivaties are smooth
since the charts are smooth.

\subsection{Canonical Symplectic structure of contangent bundle}

The physicist intuition for the following is that when we consider
a particle moving on a manifold $X$. To fully specify the state of the
particle, we need both position of the particle $x \in X$, and also
momentum $p \in T^*X$ (why does it belong to contangent bundle? Intuition:
given a velocity, it returns the momentum along it??? We can also supposedly
look at how momentum transforms).
This data is called "phase space" by physicists. Mathematically, this
is the cotangent bundle.

Now, we will see the canonical symplectic structure on the cotangent bundle.
\subsection{Tautological and Canonical forms}
Let $(U, x_1, \dots x_n)$ be a coordinate chart of $X$ and 
$(T^*U, x_1, \dots x_n, \xi_1, \dots \xi_n)$ 
be the corresponding chart of $M \equiv T^*X$. We will now define a 2-form
on $M$ on the chart $T^*U$ via:
\begin{align*}
    w = \sum_j dx_j \wedge d\xi_j
\end{align*}

\begin{theorem}
     This definition is coordinate independent
 \end{theorem}
 \begin{proof}
     Consider a one-form on $T^*U \subset M$. Let $\alpha = \sum_j \xi_j dx_j$ 
     This is called as the \emph{tautological form}.
     $\alpha$ is a one-form on $M \equiv T^*X$, when $T^*X$ is treated as a manifold.
     This is also a point on $T^* X$.
     Note that $d \alpha = - \omega$. But $\alpha$ is intrinsically defined.
     $\alpha =  \sum_j \xi_j dx_j = \sum_j \xi'_j dx'_j$.

     TODO: I don't understand why this gives us coordinate independence.
 \end{proof}

\subsubsection{Coordinate-free definition}

Let $M \equiv T^* X$. There is a canonical map $\pi: M \rightarrow X$, 
$\pi(x, \xi) = x$. We are going to pullback $T^* X$ along $T^* M = T^* (T^* X)$
$(d\pi)^*: T^*X \rightarrow T^*M$.
Let $p \in M; p = (x, \xi)$. $\xi \in T_x^* X$.
We define $\alpha$ pointwise. $\forall p \in M, \alpha_p \equiv  d\pi_p^* \xi_{\pi(p)}$.
Equivalently, let $v_p \in T_p M \equiv T_p (T^*X)$. Now, $\alpha_p (v_p) \equiv \xi( d\pi_p (v_p))$.

TODO: draw example (https://www.youtube.com/watch?v=hAX7ZCMM2kQ, 43:52)

\subsubsection{Example 1}
Let $X = \R$. Now, $M = \R \times \R$. $(x, y) \in M$ (position-momentum).
$\pi: M \rightarrow X; \pi((x, y)) = x$.

\begin{align*}
&\alpha_{(x, y)} \equiv y dx 
\qquad \omega = -d\alpha = - (\partial_y y \cdot dy \wedge dx) = dx \wedge dy \\
&\alpha_{(x, y)}(v_x, v_y) = y dx (v_x \partial_x + v_y \partial_y) = y v_x
\end{align*}


\subsubsection{Example 1}
Let $X = S^1$. Now, $M = S^1 \times \R$. $(x, y) \in M$ (position-momentum).
$\pi: M \rightarrow X; \pi((\theta, v)) = \theta$.

\begin{align*}
&\alpha_{(x, y)} \equiv y d\theta \\
&\omega_{(x, y)} \equiv d\theta \wedge dy
\end{align*}

\subsection{Naturality of Tautological form}
Let $X_1, X_2$ be n-dimensional manifolds with $T^* X_1$ and $T^* X_2$ as cotangent
bundles. Let $f: X_1 \rightarrow X_2$ is a diffeomorphism. We will show that
the tautological manifolds also match.

We show that there is a diffeomorphism $f_{\sharp}: T^*X_1 \rightarrow T^* X_2$
which lifts $f$. ie, if $f_{\sharp}((x, \xi)) = (x', \xi')$, where $x' = f(x)$,
$\xi = df^* \xi'$.

\begin{theorem} The lift $f_\sharp$ of $f: X_1 \rightarrow X_2$ pulls the
tautological form on $M_2 \equiv \coT X_2$ onto the tautological form on
$M_1 \equiv \coT X_1$.
\end{theorem}
\begin{proof}
Pointwise, we wish to show that $(df_\sharp^\star)_{p_1} (\alpha_2)_{p_2} = (\alpha_1)_{p_1}$
where $p_2 = f^\sharp p_1$.

$p_2 = f^\sharp p_1 \implies p2 = (x_2, \xi_2), x_2 = f(x_1) \land df^\star(\xi_1) = \xi_2$.

TODO!
\end{proof}

\section{Vector fields and flows}

\end{document}
