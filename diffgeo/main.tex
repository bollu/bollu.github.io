\documentclass[11pt]{book}
%\documentclass[10pt]{llncs}
%\usepackage{llncsdoc}
\usepackage{physics}
\usepackage{amsmath,amssymb}
\usepackage{graphicx}
\usepackage{makeidx}
\usepackage{algpseudocode}
\usepackage{algorithm}
\usepackage{listing}
\evensidemargin=0.20in
\oddsidemargin=0.20in
\topmargin=0.2in
%\headheight=0.0in
%\headsep=0.0in
%\setlength{\parskip}{0mm}
%\setlength{\parindent}{4mm}
\setlength{\textwidth}{6.4in}
\setlength{\textheight}{8.5in}
%\leftmargin -2in
%\setlength{\rightmargin}{-2in}
%\usepackage{epsf}
%\usepackage{url}

\usepackage{booktabs}   %% For formal tables:
                        %% http://ctan.org/pkg/booktabs
\usepackage{subcaption} %% For complex figures with subfigures/subcaptions
                        %% http://ctan.org/pkg/subcaption
\usepackage{enumitem}
%\usepackage{minted}
%\newminted{fortran}{fontsize=\footnotesize}

\usepackage{xargs}
\usepackage[colorinlistoftodos,prependcaption,textsize=tiny]{todonotes}

\usepackage{hyperref}
\hypersetup{
    colorlinks,
    citecolor=black,
    filecolor=black,
    linkcolor=black,
    urlcolor=black
}

\usepackage{epsfig}
\usepackage{tabularx}
\usepackage{latexsym}
\newcommand\ddfrac[2]{\frac{\displaystyle #1}{\displaystyle #2}}

\def\qed{$\Box$}
\newtheorem{corollary}{Corollary}
\newtheorem{theorem}{Theorem}
\newtheorem{definition}{Definition}
\newtheorem{lemma}{Lemma}
\newtheorem{observation}{Observation}
\newtheorem{proof}{Proof}

\title{General Relativity and Differential Geometry}
\author{Siddharth Bhat}
\date{Monsoon 2019}

\begin{document}
\maketitle
\tableofcontents

\chapter{Introduction}

I am following the following sources:

\begin{itemize}
    \item Susskind's General Relativity lectures as part of the Theoretical
        minimum: (\href{https://theoreticalminimum.com/courses/general-relativity/}{The link is here}).
        Note that the videos do not load. However, one can view the source to access the link
        to the iframe.
    \item Susskind's other General Relativity lectures, as part of his modern
        physics course: (\href{https://www.youtube.com/watch?v=hbmf0bB38h0&list=PL6C8BDEEBA6BDC78D}{link to playlist here}).
        These seem to be taught at a much gentler pace.
    \item The book Gravitation by Misner, Thorne and wheeler.
\end{itemize}

The notes as scribed here are a mix from all of these sources, as well
as tangential points I find interesting.

\section{The equivalence principle}

Gravity is in some sense the same thing as acceleration. First, an elementary
derivation which formalizes the intuition of the equivalence principle.

We consider an elevator moving upward. Let its distance from the bottom be $L(t)$.

\begin{align*}
    z' = z - L(t) \quad t' = t \quad x' = x
\end{align*}

\begin{align*}
F = m\frac{d^2 z'}{dt^2} \\
F = m\frac{d^2 (z - L(t))}{dt^2} \\
F = m\frac{d^2 z}{dt^2} \text{(If \frac{d^2L(t)}{dt^2} - 0)}
\end{align*}

if $\frac{d^2 L(t)}{dt^2} = 0$, then the force is the same in the new coordinate
system as that of the old coordinate system

\section{Galileo's theory of flat space and gravitation}
Newton's laws:
\begin{align*}
    &\vec F = m \vec a \\
    &\vec F = m \frac{d^2 x}{dt^2} \\
\end{align*}

Galileo's gravitation, under the approximation that the earth is flat:
If we pick downwards to be negative direction along the $2$ dimension, then his
equation can be written as ${F_2 = - m g}$ where $g$ is a constant.

This is special, because the force is proportional to the mass, which is not
the case of things like electromagnetism.


Combining the two equations, we get ${m \frac{d^2x}{dt^2} = - m g}$, or
${\frac{d^2x}{dt^2} = - m g}$.


That the acceleration induced by the graviational force is independent of the
mass of the object is known as the \textit{equivalence principle}. At this
stage, we can say that gravity is equivalent across all objects independent of
their mass.

Let's now consider a collection of point masses --- A diffuse cloud of
particles, and have it fall. Different particles maybe heavy, light, large,
small. However, since all of them have the same acceleration, the point cloud
looks unchanged as it falls. That is, the object will have no stresses or
strains as it falls. We can't tell by looking at our neighbours that there is a
force being exterted on us, since all our neighbours are moving along with us!
We cannot tell the difference between being in free space versus being in a
graviational field.

\section{Newton's theory of gravity}
all bodies exert equal and opposite forces on each other. Given two bodies $a$
and $b$ of masses $m$ and $M$ with distance $R$, the force on $a$ is
${F_a \equiv \frac{GmM}{R^2}} \hat{r}$. where $\hat r$ is the direction from $a$
to $b$.

Again, we can prove that the acceleration of an object $a$ does not depend on
its own mass.

Now that gravitation depends on distance, we can actually feel something if we
are in a gravitational field, since different parts of a given object will have
a different force on it, due to the varying distance from (say) the earth.

Gravitational field is defined as the force exerted on a test mass at every
point in space.

\subsubsection{Gauss' theorem}
${\int  \grad \cdot A dx dy dz  = \int A_{\bot} d\sigma}$ where $\sigma$ is the
differential unit of surface area of the surface. 

Show that the divergence of a field in 3 dimensions will lead to an inverse
square law.


\section{Geometry and curvature}

To describe a geometry, all we need is the distance between neighboring points
on a blackboard. In general, given a parametrization, we can draw a possibly
distorted grid of lines of constant corrdinate. The distance between two points
${(x, y)}$ and ${(x + dx, y+ dy)}$ will be ${ds^2 = g_{11} dx^2 + 2 g_{12} d_x d_y + g_{22} d_y^2}$.



\end{document}
