\documentclass[11pt]{article}
\usepackage[sc,osf]{mathpazo}   % With old-style figures and real smallcaps.
\linespread{1.025}              % Palatino leads a little more leading
% Euler for math and numbers
\usepackage[euler-digits,small]{eulervm}
\usepackage{amsmath}
\usepackage{amssymb}
\usepackage{physics}
\usepackage{tikz}
\usepackage{fancyhdr}
\newcommand{\Z}{\mathbb{Z}}
\newcommand{\R}{\mathbb{R}}
\author{Siddharth Bhat(20161105)}
\title{Optimization assignment -- Linear algebra 2}
\date{\today}

\pagestyle{fancy}
\fancyhf{}
\lhead{Siddharth Bhat (20161105)}
\rfoot{Page \thepage}

\begin{document}
\maketitle
\thispagestyle{fancy}
Prove that $(a+b)^T(a-b)^T = \norm{a}^2 + \norm{b}^2$. This question is
wrong. Let us assume the dimension of $a, b$ is $1 \times n$. Hence, $a +b, a- b$
have dimensions $n \times 1$, and can consequently not be matrix multiplied,
let alone produce a scalar as output.

The question should probably be: prove that $(a+b)(a-b)^T = \norm{a}^2 - \norm{b}^2$.
\begin{align*}
    (a+b)(a-b)^T = &
    \begin{bmatrix} a_1 + b_1 & a_2 + b_2 & \dots & a_n + b_n \end{bmatrix}
    \begin{bmatrix} a_1 - b_1 \\ a_2 - b_2 \\ \dots \\ a_n - b_n \end{bmatrix}   \\
    &= (a_1 + b_1) (a_1 - b_1) +(a_2 + b_2) (a_2 - b_2) + \dots + (a_n + b_n) (a_n - b_n) \\
    &= \sum_i a_i^2 - b_i^2 = \sum_i a_i^2 - \sum_i b_i^2 = \norm{a}^2 - \norm{b}^2
\end{align*}
\end{document}

