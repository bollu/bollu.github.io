\documentclass[11pt]{book}
%\documentclass[10pt]{llncs}
%\usepackage{llncsdoc}
\usepackage[sc,osf]{mathpazo}   % With old-style figures and real smallcaps.
\linespread{1.025}              % Palatino leads a little more leading
% Euler for math and numbers
\usepackage[euler-digits,small]{eulervm}
\usepackage{bbding} % for flower. 
\usepackage{physics}
\usepackage{amsmath,amssymb}
\usepackage{graphicx}
\usepackage{makeidx}
\usepackage{algpseudocode}
\usepackage{algorithm}
\usepackage{listing}
\usepackage{minted}
\usepackage{cancel}
% \usepackage{quiver}
\evensidemargin=0.20in
\oddsidemargin=0.20in
\topmargin=0.2in
%\headheight=0.0in
%\headsep=0.0in
%\setlength{\parskip}{0mm}
%\setlength{\parindent}{4mm}
\setlength{\textwidth}{6.4in}
\setlength{\textheight}{8.5in}
%\leftmargin -2in
%\setlength{\rightmargin}{-2in}
%\usepackage{epsf}
%\usepackage{url}

\usepackage{booktabs}   %% For formal tables:
                        %% http://ctan.org/pkg/booktabs
\usepackage{subcaption} %% For complex figures with subfigures/subcaptions
                        %% http://ctan.org/pkg/subcaption
\usepackage{enumitem}
%\usepackage{minted}
%\newminted{fortran}{fontsize=\footnotesize}

\usepackage{xargs}
\usepackage[colorinlistoftodos,prependcaption,textsize=tiny]{todonotes}

\usepackage{hyperref}
\hypersetup{
    colorlinks,
    citecolor=blue,
    filecolor=blue,
    linkcolor=blue,
    urlcolor=blue
}

\usepackage{epsfig}
\usepackage{tabularx}
\usepackage{latexsym}
\newcommand\ddfrac[2]{\frac{\displaystyle #1}{\displaystyle #2}}
\newcommand{\N}{\ensuremath{\mathbb{N}}}
\newcommand{\R}{\ensuremath{\mathbb R}}
\newcommand{\coT}{\ensuremath{T^*}}
\newcommand{\Lie}{\ensuremath{\mathfrak{L}}}
\newcommand{\Vectorfield}{\ensuremath{\mathfrak{X}}}
\newcommand{\pushforward}[1]{\ensuremath{{#1}_{\star}}}
\newcommand{\pullback}[1]{\ensuremath{{#1}^{\star}}}
\newcommand{\vectorfield}{\ensuremath{\mathfrak{X}}}

\newcommand{\pushfwd}[1]{\pushforward{#1}}
\newcommand{\pf}[1]{\pushfwd{#1}}

\newcommand{\boldX}{\ensuremath{\mathbf{X}}}
\newcommand{\boldY}{\ensuremath{\mathbf{Y}}}


\newcommand{\G}{\ensuremath{\mathcal{G}}}
% \newcommand{\braket}[2]{\ensuremath{\left\langle #1 \vert #2 \right\rangle}}


\def\qed{$\Box$}
\newtheorem{theorem}{Theorem}
\newtheorem{corollary}[theorem]{Corollary}
\newtheorem{definition}[theorem]{Definition}
\newtheorem{lemma}[theorem]{Lemma}
\newtheorem{observation}[theorem]{Observation}
\newtheorem{remark}[theorem]{Remark}
\newtheorem{example}[theorem]{Example}
\newtheorem{exercise}[theorem]{Exercise}

\newcommand{\beginproof}[1][]{\emph{Proof #1}\textbf{:} }
\newcommand{\question}[1]{\textbf{#1}}

\newcommand{\X}{\ensuremath{\mathfrak{X}}}

\title{Algebraic topology: Hatcher}
\author{Siddharth Bhat}
\date{Spring of the second Year of the Plague}


\begin{document}
\maketitle
\tableofcontents
\chapter{Categories, Functors, Natural transformations}
\section{Abstract and concrete categories}
\section{Duality}

\subsection{Musing}
How does one remember mono is is $gk = gl \implies k = l$ and vice versa?

\subsection{Solutions}
\question{Lemma 1.2.3} $f: x \to y$ is an isomorphism iff it defines a bijection $f_*: C(c, x) \to C(c, y)$.
\beginproof{($f$ is iso $\implies$ post composition with $f$ induces bijection)}
Let $f: x \to y$ be an isomorphism. Thus we have an inverse arrow $g: y \to x$ such that $fg = id_y$, $gf = id_x$.
The map: $$C(c, x) \xrightarrow{f*} C(c, y): (\alpha: c \to x) \mapsto (f\alpha: c \to y)$$
has a two sided inverse:

$$
C(c, y) \xrightarrow{g*} C(c, x): (\beta: c \to y) \mapsto (g\beta: c \to x)
$$

which can be checked as $g_*(f_*(\alpha)) = g_*(f\alpha) = gf\alpha = id_x\alpha = \alpha$, and similarly for $f_*(g_*(\beta))$.
Hence we are done, as the iso induces a bijection of hom-sets.
\qed


\beginproof{(post-composition with $f$ is bijection implies $f$ is iso)}
We are given that the post composition by $f$, $f_*: C(c, x) \rightarrow C(c, y)$ is a bijection.
We need to show that $f$ is an isomorphism, which means that there exists a function $g$ such that $fg = id_y$ and $gf = id_x$.
Since post-composition is a bijection for all $c$, pick $c = y$. This tells us that the post-composition 
$f_*: C(y, x) \rightarrow C(y, y)$ is a bijection. Since $id_y \in C(y, y)$, $id_y$ an inverse image $g \equiv f_*^{-1}(id_y)$. 
[We choose to call this map $g$]. By definition of $f_*^{-1}$, we have that $f_*(f_*^{-1}(id_y)) = id_y$ , which means
that $fg = id_y$. We also need to show that $gf = id_x$. To show this, consider $f_*(gf) = fgf = (fg)f = (1_y)f = f$.
We also have that $f_*(id_x) = f id_x = f$. Since $f_*$ is a bijection, we have that $id_x = gf$ and we are done.  \qed


\question{Q 1.2.ii:} Show that $f: x \rightarrow y$ is split epi iff for all $c \in C$, post composition
$f \circ - : C(c, x) \rightarrow C(c, y)$ is a surjection.
\beginproof
\qed

\question{Q 1.2.iii:} Mono is closed under composition, and if $gf$ is monic then so is $f$.


\beginproof[(Mono is closed under composition)]
Let $f: x \to y, g: y \to z$ be monomorphisms (Recall that $f$ is a monomorphism iff for any $\alpha, \beta$, if $f \alpha = f \beta$ then $\alpha = \beta$).
We are to show that $gf: x \to z$ is monic.
Consider this diagram which shows that $gfk = gfl$ for arbitrary $k, l: a \to x$. We wish to show that $k=l$.

\begin{minted}{text}
    a --k-> x --f--> y --g--> z
    a --l-> x --f--> y --g--> z
\end{minted}

Since $g$ is mono, we can cancel it from $gfk = gfl$, giving us $fk = fl$.
Since $f$ is mono, we can once again cancel it, giving us $k = l$ as desired.
Hence, we are done.  \qed.

\beginproof[(If $gf$ is monic then so is $f$)]
Let us assume that $fk = fl$ for arbitrary $l$. We wish to show that $k = l$. We show this
by applying $g$, giving us $fk = fl \implies gfk = gfl$. As $gf$ is monic, we can cancel, giving
us $gfk = gfl \implies k = l$. 
\qed.



\end{document}
