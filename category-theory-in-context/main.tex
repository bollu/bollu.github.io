% https://github.com/cohomolo-gy/cats-in-context/blob/master/chapter-2/Chapter%202%20Solutions.tex

\documentclass[14pt]{report}
\usepackage{classicthesis}

\usepackage{bbm}
\usepackage{bbding} % for flower. 
\usepackage{physics}
\usepackage{amsmath,amssymb}
\usepackage{graphicx}
\usepackage{makeidx}
\usepackage{algpseudocode}
\usepackage{algorithm}
\usepackage{listing}
\usepackage{minted}
\usepackage{cancel}
\usepackage{quiver}
\usepackage{booktabs}   %% For formal tables:
                        %% http://ctan.org/pkg/booktabs
\usepackage{subcaption} %% For complex figures with subfigures/subcaptions
                        %% http://ctan.org/pkg/subcaption
\usepackage{enumitem}
\usepackage{mathtools}
%\usepackage{minted}
%\newminted{fortran}{fontsize=\footnotesize}

\usepackage{xargs}
\usepackage[colorinlistoftodos,prependcaption,textsize=tiny]{todonotes}

\usepackage{hyperref}
\hypersetup{
    colorlinks,
    citecolor=blue,
    filecolor=blue,
    linkcolor=blue,
    urlcolor=blue
}

\usepackage{epsfig}
\usepackage{tabularx}
\usepackage{latexsym}
\newcommand\ddfrac[2]{\frac{\displaystyle #1}{\displaystyle #2}}
\newcommand{\N}{\ensuremath{\mathbb{N}}}
\newcommand{\Z}{\ensuremath{\mathbb{Z}}}
\newcommand{\Q}{\ensuremath{\mathbb{Q}}}
\newcommand{\R}{\ensuremath{\mathbb R}}
\newcommand{\coT}{\ensuremath{T^*}}
\newcommand{\Lie}{\ensuremath{\mathfrak{L}}}
\newcommand{\Vectorfield}{\ensuremath{\mathfrak{X}}}
\newcommand{\pushforward}[1]{\ensuremath{{#1}_{\star}}}
\newcommand{\pullback}[1]{\ensuremath{{#1}^{\star}}}
\newcommand{\vectorfield}{\ensuremath{\mathfrak{X}}}

\newcommand{\pushfwd}[1]{\pushforward{#1}}
\newcommand{\pf}[1]{\pushfwd{#1}}

\newcommand{\boldX}{\ensuremath{\mathbf{X}}}
\newcommand{\boldY}{\ensuremath{\mathbf{Y}}}


\newcommand{\cat}[1]{\mathsf{#1}}
\newcommand{\functor}[3]{#1 : \cat{#2} \to \cat{#3}}
\newcommand{\functordef}{\functor{F}{C}{D}}
\newcommand{\cc}{\cat{C}}
\renewcommand{\dd}{\cat{D}}
\newcommand{\ee}{\cat{E}}
\newcommand{\ccat}{\cat{Cat}}
\newcommand{\cset}{\cat{Set}}
\newcommand{\subcat}[2]{\bm{ \mathsf{#1}}_{\bm{ \mathsf{#2}}}}
\renewcommand{\op}[1]{#1^{\text{op}}}
\newcommand{\inv}[1]{#1^{-1}}
\newcommand{\opc}{\op{\cc}}
\newcommand{\opd}{\op{\dd}}
\newcommand{\ope}{\op{\ee}}
\newcommand{\mono}{\rightarrowtail}
\newcommand{\epi}{\twoheadrightarrow}
\newcommand{\bg}{\cat{BG}}
\newcommand{\bgg}{\cat{BG'}}
\newcommand{\nt}{\Rightarrow}
%\newcommand{\ant}[2]{\alpha : F \nt G} 
%\newcommand{\bnt}[2]{\beta : F \nt G} 
%\newcommand{\anti}[2]{\alpha : F \cong G} 
%\newcommand{\bnti}[2]{\beta : F \cong G} 
\newcommand{\zero}{\mathbbm{0}}
\newcommand{\one}{\mathbbm{1}}
\newcommand{\two}{\mathbbm{2}}
\newcommand{\three}{\mathbbm{3}}
\newcommand{\id}{\operatorname{id}}



\newcommand{\G}{\ensuremath{\mathcal{G}}}
% \newcommand{\braket}[2]{\ensuremath{\left\langle #1 \vert #2 \right\rangle}}


\def\qed{$\Box$}
\newtheorem{theorem}{Theorem}
\newtheorem{corollary}[theorem]{Corollary}
\newtheorem{definition}[theorem]{Definition}
\newtheorem{lemma}[theorem]{Lemma}
\newtheorem{observation}[theorem]{Observation}
\newtheorem{remark}[theorem]{Remark}
\newtheorem{example}[theorem]{Example}
\newtheorem{exercise}[theorem]{Exercise}
 
% \newcommand{\proof}[1][]{\emph{Proof #1}\textbf{:} }
\newcommand*{\start}[1]{\leavevmode\newline \textbf{#1} }
\newcommand*{\question}[1]{\leavevmode\newline \textbf{Question: #1.}}
\newcommand*{\proof}[1]{\leavevmode\newline \textbf{Proof #1}}
\newcommand*{\answer}{\leavevmode\newline \textbf{Answer} }

\newcommand{\X}{\ensuremath{\mathfrak{X}}}
\newcommand{\comma}{\downarrow}


% \newcommand{\hom}{\ensuremath{\operatorname{Hom}}}
% \newcommand{\Hom}{\ensuremath{\operatorname{Hom}}}

\title{Category theory in context}
\author{Siddharth Bhat}
\date{Monsoon, second year of the plague}


\begin{document}
\maketitle
\tableofcontents
\chapter{Categories, Functors, Natural transformations}
\section{Abstract and concrete categories}
\section{Duality}
\subsection{Musing}
How does one remember mono is is $gk = gl \implies k = l$ and vice versa?

\subsection{Solutions}
\question{Lemma 1.2.3} $f: x \to y$ is an isomorphism iff it defines a bijection $f_*: C(c, x) \to C(c, y)$.


\proof[($f$ is iso $\implies$ post composition with $f$ induces bijection)]
Let $f: x \to y$ be an isomorphism. Thus we have an inverse arrow $g: y \to x$ such that $fg = id_y$, $gf = id_x$.
The map: $$C(c, x) \xrightarrow{f*} C(c, y): (\alpha: c \to x) \mapsto (f\alpha: c \to y)$$
has a two sided inverse:

$$
C(c, y) \xrightarrow{g*} C(c, x): (\beta: c \to y) \mapsto (g\beta: c \to x)
$$

which can be checked as $g_*(f_*(\alpha)) = g_*(f\alpha) = gf\alpha = id_x\alpha = \alpha$, and similarly for $f_*(g_*(\beta))$.
Hence we are done, as the iso induces a bijection of hom-sets.
\qed


\proof[(post-composition with $f$ is bijection implies $f$ is iso)]
We are given that the post composition by $f$, $f_*: C(c, x) \rightarrow C(c, y)$ is a bijection.
We need to show that $f$ is an isomorphism, which means that there exists a function $g$ such that $fg = id_y$ and $gf = id_x$.
Since post-composition is a bijection for all $c$, pick $c = y$. This tells us that the post-composition 
$f_*: C(y, x) \rightarrow C(y, y)$ is a bijection. Since $id_y \in C(y, y)$, $id_y$ an inverse image $g \equiv f_*^{-1}(id_y)$. 
[We choose to call this map $g$]. By definition of $f_*^{-1}$, we have that $f_*(f_*^{-1}(id_y)) = id_y$ , which means
that $fg = id_y$. We also need to show that $gf = id_x$. To show this, consider $f_*(gf) = fgf = (fg)f = (1_y)f = f$.
We also have that $f_*(id_x) = f id_x = f$. Since $f_*$ is a bijection, we have that $id_x = gf$ and we are done.  \qed


\begin{minipage}{\textwidth}
\includegraphics[width=\textwidth]{ch1/iso-is-bijection-of-hom.png}
    \begin{center}Iso is bijection of hom-sets\end{center}
\end{minipage}

\question{Q 1.2.ii:} Show that $f: x \rightarrow y$ is split epi iff for all $c \in C$, post composition
$f \circ - : C(c, x) \rightarrow C(c, y)$ is a surjection.


\proof[(split epi implies post composition is surjective)]
Let $f: e \rightarrow b$ be split epi, and thus possess a section $s: b \rightarrow e$ such that $fs = id_b$.
We wish to show that post composition $C(c, e) \xrightarrow{f_*} C(c, b)$ is surjective.
So pick any $g \in C(c, b)$. Define $sg \in C(c, e)$. See: $$f_*(sg) = fsg = (fs)g = id_b g = g$$.
Hence, for all $g \in C(c, b)$ there exists a pre-image under $f_*$, $sg \in C(c, e)$. Thus, $f_*$ is surjective
since every element of codomain has a pre-image. \qed


\proof[(post composition is surjective implies split epi)]
Let $f: e \to b$ be a morphism such that for all $c \in C$, we have $C(c, e) \xrightarrow{f_*} C(c, b)$ is surjective.
We need to show that there exists a morpshism $s: b \rightarrow e$ such that $fs = id_b$. Set $c = b$.
This gives us a surjection $C(b, e) \xrightarrow{f_*} C(b, b)$. Pick an inverse image of $id_b \in C(b, b)$. 
That is, pick any function $s \in f_*^{-1}(id_b)$. By definition, of $s$ being in the fiber of $id_b$,
we have that $f_*(s) = fs = id_b$. Thus means that we have found a function $s$ such that $fs = id_b$. Thus we are done.
\qed

\question{Q 1.2.iii:} Mono is closed under composition, and if $gf$ is monic then so is $f$.


\proof[(Mono is closed under composition)]
Let $f: x \to y, g: y \to z$ be monomorphisms (Recall that $f$ is a monomorphism iff for any $\alpha, \beta$, if $f \alpha = f \beta$ then $\alpha = \beta$).
We are to show that $gf: x \to z$ is monic.
Consider this diagram which shows that $gfk = gfl$ for arbitrary $k, l: a \to x$. We wish to show that $k=l$.

\begin{minted}{text}
    a --k-> x --f--> y --g--> z
    a --l-> x --f--> y --g--> z
\end{minted}

Since $g$ is mono, we can cancel it from $gfk = gfl$, giving us $fk = fl$.
Since $f$ is mono, we can once again cancel it, giving us $k = l$ as desired.
Hence, we are done.  \qed.

\proof[(If $gf$ is monic then so is $f$)]
Let us assume that $fk = fl$ for arbitrary $l$. We wish to show that $k = l$. We show this
by applying $g$, giving us $fk = fl \implies gfk = gfl$. As $gf$ is monic, we can cancel, giving
us $gfk = gfl \implies k = l$. 
\qed.

\question{Q 1.2.iv} What are monomorphisms in category of fields?

\proof{} Claim: All morphisms are monomorphisms in the category of fields. Let $f: K \rightarrow L$ be an arbitrary field
morphism. Consider the kernel of $f$. It can either be $\{ 0 \}$ or $K$, since those are the only two
ideals of $K$. However, the kernel can't be $K$, since that would send $1$ to $0$ which is an illegal ring map.
Thus, the map $f$ has trivial kernel, therefore is an injection, is left-cancellable, is a monomorphism.\qed


\question{Q 1.2.v} Show that the ring map $i: \mathbb Z \rightarrow \mathbb Q$ is both monic and epic but not iso.

\proof[$i$ is not iso]
No ring map $i: \mathbb Z \rightarrow \mathbb Q$ can be iso since the rings are different (eg. $\mathbb Q$ is a field). \qed



\proof[$i$ is epic]
To show that it's epic, we must show that given for arbitrary $f, g: \mathbb Q \rightarrow R$ that $fi = gi$:

\begin{minted}{text}
Z -i-> Q --f--> R
Z -i-> Q --g--> R
\end{minted}

implies that $f = g$. Let $fi: \mathbb Z \rightarrow R = gi$. Then, the functions $f, g$ are uniquely determined
since $\mathbb Q$ is the field of fractions of $\mathbb Z$, thus a ring map $\Z \rightarrow R$ extends uniquely to a ring
map $\Q \rightarrow R$. Let's assume that $f(i(z)) = g(i(z))$ for all $z$, and show that $f = g$.
Consider arbitrary $p/q \in \mathbb \Q$ for $p, q \in \Z$. Let's evaluate:
\begin{align*}
f(p/q) = f(p)f(q)^{-1} = f(i(p)) \cdot f(i(q))^{-1} = g(i(p)) \cdot g(i(q))^{-1} = g(p/q)
\end{align*}
which shows that $f(p/q) = g(p/q)$ for all $p, q$. Thus, we can extend a ring function defined on the integers to rationals uniquely,
hence $fi = gi \implies f = g$ showing that $i$ is epic.
\qed

\proof[$i$ is monic]
given two arbitrary maps $k, l: R \rightarrow \Z$, if $ik = il$ then we must have $k = l$. Given $ik = il$, since $i$
is an injection of $\Z$ into $\Q$, we must have $k = $l.

\question{Q 1.2.vi} Mono + split epi iff iso.


\proof[Iso is mono + split epi]
Iso is both left and right cancellable. Hence it's mono and epi. It splits because the inverse of the iso splits it. \qed.


\proof[mono + split epi is iso]
Let $f: e \rightarrow b$ be mono (for all $k, l: p \to e$, $fk = fl \implies k = l$)
and split epi (there exists $s: b \rightarrow e$ such that $fs: b \rightarrow b = id_b$.
We need to show it's iso. That is, there exists a $g: b \rightarrow e$ such that $fg = id_b$ and $gf = id_e$.
I claim that $g \equiv s$. We already know that $fg = fs = id_b$ from $f$ being split epi. We need to 
check that $gf = sf = id_a$. Consider:

$$fsf = (fs)f = id_b f = f = f id_e$$

Hence, we have that $f(sf) = f(id_e)$. Since $f$ is mono, we concluce that $sf = id_e$. We are done 
since we have found a map $s$ such that $fs = id_b, sf = id_e$.


\question{1.2.vii} Regarding a poset a category, define the supremum of a subcollection, such that the dual gives the infimum.
\proof{}  We regard an arrow $a \to b$ as witnessing that $a \leq b$. First define an upper bound of a set $O$
to be an object $u$ such that for all $o \in O$, we have $o \leq u$. Now, the supremum of $O$ is the least upper bound
of $O$. That is, $s$ is a supremum iff $s$ is an upper bound, and for all other upper bounds $t$ of $O$, we have that $s \leq t$.
So we draw a diagram showing upper bounds and suprema:

\begin{minipage}{\textwidth}
\includegraphics[width=\textwidth]{ch1/sup.png}
    \begin{center}Upper bound and supremum\end{center}
\end{minipage}

\section{Functors}

\question{Exercise 1.3.i}  What is a functor between groups, when regarded as one-object categories?


\proof{} It's going to be a group homomorphism. Since, a functor preserves composition, we have that a
functor $F: C \rightarrow D$ preserves the group structure; for elements of the group / isos $f, g \in Hom(G, G)$, 
we have that the functor obeys $F(f \circ_G g) = (F f) \circ_H (F g)$, which is exactly the equation
we need to preserve group structure. For example, since a functor preserves isomorphisms, an element of the group $f \in Hom(G, G)$
is mapped to an inverbile element $F(f) \in Hom(H, H)$. \qed


\question{Exercise 1.3.ii}  What is a functor between preorders, regarded as a category?

\proof{}Going to be a preorder morphism. I don't know what these are called; If we had a partial order, these
would be called monotone maps. Recall that $a \rightarrow b$ is the encoding of $a \leq b$  within the category.
Suppose we have a functors between preorders (encoded as categories) $F: C \rightarrow D$. Since $F$ preserves
identity arrows, and $a \leq a$ is encoded as $id_a$, we have that $F(a) \leq F(a)$ as:

$$F(a \leq a) = F(id_a) = id_{F(a)} = F(a) \leq F(a)$$


Similarly, since functors take arrows to arrows, the fact that $a \leq b$ which is witnessed by an arrow $a \xrightarrow{f} b$
translates to an arrow $F(a) \xrightarrow{F f} F(b)$, which stands for the relation $F(a) \leq F(b)$. Thus, the map indeed
preserves the preorder structure. Preservation of composition of arrows preserves transitivity of the order relation. \qed


\question{Exercise 1.3.iii} Objects and morphisms in the image of a functor $F: C \rightarrow D$ do not necessarily define a subcategory of $D$.

\proof{} Recall that a morphism can \emph{smoosh} objects, thereby creating coalescing the domains and codomains
of arrows that used to be disjoint. Concretely, consider the diagram:

% https://q.uiver.app/?q=WzAsMTAsWzAsMCwiYSJdLFsxLDAsImIiXSxbMCwxLCJjIl0sWzEsMSwiZCJdLFszLDAsIngiXSxbMywxLCJ6Il0sWzQsMCwieSJdLFszLDIsIng6YSJdLFs0LDIsInk6YixjIl0sWzMsMywiejpkIl0sWzAsMSwiZiJdLFsyLDMsImciXSxbNCw1LCJnIFxcY2lyYyBmIiwyXSxbNiw1LCJsIl0sWzcsOCwiazpmIl0sWzcsOSwiZyBcXGNpcmMgZiIsMl0sWzgsOSwibDpnIl0sWzQsNiwiayJdXQ==
% https://q.uiver.app/?q=WzAsNCxbMCwwLCJhIl0sWzEsMCwiYiJdLFswLDEsImMiXSxbMSwxLCJkIl0sWzIsMywiZyJdLFswLDEsImYiXV0=
\[\begin{tikzcd}
	a & b \\
	c & d
	\arrow["g", from=2-1, to=2-2]
	\arrow["f", from=1-1, to=1-2]
\end{tikzcd}\]

Where we have a category of four objects $a, b, c, d$ with two disconnected arrow $f: a \to b$, and $g: c \to d$.
This is the domain of the functor we will build. The codomain is a three object category:

% https://q.uiver.app/?q=WzAsMyxbMCwwLCJ4Il0sWzEsMCwieSJdLFswLDEsInoiXSxbMCwxLCJrIl0sWzEsMiwibCJdLFswLDIsImxcXGNpcmMgayIsMl1d
\[\begin{tikzcd}
	x & y \\
	z
	\arrow["k", from=1-1, to=1-2]
	\arrow["l", from=1-2, to=2-1]
	\arrow["{l\circ k}"', from=1-1, to=2-1]
\end{tikzcd}\]


The functor will smoosh the four objects into three with a functor, which sends $a$ to $x$, both $b, c$ to $y$, and $d$ to $z$.
Now the image of the functor only has the arrows $k, l$, but not the composite $l \circ k$, which makes the image NOT a subcategory.

% https://q.uiver.app/?q=WzAsMyxbMCwwLCJ4OmEiXSxbMSwwLCJ5OmIsYyJdLFswLDEsIno6ZCJdLFswLDEsIms6ZiJdLFsxLDIsImw6ZyJdLFswLDIsImxcXGNpcmMgazogXFxfIiwyXV0=
\[\begin{tikzcd}
	{x:a} & {y:b,c} \\
	{z:d}
	\arrow["{k:f}", from=1-1, to=1-2]
	\arrow["{l:g}", from=1-2, to=2-1]
	\arrow["{l\circ k: \_}"', from=1-1, to=2-1]
\end{tikzcd}\]



\question{Exercise 1.3.iv} Very that the Hom-set construction is functorial.


\question{Exercise 1.3.v} What is the difference between a functor $F: C^{op} \rightarrow D$ and a functor $F: C \rightarrow D^{op}$?

\proof{} There is no difference. The functor $C^{op} \rightarrow D$ looks like:
% https://q.uiver.app/?q=WzAsNixbMSwwLCJiIl0sWzEsMSwiYSJdLFsyLDAsIkZhIl0sWzIsMSwiRmIiXSxbMCwwLCJhIl0sWzAsMSwiYiJdLFswLDJdLFsxLDNdLFswLDEsImZfe29wfSIsMCx7InN0eWxlIjp7ImJvZHkiOnsibmFtZSI6InNxdWlnZ2x5In19fV0sWzIsMywiRmZfe29wfSJdLFs0LDUsImYiLDJdXQ==
\[\begin{tikzcd}
	a & b & Fa \\
	b & a & Fb
	\arrow[from=1-2, to=1-3]
	\arrow[from=2-2, to=2-3]
	\arrow["{f_{op}}", squiggly, from=1-2, to=2-2]
	\arrow["{Ff_{op}}", from=1-3, to=2-3]
	\arrow["f"', from=1-1, to=2-1]
\end{tikzcd}\]

while the functor $G: D \rightarrow C^{op}$ looks like:

% https://q.uiver.app/?q=WzAsNixbMCwwLCJwIl0sWzAsMSwicSJdLFsxLDAsIkdwIl0sWzEsMSwiR3EiXSxbMiwwLCJHcCJdLFsyLDEsIkdxIl0sWzAsMSwiZiJdLFswLDJdLFsyLDMsIkdmIiwyLHsic3R5bGUiOnsiYm9keSI6eyJuYW1lIjoic3F1aWdnbHkifX19XSxbMSwzXSxbNSw0LCJHZiIsMl1d
\[\begin{tikzcd}
	p & Gp & Gp \\
	q & Gq & Gq
	\arrow["f", from=1-1, to=2-1]
	\arrow[from=1-1, to=1-2]
	\arrow["Gf"', squiggly, from=1-2, to=2-2]
	\arrow[from=2-1, to=2-2]
	\arrow["Gf"', from=2-3, to=1-3]
\end{tikzcd}\]

Given a functor $F: C^{op} \rightarrow D$, we can build an associated functor $G_F: C \rightarrow D^{op}$. Consider
an arrow $x \rightarrow{f} y \in C$ . Dualize it, giving us an arrow $y_{op} \xrightarrow{f_{op}} x_{op} \in C^{op}$. Find
it image under $F$, which gives us an arrow $F(y_{op}) \xrightarrow{F(f_{op})} F(x_{op}) \in D$. Dualize this
in $D$, giving us $F(x_{op})_{op} \xrightarrow{F(f_{op})}_{op} F(y_{op}) \in D^{op}$. See that the arrow
direction coincides with the domain arrow direction $x \rightarrow{f} y \in C$. So we can build a functor $H$
which sends the arrow $x \rightarrow{f} y \in C$ to the arrow $F(x_{op})_{op} \xrightarrow{F(f_{op})}_{op} F(y_{op}) \in D^{op}$.
Hence, $H: C \rightarrow D^{op}$, defined by $H(x) \equiv F(x_{op})_{op}$ and $H(f) \equiv F(f_{op})_{op}$. 
By duality, we get the other direction where we start from $F': C \rightarrow D^{op}$ and end at $H': C^{op} \rightarrow D$.
Thus, the two are equivalent.

In a nutshell, the diagram is:

% https://q.uiver.app/?q=WzAsMTUsWzEsMCwiYiJdLFsxLDEsImEiXSxbMiwwLCJGYiJdLFsyLDEsIkZhIl0sWzAsMCwiYSJdLFswLDEsImIiXSxbNCwwLCJhIl0sWzQsMSwiYiJdLFs1LDAsIkZhIl0sWzUsMSwiRmIiXSxbNywwLCJGYiJdLFs3LDEsIkZhIl0sWzMsMCwiXFxpbXBsaWVzIl0sWzMsMSwiXFxpbXBsaWVzIl0sWzYsMF0sWzAsMl0sWzEsM10sWzAsMSwiZl97b3B9IiwwLHsic3R5bGUiOnsiYm9keSI6eyJuYW1lIjoic3F1aWdnbHkifX19XSxbMiwzLCJGZl97b3B9Il0sWzQsNSwiZiIsMl0sWzYsNywiZiJdLFs2LDhdLFs4LDksIihGZilfe29wfSIsMCx7InN0eWxlIjp7ImJvZHkiOnsibmFtZSI6InNxdWlnZ2x5In19fV0sWzcsOV0sWzEwLDExLCJGZl97b3B9Il1d
\[\begin{tikzcd}
	a & b & Fb & \implies & a & Fa & {} & Fb \\
	b & a & Fa & \implies & b & Fb && Fa
	\arrow[from=1-2, to=1-3]
	\arrow[from=2-2, to=2-3]
	\arrow["{f_{op}}", squiggly, from=1-2, to=2-2]
	\arrow["{Ff_{op}}", from=1-3, to=2-3]
	\arrow["f"', from=1-1, to=2-1]
	\arrow["f", from=1-5, to=2-5]
	\arrow[from=1-5, to=1-6]
	\arrow["{(Ff)_{op}}", squiggly, from=1-6, to=2-6]
	\arrow[from=2-5, to=2-6]
	\arrow["{Ff_{op}}", from=1-8, to=2-8]
\end{tikzcd}\]



\question{Exercise 1.3.vi} Given the comma category $F \downarrow G$, define the domain and codomain projection functors $dom: F \comma G \rightarrow F$
  and $codom: F \comma G \rightarrow G$.


Recall that an object in the comma category is a a triple $(d \in D, e \in E, F(d) \xrightarrow{f} F(e))$, or diagramatically:

% https://q.uiver.app/?q=WzAsNCxbMCwwLCJkIFxcaW4gRCJdLFsxLDAsImUgXFxpbiBFIl0sWzAsMSwiRmRcXGluIEMiXSxbMSwxLCJHZSBcXGluIEMiXSxbMCwyLCJGOiBEICIsMl0sWzEsMywiRyJdLFsyLDMsImYiLDJdXQ==
\[\begin{tikzcd}
	{d \in D} & {e \in E} \\
	{Fd\in C} & {Ge \in C}
	\arrow["{F: D }"', from=1-1, to=2-1]
	\arrow["G", from=1-2, to=2-2]
	\arrow["f"', from=2-1, to=2-2]
\end{tikzcd}\]

and a morphism in such a category is a diagram:

% https://q.uiver.app/?q=WzAsNyxbMSwwLCJGZCJdLFsyLDAsIkdlIl0sWzEsMiwiRmQnIl0sWzIsMiwiR2UnIl0sWzAsMiwiKGQnLCBlJyxmJykiXSxbNSwyXSxbMCwwLCIoZCwgZSxmKSJdLFsyLDMsImYnIiwyXSxbMCwxLCJmIiwyXSxbMCwyLCJcXGFscGhhIiwxXSxbMSwzLCJcXGJldGEgIiwxXSxbNiw0LCIoXFxhbHBoYSBcXGRvd25hcnJvdyBcXGJldGEpIiwxXV0=
\[\begin{tikzcd}
	{(d, e,f)} & Fd & Ge \\
	\\
	{(d', e',f')} & {Fd'} & {Ge'} &&& {}
	\arrow["{f'}"', from=3-2, to=3-3]
	\arrow["f"', from=1-2, to=1-3]
	\arrow["\alpha"{description}, from=1-2, to=3-2]
	\arrow["{\beta }"{description}, from=1-3, to=3-3]
	\arrow["{(\alpha \downarrow \beta)}"{description}, from=1-1, to=3-1]
\end{tikzcd}\]

We constrct the domain functor $dom$ as a functor that sends an object $(d \in D, e \in E, F(d) \xrightarrow{f} F(e))$ to an object $d \in D$.
It sends the morphism between $(d, e, f)$ and $(d', e', f')$, given by $(\alpha : Fd \rightarrow Fd', \beta: Ge \rightarrow Ge')$ to
the arrow $Fd \xrightarrow{\alpha} Fd' \in D$.

In a diagram, this looks like:

% https://q.uiver.app/?q=WzAsMTAsWzEsMCwiRmQiXSxbMiwwLCJHZSJdLFsxLDIsIkZkJyJdLFsyLDIsIkdlJyJdLFswLDAsIihkLCBlLGYpIl0sWzAsMiwiKGQnLCBlJyxmJykiXSxbNSwyXSxbMywxLCJcXHhyaWdodGFycm93e2RvbX0iXSxbNCwwLCJGZCJdLFs0LDIsIkZkJyJdLFsyLDMsImYnIiwyXSxbMCwxLCJmIiwyXSxbMCwyLCJcXGFscGhhIiwxXSxbMSwzLCJcXGJldGEgIiwxXSxbNCw1LCIoXFxhbHBoYSBcXGRvd25hcnJvdyBcXGJldGEpIiwxXSxbOCw5LCJcXGFscGhhIiwxXV0=
\[\begin{tikzcd}
	{(d, e,f)} & Fd & Ge && Fd \\
	&&& {\xrightarrow{dom}} \\
	{(d', e',f')} & {Fd'} & {Ge'} && {Fd'} & {}
	\arrow["{f'}"', from=3-2, to=3-3]
	\arrow["f"', from=1-2, to=1-3]
	\arrow["\alpha"{description}, from=1-2, to=3-2]
	\arrow["{\beta }"{description}, from=1-3, to=3-3]
	\arrow["{(\alpha \downarrow \beta)}"{description}, from=1-1, to=3-1]
	\arrow["\alpha"{description}, from=1-5, to=3-5]
\end{tikzcd}\]

$codom$ will do the same thing, by stripping out the codomain of the comma instead of the domain. \qed

\question{Exercise 1.3.vii} Define slice category as special case of the comma category.


\proof{} To define the slice $C/c$ whose objects are of the form $d \rightarrow c$ for varying $d \in C$, we pick
the category $D = C, E = C$, and functors $F : C \rightarrow C = id$, $G : C \rightarrow C = \delta_c$, that is, the constant functor
which smooshes the entire $C$ category into the object $c \in C$ by mapping all objects to $c$ and all arrows to $id_{c}$.

This causes the diagram to collapse down to objects of the form $d \rightarrow c$, and the arrows to be what we'd expect \qed.

\question{Exercise 1.3.viii} Show that functors need not reflect isomorphisms. for a functor $F: C \rightarrow D$, and a 
morphisms $f \in C$ such that $F f$ is an isomorphism in $D$ but $f$ is not an isomorphism in $C$.


Pick a category $C$ and an object $o \in C$. Build the constant functor $\delta_o: C \rightarrow C$. The image
of every arrow $c \xrightarrow{a} c'$ is the identity arrow $id_o$ which is an iso. The arrow $a$ need not be iso. The functor
$\delta_o$ does not reflect isos. \qed

\question{Exercise 1.3.ix} Consider the not-yet-functors $Grp \rightarrow Grp$ that sends a group to its center, comutator subgroup, and automorphism group.
Are these functors if we limit the category $Grp$ to have (a) only isomorphisms? (b) only epimorphisms? (c) all homomorphisms?

\proof[(isos)] If we have (a) only isomorphisms, then these are indeed functors, since an isomorphism $G \simeq H$
implies that their group theoretic properties are identical. Thus, we will have
$Z(G) \simeq Z(H)$, ie, isomorphic centers.
Thus, an iso arrow $f: G \rightarrow H$ becomes an iso arrow
$Z(f) : Z(G) \rightarrow Z(H)$. The exact same happens for commutator and automorphism. \qed

\proof[(epis)] If we only have epimorphisms, we first invoke  given footnote 29, that all
epis in Group are surjections. Thus, given an epi (surjection) $\phi: G \twoheadrightarrow H$, we
identify $im(\phi) \simeq G/ker(\phi)$ or $H \simeq G/ker(\phi)$, since $H \simeq im(\phi)$ by $\phi$ being a surjection.
So we can choose to study only quotient maps $\phi: G \rightarrow G/ker \phi$.

For the center, consider the determinant map $|\cdot|: GL(2, \mathbb R) \rightarrow \mathbb R^\times$.
This map is surjective since we can pick the matrix $\begin{bmatrix} 1 & 0 \\ 0 & r \end{bmatrix}$ to get all possible
determinants  for arbitrary $r \in \mathbb R$. The center
of the group of matrices is scalar multiples of the identity, thus $Z(GL(2, \mathbb R)) = \{ k I : k \in \mathbb R \}$.
The center of the reals $Z(\mathbb R^\times)$ is the reals themselves since it's an abelian group. Now see
that the determinant of a matrix $kI$ must be $k^2$, since we get two copies of $k$ along the diagonal. Thus, the image
$\phi(Z(GL(2, \mathbb R))) = \{ k^2 : k \in \mathbb R \} = \mathbb R_{\geq 0}$ which is smaller than the center of the image,
$Z(\phi(GL(2, \mathbb R))) = Z(\mathbb R^{\times}) = \mathbb R^\times$. Thus, \textbf{the center not functorial on epis}.

\section{Natural Transformations}

\subsection{Musing}
\subsubsection{Torsion decomposition}

Let $TA$ be the subgroup of $A$ that have finite order.

\begin{itemize}
\item The idea is to first show that any natural transformation of the identity functor $\eta: 1 \implies 1$ is multiplication by some $n \in \Z$
(recall that every abelian group is a \Z-module, so this is a sensible thing to say).
\item Let's study the component of $\eta$ at $\mathbb Z$.
This means that we have an arrow at $1(\Z) \xrightarrow{\eta(id)} 1(\Z)$, which is $\Z \rightarrow{\eta(id)} \Z$ since identity functor
leaves objects and arrow invariant. Any arrow $\Z \xrightarrow{\eta(id)} \Z$ is a multiplication by some natural number.
\item Now consider a homomorphism $f: \Z \rightarrow A$. This is determined entirely by $f(1) \in A$, so any such map is 
  the same as picking an element $a \in A$. 
\item Let's now consider the isomorphism $A \epi A/TA \mono TA \oplus (A/TA) \simeq A$. If this isomorphism were natural,
    then we would have a natural endomorphism of the identity functor $\alpha: 1 \rightarrow 1$.
\item Let's observe $\alpha$ at $\mathbb Z$. We already know that such a transformation is given by $\Z \xrightarrow{\alpha} \Z$,
    which is multiplication b a number $n \neq 0$ (can't be zero since we need an isomorphism).
\item Now consider $C \equiv \Z/2n\Z$ where $n$ is the $\alpha$ scale factor. See that $T(\Z/2n\Z) = \Z/2n\Z$.
    So we get the factoring as $\Z/2n\Z \epi 0 \mono \Z/2n\Z \oplus 0 \simeq \Z/2n\Z$. Since we factor through zero,
    the full map is the zero map. However, we know from the natural transformation that the natural transformation 
    must scale all elements by $n \neq 0$. So we break naturality
\end{itemize}

The big thing I don't understand in this is why we need to factor \emph{through} the epi. If I directly define
$A \rightarrow (A/TA) \oplus TA$, given by the exact sequence $0 \mono TA \mono A \epi A/TA \epi 0$? Ah I see, this sequence
need not always split. 

\subsubsection{Walking arrow for unnatural isomorphism}

Consider the category $I \equiv (0 \rightarrow 1)$. Consider functors $F: I \rightarrow Vec(\R)$. The functor
picks out morphsisms between real vector spaces. If we consider endomorphisms, I could consider a functor $F_{id}$ that picked
out the identity map from $\R$ to $\R$, and another $F_{0}$ that picked out the constant linear function $f(x) = 0$ from $\R$ to $\R$.
These have the same domain and range, but the actual action of the arrow is wildly different. So, for a natural transformation
to be natural, it's not enough to have the same action on objects (clearly!)


\subsubsection{Permutations and total orderings for unnatural isomorphism}

Consider a subcategory of $Set$ containing only bijections.  Define the functor
$Perm: Set \rightarrow Set$ which takes a set $S$ to its set of permutations, where a permutation
is a bijection $S \rightarrow S$, and the functor $Ord: Set \rightarrow Set$ which takes a set $S$ to its total orderings, where a total ordering
is a bijection $\{1, 2, \dots |S|\} \rightarrow S$. We claim that there is no natural transformation between these two functors.
To see why, let us study the situation on the smallest non-trivial case, a two element set $\{a, b\}$.

With the chosen arrow as $id: [a \mapsto a; b \mapsto b]$, we get the commutative diagram for the naturality square as:

% https://q.uiver.app/?q=WzAsOCxbMCwxLCJbYSBcXG1hcHN0byBhOyBiIFxcbWFwc3RvIGJdIFthIFxcbWFwc3RvIGI7IGIgXFxtYXBzdG8gYV0iXSxbMCw0LCJbMSBcXG1hcHN0byBhOyAyIFxcbWFwc3RvIGJdWzEgXFxtYXBzdG8gYjsgMlxcbWFwc3RvIGFdIl0sWzUsMSwiW2EgXFxtYXBzdG8gYTsgYiBcXG1hcHN0byBiXSBbYSBcXG1hcHN0byBiOyBiIFxcbWFwc3RvIGFdIl0sWzAsMCwiaWRfQSBcXGVxdWl2W2EgXFxtYXBzdG8gYTsgYlxcbWFwc3RvIGJdIl0sWzUsNCwiWzEgXFxtYXBzdG8gYTsgMiBcXG1hcHN0byBiXVsxIFxcbWFwc3RvIGI7IDIgXFxtYXBzdG8gYV0iXSxbNSwzLCJbMVxcbWFwc3RvIGE7IDJcXG1hcHN0byBiXVsxIFxcbWFwc3RvIGI7IDIgXFxtYXBzdG8gYV0iXSxbMyw0XSxbNCw0XSxbMCwxLCJcXGV0YV9BIl0sWzIsNSwiXFxldGFfQSIsMl0sWzEsNCwiT3JkKGlkX0EpKGYpID1pZF9BIFxcY2lyYyBmID0gZiJdLFswLDIsIlBlcm0oaWRfQSkoZik9aWRfQSBcXGNpcmMgZiBcXGNpcmMgaWRfQV57LTF9ID0gZiIsMl0sWzUsNCwiXFx0ZXh0e2VxdWFsfSIsMl1d
\[\begin{tikzcd}
	{id_A \equiv[a \mapsto a; b\mapsto b]} \\
	{[a \mapsto a; b \mapsto b] [a \mapsto b; b \mapsto a]} &&&&& {[a \mapsto a; b \mapsto b] [a \mapsto b; b \mapsto a]} \\
	\\
	&&&&& {[1\mapsto a; 2\mapsto b][1 \mapsto b; 2 \mapsto a]} \\
	{[1 \mapsto a; 2 \mapsto b][1 \mapsto b; 2\mapsto a]} &&& {} & {} & {[1 \mapsto a; 2 \mapsto b][1 \mapsto b; 2 \mapsto a]}
	\arrow["{\eta_A}", from=2-1, to=5-1]
	\arrow["{\eta_A}"', from=2-6, to=4-6]
	\arrow["{Ord(id_A)(f) =id_A \circ f = f}", from=5-1, to=5-6]
	\arrow["{Perm(id_A)(f)=id_A \circ f \circ id_A^{-1} = f}"', from=2-1, to=2-6]
	\arrow["{\text{equal}}"', from=4-6, to=5-6]
\end{tikzcd}\]

While with the chosen arrow as $\sigma: [a \mapsto b; b \mapsto a]$ we get the non-commuting diagram for the naturality square as:

% https://q.uiver.app/?q=WzAsNixbMCwxLCJbYSBcXG1hcHN0byBhOyBiIFxcbWFwc3RvIGJdIFthIFxcbWFwc3RvIGI7IGIgXFxtYXBzdG8gYV0iXSxbMCw0LCJbMSBcXG1hcHN0byBhOyAyIFxcbWFwc3RvIGJdWzEgXFxtYXBzdG8gYjsgMlxcbWFwc3RvIGFdIl0sWzYsMSwiW2IgXFxtYXBzdG8gYjsgYSBcXG1hcHN0byBhXSBbYiBcXG1hcHN0byBhOyBhIFxcbWFwc3RvIGJdIl0sWzAsMCwiXFxzaWdtYSBcXGVxdWl2W2EgXFxtYXBzdG8gYjsgYlxcbWFwc3RvIGFdIl0sWzYsNCwiWzEgXFxtYXBzdG8gYjsgMiBcXG1hcHN0byBhXVsxIFxcbWFwc3RvIGE7IDIgXFxtYXBzdG8gYl0iXSxbNiwzLCJbMlxcbWFwc3RvIGI7IDEgXFxtYXBzdG8gYV1bMiBcXG1hcHN0byBhOyAxIFxcbWFwc3RvIGJdIl0sWzAsMSwiXFxldGFfQSJdLFswLDIsIlBlcm0oXFxzaWdtYSkoZikgPSBcXHNpZ21hIFxcY2lyYyBmIFxcY2lyYyBcXHNpZ21hXnstMX0iLDJdLFsxLDQsIk9yZChcXHNpZ21hKShmKSA9IFxcc2lnbWEgXFxjaXJjIGYiXSxbMiw1LCJcXGV0YV9BIiwyXSxbNSw0LCJcXHRleHR7bm90IGVxdWFsfSIsMix7InN0eWxlIjp7ImhlYWQiOnsibmFtZSI6Im5vbmUifX19XV0=
\[\begin{tikzcd}
	{\sigma \equiv[a \mapsto b; b\mapsto a]} \\
	{[a \mapsto a; b \mapsto b] [a \mapsto b; b \mapsto a]} &&&&&& {[b \mapsto b; a \mapsto a] [b \mapsto a; a \mapsto b]} \\
	\\
	&&&&&& {[2\mapsto b; 1 \mapsto a][2 \mapsto a; 1 \mapsto b]} \\
	{[1 \mapsto a; 2 \mapsto b][1 \mapsto b; 2\mapsto a]} &&&&&& {[1 \mapsto b; 2 \mapsto a][1 \mapsto a; 2 \mapsto b]}
	\arrow["{\eta_A}", from=2-1, to=5-1]
	\arrow["{Perm(\sigma)(f) = \sigma \circ f \circ \sigma^{-1}}"', from=2-1, to=2-7]
	\arrow["{Ord(\sigma)(f) = \sigma \circ f}", from=5-1, to=5-7]
	\arrow["{\eta_A}"', from=2-7, to=4-7]
	\arrow["{\text{not equal}}"', no head, from=4-7, to=5-7]
\end{tikzcd}\]

We see that we cannot define a single $\eta_A$ that works in both cases.

\subsubsection{Group as category v/s poset category}

in poset as category, objects carry most of the structure, not many arrows. In
group as category, only one object, many arrows.

\subsection{Exercises}

\question{Exercise 1.4.i} Let $\alpha: F \nt G$ be a natural isomorphism. Show that the inverses of the components define a natural isomorphism $\inv\alpha: G \nt F$.

We need to show that the square with $?$ in it commutes, given the square on top:

% https://q.uiver.app/?q=WzAsMTMsWzEsMCwiRngiXSxbMiwwLCJHeCJdLFsxLDEsIkZ5Il0sWzIsMSwiR3kiXSxbMCwwLCJ4Il0sWzAsMSwieSJdLFsyLDJdLFsxLDIsIkd4Il0sWzMsMiwiRngiXSxbMSw0LCJHeSJdLFszLDQsIkZ5Il0sWzIsMywiPyJdLFszLDNdLFswLDEsIlxcZXRhKHgpIl0sWzAsMiwiRmEiLDJdLFsyLDMsIlxcZXRhKHkpIiwyXSxbMSwzLCJHYSJdLFs0LDUsImEiLDJdLFs3LDksIkdhIiwyXSxbOCwxMCwiRmEiXSxbNyw4LCJcXGV0YV57LTF9KHgpIl0sWzksMTAsIlxcZXRhXnstMX0oeSkiLDJdXQ==
\[\begin{tikzcd}
	x & Fx & Gx \\
	y & Fy & Gy \\
	& Gx & {} & Fx \\
	&& {?} & {} \\
	& Gy && Fy
	\arrow["{\eta(x)}", from=1-2, to=1-3]
	\arrow["Fa"', from=1-2, to=2-2]
	\arrow["{\eta(y)}"', from=2-2, to=2-3]
	\arrow["Ga", from=1-3, to=2-3]
	\arrow["a"', from=1-1, to=2-1]
	\arrow["Ga"', from=3-2, to=5-2]
	\arrow["Fa", from=3-4, to=5-4]
	\arrow["{\eta^{-1}(x)}", from=3-2, to=3-4]
	\arrow["{\eta^{-1}(y)}"', from=5-2, to=5-4]
\end{tikzcd}\]

From the square, we know that $Ga \circ \eta(x) = \eta(y) \circ Fa$. Using inverses, we derive:

\begin{align*}
&Ga \circ \eta(x) = \eta(y) \circ Fa \\
&Ga \circ \eta(x) \circ \eta^{-1}(x) = \eta(y) \circ Fa \circ \eta^{-1}(x) \\
&Ga \circ id_x = \eta(y) \circ Fa \circ \eta^{-1}(x) \\
&Ga = \eta(y) \circ Fa \circ \eta^{-1}(x) \\
&\eta^{-1}(y) \circ Ga = \eta^{-1}(y) \circ \eta(y) \circ Fa \circ \eta^{-1}(x) \\
&\eta^{-1}(y) \circ Ga = id_y \circ Fa \circ \eta^{-1}(x) \\
&\eta^{-1}(y) \circ Ga = Fa \circ \eta^{-1}(x) \\
\end{align*}

which is exactly the diagram:

% https://q.uiver.app/?q=WzAsOSxbMywwXSxbMiwwLCJHeCJdLFs0LDAsIkZ4Il0sWzIsMiwiR3kiXSxbNCwyLCJGeSJdLFszLDEsIlxcZXRhXnstMX0oeSkgXFxjaXJjIEdhID0gRmEgXFxjaXJjIFxcZXRhXnstMX0oeCkiXSxbNCwxXSxbMCwwLCJ4Il0sWzAsMiwieSJdLFsxLDMsIkdhIiwyXSxbMiw0LCJGYSJdLFsxLDIsIlxcZXRhXnstMX0oeCkiXSxbMyw0LCJcXGV0YV57LTF9KHkpIiwyXSxbNyw4LCJhIiwyXV0=
\[\begin{tikzcd}
	x && Gx & {} & Fx \\
	&&& {\eta^{-1}(y) \circ Ga = Fa \circ \eta^{-1}(x)} & {} \\
	y && Gy && Fy
	\arrow["Ga"', from=1-3, to=3-3]
	\arrow["Fa", from=1-5, to=3-5]
	\arrow["{\eta^{-1}(x)}", from=1-3, to=1-5]
	\arrow["{\eta^{-1}(y)}"', from=3-3, to=3-5]
	\arrow["a"', from=1-1, to=3-1]
\end{tikzcd}\]


\question{Exercise 1.4.ii} What is a natural transformation between a parallel pair of functors between groups regarded as one object categories?

\proof{} Let $G, H$ be groups regarded as one object categories, so elements are arrows. A functor $F: G \rightarrow H$ is a group homomorphism. Two functors $F, F': G \rightarrow H$
are two group homomorphisms. An natural transformation is a map $\eta: G \rightarrow H$ which for every (the only) object $*_G \in G$, assigns
an arrow $\eta(*_G): F(*_G) \xrightarrow{\eta(*_G)} G(*_G)$ which is compatible with all arrows:

% https://q.uiver.app/?q=WzAsNSxbMCwwLCJGKCopIFxcaW4gSCJdLFsxLDAsIkYnKCopIFxcaW4gSCJdLFswLDEsIkYoKikgXFxpbiBIIl0sWzEsMSwiRicoKikgXFxpbiBIIl0sWzEsMl0sWzAsMSwiXFxldGEoKikiXSxbMiwzLCJcXGV0YSgqKSJdLFswLDIsIkYoaCkiLDFdLFsxLDMsIkYnKGgpIiwxXV0=
\[\begin{tikzcd}
	{F(*_G) \in H} & {F'(*_G) \in H} \\
	{F(*_G) \in H} & {F'(*_G) \in H} \\
	& {}
	\arrow["{\eta(*_G)}", from=1-1, to=1-2]
	\arrow["{\eta(*_G)}", from=2-1, to=2-2]
	\arrow["{F(g)}"{description}, from=1-1, to=2-1]
	\arrow["{F'(g)}"{description}, from=1-2, to=2-2]
\end{tikzcd}\]

Simplifying the diagram by substituting $F(*) = F'(*) = *$, and setting $\alpha \equiv \eta(*G) \in Hom(*_H, *_H)$, we get:

% https://q.uiver.app/?q=WzAsNSxbMCwwLCIqX0giXSxbMSwwLCIqX0giXSxbMCwxLCIqX0giXSxbMSwxLCIqX0giXSxbMSwyXSxbMCwxLCJcXGFscGhhIFxcZXF1aXYgXFxldGEoKl9HKSJdLFsyLDMsIlxcYWxwaGEgXFxlcXVpdiBcXGV0YSgqX0cpIiwyXSxbMSwzLCJGJyhoKSJdLFswLDIsIkYoaCkiLDJdXQ==
\[\begin{tikzcd}
	{*_H} & {*_H} \\
	{*_H} & {*_H} \\
	& {}
	\arrow["{\alpha \equiv \eta(*_G)}", from=1-1, to=1-2]
	\arrow["{\alpha \equiv \eta(*_G)}"', from=2-1, to=2-2]
	\arrow["{F'(g)}", from=1-2, to=2-2]
	\arrow["{F(g)}"', from=1-1, to=2-1]
\end{tikzcd}\]

So we are looking for an arrow (group element) $\alpha \in H$ such that for all $g \in G$, $F'(g) \cdot \alpha = \alpha \cdot F(g)$.
On rearranging: $\alpha^{-1} \cdot F'(g) \cdot \alpha = F(g)$. So it gives a sort of  ``inner automorphism'' from $F$ to $F'$. \qed


\question{Exercise 1.4.iii} What is a natural transformation between a parallel pair of functors between preorders regarded as categories?
\proof{} We regard preorders as thin categories, where there is an most arrow from $p \rightarrow p'$ if $p \leq p'$. A functor from $(P, \leq)$ to $(Q, \leq)$
is a monotone map. A pair of functors $F, G: P \rightarrow Q $ is a pair of monotone maps. A natural transformation $\eta: F \nt G$ makes for each $p \in $P
the diagram commute:

% https://q.uiver.app/?q=WzAsNixbMiwwLCJGKHApIl0sWzMsMCwiRyhwKSJdLFsyLDEsIkYocSkiXSxbMywxLCJHKHEpIl0sWzAsMCwicCJdLFswLDEsInEiXSxbMCwxLCJcXGV0YShwKSJdLFsyLDMsIlxcZXRhKHEpIl0sWzAsMiwiRihwPHEpIiwyXSxbMSwzLCJHKHAgPCBxKSJdLFs0LDUsInA8IHEiXV0=
\[\begin{tikzcd}
	p && {F(p)} & {G(p)} \\
	p' && {F(p')} & {G(p')}
	\arrow["{\eta(p)}", from=1-3, to=1-4]
	\arrow["{\eta(p)}", from=2-3, to=2-4]
	\arrow["{F(p<p')}"', from=1-3, to=2-3]
	\arrow["{G(p < p')}", from=1-4, to=2-4]
	\arrow["{p< p'}", from=1-1, to=2-1]
\end{tikzcd}\]

So, for every $p \leq p'$, the functor $F$ maps us to elements $F(p) \leq F(p')$, and $G$ maps us to elements $G(p) \leq G(p')$. The natural transformation $\eta$
asks to witness an arrow $F(p) \xrightarrow{\eta(p)} G(p)$, which means that we must have $F(p) \leq G(p)$ within the category $Q$, and similarly for $p'$.
Thus, it witnesses that $G$ is always \emph{above} $F$. For any element $p \in P$, we will always have $F(p) \leq G(p)$, in a way that is consistent with the
monotonicity of $F, G$.


\question{Exercise 1.4.iv} Prove that distinct parallel morphisms $f, g: c^\to_\to d$ define distinct natural transformations $f_*, g_*: C(-, c) \nt C(-, d)$ by pre-composition.

Recall that the natural transformation by $f_*$ is given for a fixed $o \xrightarrow{a} o'$ by $Hom(o, c) \xrightarrow{f_* \equiv f \circ - } Hom(o, d)$,
and similarly for $g_*$ by $Hom(o, c) \xrightarrow{g_* \equiv g \circ -} Hom(o, d)$. If we choose $o = c$, then we can consider $Hom(c, c)$.
Let' then see where $id_c \in Hom(c, c)$ gets mapped to:

\begin{align*}
&Hom(o, c) \xrightarrow{f_* \equiv f \circ - } Hom(o, d) \\
&Hom(o=c, c) \xrightarrow{_f* \equiv f \circ - } Hom(o=c, d) \\
&Hom(c, c) \xrightarrow{f_* \equiv f \circ - } Hom(c, d) \\
&id_c \in Hom(c, c) \xrightarrow{f_* \equiv f \circ - } f \circ id_c \in Hom(c, d) \\
&id_c \in Hom(c, c) \xrightarrow{f_* \equiv f \circ - } f \in Hom(c, d) \\
\end{align*}

So we map $id \in Hom(c, c)$ into $f \in Hom(c,d)$ by $f_*$. Since there was nothing special about $f$, we similarly map $id \in Hom(c, c)$
into $g \in Hom(c, d)$ by $g_*$. Since the two morphisms are distinct, we have $f \neq g$. 
Thus, the two distinct parallel morphisms $f, g$. natural transformations $f_*$ and $g_*$ are inequivalent since they have different components
on the element $c$: $f_*(c): Hom(c, c) \rightarrow Hom(c, d)$ is not the same action as $g_*(c): Hom(c, c) \rightarrow Hom(c, d)$, since they
act differently on $id_c \in Hom(c, c)$, as $f_*(c)(id_c) = f \neq g = g_*(c)(id_c)$.


\question{Exercise 1.4.v} Consider the comma cataegory $F \downarrow G$ for $F: D \rightarrow C, G: E \rightarrow C$. Construct a canonical natural transformation
$\alpha: F \circ dom \rightarrow G \circ codom$:

% https://q.uiver.app/?q=WzAsNSxbMCwwLCJGIFxcZG93bmFycm93IEciXSxbMiwwLCJFIl0sWzIsMiwiQyJdLFswLDIsIkQiXSxbMSwxLCJcXG5lYXJyb3dfXFxldGEiXSxbMCwxLCJjb2RvbSIsMl0sWzEsMiwiRyIsMl0sWzIsMywiRiIsMl0sWzMsMCwiZG9tIiwyXV0=
\[\begin{tikzcd}
	{F \downarrow G} && E \\
	& {\nearrow_\eta} \\
	D && C
	\arrow["codom"', from=1-1, to=1-3]
	\arrow["G"', from=1-3, to=3-3]
	\arrow["F"', from=3-3, to=3-1]
	\arrow["dom"', from=3-1, to=1-1]
\end{tikzcd}\]


\proof{}

Recall that elements  $k, k \in F \downarrow G$ and arrows $k \xrightarrow{a} k'$ is given by:

% https://q.uiver.app/?q=WzAsNixbMCwwLCJrXFxlcXVpdihkLGUsRmRcXHhyaWdodGFycm93e2Ffa31HZSkiXSxbMCwxLCJrJ1xcZXF1aXYoZCcsZScsRmQnXFx4cmlnaHRhcnJvd3thX3trJ319R2UnKSJdLFszLDAsIkZkIl0sWzMsMSwiRmQnIl0sWzQsMCwiR2UiXSxbNCwxLCJHZSciXSxbMCwxLCJhXFxlcXVpdihkXFx4cmlnaHRhcnJvd3thX2R9ZCcsZVxceHJpZ2h0YXJyb3d7YV9lfWUnKSJdLFsyLDMsIkYoYV9kKSIsMl0sWzIsNCwiYV9rIl0sWzMsNSwiYV9rJyIsMl0sWzQsNSwiRihhX2UnKSJdXQ==
\[\begin{tikzcd}
	{k\equiv(d,e,Fd\xrightarrow{a_k}Ge)} &&& Fd & Ge \\
	{k'\equiv(d',e',Fd'\xrightarrow{a_{k'}}Ge')} &&& {Fd'} & {Ge'}
	\arrow["{a\equiv(d\xrightarrow{a_d}d',e\xrightarrow{a_e}e')}", from=1-1, to=2-1]
	\arrow["{F(a_d)}"', from=1-4, to=2-4]
	\arrow["{a_k}", from=1-4, to=1-5]
	\arrow["{a_k'}"', from=2-4, to=2-5]
    \arrow["{G(a_e)}", from=1-5, to=2-5]
\end{tikzcd}\]

We need to make this diagram commute for all $k, k' \in F \downarrow G$

% https://q.uiver.app/?q=WzAsMTAsWzAsMCwiRlxcY2lyYyBkb20oaykiXSxbMCwyLCJGIFxcY2lyYyBkb20oaycpIl0sWzIsMCwiRyBcXGNpcmMgY29kb20oaykiXSxbMiwyLCJHIFxcY2lyYyBjb2RvbShrJykiXSxbMywxLCI9Il0sWzQsMCwiZCJdLFs0LDIsImQnIl0sWzYsMCwiZSJdLFs2LDIsImUnIl0sWzIsM10sWzAsMSwiRiBcXGNpcmMgZG9tKGEpIiwyXSxbMCwyLCJcXGV0YShrKSJdLFsyLDMsIkcgXFxjaXJjIGNvZG9tKGspIl0sWzEsMywiXFxldGEoaycpIiwyXSxbNSw2LCJGYV9kIl0sWzUsNywiXFxldGEoaykiLDJdLFs2LDgsIlxcZXRhKGsnKSJdLFs3LDgsIkdhX2UiLDJdXQ==
\[\begin{tikzcd}
	{F\circ dom(k)} && {G \circ codom(k)} && d && e \\
	&&& {=} \\
	{F \circ dom(k')} && {G \circ codom(k')} && {d'} && {e'} \\
	&& {}
	\arrow["{F \circ dom(a)}"', from=1-1, to=3-1]
	\arrow["{\eta(k)}", from=1-1, to=1-3]
	\arrow["{G \circ codom(k)}", from=1-3, to=3-3]
	\arrow["{\eta(k')}"', from=3-1, to=3-3]
	\arrow["{Fa_d}", from=1-5, to=3-5]
	\arrow["{\eta(k)}"', from=1-5, to=1-7]
	\arrow["{\eta(k')}", from=3-5, to=3-7]
	\arrow["{Ga_e}"', from=1-7, to=3-7]
\end{tikzcd}\]

To show the equality between the left square and right square, we simplify using the definitions of $k, k'$:
\begin{itemize}
    \item $k \equiv (d, e, Fd \xrightarrow{a_k} Ge)$, $k' \equiv (d', e' Fd' \xrightarrow{a_k'} Ge')$.
    \item $a : k \rightarrow k'$ is given by $a \equiv (d \xrightarrow{a_d} d', e \xrightarrow{a_e} e')$ such that the diagram commutes.
    \item $dom(a) = a_d$. $F(dom(a)) = F a_d$. Similarly, $codom(a) = a_e$, and $G(codom(a)) =G(a_e)$.
    \item $dom(k) = d$. $F(dom(k)) = Fd$. $codom(k) = e$. $G(codom(k)) = G(e)$.
\end{itemize}
By comparing the simplified naturality square to the square in the \emph{definition of arrow in the comma category}, we find
that we can pick $\eta(k) \equiv a_k$, and $\eta(k') \equiv a'_k$, the only data of $k$ and $k'$ we have not used so far!
This causes the diagram to commute by definition of what it means to have a morphism in a comma category. To be crystal
clear, we compare the two diagrams:

% https://q.uiver.app/?q=WzAsMTMsWzAsMSwiRmQiXSxbMCwzLCJGZCciXSxbMiwxLCJHZSJdLFsyLDMsIkdlJyJdLFs1LDEsImtcXGVxdWl2KGQsIGUsIEZkXFx4cmlnaHRhcnJvd3thX2t9R2UpIl0sWzUsMywiaydcXGVxdWl2KGQnLGUnRmQnXFx4cmlnaHRhcnJvd3thX3trJ319R2UnKSJdLFs1LDAsIlxcdGV4dHtpbn59IEYgXFxkb3duYXJyb3cgRyJdLFsxLDAsIlxcdGV4dHtjb25kaXRpb24gZm9yfX5hIFxcdGV4dHsgaW4gfUMiXSxbMCw1LCJGZCJdLFswLDcsIkZkJyJdLFsyLDUsIkdlIl0sWzIsNywiR2UnIl0sWzEsNCwiXFx0ZXh0e2NvbmRpdGlvbiBmb3J9fiBcXGV0YX5cXHRleHR7aW4gQ30iXSxbMCwxLCJGYV9kIl0sWzIsMywiR2FfZSJdLFswLDIsImFfayIsMV0sWzEsMywiYSdfayIsMV0sWzQsNSwiYSBcXGVxdWl2IChkIFxceHJpZ2h0YXJyb3d7YV9kfWQnLCBlIFxceHJpZ2h0YXJyb3d7YV9lfSBlJykiLDJdLFs4LDksIkZhX2QiXSxbMTAsMTEsIkdhX2UiXSxbOSwxMSwiXFxldGEoaycpIiwyXSxbOCwxMCwiXFxldGEoaykiLDFdXQ==
\[\begin{tikzcd}
	& {\text{condition for}~a \text{ in }C} &&&& {\text{in~} F \downarrow G} \\
	Fd && Ge &&& {k\equiv(d, e, Fd\xrightarrow{a_k}Ge)} \\
	\\
	{Fd'} && {Ge'} &&& {k'\equiv(d',e'Fd'\xrightarrow{a_{k'}}Ge')} \\
	& {\text{condition for}~ \eta~\text{in C}} \\
	Fd && Ge \\
	\\
	{Fd'} && {Ge'}
	\arrow["{Fa_d}", from=2-1, to=4-1]
	\arrow["{Ga_e}", from=2-3, to=4-3]
	\arrow["{a_k}"{description}, from=2-1, to=2-3]
	\arrow["{a'_k}"{description}, from=4-1, to=4-3]
	\arrow["{a \equiv (d \xrightarrow{a_d}d', e \xrightarrow{a_e} e')}"', from=2-6, to=4-6]
	\arrow["{Fa_d}", from=6-1, to=8-1]
	\arrow["{Ga_e}", from=6-3, to=8-3]
	\arrow["{\eta(k')}"', from=8-1, to=8-3]
	\arrow["{\eta(k)}"{description}, from=6-1, to=6-3]
\end{tikzcd}\]

\question{Exercise 1.4.vi} Why do extranatural transforms need a common target?

I don't understand the question. We need the same common target category to have a common space for the diagrams to live.
But this feels too naive, so I'm not sure what it is I'm missing.

\section{1.5: Equivalence of categories}

\subsection{Musings}

\start{Proof: Equivalence of categories implies full, faithful, essentially surjective}

I reproduce the proof in a way that makes sense to me, since this feels like
the first somewhat non-trivial theorem we have proven.

\start{Equivalence is faithful:} Let us have two arrows $c \xrightarrow{p} d$ and $c \xrightarrow{q} d$. We wish to show that if
$Fc \xrightarrow{Fp} Fd$ equals $Fc \xrightarrow{Fq} Fd$ , then $p$ equals $q$. So $Fp = Fq \implies p = q$. The idea is to apply $G$
to get $GFp = GFq$, at which point we can apply $\eta: 1_C \rightarrow GF$ to convert from $GFp, GFq$ into $p, q$. Witness the
diagram:

% https://q.uiver.app/?q=WzAsMjMsWzAsMywiRmMiXSxbMCwwXSxbMCwyLCJGYyJdLFsxLDIsIkZkIl0sWzEsMywiRmQiXSxbMywyLCJHRmMiXSxbNCwyLCJHRmQiXSxbNCwzLCJHRmQiXSxbMywzLCJHRmMiXSxbMyw0LCJjIl0sWzQsNCwiZCJdLFszLDEsImMiXSxbNCwxLCJkIl0sWzUsMSwiYyJdLFs1LDIsIkdGX2MiXSxbNSwzLCJjIl0sWzYsMSwiZCJdLFs2LDIsIkdGX2QiXSxbNiwzLCJkIl0sWzcsMSwiYyJdLFs3LDIsImMiXSxbOCwyLCJkIl0sWzgsMSwiZCJdLFsyLDMsIkZwIl0sWzIsMCwiPSIsMix7InN0eWxlIjp7ImhlYWQiOnsibmFtZSI6Im5vbmUifX19XSxbMCw0LCJGcSIsMl0sWzMsNCwiPSIsMCx7InN0eWxlIjp7ImhlYWQiOnsibmFtZSI6Im5vbmUifX19XSxbNSw2LCJHRnAiXSxbNiw3LCI9IiwwLHsic3R5bGUiOnsiaGVhZCI6eyJuYW1lIjoibm9uZSJ9fX1dLFs1LDgsIj0iLDIseyJzdHlsZSI6eyJoZWFkIjp7Im5hbWUiOiJub25lIn19fV0sWzgsNywiR0ZxIiwyXSxbOSw4LCJcXGV0YV9jIiwyXSxbMTAsNywiXFxldGFfZCIsMl0sWzExLDEyLCIxcCJdLFs5LDEwLCJxIiwyXSxbMTEsNSwiXFxldGFfYyIsMl0sWzEyLDYsIlxcZXRhX2QiLDJdLFsxNCwxMywiXFxldGFfY157LTF9IiwyXSxbMTUsMTQsIlxcZXRhX2MiLDJdLFsxMywxNiwicCJdLFsxNywxNiwiXFxldGFfZF57LTF9Il0sWzE4LDE3LCJcXGV0YV9kIl0sWzE1LDE4LCJxIiwyXSxbMjAsMjEsInEiLDJdLFsyMCwxOSwiXFxvcGVyYXRvcm5hbWV7aWR9X2MiXSxbMjEsMjIsIlxcb3BlcmF0b3JuYW1le2lkfV9kIiwyXSxbMTksMjIsInAiXV0=
\[\begin{tikzcd}
	{} \\
	&&& c & d & c & d & c & d \\
	Fc & Fd && GFc & GFd & {GF_c} & {GF_d} & c & d \\
	Fc & Fd && GFc & GFd & c & d \\
	&&& c & d
	\arrow["Fp", from=3-1, to=3-2]
	\arrow["{=}"', no head, from=3-1, to=4-1]
	\arrow["Fq"', from=4-1, to=4-2]
	\arrow["{=}", no head, from=3-2, to=4-2]
	\arrow["GFp", from=3-4, to=3-5]
	\arrow["{=}", no head, from=3-5, to=4-5]
	\arrow["{=}"', no head, from=3-4, to=4-4]
	\arrow["GFq"', from=4-4, to=4-5]
	\arrow["{\eta_c}"', from=5-4, to=4-4]
	\arrow["{\eta_d}"', from=5-5, to=4-5]
	\arrow["1p", from=2-4, to=2-5]
	\arrow["q"', from=5-4, to=5-5]
	\arrow["{\eta_c}"', from=2-4, to=3-4]
	\arrow["{\eta_d}"', from=2-5, to=3-5]
	\arrow["{\eta_c^{-1}}"', from=3-6, to=2-6]
	\arrow["{\eta_c}"', from=4-6, to=3-6]
	\arrow["p", from=2-6, to=2-7]
	\arrow["{\eta_d^{-1}}", from=3-7, to=2-7]
	\arrow["{\eta_d}", from=4-7, to=3-7]
	\arrow["q"', from=4-6, to=4-7]
	\arrow["q"', from=3-8, to=3-9]
	\arrow["{\operatorname{id}_c}", from=3-8, to=2-8]
	\arrow["{\operatorname{id}_d}"', from=3-9, to=2-9]
	\arrow["p", from=2-8, to=2-9]
\end{tikzcd}\]

In text, the proof proceeds as:
\begin{itemize}
    \item Start by $Fc \xrightarrow{Fp} Fd = Fc \xrightarrow{Fq}$
    \item Augment by applying $\eta: 1 \nt FG$, $\eta^{-1}: FG \nt 1$ to the left and the right, giving $$(c \xrightarrow{p} d) \xRightarrow{\eta} (Fc \xrightarrow{Fp} Fd) = (Fc \xrightarrow{Fq} Fd) \xRightarrow{\eta^{-1}} (c \xrightarrow{q} d)$$
    \item Collapse along the equality, apply composition $\eta^{-1} \circ \eta  = id$ giving: $$(c \xrightarrow{p} d) \xRightarrow{id} (c \xrightarrow q d)$$
    \item Thus, we derive $p = q$ starting from $F p = F q$. \qed
\end{itemize}

\start{Equivalence is full:} Suppose we are given an arrow $(Fc \xrightarrow{q} Fc')$ (Note that this \textbf{does not} give us an arrow $(d \xrightarrow{q} d')$ --- we know
that the objects in question are in the image of the functor). We must show that there is a pre-image of the arrow $q$, so we expect an arrow $(c \xrightarrow{p} d)$ such that $Fp = q$.
Let's do the obvious thing, and pull back along $G$ to get:

% https://q.uiver.app/?q=WzAsMTEsWzAsMCwiRmMiXSxbMSwwLCJGZCJdLFswLDEsImMiXSxbMSwxLCJkIl0sWzAsMiwiR0ZjIl0sWzEsMiwiR0ZkIl0sWzIsMSwiYyJdLFs0LDEsImQiXSxbMiwyLCJHRmMiXSxbNCwyLCJHRmQiXSxbMiwzXSxbMCwxLCJxIl0sWzIsMywiPyJdLFsyLDQsIlxcZXRhX2MiLDJdLFs0LDUsIkdxIiwyXSxbMyw1LCJcXGV0YV9kIl0sWzYsNywicD1cXGV0YV57LTF9X2RcXGNpcmMgR3EgXFxjaXJjIFxcZXRhX2MiXSxbNiw4LCJcXGV0YV9jIiwyXSxbOCw5LCJHRnA9R3EiLDJdLFs5LDcsIlxcZXRhXnstMX1fZCIsMl1d
\[\begin{tikzcd}
	Fc & Fd \\
	c & d & c && d \\
	GFc & GFd & GFc && GFd \\
	&& {}
	\arrow["q", from=1-1, to=1-2]
	\arrow["{?}", from=2-1, to=2-2]
	\arrow["{\eta_c}"', from=2-1, to=3-1]
	\arrow["Gq"', from=3-1, to=3-2]
	\arrow["{\eta_d}", from=2-2, to=3-2]
	\arrow["{p=\eta^{-1}_d\circ Gq \circ \eta_c}", from=2-3, to=2-5]
	\arrow["{\eta_c}"', from=2-3, to=3-3]
	\arrow["{GFp=Gq}"', from=3-3, to=3-5]
	\arrow["{\eta^{-1}_d}"', from=3-5, to=2-5]
\end{tikzcd}\]

So we define an arrow $p \equiv \eta^{-1}_d \circ Gq \circ \eta_c$ since it seems to be the "right arrow" for our use case. By the commutativity
of the diagram, we have that $GFp = Gq$. Since $G$ is faithful (as proven above), we have $Fp = q$ and so we are done,
as we have established a pre-image arrow $p$ for the given $q$.

\start{Equivalence is essentially surjective:} Let $d \in D$. We must find a $c \in C$ such that $F(c) \simeq d$. Let's try the obvious
candidate, $G(d) \in C$. We get $F(G(d))$, which we must show is isomorphic to $d$. Recall that we have a natural isomorphism 
$\epsilon: FG \nt 1_D$. We invoke $\epsilon_d$ to get the isomorphism $FGd \xrightarrow{\epsilon_d} d$. It is invertible
since the isomorphism $\epsilon$ is invertible, with inverse arrow $d \xrightarrow{\epsilon^{-1}_d} FGd$ such that they are
inverses of each other.

\subsection{Exercises 1.5}

\question{Exercise 1.5.i}

First, let's recall the category $\two$:

% https://q.uiver.app/?q=WzAsMixbMCwwLCIwIl0sWzIsMCwiMSJdLFswLDEsIigwIFxccmlnaHRhcnJvdyAxKSJdXQ==
\[\begin{tikzcd}
	0 && 1
	\arrow["{(0 \rightarrow 1)}", from=1-1, to=1-3]
\end{tikzcd}\]

Now when we take the product of some category $C$ with $\mathbb 2$, get as objects $\cup_{c \in C} \{ (c, 0), (c, 1) \}$ and as arrows
we get three types:
\begin{itemize}
    \item  Cross arrows from $(-, 0)$ to $(-, 1)$: $\{ (c, 0) \xrightarrow{(a, 0 \rightarrow 1)} (d, 1) : c, d \in C; a \in Hom(c, d) \}$
    \item  Arrows within the component $(-, 0)$: $\{ (c, 0) \xrightarrow{(a, id_0)} (d, 0) : c, d \in C; a \in Hom(c, d) \}$
    \item  Arrows within the component $(-, 1)$: $\{ (c, 1) \xrightarrow{(a, id_1)} (d, 1) : c, d \in C; a \in Hom(c, d) \}$
\end{itemize}

If we now have a functor $H: C \times \two \rightarrow D$, we can recover the functors $F, G$ by considering the commutative square:

% https://q.uiver.app/?q=WzAsNSxbMCwwLCJIKGMsIDApIl0sWzIsMCwiSChkLCAwKSJdLFswLDIsIkgoYywgMSkiXSxbMiwyLCJIKGQgLCAxKSJdLFsxLDJdLFsyLDMsIkgoZiwgXFxvcGVyYXRvcm5hbWV7aWR9XzEpIl0sWzAsMSwiSChmLCBcXG9wZXJhdG9ybmFtZXtpZH1fMCkiXSxbMCwyLCJIKFxcb3BlcmF0b3JuYW1le2lkfV9jLCAwIFxccmlnaHRhcnJvdyAxKSIsMV0sWzEsMywiSChcXG9wZXJhdG9ybmFtZXtpZH1fZCwgMCBcXHJpZ2h0YXJyb3cgMSkiLDFdXQ==
\[\begin{tikzcd}
	{H(c, 0)} && {H(d, 0)} \\
	\\
	{H(c, 1)} & {} & {H(d , 1)}
	\arrow["{H(f, \operatorname{id}_1)}", from=3-1, to=3-3]
	\arrow["{H(f, \operatorname{id}_0)}", from=1-1, to=1-3]
	\arrow["{H(\operatorname{id}_c, 0 \rightarrow 1)}"{description}, from=1-1, to=3-1]
	\arrow["{H(\operatorname{id}_d, 0 \rightarrow 1)}"{description}, from=1-3, to=3-3]
\end{tikzcd}\]

Where the top row is $F$, bottom row is $G$, and top-to-bottom morpshism is the natural transformation $\eta$:
% https://q.uiver.app/?q=WzAsNSxbMCwwLCJGYyBcXHNpbWVxIEgoYywgMCkiXSxbMiwwLCJIKGQsIDApIFxcc2ltZXEgRmQiXSxbMCwyLCJHYyBcXHNpbWVxIEgoYywgMSkiXSxbMiwyLCJIKGQgLCAxKVxcc2ltZXEgR2QiXSxbMSwyXSxbMiwzLCJHZiBcXHNpbWVxIEgoZiwgXFxvcGVyYXRvcm5hbWV7aWR9XzEpIl0sWzAsMSwiRmYgXFxzaW1lcSBIKGYsIFxcb3BlcmF0b3JuYW1le2lkfV8wKSJdLFswLDIsIlxcZXRhX2MgXFxzaW1lcSBIKFxcb3BlcmF0b3JuYW1le2lkfV9jLCAwIFxccmlnaHRhcnJvdyAxKSIsMV0sWzEsMywiSChcXG9wZXJhdG9ybmFtZXtpZH1fZCwgMCBcXHJpZ2h0YXJyb3cgMSlcXHNpbWVxIFxcZXRhX2QiLDFdXQ==
\[\begin{tikzcd}
	{Fc \simeq H(c, 0)} && {H(d, 0) \simeq Fd} \\
	\\
	{Gc \simeq H(c, 1)} & {} & {H(d , 1)\simeq Gd}
	\arrow["{Gf \simeq H(f, \operatorname{id}_1)}", from=3-1, to=3-3]
	\arrow["{Ff \simeq H(f, \operatorname{id}_0)}", from=1-1, to=1-3]
	\arrow["{\eta_c \simeq H(\operatorname{id}_c, 0 \rightarrow 1)}"{description}, from=1-1, to=3-1]
	\arrow["{H(\operatorname{id}_d, 0 \rightarrow 1)\simeq \eta_d}"{description}, from=1-3, to=3-3]
\end{tikzcd}\]

I haven't drawn one arrow, that of $H(f, 0 \rightarrow 1)$. The diagram we have above only tells us that the arrows have the right shape.
It does not tell us that the diagram actually \emph{commutes}. We need to prove that $G f \circ eta_c = \eta_d \circ F f$. The crux is to show that
both of these are equal to $H(f, 0 \rightarrow 1)$ by functoriality of $H$:

% https://q.uiver.app/?q=WzAsNSxbMCwwLCJGYyBcXHNpbWVxIEgoYywgMCkiXSxbMiwwLCJIKGQsIDApIFxcc2ltZXEgRmQiXSxbMCwyLCJHYyBcXHNpbWVxIEgoYywgMSkiXSxbMiwyLCJIKGQgLCAxKVxcc2ltZXEgR2QiXSxbMSwyXSxbMiwzLCJHZiBcXHNpbWVxIEgoZiwgXFxvcGVyYXRvcm5hbWV7aWR9XzEpIl0sWzAsMSwiRmYgXFxzaW1lcSBIKGYsIFxcb3BlcmF0b3JuYW1le2lkfV8wKSJdLFswLDIsIlxcZXRhX2MgXFxzaW1lcSBIKFxcb3BlcmF0b3JuYW1le2lkfV9jLCAwIFxccmlnaHRhcnJvdyAxKSIsMl0sWzEsMywiSChcXG9wZXJhdG9ybmFtZXtpZH1fZCwgMCBcXHJpZ2h0YXJyb3cgMSlcXHNpbWVxIFxcZXRhX2QiXSxbMCwzLCJIKGYsIDAgXFxyaWdodGFycm93IDEpIiwxXV0=
\[\begin{tikzcd}
	{Fc \simeq H(c, 0)} && {H(d, 0) \simeq Fd} \\
	\\
	{Gc \simeq H(c, 1)} & {} & {H(d , 1)\simeq Gd}
	\arrow["{Gf \simeq H(f, \operatorname{id}_1)}", from=3-1, to=3-3]
	\arrow["{Ff \simeq H(f, \operatorname{id}_0)}", from=1-1, to=1-3]
	\arrow["{\eta_c \simeq H(\operatorname{id}_c, 0 \rightarrow 1)}"', from=1-1, to=3-1]
	\arrow["{H(\operatorname{id}_d, 0 \rightarrow 1)\simeq \eta_d}", from=1-3, to=3-3]
	\arrow["{H(f, 0 \rightarrow 1)}"{description}, from=1-1, to=3-3]
\end{tikzcd}\]

Since in the original category we have $f \circ id_c = f$ and $\id_1 \circ (0 \rightarrow 1) = \id_1$, we combine these equations
to get $(f,  id_1) \circ (id_c, 0 \rightarrow 1) = (f, 0 \rightarrow 1)$. Similarly, we show that $(id_d, 0 \rightarrow 1) \circ (f, id_0)  = (f, 0 \rightarrow 1)$.
Thus, the diagram does indeed commute, and what we have is a natural transformation.

\question{Exercise 1.5.ii}
Define a category $\Gamma$ whose objects are finite sets, and whose morphisms from $S$ to $T$ are maps $\theta: S \rightarrow 2^T$ where $\theta(\alpha)$ and $\theta(\beta)$
are disjoint when $\alpha \neq \beta$. The composite map is given by $\psi(\alpha) = \cup_{\beta \in \theta(\alpha)} \phi(\beta)$ [set/list monad]. Prove that $\Gamma$
is equivalent to the opposite of the category $Fin_*$ of finite pointed sets. 

\begin{itemize}
\item I can see why it is the opposite of finite sets.
\item The arrow $\theta: S \rightarrow 2^T$ records the data of fibers of maps $T \rightarrow S$.
\item Define $f_\theta: (T, t_*) \rightarrow (S, s_*)$ given by
$f(t) = s$ when $t \in \theta(s)$. We are guaranteed such an $s$ is unique since all sets $\theta(s)$ is disjoint.
\item At this stage, we also see why we need
pointed sets. If there is no $s$ such that $t \in \theta(s)$, then define $f(t)
        = s_*$, the basepoint of $S$. This is the ``basepoint encoding'' of
        partial functions.
\item The above shows that the functor is full and faithful (each arrow in
    $Fin_*$ has a corresponding unique arrow in $\Gamma$), and surjective (not
        just essentially surjective), and thus the functor is an equialence of
        categories.
\end{itemize}

\question{Exercise 1.5.iii}

Recall that the data of the isomorphism of objects $a \simeq a'$ is given by morphisms $\alpha: a \rightarrow a'$ and $\alpha^{-1}: a' \rightarrow a$
such that $\alpha^{-1} \circ \alpha: a \rightarrow a \simeq id_a$ and $\alpha \circ \alpha^{-1}: a' \rightarrow a' \simeq id_{a'}$. Similarly, posit
a $\beta$ to witness $b \simeq b'$. Now the square on the left gives us the equation $\beta \circ f \circ \alpha^{-1} = f'$. We compose with $\beta^{-1}, \alpha$
to get the other squares:

% https://q.uiver.app/?q=WzAsMTIsWzEsMSwiYSJdLFsyLDEsImEnIl0sWzEsMiwiYiJdLFsyLDIsImInIl0sWzMsMSwiYSJdLFs1LDEsImEnIl0sWzMsMywiZiJdLFs1LDMsImYnIl0sWzMsMF0sWzQsMiwiXFxiZXRhIFxcY2lyYyBmIFxcY2lyYyBcXGFscGhhXnstMX0gPWYnIl0sWzAsMSwiXFxhbHBoYSBcXGNpcmMgXFxhbHBoYV57LTF9ID0gaWQiXSxbMCwyLCJcXGJldGEgXFxjaXJjIFxcYmV0YV57LTF9ID0gaWQiXSxbMCwxLCJcXGFscGhhIl0sWzEsMCwiXFxhbHBoYV57LTF9IiwwLHsib2Zmc2V0IjotMn1dLFsyLDMsIlxcYmV0YSIsMCx7Im9mZnNldCI6LTJ9XSxbMywyLCJcXGJldGFeey0xfSIsMCx7Im9mZnNldCI6LTJ9XSxbNSw0LCJcXGFscGhhXnstMX0iLDJdLFs0LDYsImYiLDJdLFs2LDcsIlxcYmV0YSIsMl0sWzUsNywiZiciXV0=
\[\begin{tikzcd}
	&&& {} \\
	{\alpha \circ \alpha^{-1} = id} & a & {a'} & a && {a'} \\
	{\beta \circ \beta^{-1} = id} & b & {b'} && {\beta \circ f \circ \alpha^{-1} =f'} \\
	&&& f && {f'}
	\arrow["\alpha", from=2-2, to=2-3]
	\arrow["{\alpha^{-1}}", shift left=2, from=2-3, to=2-2]
	\arrow["\beta", shift left=2, from=3-2, to=3-3]
	\arrow["{\beta^{-1}}", shift left=2, from=3-3, to=3-2]
	\arrow["{\alpha^{-1}}"', from=2-6, to=2-4]
	\arrow["f"', from=2-4, to=4-4]
	\arrow["\beta"', from=4-4, to=4-6]
	\arrow["{f'}", from=2-6, to=4-6]
\end{tikzcd}\]


\begin{itemize}
    \item $\beta \circ f \circ \alpha^{-1} = f'$ implies $f \circ \alpha^{-1} = \beta^{-1} \circ f'$.
    \item $\beta \circ f \circ \alpha^{-1} = f'$ implies $\beta \circ f = \circ f' \circ \alpha$.
    \item $\beta \circ f \circ \alpha^{-1} = f'$ implies $f = \beta{-1} \circ f' \circ \alpha$.
\end{itemize}

\question{1.5.iv: Full faithful functor reflects and creates isos}

\start{Proof (a) reflects isos}

\begin{itemize}
\item Let $F: C \rightarrow D$ be a full and faithful functor. Let $f: x \rightarrow y$ be an arrow in $C$.
Let $Ff: Fx \rightarrow Fy$ be an isomorphism. We must show that $f$ is an isomorphism.
\item Since $Ff$ is an iso, let it have an inverse, say $h$ such that $(Ff) \circ h = id_{Fy}$ and $h \circ (Ff) = id_{Fx}$.
\item Since $F$ is full and faithful, there is a unique $g$ such that $Fg = h$.
\item This makes the iso condition $Ff \circ Fg = id_{Fy}$. By using functoriality (a) $Ff \circ Fg = F(f \circ g)$ and (b) $id_{Fy} = F(id_y)$, we get $F (f \circ g) = F(id_y)$.
\item Since $F$ is faithful, $F (f \circ g) = F(id_y) \implies  f \circ g = id_y$.
\item Repeat proof for other sided inverse. We are done.
\end{itemize}

\start{Proof (b) creates isos}

\begin{itemize}
\item Let $x, y \in C$ such that $Fx \simeq Fy$. Then we must show that $x
    \simeq y$.
\item $Fx \simeq Fy$ means that we have an isomorphism arrow $g: Fx \rightarrow Fy$ which has inverse $g^{-1}: Fy \rightarrow Fx$.
\item Since the functor is full and faithful, there exists unique $f_1: x \rightarrow y, f_2: y \rightarrow x$ such that $Ff_1 = g$ and $Ff_2 = g^{-1}$.
\item Repeat the previous proof to see that $f_1, f_2$ witness an iso between $x$ and $y$.
\end{itemize}


\question{1.5.v A faithful functor need not reflect isos}
\begin{itemize}
\item High level idea: take a faithful functor $F: C \rightarrow D$ adjoin arrows into $D$ to make arrows in $D$ isos, see that this does not reflect.
\item consider a category $C \equiv (a \xrightarrow{p} b)$. Map into a category $D$ with arrows $x \xrightarrow{s} y$ and $y \xrightarrow{s^{-1}} x$ where $s, s^{-1}$ are inverses
of each other.
\item The functor $F: C \equiv (a \xrightarrow{p} b) \to (x \xrightarrow{s} y)$ is faithful but does not reflect isos.
\end{itemize}

\question{1.5.vi (i) Composition of full is full}
\begin{itemize}
        \item Let $F : C \rightarrow D$, $G: D \rightarrow E$ be full. We must show that $G \circ F$ is full.
        \item Pick some element $\alpha Hom(GFx, GFy)$. Since $G$ is full, there is an arrow $\beta \in Hom(Fx, Fy)$ such that $G\beta = \alpha$.
        \item Since $g$ is full, there is an arrow $\gamma \in Hom(x, y)$ such that $F\gamma = \beta$.
        \item Combining, we see that $F\gamma = \beta$ and $G\beta = \alpha$, or $GF\gamma = \alpha$.
        \item Thus $GF$ is full since for any $GFx \xrightarrow{\alpha} GFy$ we found a $x \xrightarrow{gamma} y$ such that $GF\gamma = \alpha$.
\end{itemize}

\question{1.5.vi (i) Composition of faithful is faithful}
\begin{itemize}
        \item Let $F : C \rightarrow D$, $G: D \rightarrow E$ be faithul. We must show that $G \circ F$ is full.
        \item Pick some element $\alpha, \alpha' \in Hom(x, y)$ such that $GF\alpha = GF\alpha'$ . We must show that $\alpha = \alpha'$.
        \item Since $G$ is faithful, we get that $F\alpha = F\alpha'$. Since $F$ is faithful, we get that $\alpha = \alpha'$.
        \item Thus $GF$ is faithful.
\end{itemize}

\question{1.5.vi (i) Composition of eso is eso}
\begin{itemize}
        \item Let $F : C \rightarrow D$, $G: D \rightarrow E$ be eso. We must show that $GF$ is eso.
        \item Consider som element $e \in E$. We must show that there is some $c \in C$ such that $e \simeq GFc$.
        \item Since $G$ is eso, there must be some $d \in D$ such that $Gd \simeq e$.
        \item Since $F$ is eso, there must be some $c \in C$ such that $Fc \simeq d$.
        \item Recall that singe a functor preserves isos, we must have $GFc \simeq Gd$ from $Fc \simeq d$.
        \item Combining $GFc \simeq Gd$ with $Gd \simeq e$ we get $GFc \simeq e$. Thus, we have found the $c \in C$ such that $GFc \simeq e$.
        \item Thus, $GF$ reflects isos. (Key lemma: image of iso is iso) \qed
\end{itemize}

\question{1.5.vii Construct inverse of inclusion of automorphism of some object of groupoid into groupoid}
\begin{itemize}
        \item TODO
\end{itemize}

\question{1.5.viii}
% Define affine 2D space to consist of lines in 2D. More formally, these are subspaces spanned by non-zero vectors $v$. So $Aff \equiv \mathbb A^2 \equiv \{ span((x, y)) : (x, y) \neq (0, 0) \}$.
% Define $Proj \equiv \mathbb P^2$ to be the set of 3 tuples $(x,y,z)$  where we set an equivalence relation where $(x, y, z) \sim (x', y', z') $ iff there exists a $\lambda \in \mathbb R$ such that $x' = \lambda x$, $y' = \lambda y$, $z' = \lambda z$. Choosing $z = 0$ gives us the line at infinity. Informally, we can think of this as giving us an affine plane when $z \neq 0$, since we can normalize to get $[x/z, y/z, 1]$ and when $z = 0$, we get a copy of $\mathbb P^1$. So, $\mathbb P^2 \equiv \mathbb A^2 \cup P^1$


\question{1.5.ix: Category equivalent to locally small is locally small}

\begin{itemize}
    \item Let $F: C \rightleftarrows D : G$ be an equivalence of categories. Let $D$ be locally small.
        We must show that $C$ is locally small.
    \item Recall that we must have $G: D \rightarrow C$ to be full, faithful,
        and essentially surjective as it witnesses an equivalence of
        categories. As $D$ is locally small, all hom-sets $Hom_D(X, Y)$ are
        small.
    \item Since $G: D \rightarrow C$ is full, the image $Hom_C(Gx, Gy)$ is surjective, and thus $Hom_C(Gx, Gy)$ can have cardinality at most that of $Hom_D(x, y)$ which is already small.
        Thus $Hom_C(Gx, Gy)$ is also locally small. This settles the question for all Hom-sets in the image of $G$.
    \item Consider elements $c, d \in C$  which are not in the image of $G: D \rightarrow C$. Since the functor $G$ is essentialy surjective, we must have elements $Gx, Gy$ such that
        $c \simeq Gx$ and $d \simeq Gy$. In particular, this implies that $Hom_D(c, d) = Hom_D(Gx, Gy)$. This reduces this case to the previous case, showing that these
        Hom-sets too are locally small.
\end{itemize}

\question{1.5.x: Categories equivalent to discrete categories}
\begin{itemize}
        \item TODO
\end{itemize}

\section{1.6: The art of the diagram chase}

\subsection{Musings}
\subsection{Exercises 1.6}

\start{1.6.i any map from terminal to initial is iso}
\begin{itemize}
\item Let $f: t \rightarrow i$ be map from terminal to initial. We must show that $f$ is iso.
\item Consider the unique map $g: i \rightarrow t$. This map is unique both because
    it is (a) \emph{from} the initial object and (b) \emph{to} the terminal object.
\item Consider $f \circ g: i \rightarrow i$. This is an arrow \emph{from} the initial object,
    thus is unique. But there is already another arrow $id_i: i \rightarrow i$. This implies $f \circ g = id_i$
    by uniqueness.
\item Consider $g \circ f : t \rightarrow t$. This an arrow \emph{to} the terminal object, thus is unique.
    But there is already another arrow $id_t: t \rightarrow t$. From uniqueness, we get $g \circ f = id_t$.
\item This implies $f$ is iso, and furthermore $f: t \rightarrow i$ is unique, since another such
    $f'$ will also be an inverse to $g$, and thus we will have $f = f'$.
\end{itemize}

\start{1.6.ii: Any two terminal objects are connected by unique iso}
\begin{itemize}
    \item Let $t, t'$ be two terminal objects. This gives us two unique maps $f: t \rightarrow t'$ and $g: t' \rightarrow t$,
        unique by the terminality of $t, t'$.
    \item Consider $f \circ g: t' \rightarrow t'$. There is another arrow with codomain $t'$, $id_{t'}: t' \rightarrow t'$.
        By uniqueness of arrows into $t'$, we must have $f \circ g = id_{t'}$.
    \item Similarly, $g \circ f = id_t$. Thus, $f, g$ are isomorphisms. Furthermore, $f, g$ are unique. So, $t, t'$
        are isomorphic upto \emph{unique} isomorphism.
\end{itemize}

\start{1.6.iii: Faithful functor reflects monos}
\begin{itemize}
    \item Let $F: C \rightarrow D$, let $Ff$ is mono in $D$. We must show that $f: x \rightarrow y$ is mono in $C$.
    \item So, given arrows $g, h: w \rightarrow x$, given $f \circ g = f \circ h$, we must show that $g = h$.
    \item apply the functor, giving $F f \circ F g = F f \circ F h$. 
    \item Since $F f$ is mono, it is left cancellable, giving $F g = F h$.
    \item Since $F$ is faithful, it is injective on hom-sets, thus $g = h$ (from $F g = F h$).
    \item Hence, we've shown that $f \circ g = f \circ h$ implies $g = h$, or that $f$ is mono, and thuus $F$ reflects monos.
\end{itemize}

\start{1.6.iv, 1.6.v: Faithful functor need not preserve epis}
\begin{itemize}
\item Consider $F: Ring \rightarrow Set$. This is faithful.
\item Recall that the arrow $f: \mathbb Z \rightarrow \mathbb Q$ was epi in $Ring$. However, as a set valued
    function, this is not surjective. Thus, faithful functors need not \emph{preserve} epis.
\item Consider the funcor $G: Top \rightarrow Set$ which sends a space to its set of connected components.
    This is functorial, as given an arrow $g$, the arrow $Gg$ tries to send connected components to connected components.
    This will always be the case, by virtue of continuity of $g$. Now see that the monomorphism $g: \{0, 1\} \rightarrow [0, 1]$
    where $\{0, 1 \} \subseteq \mathbb R$ is discrete and $[0, 1]$ is connected, becomes $Gg: \{0, 1 \} \rightarrow \{ * \}$
    where $\{ * \}$ is the single connected component of $[0, 1]$. $Gg$ is not mono, 
\item (Explanation: The property of being a mono or epi involves a
    quantification over all objects of a category. So if a functor is not
        essentially surjective on objects or not surjective on morphisms, such
        properties in the source category are usually not enough to guarantee
        the corresponding property in the target category.  Note that this
        objection does not apply to isomorphisms, because they are defined by
        the explicit formulas for a two-sided inverse, not by a universal
        property.)
\end{itemize}

\start{1.6.vi: terminal coalgebras}
\begin{itemize}
    \item a coalgebra \emph{for} an endofunctor $T: C \rightarrow C$ is a tuple $(c \in C, \gamma : c \rightarrow Tc)$.
        A morphism of coalgebras $f: (c, \gamma) \rightarrow (c', \gamma'$) is a commuting square of the morphism $f: c \rightarrow c'$:
% https://q.uiver.app/?q=WzAsNCxbMCwwLCJjIl0sWzEsMCwiYyciXSxbMCwxLCJUYyJdLFsxLDEsIlRjJyJdLFswLDEsImYiXSxbMCwyLCJcXGdhbW1hIiwyXSxbMSwzLCJcXGdhbW1hJyJdLFsyLDMsIlRmIiwyXV0=
\[\begin{tikzcd}
	c & {c'} \\
	Tc & {Tc'}
	\arrow["f", from=1-1, to=1-2]
	\arrow["\gamma"', from=1-1, to=2-1]
	\arrow["{\gamma'}", from=1-2, to=2-2]
	\arrow["Tf"', from=2-1, to=2-2]
\end{tikzcd}\]
\item We must show that if $(c, \gamma: c \rightarrow Tc)$ is a terminal coalgebra (that is, it is terminal in the category of coalgebras), then $\gamma$ is an isomorphism.
\item Given any other coalgbera $(d, \delta: d \rightarrow Td)$ we have a unique $f: d \rightarrow c$ that makes the coalgebra diagram commute, since $c$ is terminal. 
\item Consider the coalgebra $(Tc, T\gamma: Tc \rightarrow TTc)$. By terminality of $(c, \gamma)$, there is a morphism $f: Tc \rightarrow c$ such that the diagram commutes:

% https://q.uiver.app/?q=WzAsNSxbMCwxLCJUVGMiXSxbMSwwLCJjIl0sWzEsMSwiVGMiXSxbMCwwLCJUYyJdLFs0LDBdLFsxLDIsIlxcZ2FtbWEiXSxbMCwyLCJUVGYiLDJdLFszLDEsImYiXSxbMywwLCJUXFxnYW1tYSIsMl1d
\[\begin{tikzcd}
	Tc & c &&& {} \\
	TTc & Tc
	\arrow["\gamma", from=1-2, to=2-2]
	\arrow["TTf"', from=2-1, to=2-2]
	\arrow["f", from=1-1, to=1-2]
	\arrow["T\gamma"', from=1-1, to=2-1]
\end{tikzcd}\]

\item So we have the equation $\gamma \circ f = Tf \circ T\gamma$, or $\gamma \circ f = T (f \circ \gamma)$.

\item Now consider the map of coalgebras $(c, \gamma) \mapsto (Tc, T\gamma)$ given by $\gamma: c \rightarrow Tc$, with the following commutative diagram:
% https://q.uiver.app/?q=WzAsNCxbMCwwLCJjIl0sWzEsMCwiVGMiXSxbMCwxLCJUYyJdLFsxLDEsIlRUYyJdLFswLDEsIlxcZ2FtbWEiXSxbMCwyLCJcXGdhbW1hIiwyXSxbMSwzLCJUXFxnYW1tYSJdLFsyLDMsIlRcXGdhbW1hIiwyXV0=
\[\begin{tikzcd}
	c & Tc \\
	Tc & TTc
	\arrow["\gamma", from=1-1, to=1-2]
	\arrow["\gamma"', from=1-1, to=2-1]
	\arrow["T\gamma", from=1-2, to=2-2]
	\arrow["T\gamma"', from=2-1, to=2-2]
\end{tikzcd}\]

\item Composing the previous two diagrams, we get a map of coalgebras $(c, \gamma) \mapsto (c, \gamma)$:

% https://q.uiver.app/?q=WzAsNixbMCwwLCJjIl0sWzEsMCwiVGMiXSxbMCwxLCJUYyJdLFsxLDEsIlRUYyJdLFsyLDAsImMiXSxbMiwxLCJUYyJdLFswLDEsIlxcZ2FtbWEiXSxbMCwyLCJcXGdhbW1hIiwyXSxbMSwzLCJUXFxnYW1tYSJdLFsyLDMsIlRcXGdhbW1hIiwyXSxbMSw0LCJmIl0sWzQsNSwiXFxnYW1tYSJdLFszLDUsIlRmIiwyXV0=
\[\begin{tikzcd}
	c & Tc & c \\
	Tc & TTc & Tc
	\arrow["\gamma", from=1-1, to=1-2]
	\arrow["\gamma"', from=1-1, to=2-1]
	\arrow["T\gamma", from=1-2, to=2-2]
	\arrow["T\gamma"', from=2-1, to=2-2]
	\arrow["f", from=1-2, to=1-3]
	\arrow["\gamma", from=1-3, to=2-3]
	\arrow["Tf"', from=2-2, to=2-3]
\end{tikzcd}\]
\item From the terminality of $(c, \gamma)$, the arrow $f \circ \gamma$ must be unique, and must be isomorphic to $id_c$:

% https://q.uiver.app/?q=WzAsMTAsWzAsMCwiYyJdLFsxLDAsIlRjIl0sWzAsMSwiVGMiXSxbMSwxLCJUVGMiXSxbMiwwLCJjIl0sWzIsMSwiVGMiXSxbNCwwLCJjIl0sWzUsMCwiYyJdLFs0LDEsIlRjIl0sWzUsMSwiVGMiXSxbMCwxLCJcXGdhbW1hIl0sWzAsMiwiXFxnYW1tYSIsMl0sWzEsMywiVFxcZ2FtbWEiXSxbMiwzLCJUXFxnYW1tYSIsMl0sWzEsNCwiZiJdLFs0LDUsIlxcZ2FtbWEiXSxbMyw1LCJUZiIsMl0sWzYsNywiXFxvcGVyYXRvcm5hbWV7aWR9X2MiXSxbNiw4LCJcXGdhbW1hIiwyXSxbNyw5LCJcXGdhbW1hIl0sWzgsOSwiXFxvcGVyYXRvcm5hbWV7aWR9X3tUY30iLDJdXQ==
\[\begin{tikzcd}
	c & Tc & c && c & c \\
	Tc & TTc & Tc && Tc & Tc
	\arrow["\gamma", from=1-1, to=1-2]
	\arrow["\gamma"', from=1-1, to=2-1]
	\arrow["T\gamma", from=1-2, to=2-2]
	\arrow["T\gamma"', from=2-1, to=2-2]
	\arrow["f", from=1-2, to=1-3]
	\arrow["\gamma", from=1-3, to=2-3]
	\arrow["Tf"', from=2-2, to=2-3]
	\arrow["{\operatorname{id}_c}", from=1-5, to=1-6]
	\arrow["\gamma"', from=1-5, to=2-5]
	\arrow["\gamma", from=1-6, to=2-6]
	\arrow["{\operatorname{id}_{Tc}}"', from=2-5, to=2-6]
\end{tikzcd}\]

\item We've now proven $f \circ \gamma = id_c$. We need to show that $\gamma \circ f = id_{Tc}$.
\item From the equation $\gamma \circ f = T (f \circ \gamma)$, and $[f \circ \gamma = id_c]$ we derive $\gamma \circ f = T (id_c) = id_{Tc}$.
\item thus, $\gamma$ and $f$ are inverses of each other, and we have the desired isomorphism.
\end{itemize}

\end{document}
