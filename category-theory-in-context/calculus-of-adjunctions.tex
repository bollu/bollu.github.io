% https://github.com/cohomolo-gy/cats-in-context/blob/master/chapter-2/Chapter%202%20Solutions.tex

\documentclass[14pt]{article}
\usepackage{classicthesis}

\usepackage{bbm}
\usepackage{bbding} % for flower.
\usepackage{physics}
\usepackage{amsmath,amssymb}
\usepackage{graphicx}
\usepackage{makeidx}
\usepackage{algpseudocode}
\usepackage{algorithm}
\usepackage{listing}
\usepackage{minted}
\usepackage{cancel}
\usepackage{color}% or xcolor
\usepackage{quiver}
\usepackage{changepage}
\usepackage{booktabs}   %% For formal tables:
                        %% http://ctan.org/pkg/booktabs
\usepackage{subcaption} %% For complex figures with subfigures/subcaptions
                        %% http://ctan.org/pkg/subcaption
\usepackage{enumitem}
\usepackage{mathtools}
%\usepackage{minted}
%\newminted{fortran}{fontsize=\footnotesize}

\usepackage{xargs}
\usepackage[colorinlistoftodos,prependcaption,textsize=tiny]{todonotes}

\usepackage{hyperref}
\hypersetup{
    colorlinks,
    citecolor=blue,
    filecolor=blue,
    linkcolor=blue,
    urlcolor=blue
}

\usepackage{epsfig}
\usepackage{tabularx}
\usepackage{latexsym}
\newcommand\ddfrac[2]{\frac{\displaystyle #1}{\displaystyle #2}}
\newcommand{\N}{\ensuremath{\mathbb{N}}}
\newcommand{\Z}{\ensuremath{\mathbb{Z}}}
\newcommand{\Q}{\ensuremath{\mathbb{Q}}}
\newcommand{\R}{\ensuremath{\mathbb R}}
\newcommand{\coT}{\ensuremath{T^*}}
\newcommand{\boldX}{\ensuremath{\mathbf{X}}}
\newcommand{\boldY}{\ensuremath{\mathbf{Y}}}


\newcommand{\cat}[1]{\mathsf{#1}}
\newcommand{\functor}[3]{#1 : \cat{#2} \to \cat{#3}}
\newcommand{\functordef}{\functor{F}{C}{D}}
\newcommand{\cc}{\cat{C}}
\newcommand{\cC}{\cat{C}}
\renewcommand{\dd}{\cat{D}}
\newcommand{\ee}{\cat{E}}
\newcommand{\ccat}{\cat{Cat}}
\newcommand{\cCat}{\cat{Cat}}
\newcommand{\cset}{\cat{Set}}
\newcommand{\cSet}{\cat{Set}}
\newcommand{\cFin}{\cat{Fin}}
\newcommand{\cCAT}{\cat{CAT}}
\newcommand{\cTop}{\cat{Top}}
\newcommand{\ctop}{\cat{Top}}
\newcommand{\cgrp}{\cat{Grp}}
\newcommand{\cGrp}{\cat{Grp}}
\newcommand{\cCone}{\cat{Cone}}

\newcommand{\subcat}[2]{\bm{ \mathsf{#1}}_{\bm{ \mathsf{#2}}}}
\renewcommand{\op}{\text{op}}
\newcommand{\inv}[1]{#1^{-1}}
\newcommand{\mono}{\rightarrowtail}
\newcommand{\epi}{\twoheadrightarrow}
\newcommand{\bg}{\cat{BG}}
\newcommand{\bgg}{\cat{BG'}}
\newcommand{\nt}{\Rightarrow}
%\newcommand{\ant}[2]{\alpha : F \nt G}
%\newcommand{\bnt}[2]{\beta : F \nt G}
%\newcommand{\anti}[2]{\alpha : F \cong G}
%\newcommand{\bnti}[2]{\beta : F \cong G}
\newcommand{\zero}{\mathbbm{0}}
\newcommand{\one}{\mathbbm{1}}
\newcommand{\two}{\mathbbm{2}}
\newcommand{\three}{\mathbbm{3}}
\newcommand{\id}{\operatorname{id}}
\newcommand{\Id}{\operatorname{id}}
\newcommand{\colim}{\operatorname{colim}}



\newcommand{\G}{\ensuremath{\mathcal{G}}}
% \newcommand{\braket}[2]{\ensuremath{\left\langle #1 \vert #2 \right\rangle}}


\def\qed{$\Box$}
\newtheorem{theorem}{Theorem}
\newtheorem{corollary}[theorem]{Corollary}
\newtheorem{definition}[theorem]{Definition}
\newtheorem{lemma}[theorem]{Lemma}
\newtheorem{observation}[theorem]{Observation}
\newtheorem{remark}[theorem]{Remark}
\newtheorem{example}[theorem]{Example}
\newtheorem{exercise}[theorem]{Exercise}

% \newcommand{\proof}[1][]{\emph{Proof #1}\textbf{:} }
\newcommand*{\start}[1]{\leavevmode\newline \textbf{#1} }
\newcommand*{\question}[1]{\leavevmode\newline \par\noindent\rule{\textwidth}{0.4pt} \textbf{Question: #1.}}
\newcommand*{\proof}[1]{\leavevmode\newline \textbf{Proof #1}}
\newcommand*{\answer}{\leavevmode\newline \textbf{Answer} }

\newcommand{\X}{\ensuremath{\mathfrak{X}}}
\newcommand{\comma}{\downarrow}


% \newcommand{\hom}{\ensuremath{\operatorname{Hom}}}
\newcommand{\Hom}{\ensuremath{\operatorname{Hom}}}

\title{Category theory in context: 4.4 --- Calculus of Adjunctions}
\author{Siddharth Bhat}
\date{Monsoon, second year of the plague}


\begin{document}
\maketitle

\section{Proposition 4.4.1}

If $F, F'$ are both left adjoint to $G$, then $F \simeq F'$. Moreover, there is a unique
iso $\theta: F \simeq F'$ commuting with units and counits of adjunctions:

% https://q.uiver.app/?q=WzAsNixbMCwwLCIxX0MiXSxbMSwwLCJHRiJdLFsxLDEsIkdGJyJdLFsyLDAsIkZHIl0sWzIsMSwiRidHIl0sWzMsMCwiMV9EIl0sWzAsMSwiXFxldGEiXSxbMSwyLCJHXFx0aGV0YSJdLFswLDIsIlxcZXRhJyIsMl0sWzMsNSwiXFxlcHNpbG9uIl0sWzMsNCwiXFx0aGV0YSBHIiwyXSxbNCw1LCJcXGVwc2lsb24nIiwyXV0=
\[\begin{tikzcd}
    {1_C} & GF & FG & {1_D} \\
    & {GF'} & {F'G}
    \arrow["\eta", from=1-1, to=1-2]
    \arrow["G\theta", from=1-2, to=2-2]
    \arrow["{\eta'}"', from=1-1, to=2-2]
    \arrow["\epsilon", from=1-3, to=1-4]
    \arrow["{\theta G}"', from=1-3, to=2-3]
    \arrow["{\epsilon'}"', from=2-3, to=1-4]
\end{tikzcd}\]


\subsection{Proof by unit/counit}

Let's consider the data we need to define for an iso $\theta: F \Rightarrow F'$. Drawing out the naturality square, we need the arrows:

% https://q.uiver.app/?q=WzAsNCxbMCwwLCJGYyJdLFsxLDAsIkYnYyJdLFswLDEsIkZjJyJdLFsxLDEsIkYnYyciXSxbMCwyLCJGZiJdLFsxLDMsIkYnZiJdLFswLDEsIlxcdGhldGFfYyIsMV0sWzIsMywiXFx0aGV0YV97Yyd9IiwxXV0=
\[\begin{tikzcd}
    Fc & {F'c} \\
    {Fc'} & {F'c'}
    \arrow["Ff", from=1-1, to=2-1]
    \arrow["{F'f}", from=1-2, to=2-2]
    \arrow["{\theta_c}"{description}, from=1-1, to=1-2]
    \arrow["{\theta_{c'}}"{description}, from=2-1, to=2-2]
\end{tikzcd}\]

By adjunction, defining a commutative diagram with $Fc \rightarrow d$ is the same as defining a commutative
diagram with $c \rightarrow Gd$:

% https://q.uiver.app/?q=WzAsNCxbMCwwLCJjIl0sWzEsMCwiR0YnYyJdLFswLDEsImMnIl0sWzEsMSwiR0YnYyciXSxbMCwyLCJmIiwyXSxbMSwzLCJHRidmIl0sWzAsMSwiXFx0aGV0YV9jXlxcIyJdLFsyLDMsIlxcdGhldGFeXFwjX3tjJ30iLDJdXQ==
\[\begin{tikzcd}
    c & {GF'c} \\
    {c'} & {GF'c'}
    \arrow["f"', from=1-1, to=2-1]
    \arrow["{GF'f}", from=1-2, to=2-2]
    \arrow["{\theta_c^\#}", from=1-1, to=1-2]
    \arrow["{\theta^\#_{c'}}"', from=2-1, to=2-2]
\end{tikzcd}\]

We define $\theta^\# \equiv \eta': 1 \to G F'$, since the types match. Using this, we compute a formula for $\theta$ as the
transpose of $\theta^\#$. [TODO: how did we compute this in the first place?]

$$
\theta \equiv F \xRightarrow{F \eta'} FGF' \xRightarrow{\epsilon F'} F'
$$

Exchanging the roles of $F$ with $F'$, $\eta$ with $\eta'$, and $\epsilon$ with $\epsilon'$,
this also computes a formula for $\theta'$ given by:

$$
\theta' \equiv F' \xRightarrow{F' \eta} F'GF \xRightarrow{\epsilon' F} F
$$

The hope is that $\theta$ and $\theta'$ are inverse natural transforms. We need to check that $\theta' \circ \theta = 1_F$.
We claim that it suffices to check that $GF (\theta' \circ \theta) \circ \eta  = \eta$. [TODO: why does this suffice?]

Writing out $G (\theta' \circ \theta) \circ \eta$, which is equal to $G \theta' \circ G \theta \circ \eta$:

\begin{align*}
&1 \xRightarrow{\eta} GF {\color{red} \xRightarrow{G \theta}} GF' { \color{blue} \xRightarrow{G \theta'}} GF \\
&1 \xRightarrow{\eta} GF { \color{red} \xRightarrow{GF \eta'} GF GF' \xRightarrow{G \epsilon F'}} G F'
{\color{blue} \xRightarrow{G F' \eta} G F' G F \xRightarrow{G \epsilon' F}} GF \\
\end{align*}

We wish to swap $\eta$ with $GF\eta'$ (at the first two terms)
to bring the $\eta$ and $\epsilon$ close together (at the first three terms) so we can use the triangle identities.
To do this, we consider the commutative square, where we transport the morphism $c \xrightarrow{\eta'_c} GF' c$
along $\eta: 1_Cx \to GFx$ to give:

% https://q.uiver.app/?q=WzAsOSxbMCwxLCJjIl0sWzAsMywiR0YnYyJdLFsxLDEsIjFfQyhjKSJdLFsxLDMsIjFfQyhHRidjKSJdLFszLDEsIkdGKGMpIl0sWzMsMywiR0YoR0YnYykiXSxbMSwwLCIxX0MoeCkiXSxbMywwLCJHRih4KSJdLFsyLDIsIlxcZXRhflxcdGV4dHR0e25hdHVyYWx9Il0sWzAsMSwiXFxldGEnX2MiXSxbMiwzLCIxX0NcXGV0YSdfYyIsMl0sWzIsNCwiXFxldGFfe2N9Il0sWzMsNSwiXFxldGFfe0dGJ2N9IiwyXSxbNCw1LCJHRlxcZXRhJ19jIl0sWzYsNywiXFxldGFfeCJdXQ==
\[\begin{tikzcd}
    & {1_C(x)} && {GF(x)} \\
    c & {1_C(c)} && {GF(c)} \\
    && {\eta~\texttt{natural}} \\
    {GF'c} & {1_C(GF'c)} && {GF(GF'c)}
    \arrow["{\eta'_c}", from=2-1, to=4-1]
    \arrow["{1_C\eta'_c}"', from=2-2, to=4-2]
    \arrow["{\eta_{c}}", from=2-2, to=2-4]
    \arrow["{\eta_{GF'c}}"', from=4-2, to=4-4]
    \arrow["{GF\eta'_c}", from=2-4, to=4-4]
    \arrow["{\eta_x}", from=1-2, to=1-4]
\end{tikzcd}\]

\begin{itemize}
\item See that this square contains $1 \xRightarrow{\eta} GF \xRightarrow{GF\eta'} GFGF'$, by following right and top. The commutativity
\item of the square witnesses that this is equal to $1 \xRightarrow{\eta'} GF' \xRightarrow{\eta_{GF'}} GFGF'$.
\item See that $\eta_{GF'}$ equals $\eta GF'$ since $\eta GF'(x) \equiv \eta_{GF'} GF'x$, which is the same as $\eta_{GF'}(GF'x)$.
\item So, in total, the commutativity of this naturality square allows us to rewrite the segment
  $1 \xRightarrow{\eta} GF' {\color{red} \xRightarrow{GF \eta'} GFGF'}$ with
  $1 \xRightarrow{\eta'} GF' \xRightarrow{\eta GF'} GF GF'$.
\end{itemize}

This gives us the diagram:
\begin{align*}
&1 {\color{ForestGreen} \xRightarrow{\eta} GF \xRightarrow{GF \eta'} } GF GF' \xRightarrow{G \epsilon F'} G F'
 \xRightarrow{G F' \eta} G F' G F \xRightarrow{G \epsilon' F} GF \\
&1 { \color{ForestGreen} \xRightarrow{\eta'} GF' \xRightarrow{\eta GF'}} GF GF' \xRightarrow{G \epsilon F'} G F'
\xRightarrow{G F' \eta} G F' G F \xRightarrow{G \epsilon' F} GF \\
\end{align*}

This is regrouped using $G \epsilon \circ \eta G = 1_G$ into:

\begin{align*}
&1 \xRightarrow{\eta'} GF' \xRightarrow{\eta GF'}  GF GF' \xRightarrow{G \epsilon F'} G F'
  \xRightarrow{G F' \eta} G F' G F \xRightarrow{G \epsilon' F} GF \\
&1 \xRightarrow{\eta'} GF' { \color{ForestGreen} \xRightarrow{\eta GF'}  GF GF' \xRightarrow{G \epsilon F'} G F' }
  \xRightarrow{G F' \eta} G F' G F \xRightarrow{G \epsilon' F} GF \\
&1 \xRightarrow{\eta'} GF' { \color{ForestGreen} \xRightarrow{(\eta G; G \epsilon) F'}} G F'
  \xRightarrow{G F' \eta} G F' G F \xRightarrow{G \epsilon' F} GF \\
&1 \xRightarrow{\eta'} GF'
  \xRightarrow{G F' \eta} G F' G F \xRightarrow{G \epsilon' F} GF \\
\end{align*}

Next, we use the naturality of $\eta$ to swap $eta'$ with $GF' \eta$:

\begin{align*}
&1 \xRightarrow{\eta'} GF'
  \xRightarrow{G F' \eta} G F' G F \xRightarrow{G \epsilon' F} GF \\
&{ \color{ForestGreen} 1 \xRightarrow{\eta'} GF' \xRightarrow{G F' \eta} } G F' G F \xRightarrow{G \epsilon' F} GF \\
&{ \color{ForestGreen} 1 \xRightarrow{\eta} GF \xRightarrow{\eta' G F} } G F' G F \xRightarrow{G \epsilon' F} GF \\
\end{align*}

Finally, we use the identity $G \epsilon ' \circ \eta' G = 1_G$ to reduce the equation:

\begin{align*}
&1 \xRightarrow{\eta} GF  \xRightarrow{\eta' G F} G F' G F \xRightarrow{G \epsilon' F} GF \\
&1 \xRightarrow{\eta} GF { \color{ForestGreen}  \xRightarrow{\eta' G F} G F' G F \xRightarrow{G \epsilon' F} } GF \\
&1 \xRightarrow{\eta} GF  { \color{ForestGreen} \xRightarrow{(\eta' G; G \epsilon) F} } GF \\
&1 \xRightarrow{\eta} GF \\
\end{align*}

\subsection{Proof by Yoneda}
\begin{itemize}
    \item Since $F \vdash G$, we have that $D(Fc, d) \simeq C(c, Gd)$.
    \item Similarly, since $F' \vdash G$, we have $C(c, Gd) \simeq D(F'c, d)$.
    \item Together, this gives $D(Fc, d) \simeq D(F'c, d)$, natural in both $c$ and $d$.
    \item This implies that $D(Fc, -) \simeq D(F'c, -)$, natural in $c$, or by Yoneda, that $Fc \simeq F'c$, natural in $c$.
    \item The naturality in $c$ allows us to deduce that $F \simeq F'$.
    \item We can identify the morphism which sends $Fc$ to $F'c$ by choosing $d = Fc$. This will start at $D(Fc, d=Fc)$ and ends at $D(F'c, d=Fc)$.
\end{itemize}

We compute $\theta_c$ by contemplating the diagram below, and setting $d=Fc$ to arrive at a morphism from $1_{Fc} \in D(Fc, d=Fc)$ to
$\theta'_c \in D(F'c, d=Fc)$: [TODO: fill in the $?$]
% https://q.uiver.app/?q=WzAsOSxbMCwwLCJEKEZjLCBkKSJdLFswLDEsImY6IEZjIFxcdG8gZCJdLFsxLDAsIkMoYywgR2QpIl0sWzIsMCwiRChGJ2MsIGQpIl0sWzEsMSwiYyBcXHhyaWdodGFycm93e1xcZXRhX2N9IEdGYyBcXHhyaWdodGFycm93e0dmfSBHZCJdLFsxLDIsImc6IGMgXFx0byBHZCJdLFsyLDIsIkYnYyBcXHhyaWdodGFycm93e0YnZ30gRidHZCBcXHhyaWdodGFycm93e1xcZXBzaWxvbidfZH0gZCJdLFswLDMsIjFfe0ZjfSBcXGluIEQoRmMsIEZjKSJdLFsyLDMsIj8iXSxbMCwyXSxbMSw0LCIiLDAseyJzdHlsZSI6eyJ0YWlsIjp7Im5hbWUiOiJtYXBzIHRvIn19fV0sWzIsM10sWzUsNiwiIiwwLHsic3R5bGUiOnsidGFpbCI6eyJuYW1lIjoibWFwcyB0byJ9fX1dLFs3LDhdXQ==
\[\begin{tikzcd}
    {D(Fc, d)} & {C(c, Gd)} & {D(F'c, d)} \\
    {f: Fc \to d} & {c \xrightarrow{\eta_c} GFc \xrightarrow{Gf} Gd} \\
    & {g: c \to Gd} & {F'c \xrightarrow{F'g} F'Gd \xrightarrow{\epsilon'_d} d} \\
    {1_{Fc} \in D(Fc, Fc)} && {?}
    \arrow[from=1-1, to=1-2]
    \arrow[maps to, from=2-1, to=2-2]
    \arrow[from=1-2, to=1-3]
    \arrow[maps to, from=3-2, to=3-3]
    \arrow[from=4-1, to=4-3]
\end{tikzcd}\]

\section{Proposition 4.4.4}

Given adjunctions $F \vdash G$ and $F' \vdash G'$, their composite $FF'$ is left adjoint to the composite $GG'$:

% https://q.uiver.app/?q=WzAsNixbMCwwLCJDIl0sWzEsMCwiRCJdLFsyLDAsIkUiXSxbNCwwLCJDIl0sWzUsMCwiRSJdLFszLDAsIlxccmlnaHRzcXVpZ2Fycm93Il0sWzAsMSwiRiIsMCx7Im9mZnNldCI6LTJ9XSxbMSwwLCJHIiwwLHsib2Zmc2V0IjotMn1dLFsyLDEsIkcnIiwwLHsib2Zmc2V0IjotMn1dLFsxLDIsIkYnIiwwLHsib2Zmc2V0IjotMn1dLFszLDQsIkZGJyIsMCx7Im9mZnNldCI6LTJ9XSxbNCwzLCJHJ0ciLDAseyJvZmZzZXQiOi0yfV0sWzYsNywiIiwwLHsibGV2ZWwiOjEsInN0eWxlIjp7Im5hbWUiOiJhZGp1bmN0aW9uIn19XSxbOSw4LCIiLDAseyJsZXZlbCI6MSwic3R5bGUiOnsibmFtZSI6ImFkanVuY3Rpb24ifX1dLFsxMCwxMSwiIiwwLHsibGV2ZWwiOjEsInN0eWxlIjp7Im5hbWUiOiJhZGp1bmN0aW9uIn19XV0=
\[\begin{tikzcd}
	C & D & E & \rightsquigarrow & C & E
	\arrow[""{name=0, anchor=center, inner sep=0}, "F", shift left=2, from=1-1, to=1-2]
	\arrow[""{name=1, anchor=center, inner sep=0}, "G", shift left=2, from=1-2, to=1-1]
	\arrow[""{name=2, anchor=center, inner sep=0}, "{G'}", shift left=2, from=1-3, to=1-2]
	\arrow[""{name=3, anchor=center, inner sep=0}, "{F'}", shift left=2, from=1-2, to=1-3]
	\arrow[""{name=4, anchor=center, inner sep=0}, "{F'F}", shift left=2, from=1-5, to=1-6]
	\arrow[""{name=5, anchor=center, inner sep=0}, "{GG'}", shift left=2, from=1-6, to=1-5]
	\arrow["\dashv"{anchor=center, rotate=-90}, draw=none, from=0, to=1]
	\arrow["\dashv"{anchor=center, rotate=-90}, draw=none, from=3, to=2]
	\arrow["\dashv"{anchor=center, rotate=-90}, draw=none, from=4, to=5]
\end{tikzcd}\]

\subsection{Proof by unit/counit}


\begin{itemize}
\item The only ``reasonable'' definition of $\overline \eta: 1_C \Rightarrow GG'F'F$ is given by:

\begin{align*}
  & \overline \eta \equiv 1_c \xRightarrow{\eta} GF \xRightarrow{G\eta'F} GG'FF' \\
\end{align*}

\item A point to note: morally, the reason we build $G \eta' F$ is for the types to work; $\eta': 1_D \rightarrow G'F'$.
To mutate $GF$, it is the only type valid choice among ($\eta' GF$, $G \eta' F$, and $GF \eta'$).

\item Similarly, the only reasonable definition of $\overline \epsilon: F' F G G' \Rightarrow 1_E$
is given by the other expression as in the text.

\item I dare not perform the ``entertaining'' diagram chase.
\end{itemize}

\subsection{Proof by Yoneda}

The pleasant proof by yoneda:

\begin{align*}
  &E(F'Fc, e) \simeq D(F'c, Gd) \quad (F \vdash G) \\
  &D(F'c, d) \simeq C(c, G'Gd) \quad (F' \vdash G')
\end{align*}

which establishes a natural bijection $E(FF'c, e) \simeq C(c, G'Gd)$, which means $FF' \vdash G'G$
by the Hom-set definition of Yoneda.

\section{4.4.5: Promoting equivalence to adjoint equivalence}

Any equivalence $F: C \leftrightarrow D: G$ with $\eta: 1_C \simeq GF$ and $\epsilon: FG \simeq 1_D$
can be promoted into an adjoint equivalence. This promotion involves defining $\epsilon'$, where the natural isos ($\eta, \epsilon'$) 
now obey the triangle inequalities.

\subsection{Proof by unit/counit: (a) $G \epsilon \circ \eta' G = 1_G$}

\begin{itemize}
  \item If it really were an adjunction, then $G \epsilon \circ \eta G = 1_G$. 
  \item since we don't have an adjunction, measure the defect via $\gamma: G \Rightarrow{\eta G }  GFG \Rightarrow{G \epsilon} G$
  \item Define $\epsilon' \equiv FG \Rightarrow{F \gamma^{-1}} FG \Rightarrow{\epsilon} 1_G$
\end{itemize}

We will show that the following diagram commutes:

% https://q.uiver.app/?q=WzAsOCxbMCwxLCJHIl0sWzEsMSwiR0ZHIl0sWzIsMSwiR0ZHIl0sWzMsMSwiRyJdLFswLDAsIkciXSxbMSwwLCJHRkciXSxbMywwLCJHIl0sWzEsMiwiRyJdLFswLDEsIlxcZXRhIEciXSxbMSwyLCJHRlxcZ2FtbWFeey0xfSJdLFsyLDMsIkcgXFxlcHNpbG9uIl0sWzQsNSwiXFxldGEgRyJdLFs1LDYsIkcgXFxlcHNpbG9uJyJdLFswLDcsIlxcZ2FtbWFeey0xfSIsMl0sWzcsMywiXFxnYW1tYSIsMl0sWzcsMiwiXFxldGEgRyJdXQ==
\[\begin{tikzcd}
	G & GFG && G \\
	G & GFG & GFG & G \\
	& G
	\arrow["{\eta G}", from=2-1, to=2-2]
	\arrow["{GF\gamma^{-1}}", from=2-2, to=2-3]
	\arrow["{G \epsilon}", from=2-3, to=2-4]
	\arrow["{\eta G}", from=1-1, to=1-2]
	\arrow["{G \epsilon'}", from=1-2, to=1-4]
	\arrow["{\gamma^{-1}}"', from=2-1, to=3-2]
	\arrow["\gamma"', from=3-2, to=2-4]
	\arrow["{\eta G}", from=3-2, to=2-3]
\end{tikzcd}\]

\begin{itemize}
  \item The top row is $G \Rightarrow{\eta G} GFG \Rightarrow{G \epsilon'} G$
  \item The bottom is $G \Rightarrow{\gamma^{-1}} G \Rightarrow{\gamma} G = 1_G$.
  \item Thus, if the diagram commutes, then top equals bottom, or $G \epsilon' \circ \eta G = 1_G$, implying one of the triangle identities hold.
  \item The triangle to the right commutes by the definition of $\gamma$; $\gamma = G \epsilon \circ \eta G$.
  \item The "triangle" to the left (which actually contains 4 elements) commutes because of \emph{naturality} of $eta$. To see this, redraw
        the triangle as a commutative square:
  % https://q.uiver.app/?q=WzAsOSxbMSwxLCIxX0cgR3giXSxbMCwxLCJHeCJdLFswLDMsIkd4Il0sWzEsMywiMV9HIEd4Il0sWzEsMCwiMV9HeCJdLFszLDAsIkdGeCJdLFszLDEsIkdGeCJdLFszLDMsIkdGeCJdLFsyLDIsIlxcZXRhIH5cXHRleHR0dHtuYXR1cmFsfSJdLFsxLDIsIlxcZ2FtbWFeey0xfSIsMl0sWzAsMywiMV9HXFxnYW1tYV57LTF9IiwyXSxbNCw1LCJcXGV0YV94Il0sWzAsNiwiXFxldGFfeCJdLFs2LDcsIkdGXFxnYW1tYV57LTF9Il0sWzMsNywiXFxldGFfe0d4fSIsMl1d
\[\begin{tikzcd}
	& {1_Gx} && GFx \\
	Gx & {1_G Gx} && GFx \\
	&& {\eta ~\texttt{natural}} \\
	Gx & {1_G Gx} && GFx
	\arrow["{\gamma^{-1}}"', from=2-1, to=4-1]
	\arrow["{1_G\gamma^{-1}}"', from=2-2, to=4-2]
	\arrow["{\eta_x}", from=1-2, to=1-4]
	\arrow["{\eta_x}", from=2-2, to=2-4]
	\arrow["{GF\gamma^{-1}}", from=2-4, to=4-4]
	\arrow["{\eta_{Gx}}"', from=4-2, to=4-4]
\end{tikzcd}\]

This gives us the commutativity of the left part of the digram:

% https://q.uiver.app/?q=WzAsNyxbMCwxLCJHIl0sWzEsMSwiR0ZHIl0sWzIsMSwiR0ZHIl0sWzAsMCwiRyJdLFsxLDAsIkdGRyJdLFszLDAsIkciXSxbMSwyLCJHIl0sWzAsMSwiXFxldGEgRyJdLFsxLDIsIkdGXFxnYW1tYV57LTF9Il0sWzMsNCwiXFxldGEgRyJdLFswLDYsIlxcZ2FtbWFeey0xfSIsMl0sWzYsMiwiXFxldGEgRyJdLFs0LDUsIkcgXFxlcHNpbG9uJyJdXQ==
\[\begin{tikzcd}
	G & GFG && G \\
	G & GFG & GFG \\
	& G
	\arrow["{\eta G}", from=2-1, to=2-2]
	\arrow["{GF\gamma^{-1}}", from=2-2, to=2-3]
	\arrow["{\eta G}", from=1-1, to=1-2]
	\arrow["{\gamma^{-1}}"', from=2-1, to=3-2]
	\arrow["{\eta G}", from=3-2, to=2-3]
	\arrow["{G \epsilon'}", from=1-2, to=1-4]
\end{tikzcd}\]

\item together, we now have the left and right triangle commute, and thus the whole diagram commutes, which validates one of the triangle identities.
\end{itemize}

\subsection{Proof by unit/counit: (b) $\epsilon' F \circ F \eta = 1_F$}

\begin{itemize}
  \item This is proven by showing that $\epsilon' F \circ F \eta$ is idempotent, since an idempotent invertible map is identity. This follows from 
      $s^2 = s$ implies $s^2 s^{-1} = s s^{-1}$ or $s = id$.
    \item 
\end{itemize}

\subsection{Example of equivalence that is not adjoint equivalence}
\begin{itemize}
  \item Intuition: Pick an automorphism of a category, with $aut: C \to C$ on one side, and $aut^{-1}: C \to C$ on the other. These two should
         witness an equivalence, but they need not be adjoint.
\end{itemize}
\section{Exercises}
\end{document}
