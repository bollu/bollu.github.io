\documentclass[11pt]{book}
\usepackage[sc,osf]{mathpazo}   % With old-style figures and real smallcaps.
\linespread{1.025}              % Palatino leads a little more leading
% Euler for math and numbers
\usepackage[euler-digits,small]{eulervm}
%\documentclass[10pt]{llncs}
%\usepackage{llncsdoc}
\usepackage{amsmath,amssymb}
\usepackage{graphicx}
\usepackage{makeidx}
\usepackage{algpseudocode}
\usepackage{algorithm}
\usepackage{listing}
\usepackage{comment}
\usepackage{physics}
% look for package for quantum computing!
\evensidemargin=0.20in
\oddsidemargin=0.20in
\topmargin=0.2in
%\headheight=0.0in
%\headsep=0.0in
%\setlength{\parskip}{0mm}
%\setlength{\parindent}{4mm}
\setlength{\textwidth}{6.4in}
\setlength{\textheight}{8.5in}
%\leftmargin -2in
%\setlength{\rightmargin}{-2in}
%\usepackage{epsf}
%\usepackage{url}



\usepackage{booktabs}   %% For formal tables:
                        %% http://ctan.org/pkg/booktabs
\usepackage{subcaption} %% For complex figures with subfigures/subcaptions
                        %% http://ctan.org/pkg/subcaption
\usepackage{enumitem}
\usepackage{minted}
%\newminted{fortran}{fontsize=\footnotesize}

\usepackage{xargs}
\usepackage[colorinlistoftodos,prependcaption,textsize=tiny]{todonotes}

\usepackage{hyperref}
\hypersetup{
    colorlinks,
}

\usepackage{epsfig}
\usepackage{tabularx}
\usepackage{latexsym}
\newtheorem{lemma}{Lemma}
\newtheorem{observation}{Observation}
\newtheorem{proof}{Proof}


\def\qed{$\Box$}
\def\proof{\textit{Proof. }}
\newtheorem{corollary}{Corollary}
\newtheorem{theorem}{Theorem}
% \DeclareMathOperator{\tr}{trace}

\title{Quantum computation and information - Indranil Chakravarty}
\author{Siddharth Bhat}
\date{}

\begin{document}

% qubit
\newcommand{\qb}[1]{\ensuremath{|#1\rangle}}
% \newcommand{\braket}[2]{\ensuremath{\langle#1~\vert~#2\rangle}}
%\newcommand{\ket}[2]{\ensuremath{\langle#1~\vert~#2\rangle}}
\newcommand{\Z}{\ensuremath{\mathbb{Z}}}
\newcommand{\R}{\ensuremath{\mathbb{R}}}
\newcommand{\C}{\ensuremath{\mathbb{C}}}
\renewcommand{\H}{\ensuremath{\mathbb{H}}}
\newcommand{\tensor}{\ensuremath{\otimes}}
% \newcommand{innerprod}[2]{\ensuremath{\bra{#1}{\ket{#2}}}}
\newcommand{\E}[1]{\mathbb{E}\left[ #1 \right]}
\renewcommand{\l}{\left}
\renewcommand{\r}{\right}
\newcommand{\xor}{\ensuremath{\oplus}}
%\newcommand{\tensor}{\ensuremath{\otimes}}

\maketitle
\tableofcontents
\chapter{Lecture 1: Introduction}

Taught in collaboration with MSR Redmond for the Q\# bits.

Topics:
\begin{itemize}
    \item Intro: Transition from Classical to Quantum: Stern Gerlash, 
        Sequential Stern Gerlash, Rise of randomness.
    \item Foundations of Quantum Theory: States, Ensembles, Qubits, Pure and
        Mixed states, Multi qubit states, Tensor products, Unitary transforms,
        Spectral decomposition, SVD, Generalized measurements, Projective
        measurements, POVM, Evolution of quantum state, Krauss Representation.
    \item Quantum Entropy: Subadditivity of Entropy, Avani-Licb(?) Inequality,
        Quantum channel, Quantum channel capacity, Data compression,
        Benjamin Schumahur(?) theorem.
    \item Quantum Entanglement: EPR paradox, Schmidt decomposition, 
        Purification of entanglement, Entanglement separability problem,
        Pure and mixed entangled states, Measures of Entanglement.
    \item Quantum information processing protocols:
        Teleportation, Superdense coding, Entanglement swapping.
    \item Impossible operations in quantum information theory:
        No cloning, No deleting, No partial erasure.
    \item Quantum Computation: Introduction to Quantum Computating,
        Pauli gates, Hadamard gates, Universal gates, Quantum algorithms
        (Shor, Grover search, machine learning and optimisation).
    \item Quantum programming: Programming quantum algorithms, Q\# progtramming
        language, quantum subroutines.
\end{itemize}
Books:
\begin{itemize}
    \item Quantum computation and Quantum information --- Nielsen and Chuang.
    \item Preskill lecture notes.
\end{itemize}

Grading:
\begin{itemize}
    \item Possibility of open book take-home open ended exam for the finals.
    \item Mid 1: 15\%
    \item Mid 2: 15\%
    \item End sem (open book?) : 30\%
    \item Assignments: 15\%
    \item Projects: 25\%
\end{itemize}

\section{Stern-Gerlach: A brief, morally correct construction of qubits}
\[
\footnotesize
\verb|light rays ---> [z] ---> (z+, z-) --block (z-) --> [x] --- (x+, x-) -- block (x-) --> [z] ---> (z+, z-?!)|
\]

$[z]$ represents a polarizer along that axis. 

\begin{itemize}
    \item Since we first polarized along $z$, how did we manage to get out 
        light rays in the $x$ direction? The polarization should have killed
        everything.

    \item Since we blocked $z-$, How did we get back $z-$ after passing stuff through
        $[x]$? Something has changed drastically from our classical picture.
\end{itemize}

We can consider $\qb{z+}$ to be something like:
\[
    \qb{z+} \equiv_? \frac{1}{2}\qb{x+} + \frac{1}{2}\qb{x-}
\]
Where \qb{x+} and \qb{x-} are basis vectors for some vector space
over \R.

If we were to pass the $z+$ light rays through $[y]$, then we would get
$\qb{y+}, \qb{y-}$. So, \qb{z+} is also:
\[
    \qb{z+} \equiv_? \frac{1}{2}\qb{y+} + \frac{1}{2}\qb{y-}
\]

\subsection{Analogy with polarization of light}
Consider a monochromatic light wave in the $z$ direction. A linearly
polarized light with polarization in the $x$ direction which we call
$x$ polarized light is given by:
\[
    E_x = E_0 \hat x \cos (k z - \omega t)
\]
$\omega \equiv \text{frequency} \equiv ck$, $c \equiv \text{speed of
light}$, $k \equiv \text{wave number}$.

Similarly, $y$ polarized light is given by:
\[
    E_y = E_0 \hat y \cos (k z - \omega t)
\]

Consider the case where we have $x$ filters along direction \texttt{-}, $x'$
filter along direction \texttt{/}, $y$ filters along direction \texttt{|}.
In this case, we can have $x, x', y$ filters arranged sequentially 
give us non-zero output (contrast with just having $x, y$).

We can express the $x'$ polarization as:

\[
    E_0 \hat{x'} cos (k z - \omega t) 
    = \frac{E_0}{\sqrt 2} \hat x \cos (k z - \omega t) + \frac{E_0}{\sqrt 2} \hat y \cos (k z - \omega t)
\]

By analogy, we write:
\[
    \qb{z_+} \equiv  \frac{1}{\sqrt 2} \qb{x_+} + \frac{1}{\sqrt 2} \qb{x_-}
\]

However, we now have probability $\frac{1}{\sqrt 2}$, but we want $\frac{1}{2}$.
So, we define the probability as:
\[
    \bra{x+}\ket{x_-}^2 = \frac{1}{2}
\]
\begin{align*}
    &z_+ \equiv \text{$x$ polarization} \\
    &z_- \equiv \text{$y$ polarization} \\
    &x_+ \equiv \text{$x'$ polarization} \\
    &x_- \equiv \text{$y'$ polarization} \\
\end{align*}

This problem  can be solved again by polarization of light. This time,
we consider circularly polarized light which can be obtained by letting
linearl polarized light passing through a quarter wave plate (?)

When we pass such circularly polarized light through an $x$ or $y$ filter,
we again obtain either an $x$ polarized beam, or a $y$ polarized beam
of equal intensity. Yet, everybody knows that circularly polarized light
is totally different from $45^\circ$ linearly polarized light.

A right circularly polarized light is a linear combination of $x$ polarized
light and $y$ polarized light, where the oscillation of the electric field
for the $y$ component is $90^\circ$ out of phase with the $x$ polarized component.

\begin{align*}
    &E = \frac{E_0}{\sqrt 2} \hat x \cos (k z - \omega t) + 
    \frac{E_0}{\sqrt 2} \hat y \cos (k z - \omega t + \frac{n}{2}) \\
    %
    &\frac{E}{E_0} = \frac{1}{\sqrt 2} \hat x e^{i(kz - \omega t)} + 
        \frac{i}{\sqrt 2}\hat y e^{i (k z - \omega t)}
    \end{align*}

Similarly, left circularly polarized light is:

\[
    E = \frac{E_0}{\sqrt 2} \hat x \cos (k z - \omega t) -
    \frac{E_0}{\sqrt 2} \hat y \cos (k z - \omega t + \frac{n}{2})
\]

\section{Observable}
An observable is something that we observe.

$$
Z \ket{z+} = \frac{hbar}{\sqrt 2} \ket{z+} \qquad
Z \ket{z-} = \frac{hbar}{\sqrt 2} \ket{z-} 
$$


TODO: try to construct an operator that takes a vector $\ket{v}$ to a
vector that is orthogonal to it.

\section{Operators}

\subsection{Projectors --- $P$}

Suppose $W$ is a $k$-dimensional vector subspace of the $d$-dimensional 
vector space $V$. 

Using Gram-Schmidt, it is possible to construct an orthonormal basis
$\ket{1}, \ket{2}, \dots \ket{d}$ for $V$ such that $\ket{1} \dots \ket{k}$
is an orthonormal basis for $W$. Then the projector $P$ is defined as:
\begin{align*}
    P_W \equiv \sum_{i=1}^k \ketbra{i}
\end{align*}

\begin{itemize}
\item $P^\dagger = P$ (Immediate from writing in $\ket{i}$ basis)
\item $P^2 = P$ (Immediate from writing in $\ket{i}$ basis)
\end{itemize}

$Q = I - P$ is the projector onto orthogonal complement of the subspace that $P$.
projects into. This projects onto the $\ket{k+1} \dots \ket{d}$ basis.

\subsection{Normal operator}
\begin{align*} A A^\dagger = A^\dagger A \end{align*}


\begin{theorem}
Spectral theorem for normal operators:
Any normal operator $M$ on a vector space $V$ is diagonal with respect to some
orthonormal basis for $V$.
\end{theorem}
\begin{proof}
Let $\lambda$ be an eigenvalue of $M$. $P_\lambda$ is the projector onto
$\lambda$'s eigenvector. $Q_\lambda = P_\lambda^\bot$ is the orthogonal complement projector
of $P$.

We first establish a fact about $P M Q$:
\begin{align*}
&M M^\dagger \ket \lambda = M^\dagger (M \ket \lambda) = \lambda M^\dagger \lambda \\
&\text{Hence, $M^\dagger v \in P$.} \\
&Q (M^\dagger P) = 0 \implies (P M Q)^\dagger = 0 \implies P M Q = 0
\end{align*}

Next, we prove some properties of $QM$ and $QM^\dagger$
\begin{align*}
QM = QM(P + Q) = QMP + QMQ = QMQ \\
QM^\dagger = QM^\dagger(P + Q) = QM^\dagger P + QM^\dagger Q = (PMQ)^\dagger + QM^\dagger Q
\end{align*}

\begin{align*}
&\text{QMQ is normal:} \\
&(QMQ)^\dagger(QMQ) = Q^\dagger M^\dagger Q^\dagger Q M Q = Q M^\dagger Q M Q = Q M^\dagger M Q \\
&(QMQ)(QMQ)^\dagger = (Q M Q) (Q^\dagger M^\dagger Q^\dagger) = Q M Q M^\dagger Q = 
Q M M^\dagger Q = Q M^\dagger M Q = (QMQ)^\dagger QMQ
\end{align*}

\begin{align*}
&M = (P + Q) M (P + Q) \\
&M = P M P + P M Q + Q M P + Q M Q \\
&M = P M P + Q M Q \\
&M = \lambda_i \ketbra{i} + Q M Q \\
&\text{Since $Q M Q$ is normal, and we are performing induction on dimension, and $P \bot Q$,} \\
&M = \lambda_i \ketbra{i} + \sum_k \lambda_k \ketbra{k} \\
&\text{Hence M is normal}
\end{align*}
\end{proof}

\begin{theorem}
Any diagonalizable operator is normal
\end{theorem}
\begin{proof}
Let $M$ be diagonal with respect to basis $\ket{i}$.
Then, $M \equiv \sum_i \lambda_i \ketbra{i}$.
Now, $M^\dagger= \sum_i \lambda_i^* \ketbra{i}$. 
\begin{align*}
&M M^\dagger = \bigg(\sum_i \lambda_i \ketbra{i}\bigg)
    \bigg(\sum_j \lambda_j^* \ketbra{j}\bigg) \\
&M M^\dagger = \sum_i \lambda_i^* \lambda_i \ketbra{i} \\
&\text{Similarly,}  \quad M^\dagger M = (\sum_i \lambda_i^* \lambda_i \ketbra{i}) 
\end{align*}
\end{proof}

\subsection{Unitary operator}
\[ U U^\dagger = U^\dagger U = I \]
\begin{itemize}
\item unitary operator is normal.
\item unitary operator preserves inner products.
\begin{align*}
\bra{b'} \ket{a'} = \bra{b} U^\dagger U \ket{a} = \bra{b} I \ket{a}
\end{align*}
\end{itemize}

\subsection{Positive operator}
Special class of Hermitian operator.

\begin{align*}
 \forall v \in V, \bra v A \ket v \geq 0
\end{align*}

If the inner product is strictly greater than zero, then such an operator
is called as \emph{positive definite}. If it is greater than or equal
to zero, it is called \emph{positive semidefinite}.

\begin{theorem}
A positive operator is Hermitian
\end{theorem}
\begin{proof}
\textbf{TODO}. Proof most likely follows real case, where we use
cholesky to write it as $A^T A$ and then show that it is normal. We then
use the fact that its eigenvalues are greater than or equal to zero
to establish that it is Hermitian.
\end{proof}


\chapter{Maxwell's equations in Minkowski space}
% http://www.physics.ucc.ie/apeer/PY4112/Tensors.pdf

Let us first review Maxwell's equations:

\begin{align*}
&\div E = \frac{\rho}{\epsilon_0}~\text{(Electric charges produce fields)}\\
&\div B = 0~\text{(Only magnetic dipoles exist)}\\
&\curl E = - \pdv{B}{t}~\text{(Lenz Law / Faraday's law - time varying magnetic field induces current that opposes it)} \\
&\curl B =  \mu_0 \bigg(J + \epsilon_0 \pdv{E}{t} \bigg)~\text{(Ampere's law + fudge factor)}
\end{align*}

\section{Constructing $F$, or Tensorifying Maxwell's equations}

Begin with the equation that $\div B = 0$. This tells that $B$ can be written
as the curl of some other field:

\begin{equation}
    \boxed{B \equiv \curl A}
\end{equation}

Expanding this equation of $B$ in tensorial form:
\begin{equation}
    \boxed{ B^i = \levicevita^{ijk}  \partial_j A^k }
\end{equation}

Next, take $\curl E = - \pdv{B}{t}$.


\begin{align*}
&\curl E = - \pdv{B}{t} = \pdv{(\curl A)}{t} = \curl{\pdv{A}{t}} \\
&\curl (E + \pdv{A}{t}) = 0 \\
&\text{writing this as the gradient of some field $\phi$ scaled by $\alpha : \reals$} \\
&E + \pdv{A}{t} = \alpha \big(\grad \phi\big) \\
&E = \alpha \grad \phi - \pdv{A}{t}
\end{align*}

Since electrostatics is time-independent, we choose to think of $\alpha = -1$, 
so we can interpret $\phi$ as the potential.

\begin{equation}
     E^i = - \pdv{\phi}{x^k}  g^{ik} - \pdv{A}{t}^i
\end{equation}

A slight reformulation (since we know that in Minkowski space, $\partial_t = \partial_0$)
we get the equation:


\begin{equation}
    \boxed{ E^i = - g^{ik} \partial_k \phi - \partial_0 A^i}
\end{equation}

We get the metric $g^ik$ involved to raise the covariant $\pdv{\phi}{x^k}$
into the contravariant $E^i$.

(\textbf{Sid question:} how does one justify switching $\curl$ and $\partial$? It feels like some algebra)

\textbf{Here be magic!} We define A new rank-$2$ tensor in Minkowski space-time,
called $F$ (for Faraday),

\begin{equation}
    \boxed{F_{\mu \nu} \equiv \partial_\mu A_\nu - \partial_\nu A_\mu}
\end{equation}

(\textbf{Sid question:} why is this object $F_{\mu \nu}$ covariant? What does this \textit{mean}?)

\begin{lemma}
$F_{\mu \nu}$ is antisymmetric.
\end{lemma}

\begin{lemma}
$F_{\mu \nu}$ has 6 degrees of freedom
\end{lemma}
\begin{proof}
Number of degrees of freedom of $F$: 
\begin{align*}
\frac{4^2~\text{(total)} - 4~\text{(diagonal)}}{2~\text{(anti-symmetry)}} = 6
\end{align*}
\end{proof}

Notice that $F$ is a 1-form!

\section{Expressing $B$, $E$ in terms of $F$}
We now wish to re-expresss $B^{ij}$ and $E^{ij}$ in terms of $F$, so that
this $F$ captures all of maxwell's equations.

\begin{align*}
    B^i &= \levicevita^{ijk}  \partial_j A^k = \levicevita^{ikj} \partial_k A^j \tag*{by $k$, $j$ being free variables} \\
    B^i &= \frac{1}{2} \bigg( \levicevita^{ijk} \partial_j A^k + \levicevita^{ikj} \partial_k A^j \bigg) \\
        &\text{Substituting $\partial_j A_k - \partial_k A_j = F_{jk}$, } \\
    B^i &= \frac{1}{2} \levicevita^{ijk} F_{jk}
\end{align*}


So, $B$ in terms of $F$ is:
\begin{equation}
    \boxed{B^i = \frac{1}{2} \levicevita^{ijk} F_{jk}}
\end{equation}

Similarly, we wish to write $E$ in terms of $F$. The algebra is as follows:
\begin{align*}
    E^i &= -g^{ik} \partial_k \phi - \partial_0 A^i \\
    E^i &= -g^{ik} \partial_k \phi - \partial_0 g^{ik} A_k  \tag*{Is this allowed? Am I always allowed to insert the $g_{ik}$?} \\
    E^i &= -g^{ik} (\partial_k \phi + \partial_0 A_k) \\
\end{align*}

Since $k = \{1, 2, 3\}$ ($k$ is spacelike coordinates), and we would like to
relate $\phi$ with $A$ (to unify $E$), we \textbf{set}:

\begin{equation}
    \boxed{A_0 \equiv - \phi}
\end{equation}

Continuing the derivation,



\begin{align*}
    E^i &= -g^{ik} (\partial_k (- A_0) + \partial_0 A_k) \\
    E^i &= -g^{ik} (\partial_0 A_k - \partial_k A_0 ) \\
    E^i &= -g^{ik} F_{0k}
\end{align*}


So, finally, the relation is:

\begin{equation}
    \boxed{E^i = -g^{ik} F_{0k}}
\end{equation}


Let us reconsider what we believed $E$ to be. We had:
\begin{align*}
    E &= - \grad \phi - \pdv{A}{t}
\end{align*}
However, comparing dimensions, space derivative of $\phi$ = time
derivative of $A$. This means that 
$\frac{\delta \phi}{\delta x} = \frac{\delta A}{\delta y}$, and so
$\frac{\delta \phi}{\frac{\delta x}{\delta t}} =  \delta A$. We arbitrarily
pick $c$ as our measuring stick for $\frac{\delta x}{\delta t}$.
Also, in minkowski space, our measuring stick is actually $(ct, x, y, z)$,
so $\partial_0 = \partial_{ct}$ So, when we write the equation for $E$, we should actually write

\begin{align*}
    E &= c \bigg(- \frac{\grad \phi}{c}  - \pdv{A}{ct}\bigg)
\end{align*}

which becomes:
\begin{equation}
    \boxed{E^i = c F^{i0}}
\end{equation}

\section{Rewriting Maxwell's equations in terms of $F$}
Now that we have constructed the Faraday tensor $F$, we wish to re-expresss
Maxwell's equations in terms of this object. This will give us a compact
form of the laws which are invariant under coordinate transforms.

\subsection{Combining (1) $\grad E = \frac{\rho}{\epsilon_0}$, (4) $\curl B = \mu_0 J + \pdv{E}{t}$}
\subsubsection{1. Using (4) $\curl B = \mu_0 J + \pdv{E}{t}$}

We consider the 4th Maxwell equation:

\begin{align*}
    \curl B &= \mu_0 J + \epsilon_0 \mu_0 \pdv{E}{t} \\
    \curl B &= \mu_0 J + \frac{1}{c^2} \pdv{E}{t} \\
            &\text{Converting to indices,}\\
    (\curl B)^i &= \mu_0 J^i + \frac{1}{c} \pdv{E^i}{ct} \tag{From $\partial_{ct} = \frac{1}{c} \partial_t$} \\
                &= \mu_0 J^i + \frac{1}{c} \pdv{E^i}{X^0} \\
                &= \mu_0 J^i + \pdv{F^{i0}}{X^0} \tag{From $E^i = c F^{i0}$} \\
                &= \mu_0 J^i + \partial_0 F^{i0}
\end{align*}

Now, we start to simplify the LHS, $\curl B$:

\begin{align*}
    &(\curl B)^i = \levicevita^{ijk} \partial_j B_k \\
    %
    &\text{Since $B^k = \frac{1}{2} \levicevita^{kmn} F_{mn}$,} \\
    %
    &\text{$B_k = \frac{1}{2} \levicevita_{kmn} F^{mn}$,} \tag{\textbf{TODO:} this is scam} \\
    %
    &(\curl B)^i = \levicevita^{ijk} \partial_j \bigg( \frac{1}{2} \levicevita_{kmn} F^{mn} \bigg) =
    \frac{1}{2} \levicevita^{ijk} \levicevita_{kmn} \partial_j F^{mn}\\
\end{align*}

\textbf{Aside: We need to know how to evaluate $\levicevita^{ijk} \levicevita_{kmn}:$}
\begin{align*}
    \levicevita_{i_1, i_2, \dots, i_n} \levicevita_{j_1, j_2, \dots j_n} =  
    \det{
    \begin{vmatrix}
        \delta_{i_1 j_1} & \delta_{i_1 j_2} &\dots &\delta_{i_1 j_n} \\
        \delta_{i_2 j_1} &\delta_{i_2 j_2} &\dots &\delta_{i_2 j_n} \\
        \vdots           &\vdots  & \ddots & \vdots \\
        \delta_{i_n j_1} & \delta_{i_n j_2} & \dots & \delta_{i_n j_n}
\end{vmatrix}}
\end{align*}

$\levicevita^{ijk} \levicevita^{imn} = -1 (\delta_j^m \delta_k^n - \delta_j^n \delta_k^m)$


He argued that we get a $-1$ factor here due to the presence of the
metric. I'm not fully convinced, but I can handwave this using the
magic words "tensor density".


Plugging both equations together,

\begin{align*}
    &\frac{1}{2} \levicevita^{ijk} \levicevita_{kmn} \partial_j F^{mn} =  \mu_0 J^i + \partial_0 F^{i0}  \\
    %
    &\text{(Since $kij$ is an even permutation of $ijk$):} \\
    %
    &\frac{1}{2} \levicevita^{kij} \levicevita_{kmn} \partial_j F^{mn} =  \mu_0 J^i + \partial_0 F^{i0}  \\
    %
    &\text{(Using  $\levicevita^{kij} \levicevita^{kmn} = -1 (\delta_i^m \delta_j^n - \delta_i^n \delta_j^m)$):}\\
    %
    &\frac{1}{2} \big[ 
   - \big(\delta^i_m \delta^j_n - \delta^i_n \delta^j_m\big) \big]
   \partial_j F^{mn} =  \mu_0 J^i + \partial_0 F^{i0} \\
    %
   &- \frac{1}{2} \big[ \partial_n F^{in} - \partial_m F^{mi}  \big] = \mu_0 J^i + \partial_0 F^{i0}   \\
   %
   &\text{($F$ is anti-symmetric, so rewriting $\partial_m F^{mi} = -\partial_m F^{im}$):} \\
   %
   &-\frac{1}{2} \big[ \partial_n F^{in} + \partial_m F^{im} \big] = \mu_0 J^i + \partial_0 F^{i0}   \\
   %
   &\text{(Replacing $\partial_m F^{im} \equiv \partial_n F^{in}$ since $m$ is free):} \\
   %
   &-\big[ \partial_m F^{im} \big] = \mu_0 J^i + \partial_0 F^{i0}   \\
   % 
   &\mu_0 J^i + \partial_0 F^{i0}  + \partial_m F^{im}  = 0 \\
   % 
   &\mu_0 J^i + \partial_\mu F^{i\mu} = 0 \tag{$\mu = \{0, 1, 2, 3 \}$}
\end{align*}

This gives us a continuity-style equation, linking the current density $J$ to
the rate of change of $F$.
\begin{equation}
    \boxed{ \mu_0 J^i + \partial_\mu F^{i\mu} = 0 \tag{$\mu = \{0, 1, 2, 3 \}$} }
\end{equation}


\subsubsection{Second part, using 1st equation}

\begin{align*}
    &\grad E = \frac{\rho}{\epsilon_0} \\
    %%
    &\partial_i E^i = \frac{\rho}{\epsilon_0} \\
    %%
    &\text{(Substituting $E^i = c F^{i0}$, $c^2 = \frac{1}{\mu_0 \epsilon_0}$): } \\
    %%
    &c \partial_i F^{i0} = \frac{\rho}{\epsilon_0}  = \frac{\rho \mu_0}{\mu_0 \epsilon_0} = \rho \mu_0 c^2 \\
    %%
    &\partial_i F^{i0} = \mu_0 c \rho \\
    %%
    &\text{(Since $F$ is anti-symmetric, $F^{00} = 0$, Hence):}\\
    %%
    &\partial_0 F^{00} + \partial_i F^{i0} = \mu_0 c \rho \\
    %%
    &\partial_\mu F^{\mu 0} = \mu_0 c \rho
\end{align*}

\begin{equation}
    \boxed{ \partial_\mu F^{\mu0} = \mu_0 c \rho}
\end{equation}

\subsubsection{Combining part 1 and part 2:}


\begin{align*}
    \mu_0 J^i + \partial_\mu F^{i\mu} = 0 \tag{From $B$}  \\
    \partial_\mu F^{i\mu} = -\mu_0 J^i 
    \partial_\mu F^{\mu 0} = \mu_0 c \rho \\
    \partial_\mu F^{0 \mu} = - \mu_0 c \rho \\
\end{align*}

To combine these equations, \textbf{we set:}
\begin{equation}
    \boxed{J^0 \equiv c \rho}
\end{equation}
We arrive at the unified equation:

\begin{align*}
    \partial_\mu F^{\nu \mu} = - \mu_0 J^{\nu}
\end{align*}

Choose units such that $c = \frac{h}{2 \pi} = G_n = 1$, which gives us:


\begin{align*}
    &\partial_\mu F^{\nu \mu} = -  J^{\nu} \\
    &\text{$F$ is antisymmetric, so flipping indices} \\
    &\partial_\mu F^{\mu \nu} =  J^{\nu} \\
\end{align*}

\begin{equation}
    \boxed{ \partial_\mu F^{\mu \nu} =  J^{\nu} }
\end{equation}

Note that this is \textbf{Ampere's law!}

\subsection{Combining (2) $\curl E = - \pdv{B}{t}$, (3) $\grad B = 0$}

\begin{align*}
    %%
    &\curl E = - \pdv{B}{t} \\
    %%
    &(\curl E)^i = \levicevita^{ijk} \partial_j E_k = - \partial_0 B \\
    %%
    &\levicevita^{ijk} \partial_j E_k = - \partial_0 (\frac{1}{2} \levicevita^{ijk} F_{jk}) \\
    %%
    &\levicevita^{ijk} \partial_j E_k  + \partial_0 (\frac{1}{2} \levicevita^{ijk} F_{jk})  = 0 \\
    %%
    &2\levicevita^{ijk} \partial_j E_k  + \partial_0 (\levicevita^{ijk} F_{jk})  = 0 \\
\end{align*}

Now we begin from the other direction, and start the derivation.

We know that the equation we want is:

\begin{equation}
    \boxed{\levicevita^{\alpha \beta \mu \nu}  \partial_{\beta} F_{\mu \nu} = 0}
\end{equation}

\subsubsection{$\alpha = 0$ case:}
First, set $\alpha = 0$. So now, the other $\beta, \mu, \nu$ are forced to be
become space components --- $(i, j, k)$.

Therefore, the equation now becomes:
\begin{align*}
    \levicevita^{0 i j k}  \partial_{i} F_{j k} = 0
\end{align*}

However, note that $\levicevita{0 i j k} = \levicevita{i j k}$, because if
$(i j k)$ is an even permutation, so will $(0 i j k)$, and vice versa for odd
(since $0 < i, j, k$).

Using this, the equation becomes

\begin{align*}
    \levicevita^{i j k}  \partial_{i} F_{j k} = 0 \\
    \partial_{i} ( \levicevita^{i j k} F_{j k}) = 0 \\
    \text{Since $B^i = \frac{1}{2} \levicevita^{ijk} F_{j k}$:} \\
    \partial_{i} \bigg( \frac{B^i}{2} \bigg) = 0 \\
    \partial_{i}  B^i = 0 \\
    \grad B = 0
\end{align*}

Hence, the above equation does encode $\grad B = 0$.

\subsubsection{$\alpha = m$ case:}
Let $\alpha$ be a spatial dimension $m = \{ 1, 2, 3 \}$.
\begin{align*}
    \levicevita^{\alpha \beta \mu \nu}  \partial_{\beta} F_{\mu \nu} = 0 \\
    \levicevita^{m \beta \mu \nu}  \partial_{\beta} F_{\mu \nu} = 0
\end{align*}

Once again, we get two cases, one where $\beta = 0$, and one where $\beta = n$
where $n$ is a spatial dimension. If $\beta = 0$, then the other dimensions
are forced to be spatial dimensions, which we shall denote as $\mu \equiv x$,
$\nu \equiv y$
\begin{align*}
    \levicevita^{m \beta \mu \nu}  \partial_{\beta} F_{\mu \nu} = 0 \\
    \levicevita^{m 0 x y}  \partial_{0} F_{x y} + \levicevita^{m n \mu \nu}  \partial_{n} F_{\mu \nu}  = 0 \\
\end{align*}

Now note that $\levicevita^{m 0 \mu \nu} = - \levicevita{0 m \mu \nu} = - \levicevita{m \mu \nu}$.

Using this, we can rewrite the above equation as:

\begin{align*}
    %%%
    \levicevita^{m 0 x y}  \partial_{0} F_{x y} + \levicevita^{m n \mu \nu}  \partial_{n} F_{\mu \nu}  = 0 \\
    %%%
    - \levicevita^{m x y}  \partial_{0} F_{x y} + \levicevita^{m n \mu \nu}  \partial_{n} F_{\mu \nu}  = 0 \\
\end{align*}

We now consider cases for $\mu$ in the second term, where either $\mu = 0$ or $\mu = o \in \{1, 2, 3\}$

If $\mu = 0$, then the other dimension $\nu$ must be a spatial dimension $p$.
If $\mu = q$, then the other dimension $\nu$ must be a time dimension $0$
(This is because we are not allowed to have 4 spatial dimensions, since the $\levicevita$
evaluates to 0 on repeated dimensions).


\begin{align*}
    - \levicevita^{m x y}  \partial_{0} F_{x y} + \levicevita^{m n \mu \nu}  \partial_{n} F_{\mu \nu}  = 0 \\
    \\
    %%%
    - \levicevita^{m x y}  \partial_{0} F_{x y} + \\
    %%% mu = 0, nu = p
    \levicevita^{m n 0 p}  \partial_{n} F_{0 p}  \tag{$\mu = 0$, $\nu = p$} \\
    %%% mu = q, nu = 0
    \levicevita^{m n q 0}  \partial_{n} F_{q 0} \tag{$\mu = q$, $\nu = 0$} \\
    = 0 
\end{align*}
Rearranging, and using the fact that $F_{0 p} = - F {p 0}$,
$\levicevita{m n 0 p} = \levicevita{0 m n p} = \levicevita{m n p}$,
$\levicevita{m n q 0} = - \levicevita{0 m n q} = - \levicevita{m n q}$,

\begin{align*}
    - \levicevita^{m x y}  \partial_{0} F_{x y} + 
    %%% mu = 0, nu = p
    \levicevita^{m n p}  (- \partial_{n} F_{p 0}) +
    %%% mu = q, nu = 0
    (- \levicevita^{m n q})  \partial_{n} F_{q 0}
    = 0 
\end{align*}

Multiplying throughout by $-1$, and noticing that since $p, q$ are dummy indeces,
we can set $p = q$. This allows us to get:



\begin{align*}
    \levicevita^{m x y}  \partial_{0} F_{x y} + 
    %%% mu = 0, nu = p
    2 \levicevita^{m n p}   \partial_{n} F_{p 0} = 0
\end{align*}

First, remember that $E_p = F_{p 0}$. So, we can replace the term $F_{p 0}$
(upto fudging of constant factors that we have always done), with $E_p$.

Now, compare

\begin{align*}
    &\levicevita^{m x y}  \partial_{0} F_{x y} + 
    2 \levicevita^{m n p}   \partial_{n} E_p = 0 \tag{Our equation} \\
    \\
    &2\levicevita^{ijk} \partial_j E_k  + \partial_0 (\levicevita^{ijk} F_{jk})  = 0 \tag{Previous equation} \\
\end{align*}

Note that the two equations are identical upto variable naming, and are
hence considered equal. So, we have encoded both of Maxwell's
laws into this particular equation:
\begin{equation}
    \boxed{\levicevita^{\alpha \beta \mu \nu}  \partial_{\beta} F_{\mu \nu} = 0}
\end{equation}

\chapter{Quantum deletion}

\begin{align*}
    &\psi = \alpha \ket0 + \beta \ket 1 \\
    &\ket \psi \ket 0 \ket M \rightarrow \ket \psi \ket \psi \ket M_{\psi} \\
    &(\alpha \ket0 + \beta \ket 1) \ket 0 \ket M = (\alpha \ket{00} + \beta \ket{10}) \ket M \\
\end{align*}

Cloning is possible upto fidelity $0.83$. We get a similar theorem for
quantum deletion --- in that, we can perform approximate deletion.


If $\psi_1, \psi_2$ are two non-orthogonal states, then there is no deletion
machine by which we can delete one copy from two cpies of of $\psi_1$ and 
$\psi_2$

\begin{align*}
&\psi_1 \psi_1 \rightarrow \psi_1 \Sigma \\
&\psi_2 \psi_2 \rightarrow \psi_2 \Sigma \\
&\bra{\psi_1}\ket{\psi_2}^2 = \bra{\psi_1}\ket{\psi_2}\bra{\Sigma}\ket{\Sigma} \\
&(\bra{\psi_1}\ket{\psi_2} - 1) \bra{\psi_1}\ket{\psi_2} = 0
\end{align*}

Hence $\bra{\psi_1}\ket{\psi_2} = 0 \lor 1$


\section{No flipping}
One of the strongest impossible operations. Given a state $\ket{\psi}$, we cannot
make a state that takes it to an orthogonal state $\ket{\overline{\psi}}$.

(Take a state $a0 + b1$ to $-b0 + a1$?)


\section{No partial erasure}
$\ket{\psi(\theta, \phi)} \rightarrow \ket{\psi'(\theta)}\ket{\Sigma}$ is
impossible, where $\psi(\theta, \phi)$ is the parametrisation of a 2
qubit state on a bloch sphere.

\section{No splitting}
We cannot split quantum information.
$\ket{\psi(\theta, \phi)} \rightarrow \ket{\psi'(\theta)}\ket{\Sigma'(\phi)}$ is
impossible. That is, we cannot split the combined information in $(\theta, \phi)$
into two separate pieces of data.

\chapter{Clasical information theory}
Book recommendation: Elements of Information theory --- JJ Thomas and Thomas Cover.

\section{What is information}
\paragraph{Entropy}
Blah blah blah, define surprisal of a probability 
\begin{align*}
    I: [0, 1] \rightarrow \R \quad
    I(p) = - \log p
\end{align*}
Now, entropy of a random variable $X$ is:
\begin{align*}
    \H : \text{Random variable} \rightarrow \R \quad
    \H(X) \equiv \sum_{x \in X} p(x) I\l(p(x)\r)
\end{align*}

\paragraph{Conditional entropy}
\begin{align*}
    &\H : \text{Random variable} \times \text{Random variable} \rightarrow \R \quad
    \H(Y|X) \equiv \sum_{x \in X} p(x) H(Y|X=x) \\
\end{align*}

It can be shown that
    $\H(X, Y) = \H(X) + \H(Y|X)$
\paragraph{Mutual information}
\begin{align*}
    I(X; Y) &\equiv H(X) - H(X|Y) \\
            &= H(X) - [H(X, Y) - H(Y)] \\
            &= H(X) + H(Y) - H(X, Y) 
\end{align*}
It is a measure of the reduction of uncertainty in $X$ upon knowing $Y$.

\paragraph{Relative entropy / K-L divergence}
Suppose there are two probability distributions $P(x)$ and $Q(x)$. The
relative entropy is:

\begin{align*}
    H \l(p(x) || q(x) \r) \equiv \sum_{x \in X} p(x) \log \frac{p(x)}{q(x)}
\end{align*}

\begin{theorem}
    K-L divergence is always positive. That is, $H(p(x) || q(x)) \geq 0$,
    with $H(p(x) || q(x)) = 0 \iff p(x) = q(x)$
\end{theorem}
\begin{proof}
    \begin{align*}
        H(p(x) || q(x)) 
        &= \sum_{x \in X} p(x) \log \l( \frac{p(x)}{q(x)} \r) \\
        &= - \sum_{x \in X} p(x) \log \l( \frac{q(x)}{p(x)} \r) \\
\end{align*}

We know that $\log x \leq \frac{x - 1}{\ln 2}$.
Hence, $-\log x \geq \frac{1 - x}{\ln 2}$.

\begin{align*}
        H(p(x) || q(x)) 
        &= - \sum_{x \in X} p(x) \log \l( \frac{q(x)}{p(x)} \r) \\
        &\geq \frac{1}{\ln 2} \sum_{x \in X} p(x) \l( 1 - \frac{q(x)}{p(x)} \r) \\
        &\geq \frac{1}{\ln 2} \sum_{x \in X}\l(p(x) - q(x) \r) \\
        &\geq \frac{1}{\ln 2} (1 - 1) = 0 \\
\end{align*}
\end{proof}

%% What is the document class I need?
\documentclass{article} 
%% Some recommended packages.
\usepackage{booktabs}   %% For formal tables:
                        %% http://ctan.org/pkg/booktabs
\usepackage{subcaption} %% For complex figures with subfigures/subcaptions
                        %% http://ctan.org/pkg/subcaption
\usepackage{enumitem}
\usepackage{minted}
\newminted{fortran}{fontsize=\footnotesize}

\usepackage{xargs}
\usepackage[colorinlistoftodos,prependcaption,textsize=tiny]{todonotes}
\begin{document}
\section{Lecture 5 - 2d Convolution, Statistical signal processing}

- Using Gaussians for blurring.


\subsection{Moving Average}
Low pass conpoment: $movingaverage(x, N)$
High pass component: x - movingaverage(x, N)$

if $N$ is small, we will pick up on noise. if $N$ is large, we may smooth
way too much.

Now, we need to perform \textit{Statistics} on signals.

\subsection{Recursive moving average}
$y[n] = \frac{1}{N}\sum{m = 0}^{N - 1} x(n - m)$
$y[n] = \frac{x[n]}{N} + \frac{1}{N}\sum{m = 1}^{N - 1} x(n - m)$ + \frac{1}{N}x(n - N) - \frac{1}{N} x(n - N)
$y[n] = y[n - 1] + \frac{1}{N}(x(n) - x(n - N))$


\section{Statistical signal processing}

$Mean(n) = \frac{1}{N}\Sum{i = 0}{N - 1}x(i)$
$Mean(n) = mean_{N - 1} \cdot \frac{N - 1}{N} + \frac{1}{N} x(n - 1)$


$\sigma(n) = \frac{1}{N - 1} \sum{i = 0}{N - 1} (x(i) - \mu)^2$
$ = \frac{1}{N-1} \sum{i = 0}{N - 1} (x(i)^2 + \mu^2 - 2 \mu x(i))$
$ = \frac{1}{N-1} (\sum{i = 0}{N - 1} x(i)^2 + N \mu^2 - 2 \mu N \frac{\sum{i =0}^{N - 1} x(i)}{N} $
$ = \frac{N}{N - 1}(\frac{\sum x(i)^2}{N} - \mu^2)

(TODO: Write sigma recursively, just do this thing, not sure about computation)


\newcommand{\badpref}{\ensuremath{\textsf{BadPrefix}}}
\newcommand{\badprefix}{\badpref}

\newcommand{\tracesfin}{\ensuremath{\textsf{Traces}_{fin}}}

\chapter{Lecture 6: Liveness \& Fairness}

\begin{definition}
$E$ is a \textbf{Safety Property} iff for all words in $T \in E^c$, there is a finite bad prefix $A_0 \dots A_n$ such that \emph{no extension}
of this is in $E$. We write the set of bad prefixes for a safety property as $\badpref(E) \subseteq A^+$
\end{definition}
Formally, we write:

$$
T \models E \iff \tracesfin(T) \cap \badpref(E) = \emptyset
$$


we write $\badpref(E)$ to be the set of all finite words $A_1 \dots A_n \in A^+$ such that there is no extension which lives in $E$.

\begin{definition}
A \textbf{minimal bad prefix} is a bad prefix that itself contains no proper bad prefix.
\end{definition}


\begin{theorem}
Every invariant $E$ defined by a propositional formula $\phi$ is a safety property.
\end{theorem}
\begin{proof}
all finite words of the form $A_1 \dots A_n$ such that $A_n \not \models \phi$ is the bad prefix.
\end{proof}

\begin{definition}
The \textbf{prefix set} of an infinite word
$\sigma$ is the set of words
$pref(A_1 A_2 \dots) \equiv \{ A_1 \dots A_n : \forall n \geq 0 \}$.
\end{definition}


\begin{definition}
The \textbf{prefix set} of a property $E$ is the union of the prefix closures of all the words in it.  $pref(E) \equiv \bigcup_{\sigma \in E} pref(\sigma)$.
\end{definition}

\begin{definition}
The \textbf{prefix closure} of a property $E$ is:
$$
pref(E) \equiv \{ \sigma \in (2^{AP})^\omega : pref(\sigma) \subseteq pref(E) \}
$$
\end{definition}


\begin{theorem}
$E$ is a safety property iff $\badpref(E) \subseteq pref(E)$.
\end{theorem}
\begin{proof}
\end{proof}

\section{Safety Property as closed sets}

Let $X \equiv 2^{AP}$, our space from where we pick up events in the trace.
Define a metric on the space of infinite sequences $X^\omega$. Given two executions $\vec x, \vec y \in X^\omega$, 
we measure their similarity in the smallest index they differ (Idea from the paper ``LTL is Closed Under Topological Closure'').
We define a metric with $d(\vec x, \vec x) \equiv 0$, and $d(\vec x, \vec y) = 2^{-i}$ if $i$ is the smallest index such that $\vec x[i] \neq \vec y[i]$.
(Think why this obeys transitive).

The distance between a trace $\vec x$ and a property $S \subseteq X^\omega$ is the infimum of the distances from every element in $S$: $d(x, S) \equiv \inf_{y \in S} d(x, y)$.
Using this, we will show that safety properties correspond to closed sets, and liveness properties correspond to dense sets.

\subsection{Safety Properties}
Under this interpretation, a safety property is a closed set.
Intuitively, we are stating that every limit point of $S$ is in $S$.
Written differently, we are saying that $\forall \vec x \in X, d(\vec x, S) = 0 \implies \vec x \in S$. (Compare this to the closed interval $[0, 1]$ versus the open $(0, 1)$).
Alternatively, we can think in terms of limit points. $S$ contains all its limit points.
If we have a property $\vec x$, and we can write a sequence $\vec s_1, \vec s_2, \dots$,
where each $s_i \in S$, and $d(s_i, \vec x) < 2^i$,
then since $S$ is closed, we must have that $\lim_i \vec s_i = \vec x \in S$.
From our safety interpretation, this means that $s_1$ and $\vec x$ can diverge at step $2$, but this already tells us that $\vec x$ is safe upto 2 steps.
Similarly, $s_2$ and $\vec x$ diverge at step $4$, this tells us that $\vec x$ is safe upto 4 steps.
Repeating this, we can see that $d(s_i, \vec x) < 2^i$ establishes that $\vec x$ is safe for $2^i$ steps,
and thus it must be safe for all time.

\subsection{Liveness Properties}
Recall that a liveness property is that which can extend any finite trace.
This can be seen as a \emph{denseness} condition on the set, because intuitively, every trace is arbitrarily close to the liveness property. (Think of $\mathbb Q \in \mathbb R$).
Intuitively, suppose we have a trace $\vec x$, and let $L$ be a liveness property. Now, since every finite prefix $\vec x[:i] \in X^*$
must be extensible to a new property $\vec l_i \in X^\omega$ such that $\vec x[:i] = \vec l[:i]$ (i.e., $d(\vec x, \vec l_i) \leq 2^{-i}$), this implies that
in fact, the sequence $\vec l_1, \vec l_2, \vec l_3, \dots$ establishes that $\inf_i d(\vec x, \vec l_i) = 0$.
Therefore, any property $\vec x$ is arbitrarily close to $\vec L$.

\subsection{Decomposition Theorem}
We prove in trace semantics that any property can be written as the intersection of a safety and liveness property.
Is it true that any set of a metric space can be written as the intersection of a closed set and a dense set?
Yes.
For a given set $S$, let the closed set be its closure, $C_S \equiv \overline S$.
See that $C_S$ is an overapproximation, since it has added the limit points $C - S$. See that the set of limit points has empty interior,
so its complement will be dense. We define the dense set $D_S \equiv X - (C - S)$, or $X - \texttt{extra}$.

\chapter{Quantum Computing: Shor's algorithm}

We have $pq = N$. We wish to find $x$ such that $y = a^x \mod N$.

\begin{align*}
    &s_0 = \ket 0^{\tensor n} \\
    &s_1 = H^{\tensor n} s_0  = \frac{1}{2^n} \sum_i \ket{i} \\
    &s_2 = a^{s_1} \mod N = \frac{1}{2^n} \sum_i \ket{a^i \mod N} \\
\end{align*}

Let us now consider the function $f(x) = a^x \mod N$. This function will
be periodic with period $r$. Let us assume that $f: [0, Q-1] \to [0, Q-1]$ where
$Q$ is the domain of the function / the maximum value that is fed to $f$.

Now, note that since the function is periodic, $\l[\forall y, |f^{-1}(y)| = Q/r\r]$.

\begin{align*}
    &s_3 = measure(s_2) = \frac{1}{\sqrt\frac{Q}{r}} \l(\ket{a_0} + \ket {a_0 + r} + \dots \r)
\end{align*}

At this point, the states in $s_3$ will consists of inputs $\l[ a_0, a_0 + r, \dots a_0 + \delta r \r]$
such that $f(a_0 + \delta r) = m_0$.

We now wish to extract the $r$ from the superposition of states. A non solution
is to try and repeatedly measure the values, then what we can get is a set of
values $\l[a_0 + \delta_0 r, a_1 + \delta_1 r, a_2 + \delta_2 r, \dots \r]$.
Recovering $r$ from this set is difficult, so we try another solution.

because the Fourier transform is a change of basis, it's a unitary matrix,
and can hence be implemented as a quantum circuit. Since the function $f$
periodic and $r$ is the perid, feeding $f$ into a fourier transform will
allow us to find $r$. 

On applying the fourier transform, the function becomes a new function
such that $g \equiv FFT(f)$ such that $g(0) = g(Q/r) = g(2Q/r) = g(\lambda Q/r) = 1, g(\_) = 0~\text{otherwise}$.

    


\end{document}
