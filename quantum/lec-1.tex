\chapter{Lecture 1: Introduction}

Taught in collaboration with MSR Redmond for the Q\# bits.

Topics:
\begin{itemize}
    \item Intro: Transition from Classical to Quantum: Stern Gerlash, 
        Sequential Stern Gerlash, Rise of randomness.
    \item Foundations of Quantum Theory: States, Ensembles, Qubits, Pure and
        Mixed states, Multi qubit states, Tensor products, Unitary transforms,
        Spectral decomposition, SVD, Generalized measurements, Projective
        measurements, POVM, Evolution of quantum state, Krauss Representation.
    \item Quantum Entropy: Subadditivity of Entropy, Avani-Licb(?) Inequality,
        Quantum channel, Quantum channel capacity, Data compression,
        Benjamin Schumahur(?) theorem.
    \item Quantum Entanglement: EPR paradox, Schmidt decomposition, 
        Purification of entanglement, Entanglement separability problem,
        Pure and mixed entangled states, Measures of Entanglement.
    \item Quantum information processing protocols:
        Teleportation, Superdense coding, Entanglement swapping.
    \item Impossible operations in quantum information theory:
        No cloning, No deleting, No partial erasure.
    \item Quantum Computation: Introduction to Quantum Computating,
        Pauli gates, Hadamard gates, Universal gates, Quantum algorithms
        (Shor, Grover search, machine learning and optimisation).
    \item Quantum programming: Programming quantum algorithms, Q\# progtramming
        language, quantum subroutines.
\end{itemize}
Books:
\begin{itemize}
    \item Quantum computation and Quantum information --- Nielsen and Chuang.
    \item Preskill lecture notes.
\end{itemize}

Grading:
\begin{itemize}
    \item Possibility of open book take-home open ended exam for the finals.
    \item Mid 1: 15\%
    \item Mid 2: 15\%
    \item End sem (open book?) : 30\%
    \item Assignments: 15\%
    \item Projects: 25\%
\end{itemize}

\section{Stern-Gerlach: A brief, morally correct construction of qubits}
\[
\footnotesize
\verb|light rays ---> [z] ---> (z+, z-) --block (z-) --> [x] --- (x+, x-) -- block (x-) --> [z] ---> (z+, z-?!)|
\]

$[z]$ represents a polarizer along that axis. 

\begin{itemize}
    \item Since we first polarized along $z$, how did we manage to get out 
        light rays in the $x$ direction? The polarization should have killed
        everything.

    \item Since we blocked $z-$, How did we get back $z-$ after passing stuff through
        $[x]$? Something has changed drastically from our classical picture.
\end{itemize}

We can consider $\qb{z+}$ to be something like:
\[
    \qb{z+} \equiv_? \frac{1}{2}\qb{x+} + \frac{1}{2}\qb{x-}
\]
Where \qb{x+} and \qb{x-} are basis vectors for some vector space
over \R.

If we were to pass the $z+$ light rays through $[y]$, then we would get
$\qb{y+}, \qb{y-}$. So, \qb{z+} is also:
\[
    \qb{z+} \equiv_? \frac{1}{2}\qb{y+} + \frac{1}{2}\qb{y-}
\]

\subsection{Analogy with polarization of light}
Consider a monochromatic light wave in the $z$ direction. A linearly
polarized light with polarization in the $x$ direction which we call
$x$ polarized light is given by:
\[
    E_x = E_0 \hat x \cos (k z - \omega t)
\]
$\omega \equiv \text{frequency} \equiv ck$, $c \equiv \text{speed of
light}$, $k \equiv \text{wave number}$.

Similarly, $y$ polarized light is given by:
\[
    E_y = E_0 \hat y \cos (k z - \omega t)
\]

Consider the case where we have $x$ filters along direction \texttt{-}, $x'$
filter along direction \texttt{/}, $y$ filters along direction \texttt{|}.
In this case, we can have $x, x', y$ filters arranged sequentially 
give us non-zero output (contrast with just having $x, y$).

We can express the $x'$ polarization as:

\[
    E_0 \hat{x'} cos (k z - \omega t) 
    = \frac{E_0}{\sqrt 2} \hat x \cos (k z - \omega t) + \frac{E_0}{\sqrt 2} \hat y \cos (k z - \omega t)
\]

By analogy, we write:
\[
    \qb{z_+} \equiv  \frac{1}{\sqrt 2} \qb{x_+} + \frac{1}{\sqrt 2} \qb{x_-}
\]

However, we now have probability $\frac{1}{\sqrt 2}$, but we want $\frac{1}{2}$.
So, we define the probability as:
\[
    \bra{x+}\ket{x_-}^2 = \frac{1}{2}
\]
\begin{align*}
    &z_+ \equiv \text{$x$ polarization} \\
    &z_- \equiv \text{$y$ polarization} \\
    &x_+ \equiv \text{$x'$ polarization} \\
    &x_- \equiv \text{$y'$ polarization} \\
\end{align*}

This problem  can be solved again by polarization of light. This time,
we consider circularly polarized light which can be obtained by letting
linearl polarized light passing through a quarter wave plate (?)

When we pass such circularly polarized light through an $x$ or $y$ filter,
we again obtain either an $x$ polarized beam, or a $y$ polarized beam
of equal intensity. Yet, everybody knows that circularly polarized light
is totally different from $45^\circ$ linearly polarized light.

A right circularly polarized light is a linear combination of $x$ polarized
light and $y$ polarized light, where the oscillation of the electric field
for the $y$ component is $90^\circ$ out of phase with the $x$ polarized component.

\begin{align*}
    &E = \frac{E_0}{\sqrt 2} \hat x \cos (k z - \omega t) + 
    \frac{E_0}{\sqrt 2} \hat y \cos (k z - \omega t + \frac{n}{2}) \\
    %
    &\frac{E}{E_0} = \frac{1}{\sqrt 2} \hat x e^{i(kz - \omega t)} + 
        \frac{i}{\sqrt 2}\hat y e^{i (k z - \omega t)}
    \end{align*}

Similarly, left circularly polarized light is:

\[
    E = \frac{E_0}{\sqrt 2} \hat x \cos (k z - \omega t) -
    \frac{E_0}{\sqrt 2} \hat y \cos (k z - \omega t + \frac{n}{2})
\]

\section{Observable}
An observable is something that we observe.

$$
Z \ket{z+} = \frac{hbar}{\sqrt 2} \ket{z+} \qquad
Z \ket{z-} = \frac{hbar}{\sqrt 2} \ket{z-} 
$$


TODO: try to construct an operator that takes a vector $\ket{v}$ to a
vector that is orthogonal to it.
