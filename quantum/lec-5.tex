\chapter{Quantum Entropy}

\section{Quantum relative entropy}
$$S(\rho || \sigma) \equiv \Tr(\rho \log \rho) - \Tr (\rho \log \sigma)$$

This is a meaausre of entanglement.
$$\varepsilon(\rho) = \min_{delta} S(\rho ||\delta)$$

We will now prove that this value is always non negative.
$S(\rho || \sigma) \geq 0$.

\begin{align*}
&\rho = \sum_i p_i \ket{i} \bra{i} \quad \sigma = \sum_j q_j \ket{j} \bra{j} \\
&S(\rho || \sigma) = Tr(\rho \log \rho) - \Tr(\rho \log \sigma) = \sum_i p_i \log p_i - \sum_i \bra{i} \rho \log \sigma \ket{i} \\
&\text{Notice that $\bra{i} \rho = p_i \bra{i}$. Substituting,} \\
&\bra{i} \log \sigma \ket{i} = p_i \bra{i} \log \sigma \ket{i} = \bra{i} \l( \sum_j \log(q_j) \ket{j}\bra{j} \r) \ket{i} =
\sum_j \log q_j \bra{i}\ket{j} \bra{j}\ket{i} \\
&\text{Let $P_{ij} \equiv\bra{i}\ket{j} \bra{j}\ket{i}$} \\
&S(\rho || \sigma) = \sum_i p_i \log p_i - \sum_i p_i \bra{i} \rho \log \sigma \ket{i} 
= \sum_i p_i \l(\log p_i - \sum_j \log q_j P_{ij} \r) \\
&\text{Since $\log(\cdot)$ is a concave function,} \\
&-\sum_j P_{ij} \log q_j \geq - \log r_i \quad \text{where $r_i \equiv \sum_j P_{ij} q_j$}\\
&S(\rho || \sigma) = \sum p_i \l( \log p_i - \sum_j P_{ij} \log q_j \r) \geq
\sum_i p_i \l( \log p_i - \log r_i \r) = H(p_i || r_i)
\end{align*}

Note that $H(p_i || r_i) = 0 \iff \forall i, ~p_i = r_i $

\begin{itemize}
\item The entropy is non-negative. It is 0 iff the state is pure.
\item In a $d$ dimensional hilbert space, the entropy is at most $\log d$
\item The entropy is $\log d$ when the system is completely mixed. That is,
$I / d$. (\emph{white noise})
\item For a composite system $AB$, $S(A) = S(B)$
\item $S(\sum_i p_i \rho_i) = H(p_i) = \sum_i p_i S(rho_i)$
\item Suppose $p_i$ are the probabilities for $\{ \ket{i} \}$, and $\rho_i$ are
any density operators of system $B$. $S(\sum_i p_i \ket{i}\bra{i} \otimes \rho_i) = H(p_i) + \sum_i p_i S(\rho_i)$
\end{itemize}

\begin{align*}
&S(\rho || I/d) = -S(\rho) -tr(\rho \log(I/d)) \geq 0 \\
& -S \rho -\log(1/d) \tr \rho \geq 0 \\
&-S(\rho) + \log d \geq 0 \\
&S(\rho) \leq \log(d)
\end{align*}

Let $\lambda_i^d$ and $\ket{e_i^d}$ be the eigenvalues and eigenvectors of $\rho_i$.
Observe that $p_i \lambda_i^j$ and $\ket{e_i^j}$ are eigenvalues and 
eigenvectors of $\sum_i p_i \rho_i$. 

\begin{align*}
S \l(\sum_i p_i \rho_i\r) &= \sum_{ij} p_{i} \lambda_{i}^j \log \l( p_i \lambda_i^j \r) \\
&= - \sum_i p_i \log p_i + \sum p_i (-sum_j \lambda_i^j log \lambda_i^j \\
&= H(p_i) + \sum p_i S(\rho_i)
\end{align*}

\chapter{Locality}

$\ket{\psi} = \frac{1}{\sqrt2} \l( \ket{01}_{AB} - ket{10}_{AB} \r)$

Ontological model: Something in the background which is controlling stuff that
goes on in the foreground. Einstein was convinced that there are some hidden
variables in the ontological model which give rise to quantum weirdness.
Particularly, there is a \emph{hidden variable} $\lambda$. It is an unknown 
variable that is present to describe $\psi$, but we are unable to measure /
access it. That is, our state is $\ket{\psi(x, \lambda)}$

Any local, realistic, hidden variable model can be shown to satisfy an
inequality which at first glance seems absurd. However, it was later shown
experimentally that QM does not satisfy this inequality.
QM can also be explained with a non-local, deterministic, hidden variable
theory (pilot wave theory). (Note to self: Go read the axioms sometime).

Any local realistic model will have $(lhs \leq 2)$. Quantum mechanics can
hit $(lhs = 2 \sqrt 2)$. The inequality by construction can go up to $4$.
There are stronger correlations (example, PR-box) which can hit $4$.

There is a polytope (no signalling polytope) in which PR-box is a vertex.

\section{Bell's inequality}

\begin{minted}{text}
A <-C-> B
\end{minted}

Bell's inequality has nothing to do with QM. It is a purely mathematical
construction that is \emph{independent} of QM.

Imagine Alice recieves her particle and does a measurement on it. Imagine that
she has two different measurement apparatus, and she chooses to perform one
of the two different measurements ($P_Q$ and $P_R$). Bob has two measurements
as well, and he can perform one of the measurements ($P_S$ and $P_T$).
The measurements to be done are chosen with uniform probability $1/2$.

To make things simple, let the measurement outcomes be $+1, -1$.

\paragraph{Reality}
$Q, R, S, T$ are objective values of the particle. That is, these values exist
independent of measurement

\paragraph{Locality}
Any action performed by Alice cannot affect measurements performed by Bob.

\paragraph{The Inequality}
Now consider the function $QS + RS + RT - QT = (R + Q)S + (R - Q)T$. 
Note that each of $Q, R, S, T$ can be $\pm 1$. Hence, the maximum value
it can reach is $2$ (TODO: prove).

$P(q, r, s, t)$ is the probability of $(Q=q, R=r, S=s, T=t)$.
$\E{QS + RS + RT - QT} = \E{QS} + \E{RS} + \E{RT} - \E{QT} \leq 2$.


Now, for the QM side of the story. 

\begin{align*}
    &\ket{\psi} = \frac{\ket{01} - \ket{10}}{\sqrt 2} \quad
    Q = Z_1 \quad R = X_1 \quad S = \frac{-Z_2 -X_2}{\sqrt 2} \quad T = \frac{Z_2 - X_2}{\sqrt 2} \\
    & \E{QS + RS + RT - QT} = 2 \sqrt 2
\end{align*}

This clearly violates bell's inequality. So, we ned to lose either reality or
locality in QM (and understand which one is lost).

\section{Measures of Entangelement}
Consider a state in superposition $(\alpha \ket{0} + \beta \ket{1})$. If a state
$\psi \neq \ket{\phi_1} \otimes \ket{\phi_2} \forall \phi_1, \phi_2$, then $\psi$ is entangled.

For mixed state, $\rho \geq 0$, $\Tr(r) = 1$. We have the inequality that
$\Tr(\rho^2) = 1$ for a pure state, and $\Tr(\rho^2) < 1 $ for a mixed state.

Let $\l( \rho \neq \sum_i p_i \rho_i^A \otimes \rho_i^B \r)$ where the $\rho_i$ are
pure states. So now the question is, how do we detect if such a $\rho$ contain
entanglement. Let the states have dimensionality $d_a$ and $d_b$. 

\section{Detecting Entanglement}

\subsection{PPT criteria (Positive Partial Transposition)}
$\rho_{AB} \rightarrow \rho_{AB}^{T_{A(B)}}$.
$T_{A(B)}\l( \ket{pq}\bra{xy} \r) \equiv \ket{xq}\bra{py}$.  If it remains
a density matrix which is positive, then it is called PPT. If it 
is not positive, then it is called NPT, and is known to be entangled (sufficient).
The condition is necessary and sufficient for dimensions $2 \otimes 2$, $2 \otimes 3$.
The intuition is that when we perform transposition for these dimensions, we will
always get negative eigenvalues. 

Note that this is not physically realisable. It is physically realised by
adding white noise. This is called structural physical approximation.
It then becomes a complete positive map. (TODO: what are
the definitions of these words?)

\subsection{Entanglement Witness}
The Hahn-banach theorem states that for any convex, compact set and a given point,
either the point lies within the set, or one can construct a hyperplane which
separates the set from the point. (We only need to invoke Hahn banach 
in the infinite dimensional case. For the finite dimensional case, Farkas
lemma works)

\section{Quantification of Entanglement}
A function $Q: M(\mathbb{C}) \to \R$ is said to be an entanglement quantifier iff:
\begin{itemize}
    \item For separable states $\rho$, $Q(\rho) = 0$.
    \item Constant under local unitary: $Q(LU (\rho)) = Q(\rho)$
    \item Under LOCC, $Q(LOCC(\rho)) \leq Q(\rho)$
    \item (Desirable) Convex under classical mixing: $Q\l(\sum_i p_i \rho_i\r) \leq \sum_i p_i Q(\rho_i)$.
        That is, states can lose entanglement under classical mixing.
    \item (Desirable) Additive: $Q(\rho \otimes \sigma) = Q(\rho) + Q(\sigma)$
    \item (Desirable) Continuity: $\lim_{n \to \infty} \bra{\psi^{\otimes n}} \rho^{\otimes n} \ket{\psi^{\otimes n}} \rightarrow 1$
\end{itemize}



