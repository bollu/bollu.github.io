\documentclass[11pt]{book}
%\documentclass[10pt]{llncs}
%\usepackage{llncsdoc}
\usepackage[sc,osf]{mathpazo}   % With old-style figures and real smallcaps.
\linespread{1.025}              % Palatino leads a little more leading
% Euler for math and numbers
\usepackage[euler-digits,small]{eulervm}
\usepackage{physics}
\usepackage{amsmath,amssymb}
\usepackage{graphicx}
\usepackage{makeidx}
\usepackage{algpseudocode}
\usepackage{algorithm}
\usepackage{listing}
\usepackage{minted}
\evensidemargin=0.20in
\oddsidemargin=0.20in
\topmargin=0.2in
%\headheight=0.0in
%\headsep=0.0in
%\setlength{\parskip}{0mm}
%\setlength{\parindent}{4mm}
\setlength{\textwidth}{6.4in}
\setlength{\textheight}{8.5in}
%\leftmargin -2in
%\setlength{\rightmargin}{-2in}
%\usepackage{epsf}
%\usepackage{url}

\usepackage{booktabs}   %% For formal tables:
                        %% http://ctan.org/pkg/booktabs
\usepackage{subcaption} %% For complex figures with subfigures/subcaptions
                        %% http://ctan.org/pkg/subcaption
\usepackage{enumitem}
%\usepackage{minted}
%\newminted{fortran}{fontsize=\footnotesize}

\usepackage{xargs}
\usepackage[colorinlistoftodos,prependcaption,textsize=tiny]{todonotes}

\usepackage{hyperref}
\hypersetup{
    colorlinks,
    citecolor=blue,
    filecolor=blue,
    linkcolor=blue,
    urlcolor=blue
}

\usepackage{epsfig}
\usepackage{tabularx}
\usepackage{latexsym}
\newcommand\ddfrac[2]{\frac{\displaystyle #1}{\displaystyle #2}}
\newcommand{\N}{\ensuremath{\mathbb{N}}}
\newcommand{\R}{\ensuremath{\mathbb R}}
\newcommand{\coT}{\ensuremath{T^*}}
\newcommand{\Lie}{\ensuremath{\mathfrak{L}}}
\newcommand{\pushforward}[1]{\ensuremath{{#1}_{\star}}}
\newcommand{\pullback}[1]{\ensuremath{{#1}^{\star}}}

\newcommand{\pushfwd}[1]{\pushforward{#1}}
\newcommand{\pf}[1]{\pushfwd{#1}}

\newcommand{\boldX}{\ensuremath{\mathbf{X}}}
\newcommand{\boldY}{\ensuremath{\mathbf{Y}}}


\newcommand{\G}{\ensuremath{\mathcal{G}}}
% \newcommand{\braket}[2]{\ensuremath{\left\langle #1 \vert #2 \right\rangle}}


\def\qed{$\Box$}
\newtheorem{corollary}{Corollary}
\newtheorem{theorem}{Theorem}
\newtheorem{definition}{Definition}
\newtheorem{lemma}{Lemma}
\newtheorem{observation}{Observation}
\newtheorem{proof}{Proof}
\newtheorem{remark}{Remark}
\newtheorem{example}{Example}

\title{Game theory}
\author{Siddharth Bhat}
\date{Spring 2020}

\begin{document}
\maketitle
\tableofcontents

\chapter{Introduction}

TODO: find out and write about: 
Extrinsic form representation of a game
$\Gamma \equiv \langle N, T, Z, o, A, s, u_i, \mathcal{H}\rangle$
where $N$ is number of players, $T$ is game tree, $Z$ is leaves, $o$ is owner
function, $A$ is actions, $s$ is transition, $u_i$ is player function, $\mathcal H$
is information sets. An information set contains equivalence classes
of states that the player cannot distinguish between. There can be ambiguity
due to missing information. Perfect information game is one where all
information sets are singleton. For example, chess is perfect information.
Example of partial information is card games.

Next, we look at strategies. A strategy is a computable function by which
each player selects their actions.

Game tree for matching coins with observation:
$S_A: \{ 1 \} \rightarrow \{H, T\}$. $S_B: \{2, 3\} \rightarrow \{H, T \}$.


Game tree for matching coins without observation:
$S_A: \{ 1 \} \rightarrow \{H, T\}$. $S_B: \{\{2, 3\}\} \rightarrow \{H, T \}$.

We let $S \equiv S_1 \times S_2 \times \dots \times S_n$. This a set containing
all possible strategies, called as a "strategy profile", $s \in S$. We
write $s$ as $s \equiv (s_i, s_{-i})$, where $s_{-i}$ is cute notation for
"the rest of the players". For example, if $s \equiv (s_1, s_2, s_3)$ we can 
notate $s = (s_2, s_{-2}) = (s_2, s_1, s_3)$.

We will currently focus on pure strategies, where we have a deterministic
function per strategy.


\section{Normal form games}
Another representation for games is called as strategic form / matrix form / 
normal form games.  Here, a game $\Gamma = \langle N, (S_i)_{i \in N}, (u_i: S \rightarrow \R)_{i \in N} \rangle$.
$N$ is the number of players, $S_i$ are strategies for each player, $u_i$ 
are the utility functions / payoffs for each player. $u_i$ maps each
strategy profile $s$ how worth it it is for player $i$ if
the game proceeds with strategy profile $s$.

\subsection{Normal form for matching coins with observation}
\begin{align*}
   &S_a \equiv \{ S_a^1 (H), S_a^2 (T)\} \\
   &S_b \equiv \{ S_b^1 (HH), S_b^2 (HT), S_b^3 (TH), S_b^4 (TT) \} \\
   &u_a: S \rightarrow \R \\
   &u_a((s_a^1, s_b^1)) = +10 \qquad (H, HH) \\
   &u_a((s_a^1, s_b^2)) = +10 \qquad (H, HT) \\
   &u_a((s_a^1, s_b^3)) = -10 \qquad (H, TH) \\
   &u_a((s_a^1, s_b^4)) = -10 \qquad (H, TT) \\ 
   &u_a((s_a^2, s_b^1)) = -10 \qquad(T, HH) \\
   &u_a((s_a^2, s_b^2)) = +10 \qquad(T, HT)\\
   &u_a((s_a^2, s_b^3)) = -10 \qquad(T, TH) \\
   &u_a((s_a^2, s_b^4)) = +10 \qquad(T, TT) \\
\end{align*}

Alternate representation of same game:
$$
\begin{tabular}{ccccc}
    $s_b \rightarrow$ & HH & HT & TH & TT \\ 
    H & (10, -10) & (10, -10)  & (-10, +10)    & (-10, +10)   \\
    T &  (-10, 10)   &  (10, -10)  & (-10, 10)   &  (10, -10)  \\
\end{tabular}
$$


\subsection{Normal form for matching coins without observation}

$$
\begin{tabular}{ccccc}
    $s_a \rightarrow$  & H & T & \\
    H & (10, -10) & (-10, 10) \\ 
    T &  (-10, 10) & (10, -10)
\end{tabular}
$$


\subsection{Normal form for prisoners dilemma}

$C$ for confess, $NC$ for not confess.

$$
\begin{tabular}{ccccc}
     & C & NC & \\
    C & (-5, -5) & (-1, -10) \\ 
    NC &  (-10, -1) & (-2, -2)
\end{tabular}
$$

\chapter{Game analysis}
Rationality implies that each player is motivated to maximise his own payoff.
Intelligent implies that player can take into account all available
information. An intelligent and rational player implies that every player will
attempt to maximise their utility.

\subsection{Common knowledge and puzzles about common knowledge}
\begin{definition}
    \emph{Common knowledge} --- Player knows it. Every player knows that every player
    knows it. Every player knows that every player knows that
    every player knows it.
    $\forall k \in \N, (\text{Every player knows that})^k$ every player knows it.
\end{definition}

If we have an island with two water streams and all humans and intelligent,
rational and cannot speak. They have a rule that says that if a person has a
blue mark on their forehead, the drink water from a stream farther away. One
day, a visitor, who knows the above fact, shouts "why is a person with a blue
mark drinking water here?" The next day, no one comes to the stream. What changed?
The only difference before and after is that it is now common knowledge that
there is one person with a blue mark drinking water at the stream. This


Some one imagined two positive whole numbers $1 \leq a, b \leq 20$. He tells the sum of these
two numbers to mathematician $A$, the product of these numbers to mathematician $B$.
$A$ tells $B$ that there is no way for $B$ to know the sum. Then $B$ exclaims
"But I know the sum now!", to which $A$ exclaims "and now I know the product".

\subsection{Strongly dominated strategy}

Given a game $\Gamma \equiv \langle N, (S_i), (u_i) \rangle$, a strategy
$s_i \in S_i$ is said to be strongly dominated by a strategy $s_i' \in S_i$
if:

$$
u_i(s_i, s_{-1}) < u_i(s_i', s_{-i})~ \forall s_{-i} \in S_{-i}
$$

\subsection{Strongly dominant strategy}

Given a game $\Gamma \equiv \langle N, (S_i), (u_i) \rangle$, a strategy
$s_i^\star$ is said to be strongly dominant if it strongly dominates every
other strategy $s_i \in S_i$.

$$
\forall s_i \in S_i, s_i \neq s_i^\star \implies ~ u_i(s_i, s_{-1}) < u_i(s_i^\star, s_{-i})~ \forall s_{-i} \in S_{-i}, 
$$

Note that to analyze strongly dominated and strongly dominant strategies,
we only need $u_i$, the utility of the $i$th player. Hence, to analyze
dominance of strategies, we can get away with writing the utility of
just a single player.

$$
\begin{tabular}{ccc}
    & L & R \\
  A & 6, - & 7, - \\
  B & 5, - & 6, - \\
  C & 4, - & 6, - \\
\end{tabular}
$$

$L, R$ are moves of the player.
$A, B, C$ are stratgies with utilities filled
in.

$A$ strongly dominate $C$, $A$ strongly dominates $B$. $B$ does \emph{not}
strongly dominate $C$, since on the $R$ action, we have $6$ for both $B$ and $C$.
Hence, $A$ is the strongly dominant strategy. Note that there need not always
exist a strongly dominant strategy:

$$
\begin{tabular}{ccc}
    & L & R \\
  A & 6, - & 7, - \\
  B & 7, - & 6, - \\
\end{tabular}
$$

Neither $A$ nor $B$ are strictly better than the other.

\subsection{Weakly dominated strategy}

Given a game $\Gamma \equiv \langle N, (S_i), (u_i) \rangle$, a strategy
$s_i \in S_i$ is said to be weakly dominated by a strategy $s_i' \in S_i$
if:

$$
u_i(s_i, s_{-1}) \leq u_i(s_i', s_{-i})~ \forall s_{-i} \in S_{-i}
$$

with strict inequality for at least one $s_{-i}$. 


\subsection{Weakly dominant strategy}

Given a game $\Gamma \equiv \langle N, (S_i), (u_i) \rangle$, a strategy
$s_i^\star$ is said to be weakly dominant if it weakly dominates every
other strategy $s_i \in S_i$.

$$
\forall s_i \in S_i, s_i \neq s_i^\star \implies ~ u_i(s_i, s_{-1}) \leq u_i(s_i^\star, s_{-i})~ \forall s_{-i} \in S_{-i}, 
$$

with strict inequality for at least one $s_{-i}$. 

Once again, a weakly dominant strategy need not always exist:
$$
\begin{tabular}{ccc}
    & L & R \\
  A & 6, - & 7, - \\
  B & 7, - & 6, - \\
\end{tabular}
$$

\subsection{Very weakly dominated strategy}

Given a game $\Gamma \equiv \langle N, (S_i), (u_i) \rangle$, a strategy
$s_i \in S_i$ is said to be very weakly dominated by a strategy $s_i' \in S_i$
if:

$$
u_i(s_i, s_{-1}) \leq u_i(s_i', s_{-i})~ \forall s_{-i} \in S_{-i}
$$

Note that we do not have the strict inequality requirement anymore. This
is now a true partial order.

\subsection{Example of strong dominance in prisoners dilemma}
$$
\begin{tabular}{ccccc}
     & C & NC & \\
    C & (-5, -5) & (-1, -10) \\ 
    NC &  (-10, -1) & (-2, -2)
\end{tabular}
$$
Here, $C$ is the strongly dominant strategy for both players.


\subsection{Another game}
$$
\begin{tabular}{ccccc}
      & a & b & c \\
    A & 5 & 5 & 5 \\
    B & 4 & 5 & 5\\
    C & 4 & 4 & 4
\end{tabular}
$$

There does not exist a strongly dominant strategy.
$A$ is weakly dominant. $C$ is weakly dominated by both $A, B$.

\subsection{Can there exist two weakly dominant strategies?}
No there cannot. If $A$ is a weakly dominant strategies, then
assume $A[i] > B[i]$.If $A[i] > B[i]$, then $B$ cannot weakly dominate 
$A$, since for $B$ to dominate $A$ we need $B[j] \geq A[j]~ \forall j$,
but $B[i] < A[i]$.


\subsection{Strongly (Weakly) Dominant Strategy Equilibrium}
A strategy profile $(s_1^\star, s_2^\star, \dots, s_n^\star)$ is called as a strongly
dominant strategy equilibrium of the game $\Gamma \equiv \langle N, (S_i), (U_i) \rangle$ iff
the strategy $s_i^\star$ is a strongly dominating strategy for player $i$.

Recall the example of Prisoners dilemma ---

$$
\begin{tabular}{ccccc}
     & C & NC & \\
    C & (-5, -5) & (-1, -10) \\ 
    NC &  (-10, -1) & (-2, -2)
\end{tabular}
$$


\end{document}
