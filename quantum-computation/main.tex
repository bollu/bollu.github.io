\documentclass[11pt]{book}
\usepackage[sc,osf]{mathpazo}   % With old-style figures and real smallcaps.
\linespread{1.025}              % Palatino leads a little more leading
% Euler for math and numbers
\usepackage[euler-digits,small]{eulervm}
%\documentclass[10pt]{llncs}
%\usepackage{llncsdoc}
\usepackage{amsmath,amssymb}
\usepackage{graphicx}
\usepackage{makeidx}
\usepackage{algpseudocode}
\usepackage{algorithm}
\usepackage{listing}
\usepackage{comment}
\usepackage{physics}
% look for package for quantum computing!
\evensidemargin=0.20in
\oddsidemargin=0.20in
\topmargin=0.2in
%\headheight=0.0in
%\headsep=0.0in
%\setlength{\parskip}{0mm}
%\setlength{\parindent}{4mm}
\setlength{\textwidth}{6.4in}
\setlength{\textheight}{8.5in}
%\leftmargin -2in
%\setlength{\rightmargin}{-2in}
%\usepackage{epsf}
%\usepackage{url}



\usepackage{booktabs}   %% For formal tables:
                        %% http://ctan.org/pkg/booktabs
\usepackage{subcaption} %% For complex figures with subfigures/subcaptions
                        %% http://ctan.org/pkg/subcaption
\usepackage{enumitem}
\usepackage{minted}
%\newminted{fortran}{fontsize=\footnotesize}

\usepackage{xargs}
\usepackage[colorinlistoftodos,prependcaption,textsize=tiny]{todonotes}

\usepackage{hyperref}
\hypersetup{
    colorlinks,
}

\usepackage{epsfig}
\usepackage{tabularx}
\usepackage{latexsym}
\newtheorem{lemma}{Lemma}
\newtheorem{observation}{Observation}
\newtheorem{proof}{Proof}


\def\qed{$\Box$}
\def\proof{\textit{Proof. }}
\newtheorem{corollary}{Corollary}
\newtheorem{theorem}{Theorem}
% \DeclareMathOperator{\tr}{trace}

\title{Quantum computation and information - Indranil Chakravarty}
\author{Siddharth Bhat}
\date{}

\begin{document}

% qubit
\newcommand{\qb}[1]{\ensuremath{|#1\rangle}}
% \newcommand{\braket}[2]{\ensuremath{\langle#1~\vert~#2\rangle}}
%\newcommand{\ket}[2]{\ensuremath{\langle#1~\vert~#2\rangle}}
\newcommand{\R}{\ensuremath{\mathbb{R}}}
\newcommand{\C}{\ensuremath{\mathbb{C}}}
\newcommand{\tensor}{\texttt{(X)}}
% \newcommand{innerprod}[2]{\ensuremath{\bra{#1}{\ket{#2}}}}

\maketitle
\tableofcontents
% http://mirrors.ibiblio.org/CTAN/macros/latex/contrib/physics/physics.pdf
\newcommand{\qdot}{{\dot q}}

\chapter{Lagrangian, Hamiltonian mechanics}

Mechanics in terms of generalized coords.
\section{Lagrangian}
Define a functional. $L$ over the config. space of partibles $q^i$, $qdot^i$.
$L = L(q^i, qdot^i)$.  We have an explicit dependence on $t$.



$L = KE - PE$

Assuming a 1-particle system of unit mass,
$$L = \frac{1}{2} \qdot^2 - V(q)$$

Assuming an n-particle system of unit mass,
$$L = \sum_i \frac{1}{2} {qdot^i}^2 - V(q^i)$$ 

\section{Variational principle}

Take a minimum path from $A$ to $B$. Now notice that the path that is
slightly different from this path will have some delta from the minimum.

Action
$$S(t0, t1) = \int L \dd t = \int_{t0}^{t1} L(q^i, qdot^i) \dd t$$.
Least action: $\delta S = 0$

% \begin{align*}
%     \delta S &= \delta \int L(q^i, qdot^i) \dd{t} \\
%              &= \int \delta L(q^i, qdot^i) \dd{t} \\
%              &= \int \pdv{L}{q^i} \delta q^i + \pdv{L}{qdot^i} \delta qdot^i \dd{t} \\
% \begin{align*}




%% What is the document class I need?
\documentclass{article} 
%% Some recommended packages.
\usepackage{booktabs}   %% For formal tables:
                        %% http://ctan.org/pkg/booktabs
\usepackage{subcaption} %% For complex figures with subfigures/subcaptions
                        %% http://ctan.org/pkg/subcaption
\usepackage{enumitem}
\usepackage{minted}
\newminted{fortran}{fontsize=\footnotesize}

\usepackage{xargs}
\usepackage[colorinlistoftodos,prependcaption,textsize=tiny]{todonotes}


\begin{document}
\section{Lecture 2 - Signals and Systems: Signals}
Dennis Freeman, MIT lectures are supposedly very good.

\subsection{Brief Overview}

If $x(t)$ was our continuous time function, we wish to create
$x[n]$. $x[n]$ is discrete.

\subsection{Recovering back signals}
\begin{itemize}
    \item Zero order hold
        Keep previous value we had till the next value
    \item Nearest Neighbour (NN)
        For each point, pick nearest neighbour and use that.
    \item Linear interpolation
    \item $\frac{sin(x)}{x}$
\end{itemize}


\subsection{Operations on signals}

\begin{itemize}
    \item Shifting: $y[n] = x[n - k]$. Delay by $k$ steps
    \item Flipping: $y[n] = x[-n]$.
    \item Scaling: $y[n] = \cdot x[k * n]$. This would lead to loss of information.
\end{itemize}

Order of operations: Shift, flip, scale. For counterexample, consider
$y[n] = x[3n + 1]$. Proof by induction on $ax + b$?

Eg: $x[-2n + 2]$. Perform as:

$y[n] = x[n + 2]$.

\subsection{Characteristics of signals}

\subsubsection{Even, Odd}

Even signal $x[n] = x[-n]$
Odd signal $x[n] = -x[-n]$

Every signal can be decomposed into sum of even and odd signals
$x[n] = \frac{x[n] + x[-n]}{2} + \frac{x[n] -x[-n]}{2}$

\subsubsection{Periodic}
$\exists N \in Z, \forall n \in Z, x[n] = x[n + N]$.

\subsubsection{Energy and power}

$Energy = \sum_{k=-\infty}{\infty} |x[k]|^2$


$Power = lim_{N \larrow \infty} \frac{1}{2N + 1} \sum_{k=-N}{N} |x[k]|^2$

\subsubsubsection{Unit Step}

u(n) = 1 if n >= 0, 0 otherwise.
Energy is \infty, power is \frac{1}{2}.


\subsection{Special Signals}

\subsubsection{Dirac delta}
$\delta[n] = 0 if n \neq 0, 1 if n = 0$.

\subsubsection{Unit Step}
u[n] = 1 if n >= 0, 0 otherwise.


\subsubsection{Ramp function}
r[n] = n if n >= 0, 0 otherwise.

\subsubsection{Exponential function}
e[n] = a^n u[n]

a is a parameter.


\subsection{Writing unit step in terms of delta}
u[n] = \sum_{k=0}{\infty} delta[n - k]

\subsection{Writing delta in terms of unit step}
delta[n] = u[n + 1] - u[n]

\subsection{Writing any signal $x[n]$ in term of $delta[n]$}
$x[n] = \sum_{k=-\infty}{infty} delta[n - k] \cdot x[k]$

\chapter{Tensor product states}

\section{Postulates of QM}
\begin{itemize}
\item Associated to any isolated physical system is a complex vector space
with inner product. This space is called as the state space of the system.
This system is completely described by its state vector which is a unit
vector in the state space.
\end{itemize}

\section{Tensor product}

Let $A$ and $B$ be vector spaces with bases $A_{basis}, B_{basis}$.
$A \tensor B$ is a \emph{new vector space}, whose basis vectors are $a_i \tensor b_j$
where $a_i \in A_{basis}, b_i \in B_{basis}$.

Properties of the tensor product:
\begin{itemize}
    \item For any arbitrary scalar $z$ and element $v \in H_a$, $w \in H_b$,
        $z (\ket v \tensor \ket w) = (z \ket v) \tensor \ket w = \ket v \tensor (z \ket w)$
    \item $(\ket v_1 + \ket v_2) \tensor \ket w = \ket v_1 \tensor \ket w + \ket v_2 \tensor \ket w$
    \item $\ket w \tensor (\ket v_1 + \ket v_2)= \ket w \tensor \ket v_1 + \ket w \tensor \ket v_2$
    \textbf{TODO: what is an easy way to get correctly sized brackets?}
    \item Suppose $\ket v \in H_a, \ket w \in H_b$, and $A$ and $B$ are linear
        operators on $H_a$ and $H_b$ respectively. 
        $(A \tensor B) (\ket v \tensor \ket w) \equiv (A \ket v) \tensor (B \ket w)$.

    \item Let $C = \sum_i c_i A_i \tensor B_i$, where $A_i, B_i$ are linear
        operators on $H_a, H_b$. Now, $C (\ket v \tensor \ket w) = \sum_i c_i ((A_i \ket v) \tensor (B_i \ket w))$
    \item $\ket x = \sum_i a_i \ket v_i \tensor \ket w_i$. $\ket y = \sum_j b_j \ket v_j \tensor \ket w_j$.
        Now, $\bra{x}\ket{y} = (\sum_i a_i^* \bra v_i \tensor \bra w_i)(\sum_j b_j \ket v_j \tensor \ket w_j)$,
        which is equal to $\sum_i \sum_i a_i^* b_j \bra{v_i}\ket{v_j'} \bra{w_i}\ket{w_j'}$
\end{itemize}

This is way too redundant, \textbf{TODO:} write down the slick definition of tensor
product spaces seen in John Lee's intro to smooth manifolds, or the definition
seen in Tensor Geometry: The Geometric Viewpoint and its uses.

\begin{align*}
    \tr(A \ket \psi \bra \psi) = 
    \sum_i \bra i A \ket \psi \bra{\psi}\ket{i} =  
    \sum_i (\bra{\psi}\ket{i}) \cdot (\bra i A \ket \psi) = 
    \sum_i \bra{\psi}(\ket{i} \bra i) A \ket \psi = 
    \bra{\psi} A \ket \psi 
\end{align*}


\begin{theorem}
    Two operators $A$, $B$ are simeltanelously diagonalizable iff $[A, B] = 0$,
    where $[A, B] = AB - BA$. That is, there exists a basis where both $A$
    and $B$ are diagonal matrices.
\end{theorem}
\begin{proof}
    One direction of the proof is easy. If two operators are simeltanelously
    diagonalizable, then we can simply write both operators in this common
    basis. Diagonal matrices commute, hence $[A, B] = 0$.

    Let $\ket{a, j}$ be an orthonormal basis for the eigenspace $V_a$ of $A$
    with eigenvalue $a$ and index $j$ to label repeated eigenvalues.

    $AB \ket{a, j} = BA \ket{a, j} = a B \ket {a, j}$. Hence,
    $A (B \ket {a, j} = a (B \ket {a, j}$. Hence, $B \ket {a, j}$ is an
    eigenvector of $A$. Therefore, $B\ket{a, j} \in V_a$. 

    Define projector $P_a$ onto $V_a$. Now, define $B_a = P_a B P_a$.
\end{proof}


\end{document}
