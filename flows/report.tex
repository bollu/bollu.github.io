\documentclass[titlepage]{article}
\usepackage{amsmath, amssymb, amsthm}
\usepackage{minted}
\usepackage{graphicx}
\usepackage{tikz}
\usepackage[colorlinks=true]{hyperref}


\newmintinline[cinline]{c}{}
\newcommand{\N}{\mathbb{N}}
\newcommand{\R}{\mathbb{R}}
\newcommand{\I}{\mathbb{I}}
\newcommand{\zo}{\{0, 1\}}
\newcommand{\distinct}{\emph{distinct}}
\newcommand{\eps}{\ensuremath{\epsilon}}


\newcommand{\Xbar}{\ensuremath{\overline{X}}}
\newcommand{\xbar}{\ensuremath{\overline{x}}}

\theoremstyle{plain} 
\newtheorem{theorem}{Theorem}
\newtheorem{lemma}{Lemma}[theorem]

\theoremstyle{definition}
\newtheorem{definition}{Definition}[theorem]


\theoremstyle{remark} 
\newtheorem{note}{Note}[theorem]
\title{Flows in Networks}
\author{Siddharth Bhat ~ \texttt{siddu.druid@gmail.com}}


\date{\today}
\begin{document}
\maketitle

\section{Introduction}
This is a re-type of the RAND corporation report by Ford and Fulkerson. I hope
to re-type the essential parts of their notes, and then augment this with
push-relabel and dinic's as well.

\begin{definition}[Network]
     A directed network or a directed linear graph $G \equiv (N, A)$ consists
     of $N$ elements (nodes) $x, y, \cdots$ together with a subset $A \subseteq N \times N$ (arcs)
     of ordered pairs $(x, y)$ of elements taken from $N$.
\end{definition}

The elements of $N$ are call nodes, and the elements of $A$ are called arcs.
Our networks are directed, as each arc carries an orientation. We may sometimes
refer to undirected networks where the set $A$ carries \emph{unordered} pairs
of nodes, or mixed networks in which some arcs are directed while others are
left undirected.

We rule out arcs of the form $(x, x)$ that lead from a node to itself. Thus
all arcs are of the form $(x, y)$ where $x \neq y$. Also while the existence
of at most one arc $(x, y)$ has been postulated, the notion of a network
frequently allows multiple arcs joining $x$ to $y$. For most problems we consider
the added generality adds nothing, so we shall continue to think of at most
one arc leading from one node to the other.


\section{Chains}

\begin{definition}[Chain]
Let $x_1, x_2, \dots x_n$ ($n \geq 2)$ be a sequence of nodes \emph{distinct} of a network
such that $(x_i, x_{(i+1)})$ is an arc for each $i = 1, 2, \dots, n-1$. Then
the sequence of nodes \emph{and} arcs:
\begin{equation}
    \tag{Definition of Chain}
    x_1, (x_1, x_2), x_2, \dots, (x_{(n-1)}, x_n), x_n
    \label{defn:chain}
\end{equation}
is called a \emph{chain}; it leads from $x_1$ to $x_n$. Sometimes for emphasis
we call \eqref{defn:chain} a \emph{Directed chain}. If the definition is altered
to have $x_n = x_1$, then the displayed sequence is a \emph{Directed cycle}.
\end{definition}

\section{Paths}
\begin{definition}[Path]
    Let $x_1, x_2, \dots x_n$ be a sequence of \emph{distinct} nodes such that
    either $(x_i, x_{i+1})$ is an arc, or $(x_{i+1}, x_i)$ is an arc for each
    $i = 1, 2, \dots, n-1$. Singling out, for each $i$, one of the two
    possibilities, we call the resulting sequence of nodes and arcs a \emph{path}
    from $x_1$ to $x_n$.
\end{definition}

Thus a path differs from a chain by allowing the possibility of traversing an
arc in a direction opposite to its orientation in going from $x_1$ to $x_n$. 
Note that for undirected networks, the two notions coincide.

Arcs $(x_i, x_{i+1})$ that belong to the path are called forward arcs; the
others are reverse arcs. In the definition of a path, if we stipulate $x_n = x_1$,
then the resulting sequence of nodes and arcs is a \emph{cycle}.

We may shorten the notation and refer unambiguously to the to the chain 
$x_1, \dots, x_n$ (as opposed to being explicit,
which is $x_1, (x_1, x_2), dots, (x_{n-1}, x_n), x_n$; to be tacit, we drop the arcs ).
Occasionally, we shall also refer to some path $x_1, \dots, x_n$;
Then it is to be understood that some selection of arcs has tacitly been made.

\section{Node-Arc incidence matrix}

\begin{definition}[Node-Arc incidence matrix]
\end{definition}

\section{Set based notation for functions}
% pg 20
To simplify the notation, we adopt the following conventions. If $X$ and $Y$
are subsets of $N$ let $(X, Y)$ denote the set of all arcs that lead from
$x \in X$ to $y \in Y$. For any function $g: A \rightarrow \R$, let:

\begin{equation}
    \tag{function on arc-set}
    g(X, Y) \equiv \sum_{(x, y) \in (X, Y)} g(x, y)
    \label{defn:function-on-arc-set}
\end{equation}

Similarly, when dealing with a function $h: N \rightarrow \R$ defined on
nodes, let:

\begin{equation}
    \tag{function on node-set}
    h(X) \equiv \sum_{x \in X} h(x)
\end{equation}

If we have $X, Y, Z \subseteq N$, then:
\begin{align*}
    &g(X, Y \cup Z) = g(X, Y) + g(X, Z) - g(X, Y \cap Z)  \\
    &g(Y \cup Z, X) = g(Y, X) + g(Z, X) - g(Y \cap Z, X)
\end{align*}

Notice that:

\begin{align*}
    &(B(x), x) = (N, x) \\
    &(x, A(x)) = (x, N) \\
\end{align*}

and:


\begin{align*}
      g(N, X) = \sum_{x \in X} g(N, x) = \sum_{x \in X} g(B(x), x) \\
      g(X, N) = \sum_{x \in X} g(x, N) = \sum_{x \in X} g(x, A(x))
\end{align*}
 

\section{Cuts}
%pg 21
Progress towards a solution of maximal network flow problems is made with the
recognition of the importance of certain subsets of arcs which we shall
call cuts.

\begin{definition}[Cut]
    A cut in $[N; A]$ separating $s$ and $t$ is a set of arcs $(X, \Xbar)$.
    The capacity of the cut is $c(X, \Xbar)$.
\end{definition}

\begin{lemma}
    Every chain from $s$ to $t$ must contain some arc of every cut $(X, \Xbar)$.
\end{lemma}
\begin{proof}
    Let $x_1, \dots, x_n$ be a chain with $x_1 = s, x_n = t$. Since $x_1 \in X$,
    $x_n \in \Xbar$, there must be some transitionary $x_i$ with $(1 \leq i \leq n)$ with
    $x_i \in X, x_{i+1} \in \Xbar$.  Hence the arc $(x_i, x_{i+1})$ is a 
    member of the cut $(X, \Xbar)$.
\end{proof}

\begin{lemma}
    If all arcs of a of a cut $(X, \Xbar)$ were deleted from the network, there would be no
    chain chain from $s$ to $t$, and the maximum flow value for the new network
    is zero.
\end{lemma}
\begin{proof}
    Every chain from $s$ to $t$ must contain \emph{some} element of the
    arc $(X, \Xbar)$. We delete \emph{all} elements of the arc $(X, \Xbar)$.
    Hence no chain from $s$ to $t$ can exist; if it did, it would have some
    element of the arc $(X, \Xbar)$, which has been deleted.
\end{proof}

Since a cut blocks all chains from $s$ to $t$, it is intuitively clear
(and indeed obvious in the arc-chain version of the problem) that the value
$v$ of a flow $f$ cannot exceed the capacity of any cut, a fact that
we now prove.

\section{Flows and Cuts}
%22

\begin{lemma}[Flow-Cut equality]
    \label{lemma:flow-cut-equality}
    Let $f$ be a flow from $s$ to $t$ in a network $[N; A]$. Let $f$ have
    value $v$. If $(X, \Xbar)$ is a cut separating $s$ and $t$, then

\begin{equation}
    \tag{Flow-Cut equality}
    v = f(X, \Xbar) - f(\Xbar, X) \leq c(X, \Xbar) 
    \label{eqn:flow-cut-equality}
\end{equation}

That is, the value of a flow $f$ from $s$ to $t$ is equal to the net flow
across any cut $(X, \Xbar)$ separating $s$ and $t$.
\end{lemma}
\begin{proof}
Since $f$ is a flow, it satisfies the equations:

\begin{align*}
    &f(s, N) - f(N, s) = v \\
    &f(x, N) - f(N, x) = 0 \quad \{x \neq s, t\} \\
    &f(t, N) - f(N, t) = -v
\end{align*}

We first establish that $v = f(X, N) - f(N, X)$. To show this, we sum these
equations over $x \in X$. since $s \in X$, while $t \in \Xbar$ [hence $t \not \in X]$, we get:

\begin{align*}
    &f(X, N) - f(N, X) = \sum_{x \in X} f(x, N) - f(N, x)\\
    &= (f(s, N) - f(N, s)) + \sum_{x \in X, x \neq s} f(x, N) - f(N, x) \\
    &= v + \sum_{x \in X, x \neq s} 0 = v
\end{align*}

The above is reasonable. we have $s \in X$ contribute $v$ units. 
Since $t \not in X$, it cannot contribute to this sum. All other
$x \in X$ contribute $0$ units. Thus the total contribution is $v$.  Now, writing $N = X \cup \Xbar$ yields:

\begin{align*}
&v = f(X, N) - f(N, X) \quad \text{(From the above)} \\
&= f(X, X \cup \Xbar) - f(X \cup \Xbar, X) \\
&= f(X, X) + f(X, \Xbar) - f(X, X) - f(\Xbar, X) \\
&= f(X, \Xbar) - f(\Xbar, X)
\end{align*}

Finally, to show that this is upper bounded by capacity, recall that $0 \leq f(X, \Xbar) \leq c(X, \Xbar)$
by the definition of a valid flow, and the inequality follows.

\end{proof}

\section{Maximal Flow}
% 23
\begin{theorem}[Max-flow-min-cut]
    For any network, the maximal flow value from $s$ to $t$ is equal to the 
    minimal cut capcacity over cuts separating $s$ and $t$.
\end{theorem}
\begin{proof}
By our lemma \ref{lemma:flow-cut-equality}, it suffices to establish
a flow $f$ and a cut $(X_f, \Xbar_f)$ for which equality of
flow value $f$ and cut capacity $c(X_f, \Xbar_f)$ holds. We do this
by taking a maximal flow $f$ (why does this exist?), and defining
in terms of $f$ a cut $(X_f, \Xbar_f)$ such that $f(X_f, \Xbar_f) = c(X_f, \Xbar_f)$,
and $f(\Xbar_f, X_f) = 0$. This allows the equality to hold:

\begin{align*}
&v = f(X_f, \Xbar_f) - f(\Xbar_f, X_f) \leq c(X_f, \Xbar_f) \\
&v = f(X_f, \Xbar_f) - 0 = c(X_f, \Xbar_f)
\end{align*}

Let $f$ be a maximal flow. We build the set $X_f$ as the set of all nodes
reachable from $s$ through unsaturated arcs. Formally, we define the
the equations:
\begin{align*}
    &(a) ~ s \in X_f \\
    &(b_1) ~ \text{if $x \in X_f$ and $f(x, y) \lneq c(x, y)$, then $y \in X_f$ } \\
    &(b_2) ~ \text{if $x \in X_f$ and $f(y, x) \gneq 0$, then $y \in X_f$ } \\
\end{align*}

We assert that $t \not \in \Xbar_f$. Suppose not. it follows from the definition of $X_f$
that there is a path from $s$ to $t$: $[s = x_1], x_2, \dots, x_{n-1}, [x_n = t]$.
For all forward arcs, we have $f(x_i, x_{i+1}) < c(x_i, x_{i+1})$. For reverse
arcs, we have $f(x_{i+1}, x_i) > 0$. Let $\eps_{fwd}$ be the minimum of 
$(c - f)$ taken over all forward arcs of the path. Let $\eps_{bwd}$ be the
minimum of $f$ taken over all reverse arcs. Let $\eps = \min(\eps_{fwd}, \eps_{bwd}) > 0$.
We now alter the flow $f$ to create a new flow $f'$ by adding $\eps$ over all forward arcs and decreasing
$f$ by $\eps$ on all reverse arcs of the path. $f'$ is a valid
flow from $s$ to $t$ that has value$v + \epsilon$. Then $f$ is not maximal,
contradicting our assumption. So, $t \in \Xbar_f$. Thus, $(X_f, \Xbar_f)$ is
a separating cut of $s$ and $t$.


Thus we have that:
\begin{enumerate}
\item $\forall (x, \xbar) \in (X, \Xbar), f(x, \xbar) = c(x, \xbar)$.
\item $\forall (x, \xbar) \in (X, \Xbar), f(\xbar, x) = 0$.
\end{enumerate}

We must have $f(\xbar, x) = 0$, for otherwise, we would have $\xbar \in X$.
\end{proof}

\begin{definition}[Augmenting Path]
    a path from $s$ to $t$ with respect to a flow $f$ is called as an
    augmenting path if $f < c$ on all forward arcs on the path and $f > 0$
    on all reverse arcs of the path.
\end{definition}

\begin{theorem}[Maximal flows do not have augmenting paths]
    A flow $f$ is maximal if and only if there is no flow augmenting path
    with respect to $f$.
\end{theorem}
\begin{proof}[Proof:(Maximal implies no augmenting path)]
    If a flow $f$ is maximal, there cannot exist an augmenting path. If there
    was, we can increase the flow along the augmenting path, thereby
    contradicting that maximality of $f$.
\end{proof}
\begin{proof}[Proof:(No augmenting path implies maximal)]
    We assume that no augmenting path exists. Then the set $X_f$ as defined
    before cannot contain the sink $t$. Recall the definition of $X_f$:
    \begin{align*}
        &(a) ~ s \in X_f \\
        &(b_1) ~ \text{if $x \in X_f$ and $f(x, y) \lneq c(x, y)$, then $y \in X_f$ } \\
        &(b_2) ~ \text{if $x \in X_f$ and $f(y, x) \gneq 0$, then $y \in X_f$ } \\
    \end{align*}

    Hence, we have that $(X, \Xbar)$ is a cut separating $s$ and $t$. This cut
    has capacity equal to the value of $f$ as before. Hence the flow $f$
    is maximal.
\end{proof}

\section{Imposing lower bounds on arc flows}
We can replace the flow conditions from $0 \leq f(x, y) \leq c(x, y)$ into
$l(x, y) \leq f(x, y) \leq c(x, y)$, where $l: A \rightarrow \R^+$ is a real-valued
function on the arcs such that $0 \leq l(x, y) \leq c(x, y)$.


We can change the labelling process to handle the situation if we have an
initial flow $f_0$ that satisfies the equation.

\textbf{TODO: how do we find such a starting flow?}

\section{Node capacities}

We can set a node capacity $k: N \rightarrow \R+$ along with with the 
arc capacity constraints. We want to maximize $f(s, N)$ subject to the
constraints:

\begin{enumerate}
    \item $f(x, N) - f(N, x) = 0 \quad x \neq s, t \quad \text{[Convervation]}$
    \item $\forall (x, y) \in A: 0 \leq f(x, y) \leq c(x, y) \quad $ \text{[Capacity constraints]}
    \item $f(x, N) \leq k(x) \quad x \neq t \quad \text{[$x$ can only send (and receive, by conservation) a maximum flow of $k(x)$]}$
    \item $f(N, t)  \leq k(t) \quad \text{[$t$ can only receive a maximum flow of $k(t)$]}$
\end{enumerate}


\section{Supply-Demand theorem}

Let $(N, A)$ be a network with capacity $c$. Suppose that $N$ is partitioned into
sources $S$, sinks $T$, intermediate nodes $R$. Associate with each $s \in S$
a non-negative real number $a(s)$, which is the supply of a commodity at $s$;
similarly for each $t \in T$, associate a non-negative number $b(t)$, which
is a demand of a commodity at $t$.

\bibliographystyle{plain}
\bibliography{references}
\nocite{*}

\end{document}

