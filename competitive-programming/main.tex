% https://github.com/cohomolo-gy/cats-in-context/blob/master/chapter-2/Chapter%202%20Solutions.tex

\documentclass[14pt]{report}
\usepackage{bbm}
\usepackage{bbding} % for flower. 
\usepackage{physics}
\usepackage{amsmath,amssymb}
\usepackage{graphicx}
\usepackage{makeidx}
\usepackage{algpseudocode}
\usepackage{algorithm}
\usepackage{listing}
\usepackage{minted}
\usemintedstyle{perldoc}
\usepackage{quiver}
\usepackage{enumitem}
\usepackage{mathtools}

\usepackage{xargs}

\usepackage{hyperref}
\hypersetup{
    colorlinks,
    citecolor=blue,
    filecolor=blue,
    linkcolor=blue,
    urlcolor=blue
}

\usepackage{epsfig}
\usepackage{tabularx}
\usepackage{latexsym}
\newcommand{\N}{\ensuremath{\mathbb{N}}}
\newcommand{\Z}{\ensuremath{\mathbb{Z}}}
\newcommand{\Q}{\ensuremath{\mathbb{Q}}}
\newcommand{\R}{\ensuremath{\mathbb R}}

\def\qed{\ensuremath{\Box}}

\newcommand*{\start}[1]{\leavevmode\newline \textbf{#1} }
\newcommand*{\question}[1]{\leavevmode\newline \textbf{Question: #1.}}
\newcommand*{\proof}[1]{\leavevmode\newline \textbf{Proof #1}}
\newcommand*{\answer}{\leavevmode\newline \textbf{Answer} }
\DeclareMathOperator{\lcm}{lcm}


\usepackage[adobe-utopia]{mathdesign}
\usepackage[T1]{fontenc}

\title{Competitive programming proofs}
\author{Siddharth Bhat}
\date{Monsoon, second year of the plague}


\begin{document}
\maketitle
\tableofcontents
\section{Codeforces 1389C: woodcutters}
Problem link is \url{https://codeforces.com/problemset/problem/1389/C}.
\begin{itemize}
\item Solution is greedy.
\item It's hopefully clear that it's always optimal to have the
    first tree lean left and the last tree lean right.  So we now
    have the prove that the construction for trees `[1..n-2]` is
    optimal.

\item Let $O_*$ be the optimal solution, $O$ be the solution discovered by the
    above algorithm.  Let index $i$ be the first index where $O_*$ and $O$
        decide to do something different to a tree.  We have $i > 0, i < n-2$
        since we assume that the algorithms agree on the first and last tree.
\item We have  6 cases at index `i`: the possibilities are `L` for leaning left, `U` for standing up, and `R` for leaning right.
\begin{itemize}
\item  $O_*=L, O=U$: If there is enough space to lean left, $O$ will choose to
    lean left. This violates the construction of $O$.
\item  $O_*=L, O=R$: Same as above; If there is space to lean left, $O$ will
    choose to lean left. This violates the construction of $O$.
\item  $O_*=U, O=L$: Since $O$ and $O_*$ agree upto $i$ and $O$ chooses to lean
    left, there must be enough space to lean left for $O_*$. This creates a
        strictly better solution,  which violates the optimality of $O_*$.
\item $O_*=U, O=R$: Now what? we can't control $O$ based on $O_*$ or vice
    versa.  We'll return to this case at the end, as it's the most complex.
\item $O_*=R, O=L$: There must be enough space for $O_*$ to lean $L$. $O_*=R$
    only limits the space available for the next tree. So we can modify $O_*$
        to lean left and reach the same optimality.
\item $O_*=R, O=U$: If it's possible to lean right, then $O$ will choose to do
    so. This violates the construction of $O$.
\end{itemize}

\item $O_*=U, O=R$: This is the complex case.
    \begin{itemize}
            \item In this case, let's consider the sequence of trees felled by $O_*$ after $i$.
            \item Let us suppose that $O_*$ fells many trees rightward,
                followed by a tree that is felled in the direction $d_*$ (where
                $d_* \in \{L, U \} \neq R$.  Formally, $O_*[i, i+1, k] = U; R^k; d_*$ for
                some $d_* \neq R$. (That is, $d_* \equiv O_*[k]$).
            \item If such a $d_O$ does not exist, then $O$ can
                mimic what $O_*$ does, since $O$'s decision to move $O[i]$ right does not impact any tree in the future.
            \item If such a $d_*$ does exist, then let us consider what $d_*$ is.
            \item If $d_* \equiv O_*[k]$ is $U$, then we can have $O[k] = U$.
            \item If $d_* \equiv O_*[k]$ is $L$, then we can have $O[k] = U$.
    \end{itemize}
\end{itemize}

\section{Codeforces  1175 C: Electrification}

sliding window argument.

I will proof this by a simple contradiction argument Assume that at least one of the N nearest neighbours of x does not belong to a continuous window of size N containing x. Let us assume this nearest neighbour is Aj. Without loss of generality, assume Aj<x (same proof works for Aj>x). As the window is not continuous as per our assumption, there must exist some element Ak between Aj and x which is not a nearest neighbour of x. But distance of Ak from x is clearly less than distance of Aj from x. So if Aj is a nearest neighbour, Ak must also be a nearest neighbour. That contradicts our assumption.




\section{Codeforces 190D: Non-Secret Cypher}

\begin{quote}{Yeputons}
It's kind of optimization. Consider problem \href{https://codeforces.com/contest/190/problem/D}{190D - Non-Secret Cypher}. The
stupid solution goes over all subarrays and checks whether it's good or
not. Now we notice that if subarray is 'good', then all its superarrays is
'good' too. Let's go over all left borders $1 \leq x \leq N$
and for each of them find $r_x$. rx is a minimal number such that subarray
$x \dots  r_x$ is good.  Obviously, all subarrays which starts at $x$ and
ends after $r_x$ is good.  If you notice that $r_1 \leq r_2 \leq \dots \leq r_N$,
you can use 'two pointers' method. The first pointer is $x$ and the second one
is $r_x$. When you move the first one right, the second one
\emph{can only move right} too. No matter how many operations you perform in one step,
algo's running time is $O(n)$, because each of pointers makes $\leq N$
steps.
\end{quote}

\section{Codeforces 1389C:}
\begin{itemize}
\item It's safe to look for a longest satisfying match with DP, and if the
    match is of odd length, to decrement it to get an even length string.
\item This would NOT be safe if we had to chop of an arbitrary length $\delta$!
    \begin{itemize}
        \item To instill fear, suppose $\delta = 5$,
            and that on some problem instance, the best string had length $9$, second best had length $7$.
        \item Since the DP only keeps track of the longest length, we would
            have tracked $9$, and then subtracted $5$ to report $9-5=4$ which
            is much less than the real second-optima of $7$.
    \end{itemize}
\item The reason this is safe when we chop off length $1$ is because:
    \begin{itemize}
            \item We already know that, say, $9$ is the longest string length.
                There are no longer solutions that can compete. So we can must
                only worry about shorter solutions
            \item Amongst the second largest solutions, we must worry about the second largest solution being $8, 7, 6, \dots$.
            \item Since we only decrement by $1$, our "mutated" largest solutoin will be $8$. This is as large as the 
                maximum second-largest solution.
            \item So we do not lost any second-best optima, since we mutate the first best optima to drop by $1$, which makes it
                the best-in-class second best optima.
    \end{itemize}
\end{itemize}


\documentclass[10pt,a4paper]{article}


%\usepackage{geometry}
\usepackage{mathrsfs}
\usepackage{epsfig}
\usepackage{helvet}
\usepackage{courier}
\usepackage{amsmath, amssymb, amsthm, amsfonts, graphicx}
\usepackage{url,color}
\usepackage{tabularx}
\usepackage{amssymb}
\usepackage{amsmath}
\usepackage{amsthm}
\usepackage{nicefrac}
\usepackage{graphicx}
%\graphicspath{ {/home/vatsal/IIIT/Sem4/OM/Homework} }
\usepackage{epsfig}
\usepackage{hyperref}

\usepackage{tabu}
\usepackage{algorithm}
\usepackage[noend]{algpseudocode}
\usepackage{wrapfig}
\usepackage{empheq}
\usepackage{ragged2e}
\usepackage{multicol}
\usepackage{mathtools}
\usepackage{pstricks-add, auto-pst-pdf}
\usepackage{tikz}
\usepackage{textcomp}
\usetikzlibrary{positioning,chains,fit,shapes,calc}

\frenchspacing
%\newtheorem{theorem}{Theorem}
\newtheorem{note}{Note}
\newtheorem{lemma}{Lemma}
\newtheorem{prop}{Proposition}
\newtheorem{theorem}{Theorem}
\newtheorem{definition}{Definition}

\usepackage{tikz}
\usetikzlibrary{calc}
\usepackage{caption}
\setlength{\topmargin}{ 0.1in}
\setlength{\columnsep}{2.0pc}
\setlength{\headheight}{0.0in} \setlength{\headsep}{0.0in}
\setlength{\oddsidemargin}{.15in} \setlength{\parindent}{1pc}
\setlength{\evensidemargin}{.15in} \setlength{\parindent}{1pc}
\setlength{\parsep}{15pt}
\textheight 9.0in \textwidth 6.0in
\newcommand{\hr}{\noindent\rule{\textwidth}{.35mm}\vspace{8pt}}% 


\newcommand{\N}{\ensuremath{\mathbb{N}}}
\newcommand{\R}{\ensuremath{\mathbb R}}



\begin{document}


\begin{table}[!h]
\centering
%\resizebox{\textwidth}{!}{
\begin{tabularx}{\textwidth}{|Xll|}
\hline
& &\\
Introduction to Game Theory &  Date: & \emph{23rd March 2020}\\
 & &\\
Instructor: \emph{Sujit Prakash Gujar} & Scribes: & {Siddharth Bhat \& Harshit Sankhla} \\ 
 \hline

\end{tabularx}
%}
\end{table}

\begin{center}
\begin{LARGE}
Lecture X: Quasi-Linear games
\end{LARGE}
\end{center}

\section{Recap}
This is the a special class of environments where the Gibbard–Satterthwaite
theorem does not hold.  A popular example of quasi-linear games are actions.

\section{Introduction}
(We follow some of the exposition of
\href{http://lcm.csa.iisc.ernet.in/gametheory/ln/web-md6-quasilinear.pdf}{Game Theory by Y. Narahari: The quasilinear environment}).
The structure of the quasi-linear setting is as follows:

\begin{align*}
X \equiv \left\{ (k, t_1, \dots, t_n) : k \in K, t_i \in \R, \sum_i t_i \leq 0 \right\}.
\end{align*}


where $X$ is the space of alternatives, $K$ is the set of possible allocations.
$k \in K$ is the currently chosen allocation, and $t_i$ are monetary transfer receives by agent $i$.
By convention $t_i > 0$ implies that the agent \emph{receives money}, and
$t_i < 0$ implies that the agent \emph{is paid money}. We assume that our
agents have no external source of funding (the \emph{weakly budget-balanced}
condition). Hence, we stipulate that $\sum_i t_i \leq 0$.

A social choice function (henceforth abbreviated as SCF) in this setting
is of the form $f: \Theta \rightarrow X$, where we write 
$f(\theta \in \Theta) \equiv (k(\theta), t_1(\theta), t_2(\theta), \dots, t_n(\theta)) \in X$.
That is, we require that $k: \Theta \rightarrow K$, $t_i: \Theta \rightarrow \R$
such that for alll $\theta \in \Theta, \sum_i t_i(\theta) \leq 0$.

This setting is known as quasi-linear since the agent's utility function
is of the form:
\begin{align*}
	&u_i: X \times \Theta_i \rightarrow \R; 
	u_i(x, \theta_i) \equiv u_i((k, t_1, t_2, \dots, t_n), \theta_i) = v_i(k, \theta_i) + t_i \\
	&v_i: K \times \Theta_i \rightarrow \R \equiv \text{(Agent $i$'s valuation)} \quad t_i \equiv \text{amount paid to agent} 
\end{align*}

Here, $v_i : \Theta \rightarrow \R$ is the agent's valuation function, and $t_i$
is the amount that is paid (or is to be paid) by the agent. This informs
our choice of sign convention for $t_i$: if the agent $i$ \emph{is paid}, then
it has earned money, $t_i$ is positive, its utility is higher. 

\begin{definition}{Allocative Efficiency(AE)}
We say that a social choice function $f: \Theta \rightarrow X$
is allocatively efficient iff for all states of private information,
the SCF causes us to choose the allocation that leads to the \emph{maximum common good}.
More formally,  for all $(\theta_1, \theta_2, \dots, \theta_n) \in \Theta$, we have that:
$$k(\theta) \in \arg \max_{k \in K} \sum_{i=1}^n v_i(k, \theta_i).$$

Equivalently:

$$
\sum_{i=1}^n v_i(k(\theta), \theta_i) = \arg \max_{k \in K} \sum_{i=1}^n v_i(k, \theta_i).
$$
\end{definition}


We can think about this as saying:
\begin{quote}
``Every allocation is value-maximizing allocation. Allocations are given to
those agents that covet them.''
\end{quote}

\begin{definition}{Budget Balance(BB)}
Recall that a social choice function $f: \Theta \rightarrow X$ is said to be
\emph{budget-balanced} iff the total money is conserved for all states
of private information. Formally:

$$\forall \theta \in \Theta, ~ \sum_i t_i(\theta) = 0$$
\end{definition}

We first show that the class of quasi-linear functions is non-degenerate,
in the sense that it is non-dictatorial.

\begin{lemma}
All social choice function $f: \Theta \rightarrow X$
in the quasilinear setting are non-dictatorial.
\end{lemma}
Let us assume we have a dictator who is player $d$ (for dictator).
For every $\theta \in \Theta$, we have that:

$$
u_d(f(\theta), \theta_d) \geq u_d(x, \theta_d) ~~\forall x \in X.
$$

This models a dictator since this tells us that $u_d$ gets what he wants
for all scenarios. Written differently:

$$
u_d(f(\theta), \theta_d) = \max_{x \in X} u_d(x, \theta_d)
$$

Since our environment is quasi-linear, we have that
$u_d(f(\theta), \theta_d) = v_d(k(\theta), \theta_d) + t_d(\theta)$. Hence, 
we can an alternative $f' : \Theta \rightarrow X$:

$$
f(\theta)
\begin{cases}
(k(\theta), (t_{-d}(\theta), t_d \equiv t_d(\theta)  - \sum_i t_i(\theta))) & \sum_{i=1}^n t_i(\theta) < 0
\end{cases}
$$
% & \sum_{i=1}^n t_i(\theta) = 0
% \end{cases}
% $$

For the following outcome, we have that $u_d(x, \theta) > u_d(f'(\theta), \theta_d)$
which contradicts the assumption that $d$ is a dictator.

\qed.



\begin{definition}{Ex-post efficiency}
Recall that Ex-post efficiency is when the item is always allotted to the agents
that value it the most. Formally, we state that a
social choice function $f: \Theta \rightarrow X$ is said to be \emph{Ex-post efficient} iff:
$$
\sum_{i=1}^n u_i(k(\theta), \theta_i) = \arg \max_{k \in K} \sum_{i=1}^n u_i(k, \theta_i).
$$
\end{definition}

\begin{lemma}
A social choice function $f: \Theta \rightarrow X$
in the quasilinear setting is Ex-post efficient (EPE)
iff it is budget-balanced.
\end{lemma}

We can either relax DSIC or relax rich preference structure. We decided
to look at quasi-linear environments where we relax preferences. A popular
example of this is auctions.

$X = \{ (k, t_1, \dots, t_n) : k \in K, t_i \in \R, \sum_i t_i \leq 0 \}$

$t_i$ is monetary transfer receives by agent $i$.

$u_i(x, \theta_i) = v_i(k, \theta_i) + t_i$. Linear in $t_i$, hence
the setting is quasi-linear. Often it is even $k_i \cdot \theta_i + t_i$ --- these
settings are known as linear settings.

\section{Examples of SCF in quasi-linear settings}
\begin{itemize}
    \item \textbf{Players}: Seller and two buyers
    \item \textbf{Private information}: Seller $\Theta_0 = \{ 0 \}$. Byers = $\theta_1 =\theta_2 = [0, 1]$.
\end{itemize}

\section{Allocative efficiency}
an SCF $f(\cdot)$ is allocative efficient if it maximises sum of valuations
of agents. We assume such a maxima does exist.
$k^{\star}(\theta) \in \arg \max_{k \in K} \sum_{i=1}^n v_i(k, \theta_i)$

We also want budget balance:

$\sum_{i=1}^n t_i(\theta) = 0$.

\section{Properties of SCF(Social choice function) in quasi-linear settings}

\begin{lemma}
All SCFs in quasi-linear settings are non dictatorial.
\end{lemma}

because $\sum_i t_i < 0$, we can increase payment for the dictator by using
$t_i + \frac{e}{n - 1}$ and decrease everyone else to $t_i - \frac{e}{n-1}$.
So, there is always an outcome that is better for a dictator. Hence,
the best outcome cannot have a dictator.

\section{Ex-post efficiency}
in quasi linear, scf is exp-post efficient iff if is allocatively efficient
and strictly budget balanced. We have to prove that $EPE \implies AE + SBB$,
and also $AE + SBB \implies EPE$.

Suppose $f = (k, t)$ is EPE but not SBB. So there exists a $\theta$ such that 
$\sum_i t_i(\theta) < 0$. Hence, there exists at least
one agent $j$ such that $t_j < 0$. (If everyone is positive, sum cannot be less than 0).

Now consider a new allocation $X' = (k, t')$ where 

$t'_j(\theta) = 
\begin{cases}
    t_j(\theta) -  \sum_i t_i(\theta)/n & \text{if $t_j(\theta) < 0$} \\
    t_j(\theta) & \text{otherwise}
\end{cases}
$ 

Hence, $u'_j(k, t') > u_j(k, t)$ for such $j$ where $t_j(\theta) < 0$.
For other agents, $u'_j(k, t') = u'_j(k, t)$.

This means that $(k. t')$ pareto dominates $(k. t)$. This is a contradiction
to the assumption that $f$ was EPE, since we constructed an outcome where
one agent does better, and others don't do worse.

We now argue that f must be allocatively efficient, if  f is EPE. For contradiction,
let us assume that $f$ is not AE.
That means that there is a $k^\star$ such that
$\sum_i v_i(k^\star, \theta) > v_i (k, \theta)$.

Define $t_i'(\theta)  = v_i(k, \theta) - t_i (\theta) - \sum_j \theta_j(k^\star, \theta) + \epsilon$
where $\epsilon < \sum_j v_j(k^star, \theta) - theta_j (k, \theta)$.

Note that $v_i(k, \theta) - t_i (\theta)  = u_i(k, t)$. 
Now note that
$u_i(k^\star, t') = u_i(k, t) + \epsilon/n$, where $\epsilon$ is positive.
Hence, $u_i(k^\star, t') > u_i(k, t)$. 

We need to check that $t'$ is feasible: ie, $\sum_i t_i' < 0$.

$$
\sum_i t_i' = \sum_i v_i(k, theta) - \sum v_j(k^\star, \theta) + \sum_i t_i(\theta) \leq 0??
$$

Also note that for all $i$, $u_i(k^\star, t') > u_i(k, t)$. This is contradiction
to the fact that $f$ is APE. Hence, $f$ must be AE.

\section{Other way round: if $f$ is AE + SBB, then it is EPE}

For this, we will need to prove a lemma:

\begin{lemma}
If $f: \Theta \rightarrow X$ st $\forall \theta \in \Theta$,
$$
\sum_i u_i(f(\theta), \theta_i) \geq \sum_i u_i(x, \theta_i) \forall x \in X
$$
then $f$ is EPE.

The key idea is to write $u_i = v_i + t_i$, an we can get rid of $t_i$ since
$f$ is SBB.
\end{lemma}

\section{First price versus second price auction}
First price: reporting valuation truthfully is not an equilibrium. Second
price: truthful reporting is equilibrium.

How do we generalize this to more situations? The key idea is that in a second
price auction, our payment is independent of what we report. The allocation
might depend on our payment, but payment does not. How can we have more
DSIC mechanisms?

\section{Groves theorem}
TODO: fill up groves theorem



Three families A B C, can go to Munnar or Simla. 


\begin{tabular}{l r r l}
  &  Manali &  Shimoga & \\
Alice & -1 &  10 & \\
Bob & 5  & -2  & \\
Claire & 5  & 4  & (Claire is a kid, loves vacations) \\
\end{tabular}

We want to get this information truthfully, by using VCG/Groves mechanism.

there are two outcoomes, M or S . If we go to M, the tuility is 5+5-1=9. If
we choose S, it is 10-2+4=12. so S is allocatively efficient.



\begin{tabular}{l r r r r r r r}
    & $\{ A \}$  & $\{ B \}$   & $\{ C \}$  &  $\{ A, B \}$ & $\{ A, C \}$ & $\{ B, C  \}$ & $\{  A, B, C \}$ \\
$P_1$  & 10 & 0   & 5  &  10  &  20 &   5  &  20 \\
$P_2$  & 0  & 9   & 15  &  9  &  15 &  20  &  20 \\
$P_3$  & 10 &  2   & 2  &  10  &  12 &  2  &   28 \\
$P_4$  & 8  &  3   & 3  &  8  &   8 &   3  &    8
\end{tabular}


Giving $A$ to $P_1$ and $BC$ to $P_2$ gives $10 + 20 = 30$.


A direct revelation mechanism in which $f$ satisfies allocative efficiency
and the groves payment scheme is knows as the groves mechanism.

before this, there is another mechanism called as Clarke's mechanism

\section{Clarke's mechanism}

$h_i(\theta_i) = \sum_{j \neq i} v_j(k_{-i}^\star(\theta_{-i}, \theta_j)) \forall \theta_{-i} \in \Theta_{-i}$

That is, each agent $i$ receives
$$
t_i(\theta) = \sum_{j \neq i}(v_j(k^\star(\theta), \theta_j)) - \sum_{j \neq i} v_j(k^\star_{-i}(\theta_{-i}), \theta_j))
$$

This works for combinatorial auctions as well. It's a generalization
of second-price auction.

\begin{tabular}{l r r l}
  & M & S & \\
A &-1  &10 &  \\
B & 5  &-2 & \\
C & 5 & 4 &  (C is a kid, loves vacations) \\
\end{tabular}

For player A, first consider:

\begin{tabular}{l r r l}
   & M  & S & \\
A  & -  & -  & \\
B  & 5  & -2 & \\
C  & 5  & 4  & (C is a kid, loves vacations) \\
\end{tabular}

AE is M. 

Following Clarke Mechanism:
\begin{align*}
t_A = &[\text{valuation of remaining agents at allocatively efficient outcome without A}](-2+4)  \\
     & - [\text{valuation of remaining agents at allocatively efficient outcome with A}][5+5] \\
     = 8
\end{align*}


for player B, first consider:

\begin{tabular}{cccc}
    A    & -1 &  10 & \\
B    &  - & -1 &  -   \\
C    &  5 & -1 &  4  \\
     &  M & -1 &  S 
\end{tabular}

AE is S.  So, $t_B = 0$. Similarly, $t_C = 0$.


\bibliographystyle{apalike}
\bibliography{igt20}

\end{document}


\documentclass[10pt,a4paper]{article}


%\usepackage{geometry}
\usepackage{mathrsfs}
\usepackage{epsfig}
\usepackage{helvet}
\usepackage{courier}
\usepackage{amsmath, amssymb, amsthm, amsfonts, graphicx}
\usepackage{url,color}
\usepackage{tabularx}
\usepackage{amssymb}
\usepackage{amsmath}
\usepackage{amsthm}
\usepackage{nicefrac}
\usepackage{graphicx}
%\graphicspath{ {/home/vatsal/IIIT/Sem4/OM/Homework} }
\usepackage{epsfig}
\usepackage{hyperref}

\usepackage{tabu}
\usepackage{algorithm}
\usepackage[noend]{algpseudocode}
\usepackage{wrapfig}
\usepackage{empheq}
\usepackage{ragged2e}
\usepackage{multicol}
\usepackage{mathtools}
\usepackage{pstricks-add, auto-pst-pdf}
\usepackage{tikz}
\usepackage{textcomp}
\usetikzlibrary{positioning,chains,fit,shapes,calc}

\frenchspacing
%\newtheorem{theorem}{Theorem}
\newtheorem{note}{Note}
\newtheorem{lemma}{Lemma}
\newtheorem{prop}{Proposition}
\newtheorem{theorem}{Theorem}
\newtheorem{definition}{Definition}

\usepackage{tikz}
\usetikzlibrary{calc}
\usepackage{caption}
\setlength{\topmargin}{ 0.1in}
\setlength{\columnsep}{2.0pc}
\setlength{\headheight}{0.0in} \setlength{\headsep}{0.0in}
\setlength{\oddsidemargin}{.15in} \setlength{\parindent}{1pc}
\setlength{\evensidemargin}{.15in} \setlength{\parindent}{1pc}
\setlength{\parsep}{15pt}
\textheight 9.0in \textwidth 6.0in
\newcommand{\hr}{\noindent\rule{\textwidth}{.35mm}\vspace{8pt}}% 


\newcommand{\N}{\ensuremath{\mathbb{N}}}
\newcommand{\R}{\ensuremath{\mathbb R}}



\begin{document}


\begin{table}[!h]
\centering
%\resizebox{\textwidth}{!}{
\begin{tabularx}{\textwidth}{|Xll|}
\hline
& &\\
Introduction to Game Theory &  Date: & \emph{23rd March 2020}\\
 & &\\
Instructor: \emph{Sujit Prakash Gujar} & Scribes: & {Siddharth Bhat \& Harshit Sankhla} \\ 
 \hline

\end{tabularx}
%}
\end{table}

\begin{center}
\begin{LARGE}
Lecture X: Quasi-Linear games
\end{LARGE}
\end{center}

\section{Recap}
This is the a special class of environments where the Gibbard–Satterthwaite
theorem does not hold.  A popular example of quasi-linear games are actions.

\section{Introduction}
(We follow some of the exposition of
\href{http://lcm.csa.iisc.ernet.in/gametheory/ln/web-md6-quasilinear.pdf}{Game Theory by Y. Narahari: The quasilinear environment}).
The structure of the quasi-linear setting is as follows:

\begin{align*}
X \equiv \left\{ (k, t_1, \dots, t_n) : k \in K, t_i \in \R, \sum_i t_i \leq 0 \right\}.
\end{align*}


where $X$ is the space of alternatives, $K$ is the set of possible allocations.
$k \in K$ is the currently chosen allocation, and $t_i$ are monetary transfer receives by agent $i$.
By convention $t_i > 0$ implies that the agent \emph{receives money}, and
$t_i < 0$ implies that the agent \emph{is paid money}. We assume that our
agents have no external source of funding (the \emph{weakly budget-balanced}
condition). Hence, we stipulate that $\sum_i t_i \leq 0$.

A social choice function (henceforth abbreviated as SCF) in this setting
is of the form $f: \Theta \rightarrow X$, where we write 
$f(\theta \in \Theta) \equiv (k(\theta), t_1(\theta), t_2(\theta), \dots, t_n(\theta)) \in X$.
That is, we require that $k: \Theta \rightarrow K$, $t_i: \Theta \rightarrow \R$
such that for alll $\theta \in \Theta, \sum_i t_i(\theta) \leq 0$.

This setting is known as quasi-linear since the agent's utility function
is of the form:
\begin{align*}
	&u_i: X \times \Theta_i \rightarrow \R; 
	u_i(x, \theta_i) \equiv u_i((k, t_1, t_2, \dots, t_n), \theta_i) = v_i(k, \theta_i) + t_i \\
	&v_i: K \times \Theta_i \rightarrow \R \equiv \text{(Agent $i$'s valuation)} \quad t_i \equiv \text{amount paid to agent} 
\end{align*}

Here, $v_i : \Theta \rightarrow \R$ is the agent's valuation function, and $t_i$
is the amount that is paid (or is to be paid) by the agent. This informs
our choice of sign convention for $t_i$: if the agent $i$ \emph{is paid}, then
it has earned money, $t_i$ is positive, its utility is higher. 

\begin{definition}{Allocative Efficiency(AE)}
We say that a social choice function $f: \Theta \rightarrow X$
is allocatively efficient iff for all states of private information,
the SCF causes us to choose the allocation that leads to the \emph{maximum common good}.
More formally,  for all $(\theta_1, \theta_2, \dots, \theta_n) \in \Theta$, we have that:
$$k(\theta) \in \arg \max_{k \in K} \sum_{i=1}^n v_i(k, \theta_i).$$

Equivalently:

$$
\sum_{i=1}^n v_i(k(\theta), \theta_i) = \arg \max_{k \in K} \sum_{i=1}^n v_i(k, \theta_i).
$$
\end{definition}


We can think about this as saying:
\begin{quote}
``Every allocation is value-maximizing allocation. Allocations are given to
those agents that covet them.''
\end{quote}

\begin{definition}{Budget Balance(BB)}
Recall that a social choice function $f: \Theta \rightarrow X$ is said to be
\emph{budget-balanced} iff the total money is conserved for all states
of private information. Formally:

$$\forall \theta \in \Theta, ~ \sum_i t_i(\theta) = 0$$
\end{definition}

We first show that the class of quasi-linear functions is non-degenerate,
in the sense that it is non-dictatorial.

\begin{lemma}
All social choice function $f: \Theta \rightarrow X$
in the quasilinear setting are non-dictatorial.
\end{lemma}
Let us assume we have a dictator who is player $d$ (for dictator).
For every $\theta \in \Theta$, we have that:

$$
u_d(f(\theta), \theta_d) \geq u_d(x, \theta_d) ~~\forall x \in X.
$$

This models a dictator since this tells us that $u_d$ gets what he wants
for all scenarios. Written differently:

$$
u_d(f(\theta), \theta_d) = \max_{x \in X} u_d(x, \theta_d)
$$

Since our environment is quasi-linear, we have that
$u_d(f(\theta), \theta_d) = v_d(k(\theta), \theta_d) + t_d(\theta)$. Hence, 
we can an alternative $f' : \Theta \rightarrow X$:

$$
f(\theta)
\begin{cases}
(k(\theta), (t_{-d}(\theta), t_d \equiv t_d(\theta)  - \sum_i t_i(\theta))) & \sum_{i=1}^n t_i(\theta) < 0
\end{cases}
$$
% & \sum_{i=1}^n t_i(\theta) = 0
% \end{cases}
% $$

For the following outcome, we have that $u_d(x, \theta) > u_d(f'(\theta), \theta_d)$
which contradicts the assumption that $d$ is a dictator.

\qed.



\begin{definition}{Ex-post efficiency}
Recall that Ex-post efficiency is when the item is always allotted to the agents
that value it the most. Formally, we state that a
social choice function $f: \Theta \rightarrow X$ is said to be \emph{Ex-post efficient} iff:
$$
\sum_{i=1}^n u_i(k(\theta), \theta_i) = \arg \max_{k \in K} \sum_{i=1}^n u_i(k, \theta_i).
$$
\end{definition}

\begin{lemma}
A social choice function $f: \Theta \rightarrow X$
in the quasilinear setting is Ex-post efficient (EPE)
iff it is budget-balanced.
\end{lemma}

We can either relax DSIC or relax rich preference structure. We decided
to look at quasi-linear environments where we relax preferences. A popular
example of this is auctions.

$X = \{ (k, t_1, \dots, t_n) : k \in K, t_i \in \R, \sum_i t_i \leq 0 \}$

$t_i$ is monetary transfer receives by agent $i$.

$u_i(x, \theta_i) = v_i(k, \theta_i) + t_i$. Linear in $t_i$, hence
the setting is quasi-linear. Often it is even $k_i \cdot \theta_i + t_i$ --- these
settings are known as linear settings.

\section{Examples of SCF in quasi-linear settings}
\begin{itemize}
    \item \textbf{Players}: Seller and two buyers
    \item \textbf{Private information}: Seller $\Theta_0 = \{ 0 \}$. Byers = $\theta_1 =\theta_2 = [0, 1]$.
\end{itemize}

\section{Allocative efficiency}
an SCF $f(\cdot)$ is allocative efficient if it maximises sum of valuations
of agents. We assume such a maxima does exist.
$k^{\star}(\theta) \in \arg \max_{k \in K} \sum_{i=1}^n v_i(k, \theta_i)$

We also want budget balance:

$\sum_{i=1}^n t_i(\theta) = 0$.

\section{Properties of SCF(Social choice function) in quasi-linear settings}

\begin{lemma}
All SCFs in quasi-linear settings are non dictatorial.
\end{lemma}

because $\sum_i t_i < 0$, we can increase payment for the dictator by using
$t_i + \frac{e}{n - 1}$ and decrease everyone else to $t_i - \frac{e}{n-1}$.
So, there is always an outcome that is better for a dictator. Hence,
the best outcome cannot have a dictator.

\section{Ex-post efficiency}
in quasi linear, scf is exp-post efficient iff if is allocatively efficient
and strictly budget balanced. We have to prove that $EPE \implies AE + SBB$,
and also $AE + SBB \implies EPE$.

Suppose $f = (k, t)$ is EPE but not SBB. So there exists a $\theta$ such that 
$\sum_i t_i(\theta) < 0$. Hence, there exists at least
one agent $j$ such that $t_j < 0$. (If everyone is positive, sum cannot be less than 0).

Now consider a new allocation $X' = (k, t')$ where 

$t'_j(\theta) = 
\begin{cases}
    t_j(\theta) -  \sum_i t_i(\theta)/n & \text{if $t_j(\theta) < 0$} \\
    t_j(\theta) & \text{otherwise}
\end{cases}
$ 

Hence, $u'_j(k, t') > u_j(k, t)$ for such $j$ where $t_j(\theta) < 0$.
For other agents, $u'_j(k, t') = u'_j(k, t)$.

This means that $(k. t')$ pareto dominates $(k. t)$. This is a contradiction
to the assumption that $f$ was EPE, since we constructed an outcome where
one agent does better, and others don't do worse.

We now argue that f must be allocatively efficient, if  f is EPE. For contradiction,
let us assume that $f$ is not AE.
That means that there is a $k^\star$ such that
$\sum_i v_i(k^\star, \theta) > v_i (k, \theta)$.

Define $t_i'(\theta)  = v_i(k, \theta) - t_i (\theta) - \sum_j \theta_j(k^\star, \theta) + \epsilon$
where $\epsilon < \sum_j v_j(k^star, \theta) - theta_j (k, \theta)$.

Note that $v_i(k, \theta) - t_i (\theta)  = u_i(k, t)$. 
Now note that
$u_i(k^\star, t') = u_i(k, t) + \epsilon/n$, where $\epsilon$ is positive.
Hence, $u_i(k^\star, t') > u_i(k, t)$. 

We need to check that $t'$ is feasible: ie, $\sum_i t_i' < 0$.

$$
\sum_i t_i' = \sum_i v_i(k, theta) - \sum v_j(k^\star, \theta) + \sum_i t_i(\theta) \leq 0??
$$

Also note that for all $i$, $u_i(k^\star, t') > u_i(k, t)$. This is contradiction
to the fact that $f$ is APE. Hence, $f$ must be AE.

\section{Other way round: if $f$ is AE + SBB, then it is EPE}

For this, we will need to prove a lemma:

\begin{lemma}
If $f: \Theta \rightarrow X$ st $\forall \theta \in \Theta$,
$$
\sum_i u_i(f(\theta), \theta_i) \geq \sum_i u_i(x, \theta_i) \forall x \in X
$$
then $f$ is EPE.

The key idea is to write $u_i = v_i + t_i$, an we can get rid of $t_i$ since
$f$ is SBB.
\end{lemma}

\section{First price versus second price auction}
First price: reporting valuation truthfully is not an equilibrium. Second
price: truthful reporting is equilibrium.

How do we generalize this to more situations? The key idea is that in a second
price auction, our payment is independent of what we report. The allocation
might depend on our payment, but payment does not. How can we have more
DSIC mechanisms?

\section{Groves theorem}
TODO: fill up groves theorem



Three families A B C, can go to Munnar or Simla. 


\begin{tabular}{l r r l}
  &  Manali &  Shimoga & \\
Alice & -1 &  10 & \\
Bob & 5  & -2  & \\
Claire & 5  & 4  & (Claire is a kid, loves vacations) \\
\end{tabular}

We want to get this information truthfully, by using VCG/Groves mechanism.

there are two outcoomes, M or S . If we go to M, the tuility is 5+5-1=9. If
we choose S, it is 10-2+4=12. so S is allocatively efficient.



\begin{tabular}{l r r r r r r r}
    & $\{ A \}$  & $\{ B \}$   & $\{ C \}$  &  $\{ A, B \}$ & $\{ A, C \}$ & $\{ B, C  \}$ & $\{  A, B, C \}$ \\
$P_1$  & 10 & 0   & 5  &  10  &  20 &   5  &  20 \\
$P_2$  & 0  & 9   & 15  &  9  &  15 &  20  &  20 \\
$P_3$  & 10 &  2   & 2  &  10  &  12 &  2  &   28 \\
$P_4$  & 8  &  3   & 3  &  8  &   8 &   3  &    8
\end{tabular}


Giving $A$ to $P_1$ and $BC$ to $P_2$ gives $10 + 20 = 30$.


A direct revelation mechanism in which $f$ satisfies allocative efficiency
and the groves payment scheme is knows as the groves mechanism.

before this, there is another mechanism called as Clarke's mechanism

\section{Clarke's mechanism}

$h_i(\theta_i) = \sum_{j \neq i} v_j(k_{-i}^\star(\theta_{-i}, \theta_j)) \forall \theta_{-i} \in \Theta_{-i}$

That is, each agent $i$ receives
$$
t_i(\theta) = \sum_{j \neq i}(v_j(k^\star(\theta), \theta_j)) - \sum_{j \neq i} v_j(k^\star_{-i}(\theta_{-i}), \theta_j))
$$

This works for combinatorial auctions as well. It's a generalization
of second-price auction.

\begin{tabular}{l r r l}
  & M & S & \\
A &-1  &10 &  \\
B & 5  &-2 & \\
C & 5 & 4 &  (C is a kid, loves vacations) \\
\end{tabular}

For player A, first consider:

\begin{tabular}{l r r l}
   & M  & S & \\
A  & -  & -  & \\
B  & 5  & -2 & \\
C  & 5  & 4  & (C is a kid, loves vacations) \\
\end{tabular}

AE is M. 

Following Clarke Mechanism:
\begin{align*}
t_A = &[\text{valuation of remaining agents at allocatively efficient outcome without A}](-2+4)  \\
     & - [\text{valuation of remaining agents at allocatively efficient outcome with A}][5+5] \\
     = 8
\end{align*}


for player B, first consider:

\begin{tabular}{cccc}
    A    & -1 &  10 & \\
B    &  - & -1 &  -   \\
C    &  5 & -1 &  4  \\
     &  M & -1 &  S 
\end{tabular}

AE is S.  So, $t_B = 0$. Similarly, $t_C = 0$.


\bibliographystyle{apalike}
\bibliography{igt20}

\end{document}


\section{Codeforces 1546D}

\begin{align*}
&f[p, m] = f[p-1, m-2] + f[p, m-1] \\
&\sum_{p\geq 1, m \geq 2}f[p, m] = \sum_{p \geq 1, m \geq 2} f[p-1, m-2] + f[p, m-1] \\
&\sum_{p\geq 1, m \geq 2}f[p, m] = \sum_{p \geq 1, m \geq 2} f[p-1, m-2] + f[p, m-1] \\
\end{align*}

\begin{minted}{text}
(  p0, m0) (  p0, m1) (  p0, m2)  (  p0, m3) (  p0, m4)
(  p1, m0) (  p1, m1) (* p1, m2)  (* p1, m3) (* p1, m4)
(  p2, m0) (  p2, m1) (* p2, m2)  (* p2, m3) (* p2, m4)
(  p3, m0) (  p3, m1) (* p3, m2)  (* p3, m3) (* p3, m4)
(  p4, m0) (  p4, m1) (* p4, m2)  (* p4, m3) (* p4, m4)
(  p5, m0) (  p5, m1) (* p5, m2)  (* p5, m3) (* p5, m4)
\end{minted}

{\footnotesize
\begin{align*}
&\sum_{p\geq 1, m \geq 2}f[p, m] x^p y^m = \\
&g(x, y) - \sum_{p=0, m \geq 0} f[p, m] x^py^m - \sum_{p\geq 0, m = 0} f[p, m] x^py^m  - \sum_{p \geq 0, m = 1} f[p, m] x^py^m + f[p:0,m:0] + f[p:0, m:1] y \\
&g(x, y) - \sum_{m \geq 0} f[p=0, m] x^py^m - \sum_{p\geq 0} f[p, m=0] x^py^m  - \sum_{p \geq 0} f[p, m=1] x^py^m + f[p=0,m=0] + f[p:0, m=1] y \\
&g(x, y) - \sum_{m \geq 0} (f[p=0, m] = 1)y^m - \sum_{p\geq 0} (f[p, m=0] = \delta_0^p) x^p  - \sum_{p \geq 0} (f[p, m=1] = \delta_0^p) x^py + f[p=0,m=0] + f[p:0, m=1] y \\
&g(x, y) - 1/(1-y) - 1  - y + 1 + y \\
&g(x, y) - 1/(1-y)  \\
\end{align*}
}

(take pairwise intersection of summations. $(p=0, m \geq 0) \cap  (p=1, m \geq 0) = \emptyset$, $(p = 0, m \geq 0) \cap (m = 0, p \geq 0) = (p:0, m:0)$ and so on.

\begin{align*}
&\sum_{p \geq 1, m \geq 2} f[p-1, m-2] x^p y^m
&= xy^2 \sum_{p \geq 2, m \geq 1} f[p-1, m-2] x^{p-1} y^{m-2} = xy^2 g(x, y)
\end{align*}

\begin{align*}
&\sum_{p \geq 1, m \geq 2} f[p, m-1] x^p y^m      \\
&= y \sum_{p \geq 1, m \geq 2} f[p, m-1] x^p y^{m-1} \\
&= y \sum_{p \geq 1, m' \geq 1} f[p, m'] x^p y^{m'} \\
&= y \left( g(x, y) - \sum_{p\geq 0} f[p, m'=0]x^p - \sum_{m' \geq 0} f[p=0, m']y^{m'} + f(p=0, m=0) \right) \\
&= y \left( g(x, y) - \sum_{p\geq 0} (f[p, m'=0] = \delta_p^0 )x^p  - \sum_{m' \geq 0} (f[p=0, m']=1) \cdot y^{m'} + (f(p=0, m=0)=1) \right) \\
&= y \left( g(x, y) - x^0  - 1/(1-y) + 1 \right) \\
&= y g(x, y) - y/(1-y)  \\
\end{align*}

This means the recurrence is:

\begin{align*}
&g(x, y) - 1/(1-y) =  xy^2 g(x, y)  +  \left[ y g(x, y) - y/(1-y) \right] \\
&g(x, y) \left( 1 - xy^2 - y\right) = -y(1-y) + 1/(1-y) \\
&g(x, y) \left( 1 - xy^2 - y\right) = (1-y)/(1-y) = 1 \\
&g(x, y) = \frac{1}{1 - (xy^2 + y)} = \frac{1}{(1 - y) - xy^2} \\
&g(x, y) = \frac{1}{1 - y} \left( \frac{1}{1 - \left ( \frac{y^2}{1 - y} \right) \times x} \right) \\
&g(x, y) = \frac{1}{1-y} \left ( \sum_{p \geq 0} x^p {y^{2p}}{(1-y)^{-p}}  \right)    \\
&g(x, y) = \sum_{p \geq 0} x^p {y^{2p}}{(1-y)^{-(p+1)}} \\
&g(x, y) = \sum_{p \geq 0} x^p {y^{2p}}\sum_{j=0}^\infty \binom{(p+1)+j-1}{j}y^j  \\
&g(x, y) = \sum_{p \geq 0}\sum_{j=0}^\infty  x^p \binom{p+j}{j}y^{2p+j}  \\
\end{align*}

Let $m = 2p + j$. $p$ is fixed by the summation, so we have $j = m-2p$.

\begin{align*}
&g(x, y) = \sum_{p \geq 0}\sum_{j=0}^\infty  x^p \binom{p+j}{j}y^{2p+j}  \\
&g(x, y) = \sum_{p \geq 0}\sum_{j=0}^\infty  x^p \binom{p+(m-2p)}{j}y^{2p+(m-2p)}  \\
&g(x, y) = \sum_{p \geq 0}\sum_{j=0}^\infty  x^p \binom{m-p}{m-2p}y^{m}  \\
&g(x, y) = \sum_{p \geq 0}\sum_{j=0}^\infty  x^p \binom{m-p}{(m - p) - (m-2p)}y^{m}  \\
&g(x, y) = \sum_{p \geq 0}\sum_{j=0}^\infty  x^p \binom{m-p}{p}y^{m}  \\
\end{align*}


So the solution is $f(m, p) \equiv \binom{m-p}{p}$
\end{document}

