\chapter{Applications of Tail Inequalities}


\section{Set balancing problem}

Consider trying to divide a data set into two parts: a test set and a training
set. To make sure that both are roughly similar, we want to divide it so that
both data sets have the same number of data items for any given feature.
Similar requirelements also occur for drug trials.

\begin{itemize}
\item We will think of $n$ data items with $n$ features (we can generalize this to
$m$ features), arranged in a matrix $A$. $A$ has entries from $\{0, 1\}$.
Rows are features, columns are data items

\item The goal is to find a vector $x$ of size $n$ with entries from $\{-1, 1\}$
such that $Ax$ has the has the smallest possible maximum absolute entry.

\item Rows with $+1$ in $x$ belong to one class, and those with $-1$ belong
to another class.

\item The maximum absolute value of each entry in $Ax$ tells us how many
data items differ at feature $i$ according to the division by $x$.
\end{itemize}

\textbf{TODO: setup example}
