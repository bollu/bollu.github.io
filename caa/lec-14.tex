\chapter{Parallel Graph algorithms}

We now move to parallel graph algorithms. We will view
recent work on 1-connectivity and 2-connectivity.

A graph is 1-connected if every pair of vertices has a path between them.


In a sequential model, DFS/BFS can be used.
In the parallel setting, we know that DFS cannot be parallelized.
BFS can be, but it is inefficient to do so (We will see this later).
So, we need new approaches to this problem.

A connected component is a subset of vertices $V_i$ such that every
pair of vertices in $V_i$ have a path between them.

The algorithm we study has some resemblance with the union-find
algorithm.


\section{The algorithm for 1-connectivity}
The algorithm is by \textbf{Chandra, Sarwate, Hirschberg}.

\subsection{Intuition}
\begin{itemize}
\item Consider an initial set of rooted trees where each tree contains a single vertex.
\item Eventually, each rooted tree will correspond to a connected component.

\item Two trees can be combined into a bigger tree if these trees contain
vertices $u$ and $v$ which belong to different trees, and the edge
$(u, v) \in E(G)$. The next iteration proceeds with trees merged from the previous iterations.

\item The parallelism is in merging the trees
\end{itemize}

Instead of calling it a tree, we call it a \textit{super-vertex}.

We define the \textit{graph for an iteration} as the graph
of super-vertices and edges between the super-vertices.

\begin{itemize}
	\item $G_0 = G$
	\item $G_i$ is the graph with super-vertices at iteration $i$. We construct
	$G_{i + 1}$ from $G_i$.
\end{itemize}

Important questions:


\begin{itemize}
	\item How to represent and arrange the super-vertices?
	\item How do we build the graph for the next iteration?
	\item How many iterations do we need?
	\item What is the time and work complexity of the algorithm?
\end{itemize}

\subsection{How to represent the matrix?}
We will use an adjaceny matrix to start with. Initially, the matrix is of size
$n \times n$ where $n = |V(G)|$.

If the graph at the start of the $k$th iteration has $n_k$ vertices, then
the matrix is of size $n_k \times n_k$.

We will refer to this matrix as $A_k$.

$A_k[u, v] = 0$ means that the super-vertices do not share an edge. $A_k[u, v] = 1$ if $u$ and $v$ do share an edge.

\subsection{How do we build the graph for the next iteration?}
We will make use of \textit{concurrent writes} to create the
matrix $A_{k + 1}$ from the graph $G_k$.

In $G_k$, if there exist two distinct super-vertices $u_s$ and $v_s$ such that
a vertex $u$ in $u_s$ and a vertex $v$ in $v_s$  and the edge $(u, v)$
is in $E(G)$.

\subsection{How do we arrange the super-vertices?}
Each vertex of $G$ is given a label ($label: V(G) \to \mathbb{N}$, $label$ is injective) so that if $label(u) = label(v)$, then
$u$ and $v$ are part of the same super-vertex.

The common label used for all vertices in the super-vertex will be the label
of the smallest numbered vertex in the super-vertex.

\begin{itemize}
\item We set this up such that \textbf{the root of every tree is the node with the smallest id}.
\item As we combine two trees to make a bigger tree, we will make the tree
with the lower root id as the parent.
\item We use \textit{pointer jumping}f to adjust the labels.
\end{itemize}


\subsection{The merging algorithm}
We define a function:
\begin{align*}
&C: V \to V \\
&C(v) = min \{ label(w)~\vert~A[v, w] = 1\}
\end{align*}

In the first iteration, starting with an initial set of $n$ trees, we merge
trees as follows:

\begin{itemize}
	\item C creates a forest of trees on $V$ and with $E = \{(v, C(v))~\vert~v \in V(G)\}$.
	\item $C$ partitions $V$ such that all vertices in the same connected component are in the same partition.
	\item Each cycle in the forest is either a self-loop or of length 2.
	\item We now use pointer jumping to make everyone in a tree agree on a representative.
\end{itemize}

\subsection{The algorithm}

\begin{minted}{python}
	A_0 = A
	n_0 = n
	k = 0
	while not done:
		k = k + 1
		for v in V pardo:
			C(v) = min {w | A[k - 1][v][w] == 1}

			Shrink each tree in the forest

		pass
\end{minted}

\subsection{Analysis}
\begin{itemize}
\item Number of iterations: We can show that in each iteration till the ened,
the number of super vertices decreases by a factor of two. So,
the total number of iterations is $O(\log n)$.

\item Time spent in each iteration: In each iteration, we need
to do a pointer jumping across the forest, This takes $O(\log n)$time
and $O(n)$ work.

\item Total time: $O(\log^2 n)$.
\item Work: $O(n + m)$ ($n = |V|$, $m = |E|$)
\end{itemize}

There are better algorithm that reduce the time to $O(\log n)$ by
\textbf{Shiloach and Vishkin} --- Don't perform aggressive pointer jumping every round, but perform one step of the pointer jumping each round.


\section{k-Connectivity}
Famous result by \textbf{Cherian and Thirumella}:

A graph is $k$-connected iff the subgraph $H$ is k-connected, where $H$ is:
\begin{align*}
&T_1 = BFS(G) \\
&T_2 = BFS(G / T_1) \\
&T_3 = BFS(G / (T_1 \cup T_2) ) \\
&T_k = \dots  \\
&H = T_1 \cup T_2 \dots T_k \\
\end{align*}

Note that each of the $T_i$ are disjoint, and each $T_i$
may have $n$ edges, so $H$ has only $kn$ vertices. This is
drastically better that $|E| = O(n^2)$.

The bottleneck in practice for this is $BFS$.

Current work tries to replace the $BFS$ with other structures, and then
we repair the damage later on.