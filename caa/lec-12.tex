\chapter{Tree processing}
\section{Traversal via an Euler tour}
\begin{definition}
an \textbf{Euler tour} is a cycle of a graph that includes every edge of the
graph exactly once.
\end{definition}

\begin{lemma}
A directed graph $G$ \textbf{has an Euler tour} iff for every vertex,its in-degree
equals its out-degree.
\end{lemma}

For a tree $T = (V, E)$, to define an euler tour, we make it a directed graph.
$T_e = (V_e, E_e)$, where $V_e = V$, and $E_e = \cup_{(u,v) \in V} \{ (u, v), (v, u) \}$
That is, each $(u, v)$ in $E$ creates two edges $(u, v)$, and $(v, u)$ in $E_e$.
$T_e$ will have an Euler tour.


We have to define a successor function $s: E_e \to E_e$. Here, the successor for an edge.
For a node $u$ in $T_e$, order its \textbf{neighbours (both incoming and outgoing)}
$v_1, v_2, \dots v_d$. This can be done \textbf{independently at each node}. 
For $e = (v_i, u)$, set $s(e) = (u, v_{i + 1~\text{mod $d$}})$. This choice of $s$ is valid
since we always have both edges $(x, y)$ and $(y, x)$, and we are therefore
assured that $(v_i, u)$ will be an incoming edge, and $(u, v_{i + 1}$ will be
an outgoing edge. Also, compute $i + 1$ modulo $d$, so that we eventually
cycle.

\textbf{TODO: relabel vertices to $[0..(d - 1)]$ so that modulo works properly}
\textbf{TODO: add example}

\begin{theorem}
$s$ actually constructs a tour.
\end{theorem}
\begin{proof}
Induction on number of vertices. If $n = 1$, obviously true. 
If $n = 2$, at most one edge present. We will go along the edge and come back,
which is a valid tour.


Let the tour be well defined for $n = k$. We will prove it for $n = k + 1$.
Every tree has at least one leaf, call it $l$. Create a tree $T' = T/\{l\}$.

Let $u$ be a neighbour of $l$ in $T$.
Let $N(u) = \{ v_0, v_1, \dots v_i = l, v_{i + 1}, \dots v_d \}$.
Set $s_{new}(u, v) \equiv (v, u)$. Set $s_{new}(v_{i - 1}, u) \equiv (u, v)$.
For all other vertices, $s_{new}(e) = s(e)$.
\end{proof}


\section{Using euler tours for traversal}
Operations on a tree such a rooting, preorder, and postorder traversal
can be converted to routines on an Euler tour.

\subsection{Rooting a tree}
Designate a node in a tree as the root. All edges in the tree are
directed towards (or away) from the root.

\begin{itemize}
\item let $\{v_1, v_2, \dots v_d\}$ be the neighbours of root node $r$. 
\item we mark the final edge of the tour as \texttt{NIL}, so we get an
Euler path, and not an Euler tour.
\item the edge $(r, v_i)$ appears before $(v_i, r)$.
\item so the edge $parent \to child$ appears before $child \to parent$
\item So, if $uv$ precedes $vu$, then set $u = parent(v)$. Orient the
edge $uv$ as $v \to u$ (that is, $child \to parent$), since we want all edges towards the root.
\end{itemize}
