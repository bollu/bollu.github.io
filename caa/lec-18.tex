\chapter{Applications of Tail Inequalities - 2}


\section{Polynomial verification}
check that $P_1(x) P_2(x) =_? P_3(x)$

If both polynomials have degree $n$, we can make it work in $n \log n$ using
FFT. We will design an algorithm faster than this.


\begin{itemize}
    \item Let $S \subset F$ be a subset of size at least $2n + 1$.
    \item We evaluate $P_1(s) P_2(s)$, and $P_3(s)$ for $s \in S$, s chosen uniformly
    at random (using Horner's method, this is $O(n)$ per point). The evaluations
    are the fingerprintts.
    
    \item Clearly, if $P_3(x) = P_1(x) P_2(x)$, this item will not make a
    mistake. This algorithm \textit{makes a mistake} if $P_3(x) \neq P_1(x) P_2(x)$, but the
    points we have in $S$ fail to catch this.
    \item The probability that this makes a mistake: We create a new polynomial
    $$Q(x) \equiv P_3(x) - P_1(x) P_2(x)$$
    
    It's degree is at most $2n$. If $P_3(x) \neq P_1(x) P_2(x)$, then $Q(x)$ is a
    nonzero polynomial.

    \item The polynomial $Q(x)$ has at most $2n$ roots. So, The probability
    that $Q(r) = 0$ has probability $2n/|S|$, which is the probability of the
    error.

    \item We can make the error rate polynomially small in $n$ by using
    repeated trials, or by picking a larger $S$.
\end{itemize}

